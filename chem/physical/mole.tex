\chapter{Mole Concept}

\section{Basic Definitions}

We'll begin our advent into chemistry by considering some basic definitions.


\begin{definition}
    The smallest independent unit of matter is an \vocab{atom}.
\end{definition}

I mention the term ``independent'' to highlight the fact that an atom is an independent unit. It can exist solely by itself. This is in contrast to 
more fundamental particles, which do not exist in stable forms independently.  

\begin{definition}
    When atoms combine chemically in fixed ratios, they are called \vocab{molecules}.
\end{definition}

Some example of molecules are, \(\ce{H2O}\), \(\ce{NH3}\), \dots

Atoms don't always combine chemically, they may also be held together with
electrostatic forces. These are termed ionic compounds. 

These aren't found in some discrete units as molecular compounds are. Therefore,
we consider another term to describe them,

\begin{definition}
    A \vocab{formula unit} is the smallest unit of a non-molecular substance, such as an ionic compound, covalent network solid, or metal.
\end{definition}

Some examples of them are salts, like \(\ce{NaCl}\). 

\begin{definition}
    A species of atoms; all atoms with the same number of protons in the atomic nucleus.
\end{definition}

An element is not its atom or a molecule or any of that sort. The element 
oxygen refers to any species of atoms, all having the atomic number \(8\). It
is not necessary to distinguish between an element and its atom, molecule, etc. but we'll
do it for the sake of clarity. 

A molecule consisting of \(\ce{O2}\) and the atom \(\ce{O}\) are both referred to as
oxygen. 

\begin{definition}
    A \vocab{compound} is a chemical substance containing identical molecules of 
    more than one type of atom.
\end{definition}

An example would be something like \(\ce{CO2}\). Note that while 
some molecules like of \(\ce{CO2}\) are compounds, some like \(\ce{O2}\) aren't.

\begin{definition}
    Allotropy is the property of some chemical elements to exist in two or more different forms, 
    in which there atoms are bonded in different manner, in the same physical state, known as \vocab{allotropes} of the elements.
\end{definition}

For instance, \(\ce{O2}\) and \(\ce{O3}\) are allotropes. 

\section{Weight units}

The mass of atoms are measured in accordance to
the unified atomic mass unit, ``\uch''.

\begin{definition}
    The unified atomic mass unit, ``\uch'' is defined as
    \[1 \uch = \frac{1}{12} \text{ mass of } \ce{^12C}\] 
    atom.
\end{definition}

The \vocab{atomic mass} of atoms 
is thus measured in terms of \uch.

Experimentally, \(1 \uch = 1.66 \times 10^{-24} \si{\gram}\). 

Next, we define \vocab{Avogadro's number} to be the 
number of atoms of \(\ce{^12C}\) having mass \(12\) \si{\gram}.

This can be calculated rather simply, and comes out to be \(1/1.66 \times 10^{-24}\)
which is \(6.022 \times 10^23 = N_A\).

The \vocab{gram atomic mass} of an element is the
mass of Avogadro's number of its atoms. If its atomic mass is \(x \uch\), 
a trivial calculation leads us to conclude that its gram atomic mass is
\(x\) \si{\gram}.

By using Avogadro's number, we defined a very important quantity called
the mole.

\begin{definition}
    A \vocab{mole} of any entity consists of Avogadro's number, \(6.02214076 \times 10^{23}\) such entities.
\end{definition}

Molecular mass and gram molecular mass are defined similarly. Where we calculate
the mass of a molecule by adding the atomic masses of its atoms.

We define a very specific term using mole.

\begin{definition}
    The \vocab{molar mass} of any entity is simply the mass of a mole of that entity.
\end{definition}

We use the term \vocab{formula unit mass} for defining the mass 
of one ``formula'' of a non-molecular substance. Where the formula unit 
mass is simply the addition of the masses of the entities that make up the
substance in their lowest possible ratios.

For example, the formula unit mass of \(\ce{NaCl}\) is simply 
the mass of \(\ce{Na+}\) plus the mass of \(\ce{Cl-}\). 

A mole of such a formula unit is simply the mole of its constituent entities in that ratio.
1 mole of \(\ce{NaCl}\) is 1 mole of \(\ce{Na+}\) ions and 1 mole of \(\ce{Cl-}\) ions. 
One mole of say \(\ce{Na2SO4}\) is 2 moles of \(\ce{Na+}\) and one mole of \(\ce{SO4^{2-}}\).


\section{On Gases}

Consider the following figure.

\begin{marginfigure}
    \centering
    \begin{tikzpicture}
        \def\Ra{0.5}
        \def\Rb{1.0}
        \def\ra{0.1}
        \def\rb{0.2}
        \def\w{0.08} % wall thickness
        \def\x{2.9}  % piston position
        \def\L{3.7}  % container length
        \def\l{2.2}  % piston arm length
        \def\v{0.68} % velocity
        
        % WALL
        \draw[wall]
          (0,\Rb) -- (0,-\Rb) --++ (\L,0) arc (-90:90:{\Ra} and {\Rb}) -- cycle;
        \draw[walldark] (0,0) ellipse ({\Ra} and \Rb);
        
        % SHELL
        \draw[walldark]
          (0,\Rb) rectangle++ (\L,\w);
        \draw[walldark]
          (0,-\Rb) rectangle++ (\L,-\w);
        \draw[walldark]
          (\L,\Rb+\w) arc (90:-90:{\Ra+\w} and {\Rb+\w}) --++ (0,\w) arc (-90:90:{\Ra} and {\Rb}) -- cycle;
        \draw[walldark]
          (0,\Rb) arc (90:270:{\Ra} and {\Rb}) --++ (0,-\w) arc (-90:-270:{\Ra+\w} and {\Rb+\w}) -- cycle;
        
        % PISTON
        \draw[walldark]
          (\x,\Rb) arc (90:270:{\Ra} and {\Rb}) --++ (-2*\w,0) arc (-90:-270:{\Ra} and {\Rb}) -- cycle;
        \draw[piston] (\x,0) ellipse ({\Ra} and \Rb);
        \draw[piston]
          (\x,\rb) arc (90:270:{\ra} and {\rb}) --++ (\l,0) --++ (0,2*\rb) -- cycle;
        \draw[walldark] (\x+\l,0) ellipse ({\ra} and \rb);
        
        % LABELS
        \draw[->,very thick,orange!90!black] (\x,0.5*\Rb) --++ (0.2*\L,0)
          node[right=-2,orange!90!black] {$P$};
        \node[right,blue!60!black,above] at (\L/2-\Ra,\Rb+\w) {$V$, $P$, $T$};
        \draw[<-,thick,blue!60!black] (\x,0.7*\Rb) to[in=-30] (\x,1.2*\Rb)
          node[below=3,above left] {$A$};
        
        % GAS PARTICLE
        \pic at (-0.12*\x, 0.2*\Rb) {gasparticle={vec={ -40:0.7*\v}}};
        \pic at (-0.07*\x,-0.5*\Rb) {gasparticle={vec={  48:0.6*\v}}};
        \pic at ( 0.00*\x, 0.3*\Rb) {gasparticle={vec={ 105:0.6*\v}}};
        \pic at ( 0.05*\x,-0.5*\Rb) {gasparticle={vec={-100:0.6*\v}}};
        \pic at ( 0.08*\x, 0.0*\Rb) {gasparticle={vec={  70:0.5*\v}}};
        \pic at ( 0.07*\x, 0.7*\Rb) {gasparticle={vec={ -10:0.9*\v}}};
        \pic at ( 0.15*\x,-0.2*\Rb) {gasparticle={vec={  30:0.7*\v}}};
        \pic at ( 0.20*\x,-0.8*\Rb) {gasparticle={vec={ -10:0.6*\v}}};
        \pic at ( 0.35*\x, 0.6*\Rb) {gasparticle={vec={-110:0.7*\v}}};
        \pic at ( 0.35*\x,-0.6*\Rb) {gasparticle={vec={ 140:0.4*\v}}};
        \pic at ( 0.40*\x, 0.9*\Rb) {gasparticle={vec={ -40:0.7*\v}}};
        \pic at ( 0.43*\x,-0.2*\Rb) {gasparticle={vec={  75:0.8*\v}}};
        \pic at ( 0.50*\x, 0.5*\Rb) {gasparticle={vec={-170:0.5*\v}}};
        \pic at ( 0.52*\x,-0.7*\Rb) {gasparticle={vec={ 120:0.6*\v}}};
        \pic at ( 0.60*\x, 0.4*\Rb) {gasparticle={vec={ -80:0.5*\v}}};
        \pic at ( 0.63*\x,-0.6*\Rb) {gasparticle={vec={  42:0.5*\v}}};
        \pic at ( 0.65*\x,-0.2*\Rb) {gasparticle={vec={ 150:0.6*\v}}};
        \pic at ( 0.68*\x,-0.8*\Rb) {gasparticle={vec={ 190:0.5*\v}}};
        \pic at ( 0.72*\x, 0.8*\Rb) {gasparticle={vec={ 160:0.5*\v}}};
        \pic at ( 0.72*\x, 0.3*\Rb) {gasparticle={vec={  80:0.6*\v}}};
        
      \end{tikzpicture}
      
      
      % PISTON
      \begin{tikzpicture}
        \def\Ra{0.45}
        \def\Rb{1.10}
        \def\ra{0.20}
        \def\rb{0.25}
        \def\w{0.12}  % wall thickness
        \def\l{2}     % piston length
        \def\ang{140} % momentum angle
        \def\p{1.3}   % momentum length
        
        % PISTON
    \end{tikzpicture}
    \caption{A Piston}
\end{marginfigure}

What we have here is a piston pressing down on gas molecules.
Pistons are either freely movable, such that the external pressure
is able to equilibrate itself with the internal pressure, or they're
fixed unless another external force is applied.

Consider a fixed option now. If we press down on the piston, decreasing its volume, the pressure that the gas exerts
increases because the atoms become much more energetic since they collide much more.

Using this rather vague idea, we may say that pressure is inversely proportional to
volume under fixed temperature and suitable conditions. 

Therefore, 
\[ V \propto \frac{1}{P}\]

If we heat the container, note that if we keep the pressure constant,
the piston must move up to balance the pressure, increasing the volume of the gas.

Thus, 
\[ V \propto T\]

Also, if week temperature and pressure constance, a simple increase
in the number of moles of the gas results in an increase in volume. 

\marginnote{\(n\) is used to represent the number of moles.}

Which gives us,
\[ V \propto n\]

Putting these together, 
\[ PV = nRT\]
where \(R\), called the ideal gas constant is the proportionality constant.

This derivation is a sham but we'll do it for now and learn how many holes we went through
to derive it.

The unit of pressure we generally use are bar and atm. 

Where, 
\begin{align*}
    1 \si{\bar} &= 10^5 \, \si{\newton\per\meter\squared} \\
    1 \mathrm{atm} &= 1.013 \times 10^5 \, \si{\newton\per\meter\squared}
\end{align*}

For specific units, the values of \(R\) are,

\begin{align*}
    R &= 0.0821 \, \si{\liter} - \mathrm{atm} \si{\per\kelvin\per\mole} \\
    &= 8.314 \, \si{\joule\per\kelvin\per\mole} \\
    &= 0.8314 \, \si{\liter\bar\per\kelvin\per\mole} \\
    &= 2 \, \mathrm{cal}\,\si{\per\kelvin\per\mole}
\end{align*}

\section{Standards of temperature and pressure}

We talk about two more things here.

\begin{definition}
  \vocab{Standard Temperature and Pressure}, STP, is
  the standard at which the pressure is \(1\) \si{\bar} and 
  temperature is \(273.15\) \si{\kelvin}.
  By some unimportant calculations, the volume of 1 mole 
  of a gas at STP is \(22.7\) \si{\mole}.
\end{definition}

STP is a neat standard for gas laws and what not but it is better
to consider SATP for standard room temperature stuff.

\begin{definition}
  The \vocab{Standard Ambient Temperature and Pressure} is the 
  standard at which pressure is \(1\) atm and the temperature is \(25\) \si{\degree\celsius}.
\end{definition}