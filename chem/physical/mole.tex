\chapter{Mole Concept}

\section{Basic Definitions}

We'll begin our advent into chemistry by considering some basic definitions.


\begin{definition}
    The smallest independent unit of matter is an \vocab{atom}.
\end{definition}

I mention the term ``independent'' to highlight the fact that an atom is an independent unit. It can exist solely by itself. This is in contrast to 
more fundamental particles, which do not exist in stable forms independently.  

\begin{definition}
    When atoms combine chemically in fixed ratios, they are called \vocab{molecules}.
\end{definition}

Some example of molecules are, \(\ce{H2O}\), \(\ce{NH3}\), \dots

Atoms don't always combine chemically, they may also be held together with
electrostatic forces. These are termed ionic compounds. 

These aren't found in some discrete units as molecular compounds are. Therefore,
we consider another term to describe them,

\begin{definition}
    A \vocab{formula unit} is the smallest unit of a non-molecular substance, such as an ionic compound, covalent network solid, or metal.
\end{definition}

Some examples of them are salts, like \(\ce{NaCl}\). 

\begin{definition}
    A species of atoms; all atoms with the same number of protons in the atomic nucleus.
\end{definition}

An element is not its atom or a molecule or any of that sort. The element 
oxygen refers to any species of atoms, all having the atomic number \(8\). It
is not necessary to distinguish between an element and its atom, molecule, etc. but we'll
do it for the sake of clarity. 

A molecule consisting of \(\ce{O2}\) and the atom \(\ce{O}\) are both referred to as
oxygen. 

\begin{definition}
    A \vocab{compound} is a chemical substance containing identical molecules of 
    more than one type of atom.
\end{definition}

An example would be something like \(\ce{CO2}\). Note that while 
some molecules like of \(\ce{CO2}\) are compounds, some like \(\ce{O2}\) aren't.

\begin{definition}
    Allotropy is the property of some chemical elements to exist in two or more different forms, 
    in which there atoms are bonded in different manner, in the same physical state, known as \vocab{allotropes} of the elements.
\end{definition}

For instance, \(\ce{O2}\) and \(\ce{O3}\) are allotropes. 

\section{Weight units}

The mass of atoms are measured in accordance to
the unified atomic mass unit, ``\uch''.

\begin{definition}
    The unified atomic mass unit, ``\uch'' is defined as
    \[1 \uch = \frac{1}{12} \text{ mass of } \ce{^12C}\] 
    atom.
\end{definition}

The \vocab{atomic mass} of atoms 
is thus measured in terms of \uch.

Experimentally, \(1 \uch = 1.66 \times 10^{-24} \si{\gram}\). 

Next, we define \vocab{Avogadro's number} to be the 
number of atoms of \(\ce{^12C}\) having mass \(12\) \si{\gram}.

This can be calculated rather simply, and comes out to be \(1/1.66 \times 10^{-24}\)
which is \(6.022 \times 10^23 = N_A\).

The \vocab{gram atomic mass} of an element is the
mass of Avogadro's number of its atoms. If its atomic mass is \(x \uch\), 
a trivial calculation leads us to conclude that its gram atomic mass is
\(x\) \si{\gram}.

By using Avogadro's number, we defined a very important quantity called
the mole.

\begin{definition}
    A \vocab{mole} of any entity consists of Avogadro's number, \(6.02214076 \times 10^{23}\) such entities.
\end{definition}

Molecular mass and gram molecular mass are defined similarly. Where we calculate
the mass of a molecule by adding the atomic masses of its atoms.

We define a very specific term using mole.

\begin{definition}
    The \vocab{molar mass} of any entity is simply the mass of a mole of that entity.
\end{definition}

We use the term \vocab{formula unit mass} for defining the mass 
of one ``formula'' of a non-molecular substance. Where the formula unit 
mass is simply the addition of the masses of the entities that make up the
substance in their lowest possible ratios.

For example, the formula unit mass of \(\ce{NaCl}\) is simply 
the mass of \(\ce{Na+}\) plus the mass of \(\ce{Cl-}\). 

A mole of such a formula unit is simply the mole of its constituent entities in that ratio.
1 mole of \(\ce{NaCl}\) is 1 mole of \(\ce{Na+}\) ions and 1 mole of \(\ce{Cl-}\) ions. 
One mole of say \(\ce{Na2SO4}\) is 2 moles of \(\ce{Na+}\) and one mole of \(\ce{SO4^{2-}}\).


\section{On Gases}

Consider the following figure.

\begin{marginfigure}
    \centering
    \begin{tikzpicture}
        \def\Ra{0.5}
        \def\Rb{1.0}
        \def\ra{0.1}
        \def\rb{0.2}
        \def\w{0.08} % wall thickness
        \def\x{2.9}  % piston position
        \def\L{3.7}  % container length
        \def\l{2.2}  % piston arm length
        \def\v{0.68} % velocity
        
        % WALL
        \draw[wall]
          (0,\Rb) -- (0,-\Rb) --++ (\L,0) arc (-90:90:{\Ra} and {\Rb}) -- cycle;
        \draw[walldark] (0,0) ellipse ({\Ra} and \Rb);
        
        % SHELL
        \draw[walldark]
          (0,\Rb) rectangle++ (\L,\w);
        \draw[walldark]
          (0,-\Rb) rectangle++ (\L,-\w);
        \draw[walldark]
          (\L,\Rb+\w) arc (90:-90:{\Ra+\w} and {\Rb+\w}) --++ (0,\w) arc (-90:90:{\Ra} and {\Rb}) -- cycle;
        \draw[walldark]
          (0,\Rb) arc (90:270:{\Ra} and {\Rb}) --++ (0,-\w) arc (-90:-270:{\Ra+\w} and {\Rb+\w}) -- cycle;
        
        % PISTON
        \draw[walldark]
          (\x,\Rb) arc (90:270:{\Ra} and {\Rb}) --++ (-2*\w,0) arc (-90:-270:{\Ra} and {\Rb}) -- cycle;
        \draw[piston] (\x,0) ellipse ({\Ra} and \Rb);
        \draw[piston]
          (\x,\rb) arc (90:270:{\ra} and {\rb}) --++ (\l,0) --++ (0,2*\rb) -- cycle;
        \draw[walldark] (\x+\l,0) ellipse ({\ra} and \rb);
        
        % LABELS
        \draw[->,very thick,orange!90!black] (\x,0.5*\Rb) --++ (0.2*\L,0)
          node[right=-2,orange!90!black] {$P$};
        \node[right,blue!60!black,above] at (\L/2-\Ra,\Rb+\w) {$V$, $P$, $T$};
        \draw[<-,thick,blue!60!black] (\x,0.7*\Rb) to[in=-30] (\x,1.2*\Rb)
          node[below=3,above left] {$A$};
        
        % GAS PARTICLE
        \pic at (-0.12*\x, 0.2*\Rb) {gasparticle={vec={ -40:0.7*\v}}};
        \pic at (-0.07*\x,-0.5*\Rb) {gasparticle={vec={  48:0.6*\v}}};
        \pic at ( 0.00*\x, 0.3*\Rb) {gasparticle={vec={ 105:0.6*\v}}};
        \pic at ( 0.05*\x,-0.5*\Rb) {gasparticle={vec={-100:0.6*\v}}};
        \pic at ( 0.08*\x, 0.0*\Rb) {gasparticle={vec={  70:0.5*\v}}};
        \pic at ( 0.07*\x, 0.7*\Rb) {gasparticle={vec={ -10:0.9*\v}}};
        \pic at ( 0.15*\x,-0.2*\Rb) {gasparticle={vec={  30:0.7*\v}}};
        \pic at ( 0.20*\x,-0.8*\Rb) {gasparticle={vec={ -10:0.6*\v}}};
        \pic at ( 0.35*\x, 0.6*\Rb) {gasparticle={vec={-110:0.7*\v}}};
        \pic at ( 0.35*\x,-0.6*\Rb) {gasparticle={vec={ 140:0.4*\v}}};
        \pic at ( 0.40*\x, 0.9*\Rb) {gasparticle={vec={ -40:0.7*\v}}};
        \pic at ( 0.43*\x,-0.2*\Rb) {gasparticle={vec={  75:0.8*\v}}};
        \pic at ( 0.50*\x, 0.5*\Rb) {gasparticle={vec={-170:0.5*\v}}};
        \pic at ( 0.52*\x,-0.7*\Rb) {gasparticle={vec={ 120:0.6*\v}}};
        \pic at ( 0.60*\x, 0.4*\Rb) {gasparticle={vec={ -80:0.5*\v}}};
        \pic at ( 0.63*\x,-0.6*\Rb) {gasparticle={vec={  42:0.5*\v}}};
        \pic at ( 0.65*\x,-0.2*\Rb) {gasparticle={vec={ 150:0.6*\v}}};
        \pic at ( 0.68*\x,-0.8*\Rb) {gasparticle={vec={ 190:0.5*\v}}};
        \pic at ( 0.72*\x, 0.8*\Rb) {gasparticle={vec={ 160:0.5*\v}}};
        \pic at ( 0.72*\x, 0.3*\Rb) {gasparticle={vec={  80:0.6*\v}}};
        
      \end{tikzpicture}
      
      
      % PISTON
      \begin{tikzpicture}
        \def\Ra{0.45}
        \def\Rb{1.10}
        \def\ra{0.20}
        \def\rb{0.25}
        \def\w{0.12}  % wall thickness
        \def\l{2}     % piston length
        \def\ang{140} % momentum angle
        \def\p{1.3}   % momentum length
        
        % PISTON
    \end{tikzpicture}
    \caption{A Piston}
\end{marginfigure}

What we have here is a piston pressing down on gas molecules.
Pistons are either freely movable, such that the external pressure
is able to equilibrate itself with the internal pressure, or they're
fixed unless another external force is applied.

Consider a fixed option now. If we press down on the piston, decreasing its volume, the pressure that the gas exerts
increases because the atoms become much more energetic since they collide much more.

Using this rather vague idea, we may say that pressure is inversely proportional to
volume under fixed temperature and suitable conditions. 

Therefore, 
\[ V \propto \frac{1}{P}\]

If we heat the container, note that if we keep the pressure constant,
the piston must move up to balance the pressure, increasing the volume of the gas.

Thus, 
\[ V \propto T\]

Also, if week temperature and pressure constance, a simple increase
in the number of moles of the gas results in an increase in volume. 

\marginnote{\(n\) is used to represent the number of moles.}

Which gives us,
\[ V \propto n\]

Putting these together, 
\[ PV = nRT\]
where \(R\), called the ideal gas constant is the proportionality constant.

This derivation is a sham but we'll do it for now and learn how many holes we went through
to derive it.

The unit of pressure we generally use are bar and atm. 

Where, 
\begin{align*}
    1 \si{\bar} &= 10^5 \, \si{\newton\per\meter\squared} \\
    1 \mathrm{atm} &= 1.013 \times 10^5 \, \si{\newton\per\meter\squared}
\end{align*}

For specific units, the values of \(R\) are,

\begin{align*}
    R &= 0.0821 \, \si{\liter} - \mathrm{atm} \si{\per\kelvin\per\mole} \\
    &= 8.314 \, \si{\joule\per\kelvin\per\mole} \\
    &= 0.8314 \, \si{\liter\bar\per\kelvin\per\mole} \\
    &= 2 \, \mathrm{cal}\,\si{\per\kelvin\per\mole}
\end{align*}

\section{Standards of temperature and pressure}

We talk about two more things here.

\begin{definition}
  \vocab{Standard Temperature and Pressure}, STP, is
  the standard at which the pressure is \(1\) \si{\bar} and 
  temperature is \(273.15\) \si{\kelvin}.
  By some unimportant calculations, the volume of 1 mole 
  of a gas at STP is \(22.7\) \si{\mole}.
\end{definition}

STP is a neat standard for gas laws and what not but it is better
to consider SATP for standard room temperature stuff.

\begin{definition}
  The \vocab{Standard Ambient Temperature and Pressure} is the 
  standard at which pressure is \(1\) atm and the temperature is \(25\) \si{\degree\celsius}.
\end{definition}

\section{Solutions}

\begin{definition}
  A \vocab{solution} is a homogenous mixture of two substances.
\end{definition}

We won't talk about what exactly mixtures are here. But anyways,
a solution consists of two things --- \vocab{solute} and \vocab{solvent}.

If they are of different different states, for instance one solid and one liquid,
then the substance whose state is same as that of the solution is the solvent.

If they have the same state then the substance which is present in a larger
amount is the solvent. 

\section{Concentration Terms}

Concentration terms are used to give a measure of the concentration ---
the amount of solute in a solution.

We use many terms for it, particularly,

\begin{enumerate}
  \item Molarity (M)
  \item Molality (m)
  \item Mole Fraction (\chi)
  \item Percentage Composition --- weight by weight, volume by volume and weight by volume
  \item Parts Per Million (PPM) for extremely dilute solutions.
  \item Normality (N)
\end{enumerate}

\subsection{Molarity}

\vocab{Molarity} refers to the amount(no. of moles) of solute in \(1\) \si{\liter} solution.

Thus, 

\[\text{Molarity} = \frac{\text{No. of moles of Solvent}}{\text{Volume of Solution in litre}}\]

Molarity is an intensive property, it is homogenous across any volume of solution. 
We denote molarity by writing `M'. 

Note that if the molar mass of the solvent is \(m_B\) in grams and the total mass is 
\(M_B\), then \(m_Bn = M_B\). Rewriting in volume of solution in mililitres,

\begin{equation}
  M = \frac{1000 M_B}{m_B \times v_{ml}}
\end{equation}

\subsection{Molality}

Molality is the no. of moles of solute in \(1\) \si{\kg} \emph{solvent}.
We denote it by the letter `m'.

If the mass of solute be \(M_B\) and that of solvent be \(M_A\) both in grams,

\begin{equation}
    m = \frac{1000M_B}{m_B M_A}
\end{equation}

It is also an intensive property.

\subsection{Mole Fraction}

If two substances \(A\) and \(B\) are fixed. Then, the mole fractions of 
\(A\) and \(B\), represented \(\chi_{A}\) and \(\chi_{B}\) are,

\begin{align*}
  \chi_A &= \frac{n_A}{n_A + n_B} \\
  \chi_B &= \frac{n_B}{n_A + n_B}
\end{align*}

Clearly, \(\chi_A + \chi_B = 1\).

\subsection{Percentage Compositions}

\subsubsection{Weight by Weight}

It is the percentage of weight of solute in the weight of solution(in the same units).

\[ \% w/w = \frac{M_B}{M_A + M_B} \times 100\].

Or, \(w/w \si{\g}\) of solute in \(100 \si{\g}\) of solution.

\subsubsection{Weight by Volume}

It is percentage of weight of solute in grams in the volume of solution in mililitres.

\[ \% w/v = \frac{M_B(\si{g})}{V(\si{\ml})} \times 100\]

Or, \(w/v \si{\g}\) of solute in \(100 \si{\ml}\) of solution.

\subsubsection{Volume by Volume}

It is the percentage of volume of solute in the volume of solution(in the same units).

\[ \% = \frac{V_B}{V} \times 100\]

Or, \(v/v \si{\ml}\) of solute in \(100 \si{\ml}\) of solution.

Generally we use volume by volume for liquid solute and solution, and the others
for other solutions. 

\subsection{Parts Per Million}

It is used for extremely dilute solutions, where the molarity and other terms
are extremely small.

We talk about the weight of solute in one million\textsuperscript{th} part of solution.
It is denoted by PPM. 

\begin{equation}
  PPM = \frac{M_B}{M_A + M_B} \times 10^6
\end{equation}

\section{Molarity of Solutions}

If two solutions are mixed, one of molarity and volume, \(M_1, V_1\) (volume 
in \si{\ml}) and other 
of \(M_2, V_2\). Then we not that the solute in their mixture 
will have mass, \((M_1V_1 + M_2V_2)/1000\). Since the volume 
is just \(V_1+V_2\), 

\begin{equation}
  M = \frac{M_1V_1 + M_2V_2}{V_1 + V_2}
\end{equation}

The \(1000\)s cancel out.

\section{Some special concentration terms}

\subsection{Volume Strength of \(\ce{H2O2}\)}

It is defined and used specifically for \(\ce{H2O2}\). \(\ce{H2O2}\) is a liquid 
and specifically decomposes slowly at room temperature. 

It is particularly useful as a bleaching, cleaning agent. We are 
mainly concerned here with how much oxygen it releases. This is because 
its uses, the bleaching and cleansing action is done by nascent(very recently formed) oxygen.

It is represented as \(x\)V. Where \(x\) is the max number of litres that can be released 
by \(1\)\si{\liter} at STP.

\begin{equation*}
  \ce{2 H2O2(aq) -> 2H2O(l) + O2 (^) }
\end{equation*}

Note that \(2\) moles of \(\ce{H2O2}\) give \(1\) mole of \(\ce{O2}\). For \(M\) moles of 
\(\ce{H2O2}\) in \(1\)\si{\liter} solution, or \(M\) molarity, we have \(M/2\) moles of oxygen. Clearly at STP, this implies, 

\begin{equation}
  x \text{V} = \frac{M}{2} \times 22.7 = M \times 11.35,
\end{equation}

where \(x\)V is the volume strength of \(\ce{H2O2}\) and \(M\) is its molarity.

\subsection{Percentage labelling of Oleum}

During the preparation of sulphuric acid, \(\ce{H2SO4}\), some impurity of 
\(\ce{SO3}\) is present. This mixture is known as \vocab{Oleum}.

Consider the reaction of oleum with water, 

\begin{equation*}
  \ce{H2SO4 + SO3 + H2O -> 2H2SO4},
\end{equation*}

where \(\ce{H2SO4}\) does not react with \(\ce{H2O}\). The percentage labelling thus, is 
the maximum amount of \(\ce{H2SO4}\) that can be produced by adding sufficient water to \(100\)\si{\g} oleum.

By doing some analysis, we may equivalently say that if the percentage labelling is \(100 + y\)\%, then, 
\(y\) grams of water react with \(\ce{SO3}\) to form \(\ce{H2SO4}\).

Thus, since water is just the sufficient amount, moles of water will be exactly equal to 
moles of \(\ce{SO3}\) which will just be \(y/18\). We can find the weight of \(\ce{SO3}\) by 
doing this. 

To check if these definitions are equivalent, not that if there is \(x\)\si{\g}
\(\ce{H2SO4}\), then we will have \((100 - x)\)\si{\g} \(\ce{SO3}\) in the sample.

Thus, now you need to find the number of moles of \(\ce{SO3}\) (\(100 - x/80\)) since these 
are just equal to moles of resultant sulphuric acid, the total weight of sulphuric 
acid at the end is just going to be 
\begin{equation*}
  x + (100 - x)/80 \times 98 = 100 + y
\end{equation*}

Solve for \(x\), find moles of \(\ce{SO3}\), and since they're equal to moles of water, 
find weight of water. It will come out to be \(y\).

Note that the maximum labelling will be \(122.5\)\%, because if we exceed this, 
the weight of \(\ce{SO3}\) we will calculate will exceed \(100\)\si{\g}. 

\section{Stoichiometry}

\vocab{Stoichiometry} is essentially the study of the 
relationships in weights and thus moles of reactants and products.

For instance, consider 

\begin{equation*}
  \ce{xA + yB -> zC},
\end{equation*}

which essentially means that \(x\) moles of \(\ce{A}\) react with \(y\) moles of 
\(\ce{B}\) to give \(z\) moles of \(\ce{C}\). We can even convert this to weights, 
if required, using the definition of moles.

If we have say now \(10x\) and \(10y\) moles of both, then we will, simply get 
\(10z\) of \(\ce{C}\). 

We now define something called the \vocab{limiting reagent}. Consider if we have \(10x\) moles 
of \(\ce{A}\) and \(20y\) moles of \(\ce{B}\). How many moles of \(ce{C}\) will we have?

The thing here is that we don't have sufficient \(\ce{A}\) to react with all moles of 
\(\ce{B}\). \(\ce{A}\) is exhausted, and \(\ce{B}\) is left un-reacted, in excess. We say thus, that \(\ce{A}\) is the limiting reagent, and the 
moles of the product depend upon this quantity. We are not --- note that, concerned 
with which is present in lesser weight. We are concerned with relative to the number of moles 
that react which other, which is present in the lesser amount.

In the context of our example, if \(n_A\) are moles of \(\ce{A}\) and \(n_B\) of \(\ce{B}\),
then which of \(n_A/x\) and \(n_B/y\) is lower.

\section{Eudiometry}

An \vocab{Eudiometre} measures the volume of gases. We note that using our previous 
example, 

\begin{equation*}
  \ce{xA + yB -> zC},
\end{equation*}

we note that at STP, \(x = x22.7\text{L}\), \(y = y22.7\text{L}\) and \(z = z22.7\text{L}\).
Normalising this by dividing the equation by \(22.7\) and replacing moles with the appropriate litres,
we note that we require \(x\)\si{\l} of \(\ce{A}\), \(y\)\si{L} of \(\ce{B}\) to form 
\(z\)\si{\l} of \(\ce{C}\). Thus, we dont really need to convert into moles to 
form a relationship between their volumes. This is called \vocab{Eudiometry}.  

Note that is only true if each of the reactants and products are gasses.