\chapter{Sequence and Series}

\section{Sequences}

A sequence is essentially an ordered collection of objects in which repetitions are allowed and order matters. 
Like a set, it contains members (also called elements, or terms). 
The number of elements (possibly infinite) is called the length of the sequence. 

Unlike a set, the same elements can appear multiple times at different positions in a sequence, 
and unlike a set, the order does matter.

Formally, a sequence is a function, \(f : \NN \to \RR\) where \(f(i) = a_i\), the 
\(i\) term in the sequence. 

Sequences can be infinite or finite, depending on their length. Adding 
or subtracting the terms of the sequence results in a series, for instance, 
if \(\seq{a}{n}\), then, \(a_1 + a_2 + \dots a_n\) is a series.

\section{Progression}

The sequences whose \(n\)th term are described by any particular formula are called 
progressions. We refer to any arbitrary term in a sequence, as an \(n\)th term.

\begin{example}
    Consider the formula for the \(n\)th term, \(t_n\) to be \(t_{n-1} + t_{n-2}\). If 
    the first two terms of the sequence are \(0\) and \(1\), write down the sequence.
    
    \begin{soln}
        The sequence is simply \(0, 1, 1, 2, 3, 5, 8, 13 \dots\). This sequence is called 
        the Fibonacci sequence. The ratio the elements here tends to the golden 
        ration, \(\phi = 1.618\dots\)
    \end{soln}
\end{example}
 
\section{Arithmetic Progression}

\vocab{Arithmetic Progression}, which we shall abbreviate as A.P., is as a sequence where 
any two adjacent terms have the same difference, \(d\). 

In particular, 

\begin{equation}
    t_n = a + (n-1)d,
\end{equation}

where the sequence is just \(\seqq{a, a+d, a+2d, a+3d, \dots}\). \(a\) is the first term 
and \(d\) is called the \chvocab{Arithmetic Progression}{common difference}.

\begin{example}
    Show that \(\seqq{\log(a), \log(ab), \log(ab^2) \dots}\) is an A.P., also find its
    \(n\)th term.

    \begin{soln}
        Clearly, the \(n^th\) term is \(\log(ab^{n-1}) = \log(a) + \log(b^{n-1}) = \log(a) + (n-1)\log(b)\).
        This is simply the \(n\)th term of an A.P. with common difference \(\log(b)\) and first 
        term \(\log(a)\).
    \end{soln}
\end{example}

\begin{example}
    Consider two A.P.s, \(\seqq{2, 7, 12, 17, 22 \dots}\) and \(\seqq{1, 8, 15, 22, 29, \dots}\).
    If the first goes upto \(500\) terms, and the second upto \(300\) terms, find the number of common 
    members in them.

    \begin{soln}
        Note that the last term of the first is \(2497\) and of the second is \(2094\). 
        Thus, the last possible common term is \(2094\). 

        Consider the sequence of their common terms, \(\seqq{22, 57, 92, \dots}\). 
        It is an arithmetic progression whose common difference is equal to their \(\lcm\).

        This is so because consider their first common term, say \(a\). Clearly, 
        \(a = 2 + 5(k-1) = 1 + 7(n-1)\). Now, if their next common term is 
        at indexes \(p, m\) such that \(p > k\), \(m > n\), 
        \[2 + 5(p-1) = 1 + 7(m-1) = 2 + 5(p-k) + 5(k-1) = 1 + 7(m-n) + 7(n-1),\]

        \begin{equation*}
            a + 5(p-k) = a + 7(m-n) \iff 5(p-k) = 7(m-n)
        \end{equation*}

        Now it should be evident that \(7 \mid p-k\), \(5 \mid m-n\). Thus, 
        the common term is just \(a + 35i\) for some \(i\), which is just an A.P.!

    \end{soln}
\end{example}

\todo{complete the example}