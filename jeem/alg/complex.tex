\chapter{Complex Numbers}

We will begin our study of complex numbers in a little bit of a 
different manner. Instead of using algebraic form as the base, we will
define a complex number as,

\index{Complex Number}

\begin{definition}
    [Complex Numbers]
    An ordered pair, \((x, y)\) such that,
    \[
        z = (x, y) \in \RR^2 = \left\{(x, y) \mid x, y \in \RR \right\}
    \]
    is called a \emph{complex number}.
\end{definition}

Clearly, \(z_1(x_1, y_1) = z_2(x_1, y_1) \iff x_1 = x_2, y_1 = y_2\).

\marginnote{The letter \(z\) is often used to denote a complex number.}

The operations addition and multiplication are defined simply as, 
\[
    z_1 + z_2 = (x_1, y_1) + (x_2, y_2) = (x_1 + x_2, y_1 + y_2)
\]

\index{Complex Number!multiplication}
And multiplication is defined as,

\[
    z_1 \cdot z_2 = (x_1, y_1) \cdot (x_2, y_2) = (x_1x_2 - y_1y_2, x_1y_2 + x_2y_1)
\]

Although the definition of multiplication may seem complicated now, we will be able
to gain intuition for it after introducing the algebraic form of complex numbers.

\begin{remark}
    Note that,
    \begin{itemize}
        \item \(z_{1} = (x_{1}, 0)\), \(z_2 = (x_2, 0)\). Then, \(z_1z_2 = (x_1x_2, 0)\).
        \item \(z_1 = (0, y_1)\), \(z_2 = (0, y_2)\). Then, \(z_1z_2 = (-y_1y_2, 0)\).
    \end{itemize}
\end{remark}

The usual properties regarding addition and multiplication, commutativity, associativity, and existence of identity
and inverses are true, and can be shown through simple proofs by properties of reals.

\section{Inverses}

The additive inverse of \(z = (x, y)\) is defined \(-z = (-x, -y)\). 

The multiplicative inverse of \(z\), \(z^{-1}\), such that \(zz^{-1} = 1\) is,
\((x,y) \cdot (a,b) = (1, 0)\). We get that,
\[
    \begin{cases}
        ax - yb = 1 \\
        ay + bx = 0 \\
    \end{cases}\] 
 And we simply have,
 \[
    \frac{1}{z} = z^{-1} = \left(\frac{x}{x^2 + y^2}, \frac{y}{x^2 + y^2} \right)
\]

Two complex number \(z_1\), \(z_2\) also \emph{uniquely} determine a quotient,
\[\frac{z_1}{z_2} = z_1z_{2}^{-1} = \left(\frac{x_1x_2 + y_1y_2}{x_{2}^{2} + y_{2}^{2}},
\frac{-x_1y_2 + y_1x_2}{x_{2}^{2} + y_{2}^{2}} \right)\]

The set \(\RR^2\) with the operations addition and multiplication forms the field of 
complex numbers, \(\CC\). 

Powers are defined similar to reals,
\[
    z^n = \underbrace{z\cdot z \cdot \dots \cdot z}_{n \, \text{times}}\]%
for all integers \(n > 0\). For \(n < 0\), \(z^{n} = (z^{-1})^{-n}\). 

\marginnote{We shall use \(\CC^*\) to show \(\CC \setminus \{0\}\).}

\begin{proposition}
    For \(z_1, z_2, z \in \CC^*\), \(a, b \in \ZZ\),
    \begin{itemize}
        \item \(z^m \cdot z^n = z^{m + n}\).
        \item \(\dfrac{z^m}{z^n} = z^{m-n}\).
        \item \((z^m)^n = z^{mn}\).
        \item \((z_1z_2)^n = z_{1}^n \cdot z_{2}^n\).
    \end{itemize}
\end{proposition}

\section{Algebraic Form of Complex Numbers}

To deal with complex numbers, we introduce their algebraic form. 
Consider a function \(f\),
\[
    f \colon \RR \to \RR \times \{0\}, f(x) = (x, 0)\]%
Clearly such a \(f\) is a bijective. Moreover, 
\[(x, 0) + (y, 0) = (x + y, 0), (x,0) \cdot (y, 0) = (xy, 0)\]

Thus, \(x, 0\) behaves in the same as \(x \in \RR\). Therefore, we may write each \((x, 0) \in 
\RR \times \{0\} = x \in \RR\) since \(f\) is bijective.

\begin{definition}
    We define \(i \in \CC\) as, 
    \[i = (0, 1) \in \CC\]
\end{definition}

Note that, 
\begin{align*}
    z &= (x, y) = (x, 0) + (0, y)  \\ 
    &= (x, 0) + (y, 0) \cdot (0,1) \\
    &= x + yi = x + iy
\end{align*}

\begin{lemma}
    Every complex number, \((x, y)\) can be represented as 
    \[
        z = x + yi \mid x, y \in \RR\]
    where \(x\) is the real part of \(z\) and is denoted \(\Re(z)\). \(y\) is said
    to be the imaginary part of \(z\) and is denoted \(\Im(z)\).
\end{lemma}

\(i^2 = -1\) by definition, since, \(i \cdot i = (0,1) \cdot (0,1) = (-1,0) = -1\).

This representation of complex numbers is called the algebraic representation, where we can
say that \(\CC = \{x + iy \mid x, y \in \RR, i^2 = -1\}\). 

Complex numbers of the form \(0 + iy\) are purely imaginary and the ones of the form \(x + 0i\)
are purely real. 

\begin{tikzcd}
    a\ar[r,"A"]\ar[rr,out=-30,in=210,swap,"C"] & b\ar[r,"B"] & c
\end{tikzcd}