\chapter{Complex Numbers}

We will begin our study of complex numbers in a little bit of a 
different manner. Instead of using algebraic form as the base, we will
first consider the set \(\RR^2\),
\[\RR^2 = \Set{(x,y) \given x, y \in \RR}\]

Clearly, \((x_1, y_1) = (x_1, y_1) \iff x_1 = x_2, y_1 = y_2\) by the definition of 
ordered pairs.

Let us represent the elements as \(z = (x, y)\). Now, let us define the operations of addition and multiplication on this set as, 
\[
    z_1 + z_2 = (x_1, y_1) + (x_2, y_2) = (x_1 + x_2, y_1 + y_2)
\]

\index{Complex Number!multiplication}
And multiplication is defined as,

\[
    z_1 \cdot z_2 = (x_1, y_1) \cdot (x_2, y_2) = (x_1x_2 - y_1y_2, x_1y_2 + x_2y_1)
\]

Although the definition of multiplication may seem complicated now, we will be able
to gain intuition for it soon.

\marginnote{The letter \(z\) is used to represent a complex number.}

\index{Complex Number}
\begin{definition}
    [Complex Numbers]
    An ordered pair, \((x, y)\) such that,
    \[
        z = (x, y) \in \RR^2 = \left\{(x, y) \mid x, y \in \RR \right\}
    \]
    with the operations of addition and multiplication as defined
    above is called a \emph{complex number}.
\end{definition}

The field of complex numbers is the earlier set defined with the earlier operations,
\((\RR^2, +, \cdot)\). 

\begin{remark}
    Note that,
    \begin{itemize}
        \item \(z_{1} = (x_{1}, 0)\), \(z_2 = (x_2, 0)\). Then, \(z_1z_2 = (x_1x_2, 0)\).
        \item \(z_1 = (0, y_1)\), \(z_2 = (0, y_2)\). Then, \(z_1z_2 = (-y_1y_2, 0)\).
    \end{itemize}
\end{remark}

The usual properties regarding addition and multiplication, commutativity, associativity, and existence of identity
and inverses are true, and can be shown through simple proofs by properties of reals.

\section{Inverses}

The additive inverse of \(z = (x, y)\) is defined \(-z = (-x, -y)\). 

The multiplicative inverse of \(z\), \(z^{-1}\), such that \(zz^{-1} = 1\) is,
\((x,y) \cdot (a,b) = (1, 0)\). We get that,
\[
    \begin{cases}
        ax - yb = 1 \\
        ay + bx = 0 \\
    \end{cases}\] 
 And we simply have,
 \[
    \frac{1}{z} = z^{-1} = \left(\frac{x}{x^2 + y^2}, \frac{y}{x^2 + y^2} \right)
\]

Two complex number \(z_1\), \(z_2\) also \emph{uniquely} determine a quotient,
\[\frac{z_1}{z_2} = z_1z_{2}^{-1} = \left(\frac{x_1x_2 + y_1y_2}{x_{2}^{2} + y_{2}^{2}},
\frac{-x_1y_2 + y_1x_2}{x_{2}^{2} + y_{2}^{2}} \right)\]

The set \(\RR^2\) with the operations addition and multiplication forms the field of 
complex numbers, \(\CC\). 

Powers are defined similar to reals,
\[
    z^n = \underbrace{z\cdot z \cdot \dots \cdot z}_{n \, \text{times}}\]%
for all integers \(n > 0\). For \(n < 0\), \(z^{n} = (z^{-1})^{-n}\). 

\marginnote{We shall use \(\CC^*\) to show \(\CC \setminus \{0\}\).}

\begin{proposition}
    For \(z_1, z_2, z \in \CC^*\), \(a, b \in \ZZ\),
    \begin{itemize}
        \item \(z^m \cdot z^n = z^{m + n}\).
        \item \(\dfrac{z^m}{z^n} = z^{m-n}\).
        \item \((z^m)^n = z^{mn}\).
        \item \((z_1z_2)^n = z_{1}^n \cdot z_{2}^n\).
    \end{itemize}
\end{proposition}

\section{Algebraic Form of Complex Numbers}

To deal with complex numbers, we introduce their algebraic form. 
Consider a function \(f\),
\[
    f \colon \RR \to \RR \times \{0\}, f(x) = (x, 0)\]%
Clearly such a \(f\) is a bijective. Moreover, 
\[(x, 0) + (y, 0) = (x + y, 0), (x,0) \cdot (y, 0) = (xy, 0)\]

Thus, \((x, 0)\) behaves in the same as \(x \in \RR\). Therefore, we may write each \((x, 0) \in 
\RR \times \{0\} = x \in \RR\) for all \((x,0) \in \RR \times \Set{0}\) since \(f\) is bijective.

\begin{definition}
    We define \(i \in \CC\) as, 
    \[i = (0, 1) \in \CC\]
\end{definition}

By the definition of multiplication and using \(f\) as defined earlier, we may get that
\begin{align*}
    z &= (x, y) = (x, 0) + (0, y)  \\ 
    &= (x, 0) + (y, 0) \cdot (0,1) \\
    &= x + yi = x + iy
\end{align*}

This gives us a fascinating result! We can represent each of the complex numbers in a simple
algebraic form.

\begin{lemma}
    Every complex number, \((x, y)\) can be represented as 
    \[
        z = x + yi \mid x, y \in \RR\]
    where \(x\) is the real part of \(z\) and is denoted \(\Re(z)\). \(y\) is said
    to be the imaginary part of \(z\) and is denoted \(\Im(z)\).
\end{lemma}

\(i^2 = -1\) by definition, since, \(i \cdot i = (0,1) \cdot (0,1) = (-1,0) = -1\).

This representation of complex numbers is called the algebraic representation, where we can
say that \(\CC = \{x + iy \mid x, y \in \RR, i^2 = -1\}\). 

Complex numbers of the form \(0 + iy\) are purely imaginary and the ones of the form \(x + 0i\)
are purely real. 

Using an ordered pair first approach allows us to discern something very interesting, the difference
between \(i\) and \(-i\). Clearly, both are roots of \(x^2 + 1\). However, it is rather
difficult to differentiate them. Let's say \(i = \sqrt{-1}\) and thus, \(-i = -\sqrt{-1}\).
What exactly is the difference between them?

The answer is that it is not so obvious. We can easily differentiate between 
\(\sqrt{a}\) and \(-\sqrt{a}\) for any real \(a\) the former is positive and the latter
negative, because \(\sqrt{\bullet} : \RR \to \RR \), the principal square root function,
is defined to return a positive value. 

Unfortunately, no such thing is generally true for \(\CC\). The complex
numbers aren't ``ordered'' like the reals. You can't establish \(z_1 \ge z_2\) in general,
except for the special case when they are both purely real. So how exactly are they different?

If we don't define \(i\) to \((0,1)\) and instead \(\sqrt{-1}\), we can't
establish a difference. After all, why would \(i\) lie on the positive \(y\) axis
and not the negative one? In reals, we could this would be true if \(i \ge 0\). 
I mean, isn't that how we plot stuff on cartesian plane anyway? 

But here, the square root definition restricts us a lot. We can't know by it if
\(i\) is \((0,1)\) or \((0,-1)\). Of course, we can define our axes such that it
automatically lies on the positive \(y\) axis, but I feel that the ordered pair
definition is much better, since there we know the difference \(i\) and \(-i\). 
The former is \((0, 1)\) and the later \((0, -1)\). 

And clearly, both are still solutions of \(x^2 + 1 = 0\)! Now of course,
we can define the principal square root, \(\sqrt{-1}\) to be \(i\) but in general,
it is better to avoid it all together because there is little consensus on which of the
two complex roots of such a polynomial should be the principal root.

\subsection{Operations in algebraic form}

The addition of two complex numbers in algebraic form is simply,
\[(x + iy) + (a + ib) = (x+a) + i(y+b)\] Thus, 
\[\Re(z_1 + z_2) = \Re(z_1) + \Re(z_2)\]

Multiplication while not much more intuitive yet, is easy to calculate using \(i^2 = -1\).
\[(x+iy)(a+ib) = (xa - yb) + i(xb + ay)\] Which gives us,
\begin{align*}
    \Re(z_1z_2) &= \Re(z_1)\Re(z_2) - \Im(z_1)\Im(z_2) \\
    \Im(z_1z_2) &= \Re(z_1)\Im(z_2) - \Re(z_2)\Im(z_1)
\end{align*}

\subsection{Powers of \emph{i}}

Using some simple algebraic manipulation,

\begin{align*}
    i &= i, \quad i^2 = -1, \quad i^3 = i^2 \cdot i = -i, \quad i^4 = (i^2)^2 = 1; \\
    i^5 &= i^4 \cdot i = i, \quad i^6 = i^4 \cdot i^2 = -1, \quad \dots  
\end{align*}

Thus, for \(n \in \ZZ\), 
\[i^{4n} = 1, \quad i^{4n+1} = i, \quad i^{4n+2} = -1, \quad i^{4n+3} = -i.\]

Also, we have, 
\[i^{4n} + i^{4n+1} + i^{4n+2} + i^{4n+3} = 0\]

\section{Complex Conjugate}

\begin{definition}
    The complex conjugate, \(\conjugate{z}\) of \(z = x + iy\) is,
    \[\conjugate{z} = x - iy\]
\end{definition}

We can immediately get some properties of complex numbers,
mainly,

\begin{proposition}
    For \(z \in \CC\), and \(\conjugate{z}\) the complex conjugate of \(z\), we have,
    \begin{enumerate}
        \item \(z = \conjugate{z} \iff z \in \RR\).
        \item \(z = -\conj{z} \iff z \in \CC \setminus \RR\)
        \item \(\conj{z_1 + z_2} = \conj{z_1} + \conj{z_2}\)
        \item \(\conj{z_1z_2} = \conj{z_1} \cdot \conj{z_2}\)
        \item \(\conj{z^{-1}} = \conj{z}^{-1} \given z \ne 0\)
        \item  \(\Re(z) = \dfrac{z + \conj{z}}{2}\), \(\Im(z) = \dfrac{z - \conj{z}}{2i}\) 
    \end{enumerate}
\end{proposition}



