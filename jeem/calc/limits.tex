\chapter{Limits}

Limits are what make calculus a dynamic 
disciple. We won't very rigorously discuss limits,
that of course falls in the area of real analysis.

\section{The Neighborhood of a Real Number}

Consider an open interval in which a real number, \(a\) lies, 
\(\qty(a - \delta, a + \delta)\). This interval is 
called the \vocab{neighborhood} of \(a\) of radius \(\delta\).

\section{The Limit}

Let us now define a limit. 

\begin{definition}
If \(f : \RR \to \RR\) is a function, we say that the \vocab{limit} of \(f\), as \(x\) approaches 
\(a\) is \(L\) which is written as

\begin{equation*}
    \lim_{x \to a} f(x) = L,
\end{equation*}

if and only if the following property holds: 
for every real \(\varepsilon > 0\), there exists a real 
\(\delta > 0\) such that for all real \(x, 0 < \abs{x - p} < \delta \) implies \(\abs{f(x) - L} < \varepsilon \).
    
\end{definition}


For instance consider \(f(x) = 2x - 4\). Clearly,
 \[\abs{x - 3} < \delta \implies \abs{(2x - 4) - 2} < 2\delta\]

 Thus, \(\lim_{x \to 3} 2x - 4 = 2\). A bit more dynamically, 
 we consider the value \(f(x)\) approaches as \(x\) approaches some number \(a\). 

 The two ideas are equivalent because \(x - a\) and \(f(x) - L\) always lie on some 
 neighborhood, implied entirely by the neighborhood of \(x - a\).

 Thus, the difference between \(f(x) - L\) can be made as small as possible, by reducing the
 radius of \(x - a\), the distance \(\delta\) from the limiting point. This further implies 
 the idea of ``approach'', \(f(x) \to L\) as \(x \to a\). 
 
 We will not work with epsilon delta definition of limit at all for the remainder of this text.

 We now define two terms, in terms of the idea of approach, the \vocab{left-hand limit}(LHL)
 and the \vocab{right-hand limit}(RHL).

 The left-hand limit is the value \(f(x)\) approaches as \(x\) approaches \(a\) from behind, shown as 
 \(x \to a^-\). The right-hand limit is defined similarly, the value \(f(x)\) approaches as 
 \(x \to a^+\).

 If, 

 \begin{equation*}
    \lim_{x \to a^-} f(x) = \lim_{x \to a^+} f(x) = L,
 \end{equation*}

 we say that

 \begin{equation*}
    \lim_{x \to a} f(x) = L
 \end{equation*}

 For instance consider \(x^2 + 3\) as \(x\) approaches \(2\). We see that if 
 \(x\) approaches \(2\) from behind, \(f(x)\) approaches a number a little less than \(7\),
 denoted \(7^-\). It is important to note that such a notion isn't entirely rigorous but we 
 work in it for now.

 Similarly, as \(x\) approaches \(2\) from the right, from ahead, \(f(x)\) approaches 
 \(7^+\). It can be soon that in both ways, \(f(x)\) actually approaches \(7\).

 The behavior of the function is the same when approached by 
 either side, thus, 

 \begin{equation*}
    \lim_{x \to 2} x^2 + 3 = 7
 \end{equation*}