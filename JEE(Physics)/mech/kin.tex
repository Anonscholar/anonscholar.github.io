\chapter{Kinematics}

\para{Kinematics is the study of motion without concerning its cause. It allows
us to calculate and find out the evolution of a body. Everything in our world undergoes
motion, one way or the other. To begin our study of motion, let us first
concern ourselves with some definitions.}

\section{Reference Frame and Point Particle}

\para{Motion of any object is considered to relative to a reference frame. We consider motion of
a body with respect to any other body or co-ordinate frame.}

\begin{figure}[ht]
        \centering
        \incfig{3daxes}    
    \caption{A Reference Frame}\label{fig: 3daxes}
\end{figure}

\para{We use a reference frame according to our convenience. In general, we use the 3-dimensional cartesian system like the one
in \Cref{fig: 3daxes}.}

\index{Reference Frame}

\begin{definition}
    [Reference Frame]
    A reference frame is a co-ordinate frame relative to which motion of a particle is considered.
\end{definition}

\para{We always consider the motion of a point particle. Well, why? It is done so to simplify
the calculation of our system. The motion of a (rigid)body is discussed later.}

\begin{definition}
    [Point Particle]
    A particle whose size is negligible in the study of its motion is called a point particle.
\end{definition}

\section{Position Vector and Displacement}

\subsection{Position Vector}
\index{Position Vector}

\para{We define the position of a particle relative to a reference frame using a \emph{position 
vector}. One end of the position vector is the origin of our reference frame. The other is 
the particle itself. Consider \Cref{fig: position}, the 
vector \(\vect{r}\) from \(O\) to \(A\). It describes the position of \(A\) relative to the
co-ordinate frame. Let \(A = (x,y,z)\). Then we denoted \(\vect{r}\) as \(\vect{r}(x,y,z)\).}

\para{The position to any point is written as, \[
    \vect{r} = (x,y,z) = x\unitv{i} + y\unitv{j} + z\unitv{k}
\]}


\marginpar{\raggedleft Note that the position vector is not a \emph{true} vector. It is tied to particular
    reference frame.}

\begin{figure}
    [H]
    \centering
    \scalebox{0.6}{\incfig{position}}
    \caption{A position vector, \(\mathbf{r}\).}
    \label{fig: position}
\end{figure}


\subsection{Displacement}

\para{Consider the movement of a particle from \(A = (x_1,y_1,z_1)\) to \(B = (x_2,y_2,z_2)\). 
The \emph{displacement} defines a true vector \(\vect{S} = (x_2-x_1,y_2-y_1,z_2-z_1)\).
\(\vect{S}\) is called the displacement vector. Note that it contains no information about the
individual points, but only about the relative position of each. Thus, our choice
of reference frame, and thus position vector does not matter when we're concerned with displacement
. In \Cref{fig: displacement}, the vector
\(\vect{S}_{AB}\) defines a displacement from \(A\) to \(B\).}

\begin{figure}
    [H]
    \centering
    \scalebox{0.8}{\incfig{displacement}}
    \caption{Displacement from \(A\) to \(B\)}
    \label{fig: displacement}
\end{figure}