\chapter{Kinematics}

\para{Kinematics is the study of motion without concerning its cause. It allows
us to calculate and find out the evolution of a body. Everything in our world undergoes
motion, one way or the other. To begin our study of motion, let us first
concern ourselves with some definitions.}

\section{Reference Frame and Point Particle}

\para{Motion of any object is considered to relative to a reference frame. We consider motion of
a body with respect to any other body or co-ordinate frame.}

\begin{figure}[ht]
        \centering
        \incfig{3daxes}    
    \caption{A Reference Frame}\label{fig: 3daxes}
\end{figure}

\para{We use a reference frame according to our convenience. In general, we use the 3-dimensional cartesian system like the one
in \Cref{fig: 3daxes}.}

\index{Reference Frame}

\begin{definition}
    [Reference Frame]
    A reference frame is a co-ordinate frame relative to which motion of a particle is considered.
\end{definition}

\para{We always consider the motion of a point particle. Well, why? It is done so to simplify
the calculation of our system. The motion of a (rigid)body is discussed later.}

\index{Particle}
\begin{definition}
    [Point Particle]
    A particle whose size is negligible in the study of its motion is called a point particle.
\end{definition}

\section{Position Vector and Displacement}

\subsection{Position Vector}
\index{Position Vector}

\marginpar{\raggedleft One thing of importance is that a position vector of a particle is a function
of time. We generally denote this as \(r(t)\) to show that is a function.}

\para{We define the position of a particle relative to a reference frame using a \emph{position 
vector}. One end of the position vector is the origin of our reference frame. The other is 
the particle itself. Consider \Cref{fig: position}, the 
vector \(\vect{r}\) from \(O\) to \(A\). It describes the position of \(A\) relative to the
co-ordinate frame. Let \(A = (x,y,z)\). Then we denoted \(\vect{r}\) as \(\vect{r}(x,y,z)\).}

\para{The position to any point is written as, \[
    \vect{r} = (x,y,z) = x\unitv{i} + y\unitv{j} + z\unitv{k}
\]}



\begin{figure}
    [H]
    \centering
    \scalebox{0.6}{\incfig{position}}
    \caption{A position vector, \(\mathbf{r}\).}
    \label{fig: position}
\end{figure}


\marginpar{\raggedleft Note that the position vector is not a \emph{true} vector. It is tied to particular
    reference frame.}

\subsection{Displacement}
\index{Displacement}

\para{Consider the movement of a particle from \(A = (x_1,y_1,z_1)\) to \(B = (x_2,y_2,z_2)\). 
The \emph{displacement} defines a true vector \(\vect{S} = (x_2-x_1,y_2-y_1,z_2-z_1)\).
\(\vect{S}\) is called the displacement vector. Note that it contains no information about the
individual points, but only about the relative position of each. Thus, our choice
of reference frame, and thus position vector does not matter when we're concerned with displacement
. In \Cref{fig: displacement}, the vector
\(\vect{S}_{AB}\) defines a displacement from \(A\) to \(B\).}

\begin{figure}
    [H]
    \centering
    \scalebox{0.8}{\incfig{displacement}}
    \caption{Displacement from \(A\) to \(B\)}
    \label{fig: displacement}
\end{figure}

\section{Speed and Velocity}

\index{Speed}

\para{We are already familiar with the notion of speed. In elementary terms, speed is \(\dfrac{distance}{time}\).
Speed itself is an elementary concept. It tells us nothing about the direction of the object whose speed we are
talking into consideration. Thus, it is a \emph{scalar} quantity.}

\parbreak
\marginpar{\raggedright The symbol \(\equiv\) stands for `defined as'.}

\index{Velocity}

\para{The Velocity of a particle, contrastingly, contains information about both the speed and direction of the
object in consideration. It is a vector defined as,} 

\begin{equation} 
    \label{eq: velocity}
    \vect{v} \equiv \ndiv{\vect{r}}
\end{equation}
\marginpar{\raggedright The dot over \(r\) in \(\dot{\vect{r}}\) tells us that \(r\) has been differentiated
with respect to time.}

\para{Sometimes, we do not need to compute the derivative of the position vector, and simply need an average
estimate, or in the case of uniformly accelerated motion, it is not required that we computer derivatives.
In such a case we use the concept of \textbf{average velocity}, \[
    \avg{\vect{v}}_{12} = \frac{\vect{r}_1-\vect{r}_2}{\increment t}  
\]}

\para{In general velocity is also referred to as `instantaneous velocity'. This is done to remind of the difference
with average velocity. We will not refer to it that way since by velocity we shall always mean \(\vect{v}\) as defined in
\eqref{eq: velocity}. The same goes for speed. Speed is defined as \(\ndiv{d}\) where \(d\) is the distance.
Average speed is defined similar to average velocity.}

\marginpar{\raggedright Magnitude refers to the absolute value of a vector, i.e., its modulus. \(\abs{\vect{a}}\) is the 
magnitude of \(\vect{a}\) and is often simply denoted \(a\).}

\para{There is an interesting relation between speed and velocity. Consider the distance between \(a\) and \(b\) as 
\(a \to b\). The path between them approaches a straight line. Thus, the distance and displacement become the same
in the limiting case. Therefore, the \emph{magnitude} of velocity is the speed at that instant.} 

