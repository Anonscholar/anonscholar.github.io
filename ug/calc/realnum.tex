\chapter{Real Numbers}
\margintoc

\marginnote{\epigraph{``Give him threepence, since he must gain out of what he learns''}{\textup{Euclid} of Alexandaria}}

The real numbers form the basis of all of the calculus we are going to study. Later, we'll look at expansions of them.
To formulate a \emph{deductive} system, we must define real numbers and build up everything over them. Well, in reality
real numbers are arbitrary quantities that follow a certain set of axioms, every mathematical object that satisfies
them is a real number.

\section{Real Number Axioms}

The axioms of real numbers can be divided broadly into three different types,

\begin{itemize}
    \ii The Field axioms
    \ii The Order axioms
    \ii The Continuity axiom
\end{itemize}

\subsection{Field Axioms}

The field axioms are something like this. We consider the set of reals, \(\RR\)
along with the operations \((+, \cdot)\). This particular structure is called a 
\vocab{field}. These axioms permit us to perform our regular operations on 
the reals, namely addition, subtraction, multiplication and division by non-zero
reals.

Although we label the operations as multiplication and addition, these can in effect be 
any two operations, provided they follow the axioms. A field is a general structure, 
and we'll talk much more about it in algebra.

These operations are such that for every \(x\), \(y \in \RR\), we will have \(x + y, xy \in \RR\). 
These sums and products are unique, which means we always have one and only have one 
\(x+y, xy\) for \(x, y \in \RR\). The notion can be generalized, and we will understand what we mean by 
the operations more rigorously in the next chapter, just hold on till then.

\begin{axioms}
    \ii \chvocab{Real Numbers}{Commutativity}, \(x+y = y+x\) and \(xy = yx\).
    \ii \chvocab{Real Numbers}{Associativity}, \(x + (y+z) = (x+y) + z\), and \(x(yz) = (xy)z\).
    \ii \chvocab{Real Numbers}{Distributivity}, \(x(y + z) = xy + xz\).
    \ii \chvocab{Real Numbers}{Identity}. There exist two \emph{distinct} identity elements, \(0\) and \(1\) such that \(x + 0 = x\) and \(x \cdot 1 = x\).
    \ii \chvocab{Real Numbers}{Inverses}, for \(\forall x \in \RR \setminus \{0\}\) there exists a \(y\) such that \(xy = 1\) and \(\forall x \in \RR\),
    there exists a \(z \in \RR\), such that \(x + z = 0\). These are called the reciprocal 
    and negative of \(x\) and are denoted as \(1/x\)  (\(x^{-1}\)) and \(-x\).
\end{axioms}
 
First of all, we will show that the identity elements are unique, that is, if,

\begin{equation*}
    (\forall x),\; x + 0 = x \text{ and } (\forall x),\; x + 0' = x, \text{ then } 0 = 0'
\end{equation*}

and a similar statement for \(1\) over multiplication.

\begin{proposition}
    The identity elements are unique.
\end{proposition}

\begin{proof}
    Let us assume they aren't and \(0\), \(0'\) are two distinct identities.

    Then, 

    \begin{equation*}
        0 = 0 + 0' = 0'
    \end{equation*}

    But this leads to a contradiction! \lightning
\end{proof}

\sidenote{We use the symbol \lightning to show contradiction.}

A similar argument for the multiplicative identity, \(1\) is left to the reader. Its 
not really difficult to do it, once you understand what we did in this proof, which is 
use first use that \(0'\) is an identity for the first equality (so that \(0 + 0' = 0\)) 
and then \(0\) as the identity for the second equality. 

It is possible to formulate subtraction and division from these axioms, if for some real \(a\) and \(b\) we have a real
\(x\) such that \(x + a = b\), then it is denoted as \(b - a\). 
And if for some real \(a\) and \(b\) we have a real \(x\) such that \(xa = b\), then it is denoted as \(ba^{-1}\). 

Such \(x\)'s are also unique. We can show their uniqueness by showing that the inverses are unique.

\begin{proposition}
    The inverses are unique.
\end{proposition}

\begin{proof}
    The idea is this the identity elements are unique, and the multiplication or addition
    by an inverse results in an identity. Let the multiplicative inverse of \(x\) be 
    distinct reals, \(y\) and \(z\), then  

    \begin{equation*}
        y \cdot 1 = y \cdot (x \cdot z) = (y \cdot x) \cdot z = 1 \cdot z = 1
    \end{equation*}

    What did we do here? the first equality follows from the definition 
    of inverse, the second follows from Associativity and the third uses the 
    definition of identity. This shows that \(y = z\), but \(y\) and \(z\) 
    were, infact, distinct! \lightning

\end{proof}

It is also possible to treat subtraction as a shorthand for addition, \(a + (-b) = a - b\) using 
inverses. Division can be treated similarly.

These properties allow us to form laws of addition and multiplication --- such as 
cancellation. We can form an outline of a proof of cancellation as, 

\begin{proof}
    We need to show that \(a + b = a + c \iff b = c\). 
    This can be done simply by adding \(-a\) on both sides. Since addition is unique, 
    and both numbers are equal, so must be their addition with another number. 
\end{proof}

A similar procedure can be done for multiplication.

We see that this can also be used to prove that the product of any two negative numbers is 
positive. Let us first prove that \((-a)b = -ab\). This is rather easy as well, and follows 
from distributivity,
\begin{align*}
    (-a)b + ab &= b[(-a) + a] \\
    (-a)b + ab &= 0 \\
    (-a)b + ab &- ab = -ab\\
    (-a)b = &-ab
\end{align*}

A similar proof can be used to show that \((-a)(-b) = ab\), as 
\begin{align*}
    (-a)(-b) + (-a)b &= -a[(-b) + b] \\
    (-a)(-b) - ab &= 0\\
    (-a)(-b) - ab + ab &= ab\\
    (-a)(-b) &= ab
\end{align*}

Here we use our previous result to prove this one. Now it is obvious that the product of 
any two negatives will be positive --- however, let us first discuss what exactly ``positive''
and ``negative'' mean.

\subsection{Order Axioms}

We now again give some axioms which allow us to \emph{order} our field. 
In this case, the field being the real numbers. This allows us to formulate notions 
of positive and negative. 

\begin{axioms}
    \ii There exists a subset of real numbers, \(\RR^+\) such that \(\forall x \in \RR \setminus \Set{0}\),
     either \(x \in \RR^+\) or \(-x \in \RR^+\).
    \ii If \(x, y \in \RR^+\), both \(xy\) and \(x + y\) are also in \(\RR^+\).
    \ii \(0\) is not in \(\RR^+\).
\end{axioms}

We say that a number \(x\) is positive if it is in \(\RR^+\). We also know define some symbols, 
\begin{align*}
    x > y \text{ means that } x - y \text{ is in } \RR^+ \\
    x < y \text{ means that } y - x \text{ is in } \RR^+ 
\end{align*}

The second definition is unnecessary to be precise, \(x < y\) may as well be 
interpreted as \(y > x\), which is something we have already defined. In any case, 
a field which is imbued with an \emph{ordering relation} such as \(>\) is called 
an \vocab{ordered field}. We will again, talk about relations and what this exactly 
means later.

The symbols \(\ge\) and \(\le\) just imply that ``\dots'' is either greater/lesser or equal to ``\dots''.
Note that weird as it may appear, statements like \(6 \ge 6\) are completely valid. 
The real use of these symbols lie in terms of dealing with variables, or general 
statements.

Consider the following proposition, 

\begin{plainprop}[Trichotomy]
    If we have \(x, y \in \RR\), the one of the following must hold, 
    \begin{enumerate}
        \item \(x < y\)
        \item \(x > y\)
        \item \(x = y\)
    \end{enumerate}
\end{plainprop}

This can be inferred rather directly from the order axioms. Clearly, by the order 
axioms we have that either \(a = 0\), \(a \in \RR^+\) or \(-a \in \RR^+\).
Replace \(a\) by \(x - y\) and we're done.

This also forms the basis of the idea of the absolute value, which is, 

\begin{equation}
    \abs{a} = 
    \begin{dcases}
        a, & a \ge 0 \\
        -a, & a \le 0
    \end{dcases}
\end{equation}

We see that this clearly a positive number. It is also referred to as the euclidean length of 
a number from \(0\) on the number line. Let us consider some serious properties now, 

Consider the following two properties regarding the absolute value,

\begin{plaintheo}
    For \(x, a \in \RR\), \(\abs{x} \le a\) if and only if \( -a \le x \le a\).
\end{plaintheo}

\begin{proof}
    Let us suppose \(\abs{x} \le a\). Then clearly, \(-\abs{x} \ge -a\). But 
    \(\abs{x}\) can only take two values, and is obviously positive. Thus, \(-\abs{x} \le 
    x \le \abs{x}\). Thus, we have that, \(-a \le -\abs{x} \le x \le \abs{x} \le a\) which 
    proves one direction.
    
    For the other direction, suppose \(-a \le x \le a\). Then, if \(x \ge 0\), 
    \(\abs{x} = x \le a\). If \(x < 0\), \(\abs{x} = -x \le a\) since \(x \ge -a \implies -x \le a\).
    In both cases, \(\abs{x} \le a\). 
\end{proof}