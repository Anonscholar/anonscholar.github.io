\chapter{Real Numbers}

The real numbers form the basis of all of the calculus we are going to study. Later, we'll look at expansions of them.
To formulate a \emph{deductive} system, we must define real numbers and build up everything over them. Well, in reality
real numbers are arbitrary quantities that follow a certain set of axioms, every mathematical object that satisfies
them is a real number.

\section{Real Number Axioms}


The axioms of real numbers can be divided broadly into three different types,

\begin{itemize}
    \ii The Field axioms
    \ii The Order axioms
    \ii The Continuity axiom
\end{itemize}

\subsection{Field Axioms}

The field axioms are something like this. We consider the set of reals, \(\RR\) along with the operations \((+, \cdot)\). Broadly,
the \vocab{field} is \((\RR, +, \cdot)\). I'll define what a field is after calculus. These operations are such that for every
\(x\), \(y \in \RR\), we will have \(x + y, xy \in \RR\). These sum and product are unique, which means we always have one 
and only have one \(x+y, xy\) for \(x, y \in \RR\). Then,

\begin{axioms}
    \ii \chvocab{Real Numbers}{Commutativity}, \(x+y = y+x\) and \(xy = yx\).
    \ii \chvocab{Real Numbers}{Associativity}, \(x + (y+z) = (x+y) + z\), and \(x(yz) = (xy)z\).
    \ii \chvocab{Real Numbers}{Distributivity}, \(x(y + z) = xy + xz\).
    \ii \chvocab{Real Numbers}{Identity}. There exist two \emph{distinct} identity elements, \(0\) and \(1\) such that \(x + 0 = x\) and \(x \cdot 1 = x\).
    \ii \chvocab{Real Numbers}{Inverses}, for \(\forall x \in \RR \setminus \{0\}\) there exists a \(y\) such that \(xy = 1\) and \(\forall x \in \RR\),
    there exists a \(z \in \RR\), such that \(x + z = 0\). These are called the reciprocal 
    and negative of \(x\) and are denoted as \(1/x\)  (\(x^{-1}\)) and \(-x\).
\end{axioms}
 
It is possible to formulate subtraction and division from these axioms, if \(x + a = b\), then it is denoted as \(b - a\). 
And if \(xa = b\), it is denoted as \(ba^{-1}\). 

These are also unique. It is not difficult to show that, by the uniqueness of inverses.

It is also possible to treat subtraction as a shorthand for addition, \(a + (-b) = a - b\) using 
inverses. Division can be treated similarly.

These properties allow us to form laws of addition and multiplication --- such as 
cancellation. We can form an outline of a proof of cancellation as, 

\begin{proof}
    We need to show that \(a + b = a + c \iff b = c\). 
    This can be done simply by adding \(-a\) on both sides. Since addition is unique, 
    and both numbers are equal, so must be their addition with another number. 
\end{proof}

A similar procedure can be done for multiplication.

We see that this can also be used to prove that the product of any two negative numbers is 
positive. Let us first prove that \((-a)b = -ab\). This is rather easy as well, and follows 
from distributivity,

\begin{align*}
    (-a)b + ab &= b[(-a) + a] \\
    (-a)b + ab &= 0 \\
    (-a)b + ab - ab &= -ab\\
    (-a)b &= -ab
\end{align*}

A similar proof can be used to show that \((-a)(-b) = ab\), as 

\begin{align*}
    (-a)(-b) + (-a)b &= -a[(-b) + b] \\
    (-a)(-b) - ab &= 0\\
    (-a)(-b) - ab + ab &= ab\\
    (-a)(-b) &= ab
\end{align*}

Here we use our previous result to prove this one. Now it is obvious that the product of 
any two negatives will be positive --- however, let us first discuss what exactly ``positive''
and negative mean.

\subsection{Order Axioms}

The real numbers are ordered, we now present some axioms to look at what we mean by order.

\begin{axioms}
    \ii There exists a subset of real numbers, \(\RR^+\) such that \(\forall x \in \RR \setminus \Set{0}\),
     either \(x \in \RR^+\) or \(-x \in \RR^+\).
    \ii If \(x, y \in \RR^+\), both \(xy\) and \(x + y\) are also in \(\RR^+\).
    \ii 0 is not in \(\RR^+\).
\end{axioms}

We say that a number \(x\) is positive if it is in \(\RR^+\). We also know define some symbols, 

\begin{align*}
    x > y \text{ means that } x - y \text{ is in } \RR^+ \\
    x < y \text{ means that } y - x \text{ is in } \RR^+
\end{align*}

The symbols \(\ge\) and \(\le\) just imply that ``\dots'' is either greater/lesser or equal to ``\dots''.
Note that weird as it may appear, statements like \(6 \ge 6\) are completely valid. 
The real use of these symbols lie in terms of dealing with variables, or general 
statements.

Consider the following proposition, 

\begin{plainprop}[Trichotomy]
    If we have \(x, y \in \RR\), the one of the following must hold, 
    \begin{enumerate}
        \item \(x < y\)
        \item \(x > y\)
        \item \(x = y\)
    \end{enumerate}
\end{plainprop}

This can be inferred rather directly from the order axioms. Clearly, by the order 
axioms we have that either \(a = 0\), \(a \in \RR^+\) or \(-a \in \RR^+\).
Replace \(a\) by \(x - y\) and we're done.

This also forms the basis of the idea of the absolute value, which is, 

\begin{equation}
    \abs{a} = 
    \begin{cases}
    a & a \ge 0 \\
    -a & a \le 0    
    \end{cases}
\end{equation}

We see that this clearly a positive number. It is also referred to as the euclidean length of 
a number from \(0\) on the number line. Let us consider some serious properties now, 

Consider the following two properties regarding the absolute value,

\begin{plaintheo}
    For \(x, a \in \RR\), \(\abs{x} \le a\) if and only if \( -a \le x \le a\).
\end{plaintheo}

\begin{proof}
    Let us suppose \(\abs{x} \le a\). Then clearly, \(-\abs{x} \ge -a\). But 
    \(\abs{x}\) can only take two values, and is obviously positive. Thus, \(-\abs{x} \le 
    x \le \abs{x}\). Thus, we have that, \(-a \le -\abs{x} \le x \le \abs{x} \le a\) which 
    proves one direction.
    
    For the other direction, suppose \(-a \le x \le a\). Then, if \(x \ge 0\), 
    \(\abs{x} = x \le a\). If \(x < 0\), \(\abs{x} = -x \le a\) since \(x \ge -a \implies -x \le a\).
    In both cases, \(\abs{x} \le a\). 
\end{proof}