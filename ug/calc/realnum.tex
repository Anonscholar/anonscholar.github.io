\chapter{Real Numbers}

The real numbers form the basis of all of the calculus we are going to study. Later, we'll look at expansions of them.
To formulate a \emph{deductive} system, we must define real numbers and build up everything over them. Well, in reality
real numbers are arbitrary quantities that follow a certain set of axioms, every mathematical object that satisfies
them is a real number.

\section{Real Number Axioms}

The axioms of real numbers can be divided broadly into three different types,

\begin{itemize}
    \ii The Field axioms
    \ii The Order axioms
    \ii The Continuity axiom
\end{itemize}

The field axioms are something like this. We consider the set of reals, \(\RR\) along with the operations \(+, \cdot\). Broadly,
the \emph{field} \((\RR, +, \cdot)\). I'll define what a field is after calculus. These operations are such that for every
\(x\), \(y \in \RR\), we will have \(x + y, xy \in \RR\). These sum and product are unique, which means we always have one 
and only have one \(x+y, xy\) for \(x, y \in \RR\). Then,

\begin{itemize}
    \ii Commutativity, \(x+y = y+x\) and \(xy = yx\).
    \ii Associativity, \(x + (y+z) = (x+y) + z\), and \(x(yz) = (xy)z\).
    \ii Distributivity, \(x(y + z) = xy + xz\).
    \ii Identity. There exits two \emph{distinct} identity elements, \(0\) and \(1\) such that \(x + 0 = x\) and \(x \cdot 1 = x\).
    \ii Inverses, for \(\forall x \in \RR \setminus \{0\}\) there exists a \(y, z\) such that \(xy = 1\) and \(\forall x \in \RR\), \(x + z = 0\). These are called the reciprocal 
    and negative of \(x\) and are denoted as \(1/x\) and \(-x\).
\end{itemize}
 
It is possible to formulate subtraction and division from these axioms, as \(x + a = b\) to be denoted as \(b - a\). And for 
\(xa = b\) to be denoted as \(ba^{-1}\). 