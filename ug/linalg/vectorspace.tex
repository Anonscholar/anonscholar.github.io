\chapter{Vector Spaces}

This part of linear algebra is almost entirely based on \cite{axlerlin}.

\section{Lists}

In most of the chapters, I'll adopt the convention that \(\FF\) is a field. Some specific results require \(\FF = \CC\) or \(\FF = \RR\)
but they'll be explicitly mentioned.

\index{Scalars}

Elements of \(\FF\) are called \emph{scalars}. We call these scalars to form a distinction with vectors, which will be defined soon.

Cartesian products of \(\FF\) with itself \(n\) times are denoted as \(\FF^{n}\). For something like
\(\FF^3\) this is equivalent to saying,

\[
    \FF^{3} = \left\{(x, y, z) \mid x, y, z \in \FF\right\}
\]

It is the set of all ordered triples of the elements of \(\FF\). If we have \(\FF = \RR\),
for instance, \(\RR^2\) can be though to be a plane while \(\RR^3\) can be thought to be ordinary space.

To generalise these results to \(n\) dimensions, we use the concept of \(n\)-tuples or \emph{lists}.

\index{lists} \index{n-tuples}

\marginnote{We denote lists surrounded by paratheses and their elements separated by commas.}

\begin{definition}[Lists]
    For a non-negative integer, \(n\), an \(n\)-tuple or a list of length \(n\) is an ordered collection of \(n\) elements.
    
    A list \((a_1, a_2, \dots, a_m) = (b_1, b_2, \dots, b_n)\) if and only if \(n = m\) and \(a_1 = b_1, a_2 = b_2, \dots,
    a_m = b_n\).
\end{definition}

Thus, two lists are equal iff their length is equal and they have the same elements in the same order.
A list has a finite number of elements, thus, even though \(a_1, a_2, \dots, a_m\) is a list, \(a_1, a_2, \dots\) is not.
The list of length \(0\) is denoted by \(()\). 

\begin{example}
    [Lists vs sets]
    The lists \((3,5)\) and \((5,3)\) are not equal but \(\left\{3,5\right\} = \left\{5,3\right\}\). 
    The list \((4,4,4) \ne (4,4) \ne (4)\) but \(\left\{4,4,4\right\} = \left\{4,4\right\} = \left\{4,\right\}\).
\end{example}

\section{Higher Products}

Generalising the \(n\)th cartesian product of \(\FF\) with itself, 
\[
    \FF^n = \underbrace{\FF \times \FF \times \dots \times \FF}_{n\textup{-times}}
\]

\begin{example}
    [Lists vs sets]
    The lists \((3,5)\) and \((5,3)\) are not equal but \(\left\{3,5\right\} = \left\{5,3\right\}\). 
    The list \((4,4,4) \ne (4,4) \ne (4)\) but \(\left\{4,4,4\right\} = \left\{4,4\right\} = \left\{4,\right\}\).
\end{example}


We may say,
\begin{definition}
    \(\FF^n\) is the set of all lists of length \(n\) such that the elements of the list are in \(\FF\),
    \[
        \FF^n = \left\{(a_1, a_2, \dots, a_n) \mid a_1, a_2, \dots, a_n \in \FF\right\}
    \]
\end{definition}

Where \(n \in \ZZO\).

\begin{definition}
    [Co-ordinate]
    If \((a_1, a_2, \dots, a_n) \in \FF^n\) and \(1 \le i \le n\), \(x_i\) is called the \(i\)th co-ordinate of
    \((a_1, a_2, \dots, a_n)\).  
\end{definition}

While \(\RR^2\) can be visualised as a plane, and \(\RR^3\) as space, it is not possible to visualise them
for \(n \ge 4\). Similarly, \(\CC^1\) can be thought of as a plane, but cannot be visualised for \(n \ge 2\).

However, we can perform algebraic operations on the lists of some arbitrary length \(n\) which may even be 
very great. 

\begin{definition}
    [Addition in \(\FF^n\)]
    \label{def: addition in fn}
    Addition in \(\FF^n\) is defined as,
    \[
        \listvec{x}{n} + \listvec{y}{n} = (x_1 + y_1, x_2 + y_2, \dots, x_n + y_n)
    \]
\end{definition}

To avoid the cumbersome notation of writing out \(\listvec{x}{n}\) we will adopt the notation that \(x = \listvec{x}{n}\).

\begin{proposition}
    \(x + y = y + x\)
\end{proposition}

\begin{proof}
    \(x = \listvec{x}{n}\), \(y = \listvec{y}{n}\). Thus,
    \begin{align*}
        x + y &= \listvec{x}{n} + \listvec{y}{n}\\
        &= (x_1 + y_1, x_2 + y_2, \dots, x_n + y_n)\\
        &= (y_1 + x_1, y_2 + x_2, \dots, x_n + y_n)\\
        &= y + x
    \end{align*}
\end{proof}

The proof is based on the commutativity of reals and \Cref{def: addition in fn}. 
The elements of \(\FF^2\), \(x, y\) can be thought of as points or vectors. Disregarding
the axes, vectors may independently be thought as some objects.

We define two more things in \(\FF^n\),

\begin{definition}
    [Additive Inverse in \(\FF^n\)]
    The additive inverse of \(x \in \FF^n\) is \(-x \in \FF^n\) where, 
    \[
        x + (-x) = 0
    \]
    And if \(x = \listvec{x}{n}\), 
    \[
        -x = \listvec{-x}{n}
    \]
\end{definition}

The additive inverse of \(x\) in \(\RR^2\) is the vector of equal length but opposite direction.

The final operation is \emph{scalar multiplication}

\index{Scalar multiplication}

\begin{definition}
    [Scalar multiplication]
    For a scalar \(\lambda\), and \(x \in \RR^n\), their product is defined as
    \[
        \lambda \listvec{x}{n} = \listvec{\lambda x}{n}
    \]
\end{definition}

\section{Vector Space}

The definition of a vector space comes off from the properties we have kind of discussed
above for \(\RR^n\). Defining our operations for any vector space,

\begin{definition}
    [Operators on a vector space]
    \emph{Addition} on \(V\) is a function \(+ : V^2 \to V\) where \(\vv + \ww \in V\) for 
    any \(\vect{\vv}, \vect{\ww} \in V\).

    \noindent\emph{Scalar multiplication} on \(V\) is a function \(\bullet : V \times \RR \to V\) where 
    \(\lambda \vv \in V\) for any \(\vv \in V\) and \(\lambda \in \RR\). 
\end{definition}

Now, let us formally define \(V\). 

\begin{definition}
    [Vector Space]
    A set \(V\) along with the operations of addition and scalar multiplication is a vector
    space if the following properties hold for \(\uu, \vv, \ww \in V\) and \(a, b \in \RR\).

    \begin{axioms}
        \item Commutativity, \(\vv + \ww = \ww + \vv\).
        \item Associativity, \((\uu + \ww) + \vv = \uu + (\ww + \vv)\) and \(a(bv) = (ab)\vv\).
        \item Additive Identity, there an element \(0 \in \vv\) such that \(\vv + 0 = \vv\).
        \item Additive Inverse, there exist an element \(-\vv \in V\) for each \(\vv\) such that 
        \(\vv + (-\vv) = 0\).
        \item Multiplicative Identity, \(1v = \vv\)
        \item Distributive Property, \(a(\uu + \vv) = au + av\) and \(\vv(a + b) = va + vb\). 
    \end{axioms}

\end{definition}