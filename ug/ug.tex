\documentclass[twoside, a4paper, 10pt]{memoir}

\usepackage[no-math]{fontspec}
\setmainfont{Baskerville}[
  BoldFont={* Semibold},
]

\usepackage{amsmath, amssymb}

\nonzeroparskip
\usepackage[right=7.5cm,left=2.5cm,bottom=1cm,top=1cm,marginparwidth=5.5cm,marginparsep=1cm,includeheadfoot,asymmetric]{geometry}

\usepackage[texindy]{imakeidx}
\makeindex[intoc]


\usepackage[dvipsnames]{xcolor}

\usepackage{etoolbox}

\usepackage{booktabs, tabularx}
\setlength{\parindent}{0em}

\usepackage{dsfont}

%Chapter Style
\usepackage{calc}
%chapter style

\makeatletter
\def\@makechapterhead#1{%
  \vspace*{2\p@}%
  {\parindent \z@ \raggedleft \reset@font
            \scshape \@chapapp{} \thechapter
        \par\nobreak
        \interlinepenalty\@M
    \Huge \bfseries #1\par\nobreak
    %\vspace*{1\p@}%
    \hrulefill
    \par\nobreak
    \vskip 10\p@
  }}
\def\@makeschapterhead#1{%
  \vspace*{2\p@}%
  {\parindent \z@ \raggedleft \reset@font
            \scshape \vphantom{\@chapapp{} \thechapter}
        \par\nobreak
        \interlinepenalty\@M
    \Huge \bfseries #1\par\nobreak
    %\vspace*{1\p@}%
    \par\nobreak
    \vskip 10\p@
  }}

\usepackage{xparse}
\usepackage{lipsum}

%%%%%%%%%%%%%%%%%%%%%%%%%%%%%%%%%%%%%%%%%%%%%%%%%%%%%%%%%%%%%%%%%%%%%%%%%%%%%%%%%%%%%%%%%%%%%%%%%%%%%%%%%%%%%%%%%%%%%%%%

%page style

\nouppercaseheads
\makepagestyle{mystyle}
\makeevenhead{mystyle}{\textbf{\thepage}\;\;\;{\emph{\leftmark}}}{}{}
\makeoddhead{mystyle}{}{}{\emph{\rightmark}\;\;\;\textbf{\thepage}}

\makeevenfoot{mystyle}{}{}{}
\makeoddfoot{mystyle}{}{}{}
\makepsmarks{mystyle}{%
  \createmark{chapter}{left}{nonumber}{}{}
  \createmark{part}{right}{nonumber}{}{}}
  

\pagestyle{mystyle}

\copypagestyle{chapter}{plain}
\makeoddfoot{chapter}{}{}{\rule[0cm]{0.8cm}{0.12cm}\\\emph{\thepage}}

\copypagestyle{part}{plain}
\makeoddfoot{part}{}{}{\rule[0cm]{0.8cm}{0.12cm}\\\emph{\thepage}}


%some graph stuff
\usepackage{todonotes}
\usepackage{tikz}
\usepackage{tikz-3dplot}
\usetikzlibrary{mindmap}
  \usetikzlibrary{arrows.meta,backgrounds}
  \usetikzlibrary{decorations.pathmorphing,patterns}
  \usetikzlibrary{angles, quotes, intersections}
\usepackage{pgfplots}
\pgfplotsset{width=7cm,compat=1.18}
\pgfplotsset{
    funcgraphbare/.style={
        axis x line=center,
        axis y line=center,
        ticks=none,
    },
    funcgraph/.style={
        funcgraphbare,
        xlabel={\(x\)},
        ylabel={\(y\)},
    },
}
%%%%%%%%%%%%%%%%%%%%%%%%%%%%%%%%%%%%%%%%%%%%%%%%%%%%%%%%%%%%%%%%%%%%%%%%%%%%%%%%%%%%%%%%%%%%%%%%%%%%%%%%%%%%%%%%%%%%%%%%
%packages
\usepackage{physics}
\usepackage{tkz-euclide}
\usepackage{babel}
\usepackage{mhchem}
\usepackage[ddmmyyyy]{datetime}
\renewcommand{\dateseparator}{.}
\usepackage{caption}
\usepackage{paralist}
\usepackage{graphicx}
\usepackage{microtype}
\usepackage{wrapfig}
\usepackage{import}
\usepackage{xifthen}
\usepackage{appendix}

\usepackage[per-mode=symbol]{siunitx}
\usepackage{bohr}
\usepackage{modiagram}
\usepackage{subcaption}


\usepackage{nameref}
\usepackage{tzplot}
\usepackage{enumitem}
\newlist{axioms}{enumerate}{10}
\setlist[axioms]{label=\textbf{A\arabic*.},ref=Axiom\arabic*,leftmargin=*}

\newlist{casework}{enumerate}{10}
\setlist[casework]{label=\textbf{Case\,\Roman*},ref=Axiom\arabic*,leftmargin=*}


\usepackage{ragged2e}
\usepackage{empheq}

\usepackage[linktoc=page,hyperfootnotes=false]{hyperref}
\hypersetup{
colorlinks=true,
linkcolor=TealBlue!60!black,
citecolor=TealBlue!60!black,
urlcolor=TealBlue!60!black,
}
\usepackage[nameinlink]{cleveref}
\usepackage{float}


\usepackage[citestyle=alphabetic]{biblatex}

\usepackage{listings}
\definecolor{codegreen}{rgb}{0,0.6,0}
\definecolor{codegray}{rgb}{0.5,0.5,0.5}
\definecolor{codepurple}{rgb}{0.58,0,0.82}
\definecolor{backcolour}{rgb}{0.95,0.95,0.92}

\lstdefinestyle{mystyle}{
    backgroundcolor=\color{backcolour},   
    commentstyle=\color{codegreen},
    keywordstyle=\color{magenta},
    numberstyle=\tiny\color{codegray},
    stringstyle=\color{codepurple},
    basicstyle=\ttfamily\footnotesize,
    breakatwhitespace=false,         
    breaklines=true,                 
    captionpos=b,                    
    keepspaces=true,                 
    numbers=left,                    
    numbersep=5pt,                  
    showspaces=false,                
    showstringspaces=false,
    showtabs=false,                  
    tabsize=2
}

\lstset{style=mystyle}


\usetikzlibrary{decorations.markings}
\tikzset{
    arrow inside/.style = {
        postaction = {
            decorate,
            decoration={
                markings,
                mark=at position #1 with {\arrow{>}}
            }
        }
    },
    arrow inside/.default = 0.5
}

\raggedbottom
\newcommand{\incfig}[1]{%
    \def\svgwidth{2.4in}
    \scalebox{.5}{\import{../figures/}{#1.pdf_tex}}
}


\epigraphfontsize{\small\itshape}
\setlength\epigraphwidth{6cm}
\setlength\epigraphrule{0pt}

\usepackage{collectbox}

\makeatletter
\newcommand{\mybox}{%
    \collectbox{%
        \setlength{\fboxsep}{1pt}%
        \fbox{\BOXCONTENT}%
    }%
}
\makeatother

\usepackage{cleveref}


\crefformat{section}{\S#2#1#3} % see manual of cleveref, section 8.2.1
\crefformat{subsection}{\S#2#1#3}
\crefformat{subsubsection}{\S#2#1#3}


%%%%%%%%%%%%%%%%%%%%%%%%%%%%%%%%%%%%%%%%%%%%%%%%%%%%%%%%%%%%%%%%%%%%%%%%%%%%%%%%%%%%%%%%%%%%%%%%%%%%%%%%%%%%%%%%%%%%%%%%

%section, part formatting

\renewcommand*{\thepart}{\color{black}\Alph{part}}
\renewcommand*{\parttitlefont}{\color{black}\normalfont\bfseries\HUGE}
\renewcommand*{\partnamefont}{\normalfont\LARGE\itshape}
\renewcommand*{\partnumfont}{\color{black}\normalfont\itshape}

\setsechook{\hangsecnum}

\makeatletter

\renewcommand{\hangsecnum}{%
  \def\@seccntformat##1{%
    \makebox[0pt][r]{%
      \color{black} 
      {%
      \csname the##1\endcsname
      }
      \, 
    }%
  }%
}

\setsecnumdepth{subsection}

\setsubsechook{\hangsecnum}
\setsubsubsechook{\hangsecnum}


\makeatletter                   
\def\printauthor{%                  
    {\large \@author}}          
\makeatother

\usepackage{fontawesome5}

\makeatletter
\newcommand{\github}[1]{%
   \href{#1}{\textcolor{black}{\faGithub}}%
}
\makeatother

\usepackage{authblk}

\author{
    \github{https://github.com/Anonscholar} Ascholar
}

\newcommand*{\email}[1]{%
    \normalsize\href{mailto:#1}{#1}\par
    }

\affil{\email{adyanshmishra@proton.me}}

%%%%%%%%%%%%%%%%%%%%%%%%%%%%%%%%%%%%%%%%%%%%%%%%%%%%%%%%%%%%%%%%%%%%%%%%%%%%%%%%%%%%%%%%%%%%%%%%%%%%%%%%%%%%%%%%%%%%%%%%

%title page

\newcommand*{\titleSW}{\begin{titlingpage}
\newgeometry{left=10cm}
\begingroup% Story of Writing
\raggedleft
\vspace*{\baselineskip}
{\HUGE\itshape \thetitle}\\[\baselineskip]
\vspace{3em}
{\printauthor ~ \\ Version: 0.1\today}
\vfill
\vspace*{\baselineskip}
\endgroup
\restoregeometry
\end{titlingpage}}

\usepackage{ellipsis}

\usetikzlibrary{patterns,decorations.pathmorphing}
\usetikzlibrary{arrows.meta}
\tikzset{>=latex}

\usepackage{framed}
% colors to be used
\definecolor{myred}{RGB}{127,0,0}
\definecolor{myyellow}{RGB}{169,121,69}

% a modification of the leftbar environment defined by the framed package
% will be used to place a vertical colored bar separating the page number and the
% title in chapter entries
\renewenvironment{leftbar}{%
  \def\FrameCommand{\textcolor{myyellow}{\vrule width 1.4pt depth 5pt}\hspace*{15pt}}%
  \MakeFramed{\advance\hsize-\width\FrameRestore}}%
 {\endMakeFramed}

% redefinition of the name of the ToC
\makeatletter
% redefinitions for chapter entries

% redefinitions for part entries
\renewcommand{\partnumberline}[1]{\mbox{\centering\normalfont\itshape\rmfamily Part\thepart~#1}\par\noindent\Large}
\renewcommand\cftpartafterpnum{\vskip1ex}

\makeatletter
\renewcommand*{\l@part}[2]{%
  \ifnum \c@tocdepth >-2\relax
    \cftpartbreak
    \begingroup
      {
        \setlength{\memRTLleftskip}{0pt}
        \setlength{\memRTLrightskip}{0pt}
        \interlinepenalty\@M
        \centering
        \cftpartfont #1
        \par
      }
      \nobreak
        \global\@nobreaktrue
        \everypar{\global\@nobreakfalse\everypar{}}%
    \endgroup
  \fi}
\makeatother
\renewcommand{\cftpartfont}{\bfseries}

% redefinitions for section entries
\renewcommand\cftsectionfont{\rmfamily}
\renewcommand\cftsectionpagefont{\rmfamily\itshape\color{myred}}

\cftsetindents{section}{1em}{2em}


% text styling of all side footnotes
% styling and placement of mark
\footmarkstyle{{\itshape\footnotesize#1. }}
\setlength{\footmarkwidth}{0em}
\setlength{\footmarksep}{-\footmarkwidth}
% memoir command - do all footnotes in margin
\footnotesinmargin

% SIDECAPTIONS
\setsidecaps{\marginparsep}{\marginparwidth}
\sidecapmargin{outer}
\setsidecappos{t}
\renewcommand*{\sidecapstyle}{%
\captionnamefont{\scshape}
\ifscapmargleft
\captionstyle{\RaggedLeft\footnotesize\foottextfont}%
\else
\captionstyle{\RaggedRight\footnotesize\foottextfont}%
\fi}

 \makeatletter
  \renewcommand{\fnum@figure}{
  {\large\scshape\figurename~\thefigure}}
\makeatother
\renewcommand{\footnotesize}{\fontsize{10pt}{12pt}\selectfont} % Change footnotesize to 10pt


%%%%%%%%%%%%%%%%%%%%%%%%%%%%%%%%%%%%%%%%%%%%%%%%%%%%%%%%%%%%%%%%%%%%%%%%%%%%%%%%%%%%%%%%%%%%%%%%%%%%%%%%%%%%%%%%%%%%%%%%
\usepackage{minitoc}     % For generating chapter-local TOCs

% Initialize the minitoc package
\dominitoc[n]
\setcounter{minitocdepth}{1}   % Show until sections in minitoc
\nomtcrule    % removes rules = horizontal lines  Define the command to create a local TOC in the margin without subsections

\undottedmtctrue

\newcommand{\margintoc}{
  \marginnote{
    \vspace{-10em}
    \begin{minipage}[t]{7.5cm}
      {\minitoc} % This will use the depth set previously in \dominitoc
    \end{minipage}
  }
}

% Start of subappendices environment
\AtBeginEnvironment{subappendices}{%
\chapter*{Appendices}
\addcontentsline{toc}{chapter}{Appendices}
\counterwithin{figure}{section}
\counterwithin{table}{section}
}

% End of subappendices environment
\AtEndEnvironment{subappendices}{%
\counterwithout{figure}{section}
\counterwithout{table}{section}
}

\tikzset{axis/.style={thick,-latex}}
\tikzset{vec/.style={thick,blue}}
\tikzset{univec/.style={thick,black,-latex}}

%Modified from Evan's evan.sty.
%
%Boost Software License - Version 1.0 - August 17th, 2003
%
%Permission is hereby granted, free of charge, to any person or organization
%obtaining a copy of the software and accompanying documentation covered by
%this license (the "Software") to use, reproduce, display, distribute,
%execute, and transmit the Software, and to prepare derivative works of the
%Software, and to permit third-parties to whom the Software is furnished to
%do so, all subject to the following:

%The copyright notices in the Software and this entire statement, including
%the above license grant, this restriction and the following disclaimer,
%must be included in all copies of the Software, in whole or in part, and
%all derivative works of the Software, unless such copies or derivative
%works are solely in the form of machine-executable object code generated by
%a source language processor.

%THE SOFTWARE IS PROVIDED "AS IS", WITHOUT WARRANTY OF ANY KIND, EXPRESS OR
%IMPLIED, INCLUDING BUT NOT LIMITED TO THE WARRANTIES OF MERCHANTABILITY,
%FITNESS FOR A PARTICULAR PURPOSE, TITLE AND NON-INFRINGEMENT. IN NO EVENT
%SHALL THE COPYRIGHT HOLDERS OR ANYONE DISTRIBUTING THE SOFTWARE BE LIABLE
%FOR ANY DAMAGES OR OTHER LIABILITY, WHETHER IN CONTRACT, TORT OR OTHERWISE,
%ARISING FROM, OUT OF OR IN CONNECTION WITH THE SOFTWARE OR THE USE OR OTHER
%DEALINGS IN THE SOFTWARE.



\usepackage{amsthm}
\usepackage{thmtools}
\usepackage[framemethod=TikZ]{mdframed}
\usetikzlibrary{shadows}
% https://tex.stackexchange.com/a/292090/76888
% https://github.com/marcodaniel/mdframed/issues/12
\xpatchcmd{\endmdframed}
{\aftergroup\endmdf@trivlist\color@endgroup}
{\endmdf@trivlist\color@endgroup\@doendpe}
{}{}

\mdfdefinestyle{mdbluebox}{%
    roundcorner=0pt,
    linewidth=1pt,
    skipabove=12pt,
    innertopmargin=9pt,
    innerbottommargin=9pt,
    skipbelow=2pt,
    linecolor=TealBlue!35,
    nobreak=false,
    backgroundcolor=TealBlue!5,
}
\declaretheoremstyle[
headfont=\rmfamily\bfseries\color{TealBlue},
mdframed={style=mdbluebox},
headpunct={\\[3pt]},
postheadspace={0pt},
postheadhook={\textcolor{TealBlue!80}{\rule[.6ex]{\linewidth}{0.4pt}} \\ }, % \\ removed
]{thmbluebox}

\mdfdefinestyle{mdredbox}{%
    roundcorner=0pt,
    linewidth=1pt,
    skipabove=12pt,
    skipbelow=12pt,
    innertopmargin=9pt,
    innerbottommargin=9pt,
    backgroundcolor=Salmon!5,
    linecolor=Salmon!35,
    frametitleaboveskip=8pt,
    frametitlebelowskip=8pt,
    frametitlebackgroundcolor=Violet!50!black,
    frametitlefont=\bfseries\sffamily\color{white},
    frametitlerule=true,    
    nobreak=false,
}
\declaretheoremstyle[
headfont=\bfseries\color{RawSienna},
mdframed={style=mdredbox},
headpunct={\\[3pt]},
postheadspace={0pt},
postheadhook={\textcolor{RawSienna!80}{\rule[.6ex]{\linewidth}{0.4pt}} \\ }, % \\ removed
]{thmredbox}

\mdfdefinestyle{mdgreenbox}{%
    skipabove=8pt,
    linewidth=2pt,
    innertopmargin=9pt,
    innerbottommargin=9pt,
    rightline=false,
    leftline=true,
    topline=false,
    bottomline=false,
    linecolor=ForestGreen!40,
    backgroundcolor=ForestGreen!4,
}
\declaretheoremstyle[
headfont=\bfseries\rmfamily\color{ForestGreen},
bodyfont=\normalfont,
postheadspace={0pt},
mdframed={style=mdgreenbox},
headpunct={ --- },
]{thmgreenbox}

\mdfdefinestyle{mdblackbox}{%
    skipabove=8pt,
    linewidth=3pt,
    rightline=false,
    leftline=true,
    innertopmargin=9pt,
    innerbottommargin=9pt,
    topline=false,
    bottomline=false,
    linecolor=black,
    backgroundcolor=RedViolet!5!gray!5,
}
\declaretheoremstyle[
headfont=\bfseries,
bodyfont=\normalfont\small,
spaceabove=2pt,
spacebelow=0pt,
mdframed={style=mdblackbox}
]{thmblackbox}

\mdfdefinestyle{mdpurplebox}{%
    roundcorner=0pt,
    linewidth=1pt,
    skipabove=12pt,
    skipbelow=12pt,
    innertopmargin=9pt,
    innerbottommargin=9pt,
    linecolor=Orchid!35,
    nobreak=false,
    backgroundcolor=Orchid!5,
    frametitleaboveskip=8pt,
    frametitlebelowskip=8pt,
    frametitlebackgroundcolor=Violet!50!black,
    frametitlefont=\bfseries\sffamily\color{white},
    frametitlerule=true,
}
\declaretheoremstyle[
headfont=\rmfamily\bfseries\color{Orchid},
mdframed={style=mdpurplebox},
headpunct={\\[3pt]},
postheadspace={0pt},
postheadhook={\textcolor{Orchid!80}{\rule[.6ex]{\linewidth}{0.4pt}} \\ }, % \\ removed
]{thmpurplebox}
\newcommand{\listhack}{$\empty$\vspace{-2em}}


\mdfdefinestyle{mdsiennamargin}{%
    skipabove=8pt,
    linewidth=2pt,
    rightline=false,
    leftline=true,
    topline=false,
    bottomline=false,
    linecolor=RoyalBlue!40,
    backgroundcolor=RoyalBlue!5,
}
\declaretheoremstyle[
headfont=\bfseries\rmfamily\color{purple},
bodyfont=\normalfont,
postheadspace={0pt},
mdframed={style=mdsiennamargin},
headpunct={ \\[3pt] },
]{thmmarginbox}

\mdfdefinestyle{mdgreymargin}{%
    skipabove=8pt,
    linewidth=2pt,
    rightline=false,
    leftline=false,
    topline=false,
    bottomline=false,
    backgroundcolor=JungleGreen!15,
}
\declaretheoremstyle[
headfont=\bfseries\rmfamily\color{purple},
bodyfont=\normalfont,
postheadspace={0pt},
mdframed={style=mdgreymargin},
headpunct={ \\[3pt] },
]{thmmarginref}


\theoremstyle{definition}
\declaretheorem[style=thmbluebox,name=Theorem,numberwithin=section]{theorem}


\declaretheorem[style=thmbluebox,name=Theorem,numbered=no]{theorem*}
\declaretheorem[style=thmbluebox,name=Lemma,numbered=no]{lemma*}
\declaretheorem[style=thmbluebox,name=Lemma,sibling=theorem]{lemma}


\declaretheorem[style=thmgreenbox,name=Proposition,numbered=no]{proposition*}
\declaretheorem[style=thmgreenbox,name=Corollary,numbered=no]{corollary*}
\declaretheorem[style=thmgreenbox,name=Assumption,numbered=no]{assume*}
\declaretheorem[style=thmgreenbox,name=Proposition,sibling=theorem]{proposition}
\declaretheorem[style=thmgreenbox,name=Corollary,sibling=theorem]{corollary}
\declaretheorem[style=thmgreenbox,name=Assumption,sibling=theorem]{assume}
\declaretheorem[style=thmgreenbox,name=Algorithm,sibling=theorem]{algorithm}
\declaretheorem[style=thmgreenbox,name=Algorithm,numbered=no]{algorithm*}
\declaretheorem[style=thmgreenbox,name=Claim,sibling=theorem]{claim}
\declaretheorem[style=thmgreenbox,name=Claim,numbered=no]{claim*}

\declaretheorem[style=thmredbox,name=Example,sibling=theorem]{example}
\declaretheorem[style=thmredbox,name=Example,numbered=no]{example*}


% Remark-style theorems
\declaretheorem[style=thmblackbox,name=Remark,sibling=theorem]{remark}
\declaretheorem[style=thmblackbox,name=Remark,numbered=no]{remark*}
\declaretheorem[style=thmblackbox,name=Problem,numbered=no]{problem*}
\declaretheorem[style=thmblackbox,name=Question,sibling=theorem]{ques}


\declaretheorem[style=thmmarginbox,name=\begingroup\color{RoyalBlue}\blacktriangleright\endgroup,numbered=no]{marginnotebox}
\declaretheorem[style=thmmarginref,name=\begingroup\color{JungleGreen}\blacksquare\endgroup,numbered=no]{marginrefbox}


\declaretheoremstyle[
headfont=\color{blue!40!black}\normalfont\bfseries,
spaceabove=8pt,
spacebelow=8pt,
bodyfont=\normalfont
]{basehead}

\declaretheoremstyle[spaceabove=6pt,spacebelow=6pt]{basehead}


\declaretheorem[style=basehead,name=Answer,sibling=theorem]{answer}
\declaretheorem[style=basehead,name=Answer,numbered=no]{answer*}
\declaretheorem[style=basehead,name=Proposition,sibling=theorem]{plainprop}
\declaretheorem[style=basehead,name=Proposition,numbered=no]{plainprop*}
\declaretheorem[style=basehead,name=Theorem,sibling=theorem]{plaintheo}
\declaretheorem[style=basehead,name=Theorem,numbered=no]{plaintheo*}
\declaretheorem[style=basehead,name=Lemma,sibling=theorem]{plainlem}
\declaretheorem[style=basehead,name=Lemma,numbered=no]{plainlem*}


\declaretheorem[style=thmpurplebox,name=Conjecture,sibling=theorem]{conjecture}
\declaretheorem[style=thmpurplebox,name=Conjecture,numbered=no]{conjecture*}
\declaretheorem[style=thmpurplebox,name=Definition,sibling=theorem]{definition}
\declaretheorem[style=thmpurplebox,name=Definition,numbered=no]{definition*}

\declaretheorem[style=basehead,name=Exercise,sibling=theorem]{exercise}
\declaretheorem[style=basehead,name=Exercise,numbered=no]{exercise*}
\declaretheorem[style=basehead,name=Fact,sibling=theorem]{fact}
\declaretheorem[style=basehead,name=Fact,numbered=no]{fact*}
\declaretheorem[style=basehead,name=Problem,sibling=theorem]{problem}
\declaretheorem[style=basehead,name=Question,numbered=no]{ques*}

\Crefname{answer}{Answer}{Answers}
\Crefname{assume}{Assumption}{Assumptions}
\Crefname{claim}{Claim}{Claims}
\Crefname{conjecture}{Conjecture}{Conjectures}
\Crefname{exercise}{Exercise}{Exercises}
\Crefname{fact}{Fact}{Facts}
\Crefname{problem}{Problem}{Problems}
\Crefname{ques}{Question}{Questions}

\newcommand{\motiv}[1]{
    \emph{{\color{red} Motivation:} #1} \par\medskip
}
\newenvironment{moral}{%
    \begin{mdframed}[linecolor=green!70!black]%
        \bfseries\color{green!50!black}}%
    {\end{mdframed}}

    \usepackage{exsheets}
%Modified from Evan's evan.sty.
%
%Boost Software License - Version 1.0 - August 17th, 2003
%
%Permission is hereby granted, free of charge, to any person or organization
%obtaining a copy of the software and accompanying documentation covered by
%this license (the "Software") to use, reproduce, display, distribute,
%execute, and transmit the Software, and to prepare derivative works of the
%Software, and to permit third-parties to whom the Software is furnished to
%do so, all subject to the following:

%The copyright notices in the Software and this entire statement, including
%the above license grant, this restriction and the following disclaimer,
%must be included in all copies of the Software, in whole or in part, and
%all derivative works of the Software, unless such copies or derivative
%works are solely in the form of machine-executable object code generated by
%a source language processor.

%THE SOFTWARE IS PROVIDED "AS IS", WITHOUT WARRANTY OF ANY KIND, EXPRESS OR
%IMPLIED, INCLUDING BUT NOT LIMITED TO THE WARRANTIES OF MERCHANTABILITY,
%FITNESS FOR A PARTICULAR PURPOSE, TITLE AND NON-INFRINGEMENT. IN NO EVENT
%SHALL THE COPYRIGHT HOLDERS OR ANYONE DISTRIBUTING THE SOFTWARE BE LIABLE
%FOR ANY DAMAGES OR OTHER LIABILITY, WHETHER IN CONTRACT, TORT OR OTHERWISE,
%ARISING FROM, OUT OF OR IN CONNECTION WITH THE SOFTWARE OR THE USE OR OTHER
%DEALINGS IN THE SOFTWARE.

%use l instead of ell in math mode.
\mathcode`l="8000
\begingroup
\makeatletter
\lccode`\~=`\l
\DeclareMathSymbol{\lsb@l}{\mathalpha}{letters}{`l}
\lowercase{\gdef~{\ifnum\the\mathgroup=\m@ne \ell \else \lsb@l \fi}}%
\endgroup

\usepackage{xargs}
\usepackage{centernot}
\usepackage{mathtools}
\usepackage{tikz-cd}

\renewcommand{\mathbb}[1]{\mathds{#1}}

\newcommand{\increment}{\Delta}
\renewcommand{\vec}[1]{\boldsymbol{\mathbf{#1}}}
\renewcommand{\Vec}[1]{\mathbf{#1}}
\newcommand{\unitv}[1]{\vec{\hat{#1}}}
\let\conjugate\overline
\newcommand{\conj}[1]{\overline{#1}}
\newcommand{\seq}[2]{\left\langle#1_1, #1_2, \dots, #1_{#2}\right\rangle}
\newcommand{\seqq}[1]{\left\langle#1\right\rangle}

\usepackage{braket}
\renewcommand*{\Set}[1]{\left\{#1\right\}}
\newcommand{\given}{\mid}
\newcommand{\cgiven}{:}

\newcommand{\avg}[1]{\langle#1\rangle}
\newcommand{\bavg}[1]{\overline{#1}}
\newcommand{\notiff}{%
  \mathrel{{\ooalign{\hidewidth$\not\phantom{"}$\hidewidth\cr$\iff$}}}}
  
\newcommand{\listvec}[2]{\left(#1_{1}, #1_{2}, \dots, #1_{#2}\right)}
\newcommand{\many}[2]{#1_{1}#1_{2}\cdots#1_{#2}}
\newcommand{\manys}[2]{\{#1_{1},#1_{2}\ldots,#1_{#2}\}}
\newcommand{\cbrt}[1]{\sqrt[3]{#1}}
\newcommand{\floor}[1]{\left\lfloor #1 \right\rfloor}
\newcommand{\ceiling}[1]{\left\lceil #1 \right\rceil}
\newcommand{\mailto}[1]{\href{mailto:#1}{\texttt{#1}}}
\newcommand{\ol}{\overline}
\newcommand{\ul}{\underline}
\newcommand{\wt}{\widetilde}
\newcommand{\wh}{\widehat}
\newcommand{\eps}{\varepsilon}
\newcommand{\vocab}[1]{\sffamily #1}
\providecommand{\alert}{\vocab}
\providecommand{\half}{\frac{1}{2}}
\newcommand{\catname}{\mathsf}
\newcommand{\hrulebar}{
    \par\hspace{\fill}\rule{0.95\linewidth}{.7pt}\hspace{\fill}
    \par\nointerlineskip \vspace{\baselineskip}
}
\providecommand{\arc}[1]{\wideparen{#1}}

%For use in author command
\newcommand{\plusemail}[1]{\\ \normalfont \texttt{\mailto{#1}}}

%More commands and math operators
\DeclareMathOperator{\cis}{cis}
\DeclareMathOperator*{\lcm}{lcm}
\DeclareMathOperator*{\argmin}{arg min}
\DeclareMathOperator*{\argmax}{arg max}

%Convenient Environments
\newenvironment{soln}{\begin{proof}[Solution]}{\end{proof} \hrule}
\newenvironment{parlist}{\begin{inparaenum}[(i)]}{\end{inparaenum}}
\newenvironment{gobble}{\setbox\z@\vbox\bgroup}{\egroup}

%Inequalities
\newcommand{\cycsum}{\sum_{\mathrm{cyc}}}
\newcommand{\symsum}{\sum_{\mathrm{sym}}}
\newcommand{\cycprod}{\prod_{\mathrm{cyc}}}
\newcommand{\symprod}{\prod_{\mathrm{sym}}}

%From H113 "Introduction to Abstract Algebra" at UC Berkeley
\newcommand{\CC}{\mathbb C}
\newcommand{\FF}{\mathbb F}
\newcommand{\NN}{\mathbb N}
\newcommand{\NNO}{\mathbb N_{0}}
\newcommand{\ZZO}{\mathbb Z_{\ge 0}}
\newcommand{\RRO}{\mathbb R_{\ge 0}}
\newcommand{\QQ}{\mathbb Q}
\newcommand{\RR}{\mathbb R}
\newcommand{\ZZ}{\mathbb Z}

\newcommand{\charin}{\text{ char }}
\DeclareMathOperator{\sign}{sign}
\DeclareMathOperator{\Aut}{Aut}
\DeclareMathOperator{\Inn}{Inn}
\DeclareMathOperator{\Syl}{Syl}
\DeclareMathOperator{\Gal}{Gal}
\DeclareMathOperator{\GL}{GL} % General linear group
\DeclareMathOperator{\SL}{SL} % Special linear group
\DeclareMathOperator{\Vol}{Vol} % Special linear group

%From Kiran Kedlaya's "Geometry Unbound"
\newcommand{\dang}{\measuredangle} %% Directed angle
\newcommand{\ray}[1]{\overrightarrow{#1}}
\newcommand{\seg}[1]{\overline{#1}}

\newcommand{\comp}[1]{\widebar{#1}}
\newcommand{\ndiv}[1]{\dot{#1}}
\newcommand{\nddiv}[1]{\ddot{#1}}

%From M275 "Topology" at SJSU
\DeclareMathOperator{\id}{id}
\newcommand{\taking}[1]{\xrightarrow{#1}}
\newcommand{\inv}{^{-1}}

%From M170 "Introduction to Graph Theory" at SJSU
\DeclareMathOperator{\diam}{diam}
\DeclareMathOperator{\ord}{ord}
\newcommand{\defeq}{\overset{\mathrm{def}}{=}}

%From the USAMO .tex files
\newcommand{\ts}{\textsuperscript}
\newcommand{\dg}{^\circ}
\newcommand{\ii}{\item}

% From Math 55 and Math 145 at Harvard
\newenvironment{subproof}[1][Proof]{%
    \begin{proof}[#1] \renewcommand{\qedsymbol}{$\blacksquare$}}%
    {\end{proof}}

\newcommand{\liff}{\leftrightarrow}
\newcommand{\lthen}{\rightarrow}

\DeclareMathOperator{\Img}{Im} % Image
\DeclareMathOperator{\coker}{coker} % Cokernel
\DeclareMathOperator{\Coker}{Coker} % Cokernel
\DeclareMathOperator{\Ker}{Ker} % Kernel
\DeclareMathOperator{\Spec}{Spec} % spectrum
\DeclareMathOperator{\pr}{pr} % projection
\DeclareMathOperator{\ext}{ext} % extension
\DeclareMathOperator{\pred}{pred} % predecessor
\DeclareMathOperator{\dom}{dom} % domain
\DeclareMathOperator{\ran}{ran} % range
\DeclareMathOperator{\Hom}{Hom} % homomorphism
\DeclareMathOperator{\Mor}{Mor} % morphisms
\DeclareMathOperator{\End}{End} % endomorphism

% Things Lie
\newcommand{\kb}{\mathfrak b}
\newcommand{\kg}{\mathfrak g}
\newcommand{\kh}{\mathfrak h}
\newcommand{\kn}{\mathfrak n}
\newcommand{\ku}{\mathfrak u}
\newcommand{\kz}{\mathfrak z}
\DeclareMathOperator{\Ext}{Ext} % Ext functor
\DeclareMathOperator{\Tor}{Tor} % Tor functor
\newcommand{\SC}{\mathcal{S}}
\newcommand{\SCF}{\mathscr F}
\newcommand{\SCG}{\mathscr G}
\newcommand{\SCH}{\mathscr H}

% Mathfrak primes
\newcommand{\km}{\mathfrak m}
\newcommand{\kp}{\mathfrak p}
\newcommand{\kq}{\mathfrak q}


%aliases
\renewcommand{\ge}{\geqslant}
\renewcommand{\le}{\leqslant}
\renewcommand{\subset}{\subsetneq}
\newcommand{\auth}[1]{\emph{#1}}
\newcommand{\para}[1]{#1 \par}
\newcommand{\lpara}[1]{\par}
\newcommand{\parbreak}{\smallskip}

%some stuff

%geo
\newcommand{\rantri}{\tkzDefPoints{0/1/A,4/3/B,5/1/C}%
\tkzDrawPolygon(A,B,C)}
\newcommand{\coor}[2]{\tkzDefPoint(#1){#2}}
\newcommand{\Triangle}[1]{\tkzDrawPolygon(#1)}
\newcommand{\polygon}[1]{\tkzDrawPolygon(#1)}



\newcommand{\Line}[1]{\tkzDrawSegment(#1)}

\newcommand{\equi}[1]{\tkzDefTriangle[equilateral](#1)}
\newcommand{\twoang}[2]{\tkzDefTriangle[two angles = #1](#2)}
\newcommand{\isoright}[1]{\tkzDefTriangle[isosceles right](#1)}
\newcommand{\getp}[1]{\tkzGetPoint{#1}}

\newcommand{\centroid}[2]{\tkzDefTriangleCenter[centroid](#1)
\tkzGetPoint{G}\tkzDrawPoints(G)\tkzLabelPoints[#2](G)}

\newcommand{\incentre}[2]{\tkzDefCircle[in](#1) \tkzGetPoints{I}{a}
\tkzDrawPoints(I)\tkzLabelPoints[#2](I)}

\newcommand{\incircle}[1]{\tkzDefCircle[in](#1) \tkzGetPoints{I}{a}
\tkzDrawCircle(I,a)}

\newcommand{\circumcentre}[2]{\tkzDefCircle[circum](#1) 
\tkzGetPoint{O}\tkzDrawPoints(I)\tkzLabelPoints[#2](I)}

\newcommand{\circumcircle}[2]{\tkzDefCircle[circum](#1) \tkzGetPoint{O}
\tkzDrawCircle(O,#2)}

\newcommand{\orthocentre}[2]{\tkzDefTriangleCenter[ortho](#1)
\tkzGetPoint{H}\tkzDrawPoints(H)\tkzLabelPoints[#2](H)}

\newcommand{\orthopoints}[3]{\tkzDefSpcTriangle[orthic](#1,#2,#3){H_#1,H_#2,H_#3}}
\newcommand{\ortho}[3]{\tkzDefSpcTriangle[orthic](#1,#2,#3){H_#1,H_#2,H_#3}
\tkzDrawSegments(#1,H_#1 #2,H_#2 #3,H_#3)
\tkzMarkRightAngles[fill=gray!20,
opacity=.5](#1,H_#1,#3 #2,H_#2,#1 #3,H_#3,#1)}

\newcommand{\rightang}[1]{\tkzMarkRightAngles[fill=gray!20,
opacity=.5](#1)}

\newcommand{\drawsquare}[1]{\tkzDefSquare(#1)
\tkzDrawPolygon(#1,tkzFirstPointResult,%
tkzSecondPointResult)}

\newcommand{\angname}[2]{\tkzLabelAngle[pos=1](#2){$#1$}}
\newcommand{\foot}[3]{\tkzDefLine[perpendicular=through #1,K=-.5](#2,#3)\tkzGetPoint{c}
\tkzDefPointBy[projection=onto #2--#3](c)\tkzGetPoint{h}}

\newcommand{\project}[3]{\tkzDefPointBy[projection=onto #2](#1) \tkzGetPoint{#3}}

\newcommandx{\empangle}[4][1=0.5,2=black,3=|]{\tkzMarkAngle[size=#1,color=#2,mark=#3](#4)}
\newcommandx{\vertice}[2][1=left]{\tkzDrawPoints(#2)\tkzLabelPoints[#1](#2)}

\newcommandx{\fillangle}[3][1=orange]{\tkzDrawSector[R with nodes,fill=#1!20](#2,0.25)(#3)}

\newcommand*\len[1]{\overline{#1}}

%framed
\mdfdefinestyle{MyFrame}{%
    linecolor=black,
    outerlinewidth=0.05pt,
    %roundcorner=20pt,
    %backgroundcolor=gray!50!white}
        }

\newcommand\header[1]{
  \newlength{\headerwidth}
  \setlength{\headerwidth}{\widthof{#1}}
  \addtolength{\headerwidth}{8pt}
  \begin{mdframed}[style=MyFrame,userdefinedwidth=\headerwidth]
    #1
  \end{mdframed}
}

\makeatletter
\let\save@mathaccent\mathaccent
\newcommand*\if@single[3]{%
  \setbox0\hbox{${\mathaccent"0362{#1}}^H$}%
  \setbox2\hbox{${\mathaccent"0362{\kern0pt#1}}^H$}%
  \ifdim\ht0=\ht2 #3\else #2\fi
  }
%The bar will be moved to the right by a half of \macc@kerna, which is computed by amsmath:
\newcommand*\rel@kern[1]{\kern#1\dimexpr\macc@kerna}
%If there's a superscript following the bar, then no negative kern may follow the bar;
%an additional {} makes sure that the superscript is high enough in this case:
\newcommand*\widebar[1]{\@ifnextchar^{{\wide@bar{#1}{0}}}{\wide@bar{#1}{1}}}
%Use a separate algorithm for single symbols:
\newcommand*\wide@bar[2]{\if@single{#1}{\wide@bar@{#1}{#2}{1}}{\wide@bar@{#1}{#2}{2}}}
\newcommand*\wide@bar@[3]{%
  \begingroup
  \def\mathaccent##1##2{%
%Enable nesting of accents:
    \let\mathaccent\save@mathaccent
%If there's more than a single symbol, use the first character instead (see below):
    \if#32 \let\macc@nucleus\first@char \fi
%Determine the italic correction:
    \setbox\z@\hbox{$\macc@style{\macc@nucleus}_{}$}%
    \setbox\tw@\hbox{$\macc@style{\macc@nucleus}{}_{}$}%
    \dimen@\wd\tw@
    \advance\dimen@-\wd\z@
%Now \dimen@ is the italic correction of the symbol.
    \divide\dimen@ 3
    \@tempdima\wd\tw@
    \advance\@tempdima-\scriptspace
%Now \@tempdima is the width of the symbol.
    \divide\@tempdima 10
    \advance\dimen@-\@tempdima
%Now \dimen@ = (italic correction / 3) - (Breite / 10)
    \ifdim\dimen@>\z@ \dimen@0pt\fi
%The bar will be shortened in the case \dimen@<0 !
    \rel@kern{0.6}\kern-\dimen@
    \if#31
      \overline{\rel@kern{-0.6}\kern\dimen@\macc@nucleus\rel@kern{0.4}\kern\dimen@}%
      \advance\dimen@0.4\dimexpr\macc@kerna
%Place the combined final kern (-\dimen@) if it is >0 or if a superscript follows:
      \let\final@kern#2%
      \ifdim\dimen@<\z@ \let\final@kern1\fi
      \if\final@kern1 \kern-\dimen@\fi
    \else
      \overline{\rel@kern{-0.6}\kern\dimen@#1}%
    \fi
  }%
  \macc@depth\@ne
  \let\math@bgroup\@empty \let\math@egroup\macc@set@skewchar
  \mathsurround\z@ \frozen@everymath{\mathgroup\macc@group\relax}%
  \macc@set@skewchar\relax
  \let\mathaccentV\macc@nested@a
%The following initialises \macc@kerna and calls \mathaccent:
  \if#31
    \macc@nested@a\relax111{#1}%
  \else
%If the argument consists of more than one symbol, and if the first token is
%a letter, use that letter for the computations:
    \def\gobble@till@marker##1\endmarker{}%
    \futurelet\first@char\gobble@till@marker#1\endmarker
    \ifcat\noexpand\first@char A\else
      \def\first@char{}%
    \fi
    \macc@nested@a\relax111{\first@char}%
  \fi
  \endgroup
}
\makeatother




\newcommand{\irrev}[1]{%
    \Ifthispageodd{%
    \reversemarginpar\marginpar{\RaggedLeft\large \bfseries \color{purple}Extra}\normalmarginpar}{%
    \reversemarginpar\marginpar{\RaggedLeft\large \bfseries \color{purple}Extra}\normalmarginpar}%
    }


\usepackage{scrextend}

\definecolor{boldcolor}{gray}{0.18} % range from [0,1]
\newcommand{\lightbold}[1]{\textcolor{boldcolor}{#1}}

\newcommand{\cautionmark}{{\Huge\color{red}!}}
\newlist{Caution}{enumerate}{1}
\setlist[Caution]{label=\raisebox{-0.5cm}[0pt][0pt]{\cautionmark},leftmargin=1cm}

\newcommand{\alignedmarginpar}[1]{%
    \Ifthispageodd{%
        \marginpar{\RaggedRight#1}}{%
        \marginpar{\RaggedLeft#1}}%
    }


\newcommand{\caution}[1]{  
  \alignedmarginpar{\bigskip \cautionmark}
  
  \begin{mdframed}[linecolor=red!70!black]%
      \bfseries\color{red!50!black}%
      #1
    \end{mdframed}
}

%From Knzhou

\newcommand{\union}{\cup}
\newcommand{\intersect}{\cap}
\newcommand{\subgr}{\subseteq}
\newcommand{\subr}{\subseteq}
\newcommand{\nsubgr}{\trianglelefteq} % normal subgroup
\newcommand{\dunion}{\sqcup}
\newcommand{\incl}{\iota}
\renewcommand{\mod}{\, \mathrm{mod}\, } % modular arithmetic
\newcommand{\sdprod}{\rtimes} % semidirect product

\def\rcurs{{\mbox{$\resizebox{.09in}{.08in}{\includegraphics[trim= 1em 0 14em 0,clip]{../script_r/ScriptR.pdf}}$}}}
\def\brcurs{{\mbox{$\resizebox{.09in}{.08in}{\includegraphics[trim= 1em 0 14em 0,clip]{../script_r/BoldR.pdf}}$}}}
\def\hrcurs{{\mbox{$\hat \brcurs$}}}

\newcommand*\widefbox[1]{\fbox{\hspace{2em}#1\hspace{2em}}}

\renewcommand*{\Re}{\mathfrak{R}}
\renewcommand*{\Im}{\mathfrak{I}}

\newcommand{\spart}[1]{\newgeometry{left=2cm,right=2cm} \part{#1} \restoregeometry} 

\renewcommand*{\vocab}[1]{{\sffamily#1}\index{#1}}
\newcommand{\chvocab}[2]{{\sffamily#2}\index{#1!#2}}
\newcommand{\vv}{\vec{v}}
\newcommand{\oo}{\vec{0}}
\newcommand{\Aa}{\vec{A}}
\newcommand{\BB}{\vec{B}}
\newcommand{\uu}{\vec{u}}
\newcommand{\ww}{\vec{w}}

\newcommand{\dtp}{\dotproduct}

\newcommand{\uch}{\,{\text{u}}}
\renewcommand{\dd}{\mathop{}\!{d}}
\renewcommand{\diffd}{\mathop{}\!{d}}

\newcommand{\sidenote}[1]{%
    \marginpar{\RaggedRight \itshape #1 }}

\newcommand{\marginnote}[1]{\sidenote{#1}}
\newcommand{\marginref}[1]{\sidenote{#1}}

\captionsetup[figure]{labelfont={bf},name={Fig.},labelsep=period}

\renewcommand{\figurename}{\normalsize\textbf{Fig.}}
\makeatletter
\makeatletter
\renewcommand{\@makecaption}[2]{%
  \vskip\abovecaptionskip
  \sbox\@tempboxa{#1. {\normalfont #2}}% Add a period after the figure name
  \ifdim \wd\@tempboxa >\hsize
    #1. \RaggedRight #2\par % Apply \RaggedRight here
  \else
    \global \@minipagefalse
    \hb@xt@\hsize{\hfil\box\@tempboxa\hfil}%
  \fi
  \vskip\belowcaptionskip
}
\makeatother

\author{
    Adyansh Mishra \\
    \mailto{adyanshmishra@proton.me}
}\date{\today}
\title{An Undergraduate Notebook}
\addbibresource{ugreferences.bib}

\begin{document}
%https://tex.stackexchange.com/questions/249475/index-hyperlink-not-pointing-to-correct-page
\pagenumbering{roman}

\begin{titlingpage}
    \BgThispage
    \newgeometry{left=1cm,right=6cm,bottom=2cm}
    \vspace*{0.4\textheight}
    \noindent
    \textcolor{white}{\Huge\textbf{\textsf{\thetitle}}}
    \vspace*{2cm}\par
    \noindent
    \begin{minipage}{0.35\linewidth}
        \begin{flushright}
            \printauthor
        \end{flushright}
    \end{minipage} \hspace{15pt}
    %
    \begin{minipage}{0.02\linewidth}
        \rule{1pt}{175pt}
    \end{minipage} \hspace{-10pt}
    %
    \begin{minipage}{0.63\linewidth}
    \vspace{5pt}
        \begin{abstract} 
    An abstract is a brief summary of a research article, thesis, review, conference proceeding or any in-depth analysis of a particular subject or discipline, and is often used to help the reader quickly ascertain the paper's purpose. When used, an abstract always appears at the beginning of a manuscript, acting as the point-of-entry for any given scientific paper or patent application. Abstracting and indexing services for various academic disciplines are aimed at compiling a body of literature for that particular subject.
        \end{abstract}
    \end{minipage}
\end{titlingpage}
    \restoregeometry

    \frontmatter

    \newgeometry{right=1.5cm,left=1.5cm}
    \renewcommand*{\contentsname}{Short contents}
    \setcounter{tocdepth}{0}% chapters and above
    \tableofcontents

    \clearpage
    \renewcommand*{\contentsname}{Contents}
    \setcounter{tocdepth}{1}
    \tableofcontents
    
    \clearpage
    \listoffigures
    \clearpage

    \mainmatter
    \pagestyle{mystyle}
    \setcounter{chapter}{-1}
    \chapter{Proofs}
\motiv{The basis of mathematics relies on proofs.}

\noindent Proofs are necessary for the study of higher mathematics. The notes for proofs are almost entirely based on \cite{proveit}. 
Mathematics itself is as a subject is based on \emph{deductive} logic. We start with a set of axioms and end with theorem,
conjectures, theorems, lemmas, etc. 

The language of proofs is therefore necessary for the study of higher mathematics, and thus this chapter is included at the 
start instead of being chucked away at the distance.

    \newgeometry{right=1.5cm,left=1.5cm}
    \part{Calculus}
    \parttoc
    \restoregeometry
    \chapter{Real Numbers}

\marginnote{\epigraph{``Give him threepence, since he must gain out of what he learns''}{\textup{Euclid} of Alexandaria}}

\begin{equation*}
\end{equation*}

The real numbers form the basis of all of the calculus we are going to study. Later, we'll look at expansions of them.
To formulate a \emph{deductive} system, we must define real numbers and build up everything over them. Well, in reality
real numbers are arbitrary quantities that follow a certain set of axioms, every mathematical object that satisfies
them is a real number.

\section{Real Number Axioms}

The axioms of real numbers can be divided broadly into three different types,

\begin{itemize}
    \ii The Field axioms
    \ii The Order axioms
    \ii The Continuity axiom
\end{itemize}

\subsection{Field Axioms}

Consider the set \(\mathbb{F}\) along with two operations, \(+, \cdot\).
This structure, \((\FF, +, \cdot)\) is called a \vocab{field} if it follows the following
set of axioms.

Although we label the operations as symbols multiplication and addition, these can in effect be 
any two operations, provided they follow the axioms. A field is a general structure, 
and we'll talk much more about it in algebra.

These operations are such that for every \(x\), \(y \in \FF\), we will have \(x + y, xy \in \FF\),
which is what we call \(\emph{closure}\).

These sums and products are unique, which means we always have one and only have one 
\(x+y, xy\) for \(x, y \in \FF\). We will talk about what these operations precisely are in 
a moment.

\begin{axioms}
    \ii \chvocab{Real Numbers}{Commutativity}, \(x+y = y+x\) and \(xy = yx\).
    \ii \chvocab{Real Numbers}{Associativity}, \(x + (y+z) = (x+y) + z\), and \(x(yz) = (xy)z\).
    \ii \chvocab{Real Numbers}{Distributivity}, \(x(y + z) = xy + xz\).
    \ii \chvocab{Real Numbers}{Identity}. There exist two \emph{distinct} identity elements, \(0\) and \(1\) such that \(x + 0 = x\) and \(x \cdot 1 = x\).
    \ii \chvocab{Real Numbers}{Inverses}, for \(\forall x \in \RR \setminus \{0\}\) there exists a \(y\) such that \(xy = 1\) and \(\forall x \in \RR\),
    there exists a \(z \in \RR\), such that \(x + z = 0\). These are called the reciprocal 
    and negative of \(x\) and are denoted as \(1/x\)  (\(x^{-1}\)) and \(-x\).
\end{axioms}
 
First of all, we will show that the identity elements are unique, that is, if,
\begin{equation*}
    (\forall x),\; x + 0 = x \text{ and } (\forall x),\; x + 0' = x, \text{ then } 0 = 0'
\end{equation*}
and a similar statement for \(1\) over multiplication.

\begin{proposition}
    The identity elements are unique.
\end{proposition}

\begin{proof}
    Let us assume they aren't and \(0\), \(0'\) are two distinct identities.

    Then, 

    \begin{equation*}
        0 = 0 + 0' = 0'.
    \end{equation*}

    But this leads to a contradiction! \lightning
\end{proof}

\sidenote{We use the symbol \lightning to show contradiction.}

A similar argument for the multiplicative identity, \(1\) is left to the reader. Its 
not really difficult to do it, once you understand what we did in this proof, which is 
use first use that \(0'\) is an identity for the first equality (so that \(0 + 0' = 0\)) 
and then \(0\) as the identity for the second equality. 

Such \(x\)'s are also unique. We can show their uniqueness by showing that the inverses are unique.

\begin{proposition}
    The inverses are unique.
\end{proposition}

\begin{proof}
    The idea is this the identity elements are unique, and the multiplication or addition
    by an inverse results in an identity. Let the multiplicative inverse of \(x\) be 
    distinct reals, \(y\) and \(z\), then  

    \begin{equation*}
        y \cdot 1 = y \cdot (x \cdot z) = (y \cdot x) \cdot z = 1 \cdot z = 1.
    \end{equation*}

    What did we do here? the first equality follows from the definition 
    of inverse, the second follows from Associativity and the third uses the 
    definition of identity. This shows that \(y = z\), but \(y\) and \(z\) 
    were, in fact, distinct! \lightning

\end{proof}

Subtraction is treated as a shorthand for addition, \(a + (-b) = a - b\) using 
inverses. Similarly, one writes \(a \cdot b\inv = a/b\).

These properties allow us to form laws of addition and multiplication --- such as 
cancellation. We can form an outline of a proof of cancellation as, 

\begin{proof}
    We need to show that \(a + b = a + c \iff b = c\). 
    This can be done simply by adding \(-a\) on both sides. Since addition is unique, 
    and both numbers are equal, so must be their addition with another number. 
    
    By a formal manipulation,
    \begin{align*}
        b &= b + (a - a) \\
        b &= (b + a) - a \\
        b &= (c + a) - a \\
        b &= c + (a - a)\\
        b &= c.
    \end{align*}
\end{proof}

Let us prove a fundamental property of the additive inverse.

\begin{proposition}
    We have for all \(a \in \FF\), 
    \begin{equation*}
        a \cdot 0 = 0.
    \end{equation*}
\end{proposition}

\begin{proof}
    How do we go about proving this? We use the fundamental property of \(0\), as 
    the additive inverse, \(0 + 0 = 0\).
    \begin{equation*}
        a \cdot 0 = a \cdot (0 + 0) = a \cdot 0 + a \cdot 0.
    \end{equation*}

    The left-hand side is the same as \(0 + a \cdot 0\). Now, we have already justified 
    cancellation for addition, so that cancelling \(a \cdot 0\) from both sides, we have, 

    \begin{equation*}
        0 = a\cdot 0.
    \end{equation*}
\end{proof}

A similar procedure can be done for multiplication.

We see that this can also be used to prove that the product of any two negative numbers is 
positive. Let us first prove that \((-a)b = -ab\). This is rather easy as well, and follows 
from distributivity,
\begin{align*}
    (-a)b &= (-a)b + ab - ab \\
    (-a)b &= b[(-a) + a] - ab \\
    (-a)b & = 0 \cdot b - ab\\
    (-a)b &= -ab
\end{align*}

A similar proof can be used to show that \((-a)(-b) = ab\), as 
\begin{align*}
    (-a)(-b) &= (-a)(-b) + (-a)b - (-a)b \\
    (-a)(-b) &= -a[(-b) + b] - (-ab) \\
    (-a)(-b) &= -a \cdot 0 + ab\\
    (-a)(-b) &= ab
\end{align*}

Note that none of our proofs are reliant on the fact that these numbers are 
real numbers. These properties hold for \emph{any} field. In particular, 
they also hold for the real numbers, with the operations of addition and multiplication.

As such, real numbers form a \emph{field}. These axioms permit us to perform our regular operations on 
the reals, namely addition, subtraction, multiplication and division by non-zero
reals.

In the last proof, we use our previous result\footnote{In the second line, \(-(-a)b = -(-ab) = ab\). A proof for \(-(-a) = a\) is left as 
an exercise to the reader.} to prove this one. Now it is obvious that for 
the reals, the product of any two negatives will be positive --- however, let us first discuss what exactly ``positive''
and ``negative'' mean.

\begin{exc}
\begin{exercise}[points=1]
    Complete the other half of the proofs for uniqueness of identity and inverses.
\end{exercise}

\begin{exercise}
    Show that \(-(-a) = a\) and for \(a \ne 0\), \((a\inv)\inv = a\).
\end{exercise}

\begin{solution}
    The idea behind this is the definition of inverse. \(x = -(-a)\) iff, 
    \begin{equation*}
        x + (-a) = 0,
    \end{equation*}
    but we already know that \(x = a\) satisfies this equation! From the uniqueness of 
    inverses, \(x = a\). A similar proof follows for \((a\inv)\inv\).
\end{solution}

\begin{exercise}
    Consider the equation \(xy = 0\), show that it is only satisfied when \(x = 0\) or 
    \(y = 0\). Hint: Consider the case when \(x\) and \(y\) are both not zero.
\end{exercise}

\begin{solution}
    If \(x\) and \(y\) are both not zero, then we know \(x\inv\) and \(y\inv\) exist, so that,
    \begin{equation*}
        xy \cdot (x\inv \cdot y\inv) = (x\cdot x\inv)(y \cdot y\inv) = 1 \cdot 1 = 1 
    \end{equation*}

    But we know that \(0 \cdot a\) for any \(a \in \FF\) is \(0\) \lightning.
\end{solution}

\begin{exercise}
    Formulate a law for cancellation over multiplication. That is, given non-zero \(c \in \FF\) and, 
    \begin{equation*}
        ac = bc,
    \end{equation*}
    Show that \(a = b\).
\end{exercise}

\begin{exercise}
    We now consider the natural properties of division, or quotients. A quotient, \(a/b\) 
    is a shorthand for \(a \cdot b\inv\). Prove the following familiar properties of 
    quotients. Assume accordingly that the numbers having inverses are non-zero. (for instance 
    both \(a\) and \(b\) are non-zero in 1).
    
    \begin{enumerate}
        \item \((ab)\inv = a\inv b\inv\). Hint: How is \((ab)\inv\) defined?
        \item \(\displaystyle \frac{a}{b} = \frac{ac}{bc}\).
        \item \(\displaystyle \frac{a}{b} + \frac{c}{d} = \frac{ad + bc}{bd}\).
        \item \(\displaystyle \frac{a}{b} \cdot \frac{c}{d} = \frac{ac}{bd}\).
        \item \(\displaystyle \frac{a}{b} {\bigg/} \frac{c}{d} = \frac{ad}{bc}\). 
        \item Show that \(\displaystyle\frac{a}{b} = \frac{c}{d} \iff ad = bc\).  
    \end{enumerate}
\end{exercise}

\end{exc}

\subsection{Order Axioms}

We now again give some axioms which allow us to \emph{order} our field. This allows us to formulate notions 
of positive and negative. 

\begin{axioms}
    \ii There exists a subset of our field, \(\FF^+\) such that \(\forall x \in \FF \setminus \Set{0}\),
     either \(x \in \FF^+\) or \(-x \in \FF^+\).
    \ii If \(x, y \in \FF^+\), both \(xy\) and \(x + y\) are also in \(\FF^+\).
    \ii \(0\) is not in \(\FF^+\).
\end{axioms}

A field in which these axioms is true is called a \vocab{ordered field}.

We say that a number \(x\) is positive if it is in \(\RR^+\). We also know define some symbols, 
\begin{align*}
    x > y \text{ means that } x - y \text{ is in } \FF^+ \\
    x < y \text{ means that } y - x \text{ is in } \FF^+ 
\end{align*}

The second definition is unnecessary to be precise, \(x < y\) may as well be 
interpreted as \(y > x\), which is something we have already defined. In any case, 
we could equivalently say a field which is imbued with an \emph{ordering relation} such as \(>\) is called 
an \vocab{ordered field}. We will again, talk about relations and what this exactly 
means later.

The symbols \(\ge\) and \(\le\) just imply that ``\dots'' is either greater/lesser or equal to ``\dots''.
Note that weird as it may appear, statements like \(6 \ge 6\) are completely valid. 
The real use of these symbols lie in terms of dealing with variables, or general 
statements.

Consider the following proposition, 

\begin{plainprop}[Trichotomy]
    If we have \(x, y \in \FF\), then one of the following must hold, 
    \begin{enumerate}
        \item \(x < y\)
        \item \(x > y\)
        \item \(x = y\)
    \end{enumerate}
\end{plainprop}

This can be inferred rather directly from the order axioms. Clearly, by the order 
axioms we have that either \(a = 0\), \(a \in \RR^+\) or \(-a \in \RR^+\).
Replace \(a\) by \(x - y\) and we're done.

\begin{proposition}
    In an ordered field, we have,
    \begin{enumerate}
        \item \(a > b \iff a + c > b + c\), for some \(c \in \FF\).
        \item \(a > b \iff ac > bc\), for \(c > 0\).
        \item For all \(a \ne 0\), \(a^2 > 0\).
        \item \(1 > 0\).
    \end{enumerate}
\end{proposition}

\begin{proof}
    1. By definition, \(x > y \iff x - y \in \FF^+\). Now, note that,
    \begin{equation*}
        x - y = x - y + c - c = (x + c) - (y + c) \in \FF^+,
    \end{equation*}
    so we are done.

    2. Similar to the previous proof, use the fact that \(c \in \FF^+\) and 
    \(x - y \in \FF^+\) imply that \(c(x - y) \in \FF^+\) by the axioms.

    3. If \(a > 0\), then \(a \cdot a = a^2> 0\) (since \(x, y > 0 \implies xy > 0\)). 
    If \(a > 0\), then by the axioms \(-a > 0\). Thus, \(-a \cdot -a = a^2 > 0\).
    Since \((-a)(-b) = ab\).

    4. Trivially follows from \(1 = 1^2\).
\end{proof}

Let us actually discuss some examples of fields and ordered fields now.

\begin{example}
    The sets \(\QQ, \RR\) along with multiplication and addition satisfy the field axioms. 
    In fact, they also satisfy the order axioms. Thus, they're ordered fields. 
\end{example}

\begin{example}
    Consider the set \(\FF = \Set{0, 1}\). We define \(0 + 0 = 0, 0 + 1 = 1 + 0 = 1, 1 + 1 = 0, 0 \cdot 0 = 0 \cdot 1 = 1 \cdot 0 = 0, 1 \cdot 1 = 1\).
    Then it follows the field axioms. However, we cannot have some ordering relation.
    
    This follows because (as we will show), \(1 > 0\) in an ordered field, but \(1 + 1 = 0 > 0 + 1 = 1\) \lightning.
\end{example}

\sidenote{We are just adding \(1\) to both sides, as permitted by the axioms.}

This also forms the basis of the idea of the absolute value, which is, 

\begin{equation}
    \abs{a} = 
    \begin{dcases}
        a, & a \ge 0 \\
        -a, & a \le 0
    \end{dcases}
\end{equation}

We see that this clearly a positive number. It is also referred to as the euclidean length of 
a number from \(0\) on the number line. Let us consider some serious properties now, 

Consider the following two properties regarding the absolute value,

\begin{plaintheo}
    For \(x, a \in \RR\), \(\abs{x} \le a\) if and only if \( -a \le x \le a\).
\end{plaintheo}

\begin{proof}
    Let us suppose \(\abs{x} \le a\). Then clearly, \(-\abs{x} \ge -a\). But 
    \(\abs{x}\) can only take two values, and is obviously positive. Thus, \(-\abs{x} \le 
    x \le \abs{x}\). Thus, we have that, \(-a \le -\abs{x} \le x \le \abs{x} \le a\) which 
    proves one direction.
    
    For the other direction, suppose \(-a \le x \le a\). Then, if \(x \ge 0\), 
    \(\abs{x} = x \le a\). If \(x < 0\), \(\abs{x} = -x \le a\) since \(x \ge -a \implies -x \le a\).
    In both cases, \(\abs{x} \le a\). 
\end{proof}

We will not look at the least upper bound axiom as of now. Note we still haven't 
shown the existence of radicals, which you must take for granted for now.

\section{Functions}

The idea of a function is very simple. You have an input, and you want to get 
an output. For instance, consider the standard operations we defined on the reals,
\(+, \cdot\). These take two inputs, and produce the appropriate output, the sum 
and the product, respectively.

More precisely, functions are maps. They map an element of a set, to another 
element of a different set. The basis of such a mapping is the definition of a 
function.

For instance, we could consider a mapping from \(\RR\) to \(\RR\), \(f\) 
\begin{equation*}
    x \overset{f}{\mapsto} x + 1,
\end{equation*}
which maps an element from the set of real numbers, to its successor.

We also denote this mapping by writing \((x, x + 1) \in f\). The first element 
is our input, which we map to second element. 

There are certain restriction, however, that we must impose, for such an object 
to be useful. In the case of functions we impose the restriction 
that the mappings are \emph{unique}. 

That is, if 
\begin{equation*}
    (a, b) \in f \text{ and } (a, c) \in f, 
\end{equation*}
then, \(b = c\).

Now, we are finally ready to give the formal definition.

\begin{definition}
    A \vocab{function} from \(A \to B\) is a set such that,
    \begin{equation}
        f \subseteq \Set{A \times B} \given (a, b), (a, c) \in f \iff b = c.
    \end{equation}
\end{definition}

Now we can define the operations of multiplication and addition rigorously, 

\begin{align*}
    + \colon \; &\RR^2 \to \RR \given (x, y) \to x + y, \\
    \times \colon \; &\RR^2 \to \RR \given (x, y) \to x \times y.
\end{align*}

We usually write \(xy\) or \(x \cdot y\) instead of \(x \times y\).

\begin{equation}
    \pdv{\varphi}{x^i} \dd{x^i} = \nabla \varphi = \vec{\varphi} 
\end{equation}

    \newgeometry{right=1.5cm,left=1.5cm}
    \part{Linear Algebra}
    \parttoc
    \restoregeometry
    \chapter{Vector Spaces}

This part of linear algebra is almost entirely based on \cite{axlerlin}.

\section{Lists}

In most of the chapters, I'll adopt the convention that \(\FF\) is a field. Some specific results require \(\FF = \CC\) or \(\FF = \RR\)
but they'll be explicitly mentioned.

\index{Scalars}

Elements of \(\FF\) are called \emph{scalars}. We call these scalars to form a distinction with vectors, which will be defined soon.

Cartesian products of \(\FF\) with itself \(n\) times are denoted as \(\FF^{n}\). For something like
\(\FF^3\) this is equivalent to saying,

\[
    \FF^{3} = \left\{(x, y, z) \mid x, y, z \in \FF\right\}
\]

It is the set of all ordered triples of the elements of \(\FF\). If we have \(\FF = \RR\),
for instance, \(\RR^2\) can be though to be a plane while \(\RR^3\) can be thought to be ordinary space.

To generalise these results to \(n\) dimensions, we use the concept of \(n\)-tuples or \emph{lists}.

\index{lists} \index{n-tuples}

\marginnote{We denote lists surrounded by paratheses and their elements separated by commas.}

\begin{definition}[Lists]
    For a non-negative integer, \(n\), an \(n\)-tuple or a list of length \(n\) is an ordered collection of \(n\) elements.
    
    A list \((a_1, a_2, \dots, a_m) = (b_1, b_2, \dots, b_n)\) if and only if \(n = m\) and \(a_1 = b_1, a_2 = b_2, \dots,
    a_m = b_n\).
\end{definition}

Thus, two lists are equal iff their length is equal and they have the same elements in the same order.
A list has a finite number of elements, thus, even though \((a_1, a_2, \dots, a_m)\) is a list, \((a_1, a_2, \dots)\) is not.
The list of length \(0\) is denoted by \(()\). 

\begin{example}
    [Lists vs sets]
    The lists \((3,5)\) and \((5,3)\) are not equal but \(\left\{3,5\right\} = \left\{5,3\right\}\). 
    The list \((4,4,4) \ne (4,4) \ne (4)\) but \(\left\{4,4,4\right\} = \left\{4,4\right\} = \left\{4,\right\}\).
\end{example}

\section{Higher Products}

Generalising the \(n\)th cartesian product of \(\FF\) with itself, 
\[
    \FF^n = \underbrace{\FF \times \FF \times \dots \times \FF}_{n\textup{-times}}
\]


We may say,
\begin{definition}
    \(\FF^n\) is the set of all lists of length \(n\) such that the elements of the list are in \(\FF\),
    \[
        \FF^n = \left\{(a_1, a_2, \dots, a_n) \mid a_1, a_2, \dots, a_n \in \FF\right\}
    \]
\end{definition}

Where \(n \in \ZZO\).

\begin{definition}
    [Co-ordinate]
    If \((a_1, a_2, \dots, a_n) \in \FF^n\) and \(1 \le i \le n\), \(x_i\) is called the \(i\)th co-ordinate of
    \((a_1, a_2, \dots, a_n)\).  
\end{definition}

While \(\RR^2\) can be visualised as a plane, and \(\RR^3\) as space, it is not possible to visualise them
for \(n \ge 4\). Similarly, \(\CC^1\) can be thought of as a plane, but cannot be visualised for \(n \ge 2\).

However, we can perform algebraic operations on the lists of some arbitrary length \(n\) which may even be 
very great. 

\begin{definition}
    [Addition in \(\FF^n\)]
    \label{def: addition in fn}
    Addition in \(\FF^n\) is defined as,
    \[
        \listvec{x}{n} + \listvec{y}{n} = (x_1 + y_1, x_2 + y_2, \dots, x_n + y_n)
    \]
\end{definition}

To avoid the cumbersome notation of writing out \(\listvec{x}{n}\) we will adopt the notation that \(x = \listvec{x}{n}\).

\begin{proposition}
    \(x + y = y + x\)
\end{proposition}

\begin{proof}
    \(x = \listvec{x}{n}\), \(y = \listvec{y}{n}\). Thus,
    \begin{align*}
        x + y &= \listvec{x}{n} + \listvec{y}{n}\\
        &= (x_1 + y_1, x_2 + y_2, \dots, x_n + y_n)\\
        &= (y_1 + x_1, y_2 + x_2, \dots, x_n + y_n)\\
        &= y + x
    \end{align*}
\end{proof}

The proof is based on the commutativity of reals and \cref{def: addition in fn}. 
The elements of \(\FF^2\), \(x, y\) can be thought of as points or vectors. Disregarding
the axes, vectors may independently be thought as some objects.

We define two more things in \(\FF^n\),

\begin{definition}
    [Additive Inverse in \(\FF^n\)]
    The additive inverse of \(x \in \FF^n\) is \(-x \in \FF^n\) where, 
    \[
        x + (-x) = 0
    \]
    And if \(x = \listvec{x}{n}\), 
    \[
        -x = \listvec{-x}{n}
    \]
\end{definition}

The additive inverse of \(x\) in \(\RR^2\) is the vecor of equal length but opposite direction.

The final operation is \emph{scalar multiplication}

\index{Scalar multiplication}

\begin{definition}
    [Scalar multiplication]
    For a scalar \(\lambda\), and \(x \in \RR^n\), their product is defined as
    \[
        \lambda \listvec{x}{n} = \listvec{\lambda x}{n}
    \]
\end{definition}

\section{Vector Space}

The definition of a vecor space comes off from the properties we have kind of discussed
above for \(\RR^n\). Defining our operations for any vecor space,

\begin{definition}
    [Operators on a vecor space]
    \emph{Addition} on \(V\) is a function \(+ : V^2 \to V\) where \(\vv + \ww \in V\) for 
    any \(\vec{\vv}, \vec{\ww} \in V\).

    \noindent\emph{Scalar multiplication} on \(V\) is a function \(\bullet : V \times \RR \to V\) where 
    \(\lambda \vv \in V\) for any \(\vv \in V\) and \(\lambda \in \RR\). 
\end{definition}

Now, let us formally define \(V\). 

\begin{definition}
    [Vector Space]
    A set \(V\) along with the operations of addition and scalar multiplication is a vecor
    space if the following properties hold for \(\uu, \vv, \ww \in V\) and \(a, b \in \RR\).

    \begin{axioms}
        \item Commutativity, \(\vv + \ww = \ww + \vv\).
        \item Associativity, \((\uu + \ww) + \vv = \uu + (\ww + \vv)\) and \(a(bv) = (ab)\vv\).
        \item Additive Identity, there an element \(0 \in \vv\) such that \(\vv + 0 = \vv\).
        \item Additive Inverse, there exist an element \(-\vv \in V\) for each \(\vv\) such that 
        \(\vv + (-\vv) = 0\).
        \item Multiplicative Identity, \(1\vv = \vv\)
        \item Distributive Property, \(a(\uu + \vv) = a\uu + a\vv\) and \(\vv(a + b) = \vv a + \vv b\). 
    \end{axioms}

\end{definition}
    
    \newgeometry{right=1.5cm,left=1.5cm}
    \part{Appendices}
    \parttoc
    \restoregeometry
    \appendix

    \backmatter
    \printindex
\end{document}

latex