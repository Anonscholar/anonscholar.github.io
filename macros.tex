%Modified from Evan's evan.sty.
%
%Boost Software License - Version 1.0 - August 17th, 2003
%
%Permission is hereby granted, free of charge, to any person or organization
%obtaining a copy of the software and accompanying documentation covered by
%this license (the "Software") to use, reproduce, display, distribute,
%execute, and transmit the Software, and to prepare derivative works of the
%Software, and to permit third-parties to whom the Software is furnished to
%do so, all subject to the following:

%The copyright notices in the Software and this entire statement, including
%the above license grant, this restriction and the following disclaimer,
%must be included in all copies of the Software, in whole or in part, and
%all derivative works of the Software, unless such copies or derivative
%works are solely in the form of machine-executable object code generated by
%a source language processor.

%THE SOFTWARE IS PROVIDED "AS IS", WITHOUT WARRANTY OF ANY KIND, EXPRESS OR
%IMPLIED, INCLUDING BUT NOT LIMITED TO THE WARRANTIES OF MERCHANTABILITY,
%FITNESS FOR A PARTICULAR PURPOSE, TITLE AND NON-INFRINGEMENT. IN NO EVENT
%SHALL THE COPYRIGHT HOLDERS OR ANYONE DISTRIBUTING THE SOFTWARE BE LIABLE
%FOR ANY DAMAGES OR OTHER LIABILITY, WHETHER IN CONTRACT, TORT OR OTHERWISE,
%ARISING FROM, OUT OF OR IN CONNECTION WITH THE SOFTWARE OR THE USE OR OTHER
%DEALINGS IN THE SOFTWARE.


\usepackage{amsmath, amssymb}
\usepackage{xargs}


\newcommand{\many}[2]{#1_{1}#1_{2}\ldots#1_{#2}}
\newcommand{\manys}[2]{\{#1_{1},#1_{2}\ldots,#1_{#2}\}}
\newcommand{\cbrt}[1]{\sqrt[3]{#1}}
\newcommand{\floor}[1]{\left\lfloor #1 \right\rfloor}
\newcommand{\ceiling}[1]{\left\lceil #1 \right\rceil}
\newcommand{\mailto}[1]{\href{mailto:#1}{\texttt{#1}}}
\newcommand{\ol}{\overline}
\newcommand{\ul}{\underline}
\newcommand{\wt}{\widetilde}
\newcommand{\wh}{\widehat}
\newcommand{\eps}{\varepsilon}
\newcommand{\vocab}[1]{\textbf{\color{blue}\sffamily #1}}
\providecommand{\alert}{\vocab}
\providecommand{\half}{\frac{1}{2}}
\newcommand{\catname}{\mathsf}
\newcommand{\hrulebar}{
    \par\hspace{\fill}\rule{0.95\linewidth}{.7pt}\hspace{\fill}
    \par\nointerlineskip \vspace{\baselineskip}
}
\providecommand{\arc}[1]{\wideparen{#1}}

%For use in author command
\newcommand{\plusemail}[1]{\\ \normalfont \texttt{\mailto{#1}}}

%More commands and math operators
\DeclareMathOperator{\cis}{cis}
\DeclareMathOperator*{\lcm}{lcm}
\DeclareMathOperator*{\argmin}{arg min}
\DeclareMathOperator*{\argmax}{arg max}

%Convenient Environments
\newenvironment{soln}{\begin{proof}[Solution]}{\end{proof}}
\newenvironment{parlist}{\begin{inparaenum}[(i)]}{\end{inparaenum}}
\newenvironment{gobble}{\setbox\z@\vbox\bgroup}{\egroup}

%Inequalities
\newcommand{\cycsum}{\sum_{\mathrm{cyc}}}
\newcommand{\symsum}{\sum_{\mathrm{sym}}}
\newcommand{\cycprod}{\prod_{\mathrm{cyc}}}
\newcommand{\symprod}{\prod_{\mathrm{sym}}}

%From H113 "Introduction to Abstract Algebra" at UC Berkeley
\newcommand{\CC}{\mathbb C}
\newcommand{\FF}{\mathbb F}
\newcommand{\NN}{\mathbb N}
\newcommand{\NNO}{\mathbb N_0}
\newcommand{\QQ}{\mathbb Q}
\newcommand{\RR}{\mathbb R}
\newcommand{\ZZ}{\mathbb Z}
\newcommand{\charin}{\text{ char }}
\DeclareMathOperator{\sign}{sign}
\DeclareMathOperator{\Aut}{Aut}
\DeclareMathOperator{\Inn}{Inn}
\DeclareMathOperator{\Syl}{Syl}
\DeclareMathOperator{\Gal}{Gal}
\DeclareMathOperator{\GL}{GL} % General linear group
\DeclareMathOperator{\SL}{SL} % Special linear group
\DeclareMathOperator{\Vol}{Vol} % Special linear group

%From Kiran Kedlaya's "Geometry Unbound"
\newcommand{\dang}{\measuredangle} %% Directed angle
\newcommand{\ray}[1]{\overrightarrow{#1}}
\newcommand{\seg}[1]{\overline{#1}}


%From M275 "Topology" at SJSU
\DeclareMathOperator{\id}{id}
\newcommand{\taking}[1]{\xrightarrow{#1}}
\newcommand{\inv}{^{-1}}

%From M170 "Introduction to Graph Theory" at SJSU
\DeclareMathOperator{\diam}{diam}
\DeclareMathOperator{\ord}{ord}
\newcommand{\defeq}{\overset{\mathrm{def}}{=}}

%From the USAMO .tex files
\newcommand{\ts}{\textsuperscript}
\newcommand{\dg}{^\circ}
\newcommand{\ii}{\item}

% From Math 55 and Math 145 at Harvard
\newenvironment{subproof}[1][Proof]{%
    \begin{proof}[#1] \renewcommand{\qedsymbol}{$\blacksquare$}}%
    {\end{proof}}

\newcommand{\liff}{\leftrightarrow}
\newcommand{\lthen}{\rightarrow}
\newcommand{\opname}{\operatorname}
\newcommand{\surjto}{\twoheadrightarrow}
\newcommand{\injto}{\hookrightarrow}
\newcommand{\On}{\mathrm{On}} % ordinals
\DeclareMathOperator{\img}{im} % Image
\DeclareMathOperator{\Img}{Im} % Image
\DeclareMathOperator{\coker}{coker} % Cokernel
\DeclareMathOperator{\Coker}{Coker} % Cokernel
\DeclareMathOperator{\Ker}{Ker} % Kernel
\DeclareMathOperator{\Spec}{Spec} % spectrum
\DeclareMathOperator{\pr}{pr} % projection
\DeclareMathOperator{\ext}{ext} % extension
\DeclareMathOperator{\pred}{pred} % predecessor
\DeclareMathOperator{\dom}{dom} % domain
\DeclareMathOperator{\ran}{ran} % range
\DeclareMathOperator{\Hom}{Hom} % homomorphism
\DeclareMathOperator{\Mor}{Mor} % morphisms
\DeclareMathOperator{\End}{End} % endomorphism

% Things Lie
\newcommand{\kb}{\mathfrak b}
\newcommand{\kg}{\mathfrak g}
\newcommand{\kh}{\mathfrak h}
\newcommand{\kn}{\mathfrak n}
\newcommand{\ku}{\mathfrak u}
\newcommand{\kz}{\mathfrak z}
\DeclareMathOperator{\Ext}{Ext} % Ext functor
\DeclareMathOperator{\Tor}{Tor} % Tor functor
\newcommand{\gl}{\opname{\mathfrak{gl}}} % frak gl group
\def\sl{\opname{\mathfrak{sl}}} % frak sl group chktex 6

% More script letters etc.
\newcommand{\SA}{\mathcal A}
\newcommand{\SB}{\mathcal B}
\newcommand{\SC}{\mathcal C}
\newcommand{\SF}{\mathcal F}
\newcommand{\SG}{\mathcal G}
\newcommand{\SH}{\mathcal H}
\newcommand{\OO}{\mathcal O}

\newcommand{\SCA}{\mathscr A}
\newcommand{\SCB}{\mathscr B}
\newcommand{\SCC}{\mathscr C}
\newcommand{\SCD}{\mathscr D}
\newcommand{\SCE}{\mathscr E}
\newcommand{\SCF}{\mathscr F}
\newcommand{\SCG}{\mathscr G}
\newcommand{\SCH}{\mathscr H}

% Mathfrak primes
\newcommand{\km}{\mathfrak m}
\newcommand{\kp}{\mathfrak p}
\newcommand{\kq}{\mathfrak q}


%aliases
\newcommand{\auth}[1]{\emph{#1}}
\newcommand{\para}[1]{#1 \par}
\newcommand{\parbreak}{\medskip}

%some stuff
\newcommand{\pointspades}[1]{[{\color{red}#1}\spadesuit]}
\newcommand{\irrevr}{\marginpar{\large \bfseries Extra}}
\newcommand{\irrevl}{\reversemarginpar\marginpar{\raggedleft\large\bfseries Extra}\normalmarginpar}

%geo
\newcommand{\rantri}{\tkzDefPoints{0/1/A,4/3/B,5/1/C}%
\tkzDrawPolygon(A,B,C)}
\newcommand{\coor}[2]{\tkzDefPoint(#1){#2}}
\newcommand{\Triangle}[1]{\tkzDrawPolygon(#1)}
\newcommand{\polygon}[1]{\tkzDrawPolygon(#1)}



\newcommand{\Line}[1]{\tkzDrawSegment(#1)}

\newcommand{\equi}[1]{\tkzDefTriangle[equilateral](#1)}
\newcommand{\twoang}[2]{\tkzDefTriangle[two angles = #1](#2)}
\newcommand{\isoright}[1]{\tkzDefTriangle[isosceles right](#1)}
\newcommand{\getp}[1]{\tkzGetPoint{#1}}

\newcommand{\centroid}[2]{\tkzDefTriangleCenter[centroid](#1)
\tkzGetPoint{G}\tkzDrawPoints(G)\tkzLabelPoints[#2](G)}

\newcommand{\incentre}[2]{\tkzDefCircle[in](#1) \tkzGetPoints{I}{a}
\tkzDrawPoints(I)\tkzLabelPoints[#2](I)}

\newcommand{\incircle}[1]{\tkzDefCircle[in](#1) \tkzGetPoints{I}{a}
\tkzDrawCircle(I,a)}

\newcommand{\circumcentre}[2]{\tkzDefCircle[circum](#1) 
\tkzGetPoint{O}\tkzDrawPoints(I)\tkzLabelPoints[#2](I)}

\newcommand{\circumcircle}[2]{\tkzDefCircle[circum](#1) \tkzGetPoint{O}
\tkzDrawCircle(O,#2)}

\newcommand{\orthocentre}[2]{\tkzDefTriangleCenter[ortho](#1)
\tkzGetPoint{H}\tkzDrawPoints(H)\tkzLabelPoints[#2](H)}

\newcommand{\orthopoints}[3]{\tkzDefSpcTriangle[orthic](#1,#2,#3){H_#1,H_#2,H_#3}}
\newcommand{\ortho}[3]{\tkzDefSpcTriangle[orthic](#1,#2,#3){H_#1,H_#2,H_#3}
\tkzDrawSegments(#1,H_#1 #2,H_#2 #3,H_#3)
\tkzMarkRightAngles[fill=gray!20,
opacity=.5](#1,H_#1,#3 #2,H_#2,#1 #3,H_#3,#1)}

\newcommand{\rightang}[1]{\tkzMarkRightAngles[fill=gray!20,
opacity=.5](#1)}

\newcommand{\drawsquare}[1]{\tkzDefSquare(#1)
\tkzDrawPolygon(#1,tkzFirstPointResult,%
tkzSecondPointResult)}

\newcommand{\angname}[2]{\tkzLabelAngle[pos=1](#2){$#1$}}
\newcommand{\foot}[3]{\tkzDefLine[perpendicular=through #1,K=-.5](#2,#3)\tkzGetPoint{c}
\tkzDefPointBy[projection=onto #2--#3](c)\tkzGetPoint{h}}

\newcommand{\project}[3]{\tkzDefPointBy[projection=onto #2](#1) \tkzGetPoint{#3}}

\newcommandx{\empangle}[4][1=0.5,2=black,3=|]{\tkzMarkAngle[size=#1,color=#2,mark=#3](#4)}
\newcommandx{\vertice}[2][1=left]{\tkzDrawPoints(#2)\tkzLabelPoints[#1](#2)}

\newcommandx{\fillangle}[3][1=orange]{\tkzDrawSector[R with nodes,fill=#1!20](#2,0.25)(#3)}

\newcommand*\len[1]{\overline{#1}}

%framed
\mdfdefinestyle{MyFrame}{%
    linecolor=black,
    outerlinewidth=0.05pt,
    %roundcorner=20pt,
    %backgroundcolor=gray!50!white}
        }

\newcommand\header[1]{
  \newlength{\headerwidth}
  \setlength{\headerwidth}{\widthof{#1}}
  \addtolength{\headerwidth}{8pt}
  \begin{mdframed}[style=MyFrame,userdefinedwidth=\headerwidth]
    #1
  \end{mdframed}
}
