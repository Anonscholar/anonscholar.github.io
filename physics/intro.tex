\chapter{Preface}

This is a set of notes I am writing (as of now) in my high-school years --- part of which I hope to 
publish one day. These are mostly meant to cover the whole of undergrad physics, covering, essentially  
newtonian mechanics(including special relativity), classical electromagnetism, waves and optics, thermodynamics and statistical mechanics, 
quantum mechanics and analytical mechanics. There will also be some introductory sections on more advanced graduate 
topics and generally the depth of material covered will be higher than the standard undergrad. 

Out of these the sections that I wish to publish include everything but higher introductions, and analytical mechanics. 
I may write some sections on other topics as well and even in that case, most of them will be excluded.

\section{Philosophy and Style}

The book is particularly written for advanced high-schoolers, perhaps ones 
interested in physics olympiads who want to experience a broader coverage of physics in general.
The whole text will also be certainly useful for undergrads with a strong background. 

Frankly, I am not a master of the subjects I am writing on, all that I have written here is either 
borrowed from others and the perhaps only difference from each of them is in the way of exposition.
The real catch is that it because it borrows from a lot of places --- it is perhaps more complete 
than each of them, where I have tried to omit the weakest section of the books I have borrowed from,
while keeping intact the wonderful ones.

These notes are meant to adhere to a particular style, they're meant to be 
rigorous, both physically and mathematically. I try not to pull any wool over your eyes, and 
we experience how to think, in essence, somewhat like a physicist. Tools such as approximations 
are particularly talked about so that you know how to use them. 

This also means that some of the sections introduce terms you may not be typically introduced to in an introductory text.
The math background required for the text is \textbf{not} covered in the text itself, calculus(both single and multi) is assumed 
a prerequisite. If you are not familiar with either --- I would recommend going through a calculus course 
such as MIT OCW 18.01 and 18.02. The other prereqs, like probability theory for statistical mechanics 
and linear algebra for quantum mechanics, will be given an intro. They will be much better covered in 
the other set of notes I have, for undergraduate mathematics. You may look at them, if you're interested. 

Hopefully this book gives you something new, and is innovative in some way or the other. If you like it, please 
let me know! It would mean a ton to me.

Good luck, and happy learning.

\clearpage
\thispagestyle{empty}
    \vspace*{\fill}
    {\em \centering \hspace{3cm} To the ones who believed in me.}
    \vspace*{\fill}
