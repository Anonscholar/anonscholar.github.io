\chapter{Electrostatics}

\motiv{The world around is very well governed largely at day to day 
scales by electromagnetic interactions. Tensions, friction 
and a lot more are results of electrical forces between 
microscopic particles. So a study of electromagnetism is 
of vital importance to us.}

\emph{In electrostatics we will deal with the study of stationary charges. We will 
discuss Coulomb's law and further alternate versions of it 
for high symmetries like Gauss's law. This forms the basis of all 
of electromagnetic theory we will encounter.}

\section{Electric Charge}

The theory of electromagnetism is centered around the idea of charge. The 
fundamental property of charge is their duality --- there are two 
particular types of charges, \emph{positive, and negative}. 

The basic \emph{independent} unit of charge is the electron, the charge of which, 
we denote \(-e\) where the negative sign represents that it is negative\footnote{This
is just convention, we could have as well denoted it positive, and it wouldn't 
have made a difference.}. 

In contrast, the positive charge of equal magnitude is found in proton, 
whose charge is \(e\). This nice set of duality is rather interesting and maybe 
linked to a more fundamental concept which we'll see in a moment. 

Charges in particular follow two nice set of properties --- \chvocab{Charges}{conservation} and 
\chvocab{Charges}{quantization}.

\subsection{Conservation of Charges}

Any isolated system of charges, which basically means that no matter is allowed 
to cross into the boundary of our system; is conserved. What we mean is that the 
algebraic sum of the charges is always constant.

The isolation isn't exactly the most general criteria, we can let a photon 
cross over into our system since it carries no charges and observe a very interesting 
phenomenon, of \vocab{pair creation}. 

A photon when exposed to gamma rays may end its existence and create an electron 
and \emph{positron}, the anti-particle of electron which carries a charge of \(e\) but 
is same in all other respects to an electron\footnote{In contrast to proton 
which has a different mass and so on.}. Note that this still ensures conservation,
the algebraic sum is still the same. Thus, it follows the ``law of charge 
conservation'', which we'll take as something fundamental.

Taking this as more fundamental also maybe allows us to answer 
why the proton has equal magnitude of charge as an electron. Some 
certain theories suggest that a proton \emph{may} decay into a positron
and some uncharged particles. This event must follow the law of charge 
conservation and thus implies that a proton has the same charge as that of 
a positron.
However, no one has observed with certainty such a decay.

\subsection{Quantization of Charge}

Charge in its fundamental form is only present as in an electron or proton, 
so a system of charges must have a charge of \(\pm ne\), where \(n \in \ZZ\).
Quarks that exist inside the proton contain a charge in multiplies of \(\pm e/3\).

In particular in the neutron two quarks consist of charge \(-e/3\) and one with 
charge \(2e/3\), rendering the overall charge neutral. 
Quarks, however, have yet to be found singularly, thus we have not observed such fractional charges
and the basic unit of charges remains \(e\). 

\section{Coulomb's Law}

\begin{marginfigure}
    \centering
    \scalebox{2}{\incfig{coulomb}}
    \caption{Two charges \(q_1\) and \(q_2\) and the 
    radial vector between them.}
\end{marginfigure}

Coulomb's law, which we take as an empirical law, states that 
for point charges of charge \(q_1\), \(q_2\), the electrical force 
on the charge \(2\) by \(1\) is,

\begin{equation}
    \label{coloumb force}
    \boxed{\vec{F}_{21} = k\frac{q_1q_2}{\rcurs}\hrcurs}
\end{equation}

where according to standard convention, \(\unitv{r}_{21}\) is the unit vector 
in the direction of the line from charge \(1\) to \(2\) and \(\vec{F}_{21}\) is 
the force on \(2\) by \(1\).

We could also argue that the line joining the two charges is the only possible direction 
the force could act at all. After-all, for stationary charges, there is no other 
\emph{unique} direction\footnote{Note of course, that this is not true for 
moving charges and in fact one component of the force acts along the direction 
\(\vec{r} \cp (\vec{r} \cp \vec{a})\).}.

The constant \(k\) here evaluates to \(8.988 \times 10^5\) \unit{\newton\meter\squared\per\coulomb\squared}.
For reasons involving easier calculations, \(k\) is often written as,

\begin{equation}
    k \equiv \frac{1}{4\pi\epsilon_0} \iff \boxed{\epsilon_0 = \frac{1}{4\pi k} = 8.854 \times 10^{-12} \unit{\coulomb\squared\per\newton\per\meter\squared}}
\end{equation}

We will see its use when talking about Gauss's law.

Note that for two alike charges, \(q_1q_2 > 0\) and the electrical force points along the direction of 
\(\vec{r}_{21}\) or from the charge \(1\) towards \(2\). Thus, it is repulsive 
in nature. The converse is clearly true for particles of unlike charge.