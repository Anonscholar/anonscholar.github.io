\chapter{Prerequisite Mathematics}

\section{Vectors}

Vectors are quantities with magnitude and direction, contrastingly to scalars which only have magnitude.
I shall use boldface to denote vectors, \(\vect{A}\). The magnitude of the vector \(\vect{A}\),
shall be denoted as \(\norm{\vect{A}}\) or simply \(A\). \(\unitv{A}\) is 
the vector of magnitude \(1\) pointing in the direction of \(\vect{A}\).

\[
    \unitv{A} = \frac{\vect{A}}{A}
\]

\emph{Minus} \(\vect{A}\), \(-\vect{A}\) is a vector with the same magnitude as \(\vect{A}\)
but point in the opposite direction. I assume familiarity with all the base operations of 
vectors.  

The scalar product of \(\vect{A}\) and \(\vect{B}\) is
\[
    \vect{A} \dotproduct \vect{B} = AB \sin \theta
\]

The vector product is defined as,
\[
    \vect{A} \times \vect{B} = AB \cos \theta \unitv{n}
\]

The direction of \(\unitv{n}\) is perpendicular to the plane of \(\vect{A}\) and \(\vect{B}\). 
To remove ambiguity, \(\vect{A}\), \(\vect{B}\) form a right hand triple. Point the
fingers of your right hand in the direction of \(\vect{A}\), and fold your fingers towards
\(\vect{B}\) in the direction of smallest angle. The direction in which your thumb points
is the direction of \(\unitv{n}\).

In component form,
\(\vect{A} = (A_x, A_y, A_z)\)

The scalar product is now,
\[
    \vect{A} \dotproduct \vect{B} = A_xB_x + A_yB_y + A_zB_z
\]

And the vector product is defined as, 

\[
\begin{vmatrix}
    \unitv{i} & \unitv{j} & \unitv{k} \\
    A_x & A_y & A_z \\
    B_x & B_y & B_z
\end{vmatrix}
\]

\subsection{Triple Products}

\emph{Scalar Triple Product}, \(\vect{A} \dotproduct (\vect{B} \times \vect{C})\) is geometrically
the volume of the parallelepiped generated by \(\vect{A}\), \(\vect{B}\) and \(\vect{C}\). 

\[
    \vect{A} \dotproduct (\vect{B} \cross \vect{C}) = \vect{B} \dotproduct (\vect{C} \cross \vect{A}) 
    = \vect{C} \dotproduct (\vect{A} \cross \vect{B}) 
\]

The dot and cross can be interchanged,
\[
    \vect{A} \dotproduct (\vect{B} \cross \vect{C}) = (\vect{A} \cross \vect{B}) \dotproduct \vect{C}
\]

\noindent \emph{Vector Triple Product}, \(\vect{A} \cross (\vect{B} \cross \vect{C})\). It can
be simplified using the \textbf{ABC-CAB} rule,

\[
    \vect{A} \cross (\vect{B} \cross \vect{C}) = \vect{B}(\vect{A} \dotproduct \vect{C}) -
    \vect{C}(\vect{A} \dotproduct \vect{B})
\]


