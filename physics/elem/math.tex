\chapter{Prerequisite Mathematics}

\section{Vectors}

Vectors are quantities with magnitude and direction, contrastingly to scalars which only have magnitude. 
For our purposes, vectors can be understood as directed line segments, although
they're more rigorously defined as elements of a vector space, or can be 
understood through rotations. We'll talk about rotations later, but will not deal with
vector spaces right now.

I shall use boldface to denote vectors, such as \(\vec{A}\). 

The utility of vectors stem from their independence from a particular co-ordinate system. They're
directed \emph{line segments}, only their length
matters and we're not concerned with their end points. This means that a vector 
can be freely translated. 

\begin{marginfigure}
    \scalebox{2}{\incfig{transvectors}}
    \caption{Identical Vectors}
\end{marginfigure}

The magnitude of a vector, its norm, is its euclidean ``length'', we denote it
by \(\norm*{\vec{A}}\) or simply \(A\). 

A unit vector, \(\unitv{A}\) (``A hat'') is a vector whose magnitude is unity
and which points in the direction of \(\vec{A}\). We use unit vectors when we're merely
concerned with the direction of a vector. It is defined as,

\[
    \unitv{A} = \frac{\vec{A}}{A}
\]

\emph{Minus} \(\vec{A}\), \(-\vec{A}\) is a vector with the same magnitude as \(\vec{A}\)
but point in the opposite direction. I assume familiarity with all the base operations of 
vectors.  

The scalar product of \(\vec{A}\) and \(\vec{B}\) is
\[
    \vec{A} \dtp \vec{B} = AB \sin \theta
\]

The vector product is defined as,
\[
    \vec{A} \times \vec{B} = AB \cos \theta \unitv{n}
\]

The direction of \(\unitv{n}\) is perpendicular to the plane of \(\vec{A}\) and \(\vec{B}\). 
To remove ambiguity, \(\vec{A}\), \(\vec{B}\) form a right hand triple. Point the
fingers of your right hand in the direction of \(\vec{A}\), and fold your fingers towards
\(\vec{B}\) in the direction of smallest angle. The direction in which your thumb points
is the direction of \(\unitv{n}\).

In component form,
\(\vec{A} = (A_x, A_y, A_z)\)

The scalar product is now,
\[
    \vec{A} \dtp \vec{B} = A_xB_x + A_yB_y + A_zB_z
\]

And the vector product is defined as, 

\[
\begin{vmatrix}
    \unitv{i} & \unitv{j} & \unitv{k} \\
    A_x & A_y & A_z \\
    B_x & B_y & B_z
\end{vmatrix}
\]

\subsection{Triple Products}

\emph{Scalar Triple Product}, \(\vec{A} \dtp (\vec{B} \times \vec{C})\) is geometrically
the volume of the parallelepiped generated by \(\vec{A}\), \(\vec{B}\) and \(\vec{C}\). 

\[
    \vec{A} \dtp (\vec{B} \cp \vec{C}) = \vec{B} \dtp (\vec{C} \cp \vec{A}) 
    = \vec{C} \dtp (\vec{A} \cp \vec{B}) 
\]

The dot and cross can be interchanged,
\[
    \vec{A} \dtp (\vec{B} \cp \vec{C}) = (\vec{A} \cp \vec{B}) \dtp \vec{C}
\]

\noindent \emph{Vector Triple Product}, \(\vec{A} \cp (\vec{B} \cp \vec{C})\). It can
be simplified using the \textbf{ABC-CAB} rule,

\[
    \vec{A} \cp (\vec{B} \cp \vec{C}) = \vec{B}(\vec{A} \dtp \vec{C}) -
    \vec{C}(\vec{A} \dtp \vec{B})
\]



