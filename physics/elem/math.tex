\chapter{Prerequisite Mathematics}

\section{Vectors}

Vectors are quantities with magnitude and direction, contrastingly to scalars which only have magnitude. 
For our purposes, vectors can be understood as directed line segments, although
they're more rigorously defined as elements of a vector space, or can be 
understood through rotations. We'll talk about rotations later, but will not deal with
vector spaces right now.

I shall use boldface to denote vectors, such as \(\vec{A}\). 

The utility of vectors stem from their independence from a particular co-ordinate system. They're
directed \emph{line segments}, only their length
matters and we're not concerned with their end points. This means that a vector 
can be freely translated. 

\begin{marginfigure}
    \scalebox{2}{\incfig{transvectors}}
    \caption{Identical Vectors}
\end{marginfigure}

The magnitude of a vector, its norm, is its euclidean ``length'', we denote it
by \(\norm*{\vec{A}}\) or simply \(A\). 

A unit vector, \(\unitv{a}\) (``a hat'') is a vector whose magnitude is unity.
We use the unit vector to often denote the direction of a vector by multiplying 
the unit vector with its magnitude. For instance, the unit vector
that point is the direction of \(A\), \(\unitv{A}\) can be calculated as,
\[ \unitv{A} = \frac{\vec{A}}{A}\] 
where we divide by the magnitude of \(A\) to ensure that the magnitude of \(\unitv{a}\) is
\(1\).

\section{Vector Algebra}    

There are two operations generally defined on vectors, \vocab{scalar multiplication}
and \vocab{vector addition}. 

\subsection{Scalar Multiplication}

Multiplying a vector \(\vec{A}\) by a scalar \(c\) results
in another vector, \(c\vec{A}\) parallel or antiparallel(pointing in the opposite 
direction) to \(\vec{A}\) scaled by a factor of \(\abs*{c}\).

If \(c\) is positive, then the resultant vector is simply scaled by it.
If we multiply a vector by \(-1\), then the result is the simple
reversal in direction of vectors. If \(c\) is negative, then the 
vector is reversed and scaled by \(\abs*{c}\). 

\begin{marginfigure}
    \centering
    \scalebox{2}{\incfig{scalarmultiplication}}
    \caption{Scalar Multiplication}
\end{marginfigure}

\subsection{Vector Addition}

We define an operation called addition of vectors which produces another vector.
To add two vectors, \(\Aa\) and \(\BB\), we stick the tail
of one vector with the head of another. 

\begin{figure}[H]
    \centering
    {\incfig{vectoradd}}
    \label{fig: vectoradd}
    \caption{Vector Addition}
\end{figure}

We can calculate the magnitude of their resultant vector, \(\norm*{\Aa + \BB}\)
by using the cosine rule. In \cref{fig: vectoradd} \(\theta\) is the angle between 
\(\Aa\) and \(\BB\).

\begin{align*}
    &\norm*{\Aa + \BB}^2 = A^2 + B^2 - 2AB\cos(\pi - \theta) \\
    &\norm*{\Aa + \BB}^2 = A^2 + B^2 + 2AB\cos\theta \\
    &\norm*{\Aa + \BB} = \sqrt{A^2 + B^2 + 2AB\cos\theta}
\end{align*}
 
To subtract two vectors, we simply reverse the vector and then add the resulting vector.

\begin{example}
    Consider the infinitesimal change vector, \(\dd{\Aa} = \Aa(t + \dd{t}) - \Aa(t)\).
    
    We have, 
    \begin{align*}
        \dd{\Aa} &= A(t + \dd{t})\unitv{A}(t + \dd{t}) - A(t)\unitv{A}(t) \\
        \dd{\Aa} &= A(t + \dd{t})\unitv{A}(t + \dd{t}) - A(t + \dd{t})\unitv{A}(t) \\ 
        & + A(t + \dd{t})\unitv{A}(t) - A(t)\unitv{A}(t) \\
        \dd{\Aa} &= A(t + \dd{t})(\unitv{A}(t + \dd{t}) - \unitv{A}(t)) + \unitv{A}(t)(A(t + \dd{t}) - A(t)) \\
        \dd{\Aa} &= A\dd{\unitv{A}} + \dd{A}\unitv{A} 
    \end{align*}
The first term represents a change in direction and the second represents a change in 
magnitude.
\end{example}

The magnitude of the first can be calculated as,

Let us first consider a geometric argument for \(A\dd{\unitv{A}}\)
which is perpendicular to \(\vec{A}\). In the limiting case, \(A\dd{\unitv{A}}\),
the change in magnitude is negligible, we only observe a change in direction. For now,
let us consider \(\increment \Aa\) perpendicular to \(\Aa\) as an approximation.

Consider the following figure,

\begin{figure}
    [H]
    \centering
    \scalebox{0.7}{\incfig{timedev}}
\end{figure}

\para{\noindent This allows us to consider a simple geometric argument which gives \(\norm{\increment \vec{A}} = 2A\sin(\theta/2)\)
where \(\theta\) is the angle between \(\vec{A}(t_1)\) and \(\vec{A}(t_2)\).}

Since we're considering values of small \(\theta\), we may use the small angle approximation,
\(\sin(\theta) \approx \theta\). Therefore,

\begin{align}
    \norm{\vec{\increment A}} &\approx 2A \times \frac{\theta}{2}\\ 
    &\approx A \times \theta
\end{align}

Which is equivalent to \(A \increment \theta\). In the limit \(\increment \theta \to 0\),

\begin{equation}
    \dd{A_{\perp}} = A \dd{\theta}
\end{equation}

Which finally gives us,
\begin{equation}
    \norm{\dv{\vec{A}_{\perp}}{t}} = A\dv{\theta}{t}
\end{equation}

which is the magnitude of \(A\dd{\unitv{A}}\). Thus,
\begin{equation}
    \norm*{\dd{\unitv{A}}} = \dv{\theta}{t}
\end{equation}

\section{Dot Product}

The \vocab{dot product} of two vectors results in a scalar and is defined as,
\[\Aa \dtp \BB = AB \cos\theta\]
where \(\theta\) is the angle between \(\Aa\) and \(\BB\). 

Note that the quantity \(\Aa \dtp \BB\) is just \(A\) times 
the magnitude of the projection of \(\BB\) on \(\Aa\). Similarly, \(\BB \dtp \Aa\) is
magnitude of the projection of \(\Aa\) on \(\BB\) times \(B\).

We may note that,

\[\Aa \dtp \Aa = A^2\] Or,
\[A = \sqrt{\Aa \dtp \Aa}\]

It can be easily shown that the scalar product distributes over vector addition. 
Another thing of importance is the identity(which is also easy to derive),

\[\dv{\Aa \dtp \BB}{t} = \dv{\Aa}{t} \dtp \BB + \dv{\BB}{t} \dtp \Aa\]

\section{Cross Product}

The \vocab{cross product} is another product operation on vectors, this product produces a vector.

It is defined as, 
\[\Aa \cp \BB = AB\sin\theta \unitv{n}\] 

Where \(\unitv{n}\) is the vector perpendicular to the plane containing \(\Aa\) and \(\BB\).
Since there are two directions that \(\unitv{n}\) can have, we define the two vectors 
and their cross product to form a right-hand triple.

Place your fingers in the direction of \(\Aa\) and curl them, along the smaller angle, towards 
\(\BB\). Then the direction in which your thumb is pointing is the direction of \(\Aa \cp \BB\). 

\section{Vectors in Component form}

Although we have worked without co-ordinate systems till now,
to actually extract any meaning from vector operations, a coordinate system is necessary
though the concept it describes is actually independent of any coordinate
system.
A spatial, three-dimensional coordinate system essentially establishes
three basis vectors.

Basis vectors are vectors that are linearly independent. That is,
their linear combination, 
\[
    c_1\vec{e}_1 + c_2\vec{e}_2 + \dots + c_n\vec{e}_n\] 

only has a trivial solution of \(c_1 = c_2 = \dots = c_n = 0\), when equated to the \(n\) dimensional null
vector, \(c_1\vec{e}_1 + c_2\vec{e}_2 + \dots + c_n\vec{e}_n = \vec{0}\). 