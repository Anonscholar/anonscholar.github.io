\chapter{Unit and Dimensions}


Units are dimension are in many ways what give the abstract ideas of physics some
ground in the real life. The dimensions of a physical quantity represent what
kind of physical quantity is, while the unit is a measure of that physical quantity.

The distinction is appreciable to know, just not always useful.

\section{Dimensions}    

There are seven fundamental quantities which are assigned specific dimensions. It is not
necessary to take these as fundamental, in some perfectly fine formulations like the Planck quantities
this is not the case. But we'll not be working in these for now.

\begin{table}[H]
    \centering
    \begin{tabular}{llcc} \toprule
        Quantity & Unit Name & Unit symbol & Dimension \\ \midrule
        Mass & kilogram & \unit{\kilogram} & M  \\
        Length & meter & \unit{\meter} & L\\
        Time & second & \unit{\second}& T \\
        Electric Current & ampere &\unit{\ampere} & I \\
        Luminous Intensity & candela & \unit{\candela}  & J \\
        Temperature & kelvin & \unit{\kelvin} & \(\Theta\) \\
        Amount of substance & mole & \unit{\mole} & N \\ \bottomrule
    \end{tabular}
    \caption{The \(7\) fundamental quantities.}
    \label{tbl: fundamental quantities}
\end{table}

For any physical quantity \(A\), we use the notation \([A]\) to denote
its dimension. For instance, \[
    [F] = MLT^{-2} 
    \]Evaluating the dimensions has a really neat use, it allows us to check
    our answers, and establish some relations through the use of \emph{dimensional}
    analysis.   

\section{Rules regarding dimensions}

There are particle constrains that help us utilise dimensions quite a lot. For one,
the dimensions on both sides of an equation must be homogenous, i.e the same. It is nonsense
to ask if \(1\) second is equal to \(1\) kelvin. 

Another constraint is that the operations of addition and subtraction can only be 
performed on quantities of the same dimension, and indeed you cannot add a metre to a kilogram.

Thus, we arrive at a third derived constraint, namely that the functions such as \(\sin\), 
\(\cos\), \(\exp\), \(\log\) must all have dimensionless constraints. 

This because for instance, let us take the taylor expansion of \(e^x\),

\begin{equation*}
    e^x = 1 + x + \frac{x^2}{2} + \frac{x^3}{6} + \dots
\end{equation*}

We note that if we \(x\) had a dimension, for example \([x] = L\), then we would be adding 
\(L\) to \(L^2\), \(L^3\) and so on. But this is not permitted, we cannot add a metre to a square metre
or a square metre to a cubic metre and so on. Thus, we have that their arguments
must be dimensionless.

\section{Dimensional Analysis}

Suppose that we know that the rest energy of a particle depends on it mass and 
the speed of light. If we were asked to find a relation determining its rest energy
we could use the idea of dimensional analysis.

We see that, 
\begin{equation*}
    E \propto m, c
\end{equation*}

We could note that the dimesion on both side of the equation must have the same dimesions. Thus 
we need to combine them in such a manner that we get the dimensions of energy. 

Since \([c] = LT^{-1}\), \([m] = M\) and \([E] = ML^{-2}T^{-2}\), we can form a guess 
and say that,

\begin{equation*}
    E = mc^2,
\end{equation*}

since the dimensions are same across both quantities. In fact this the correct formula!

In general, for some quantity \(A\) dependent on parameters \(Q_1, Q_2, \dots, Q_n\), 

\begin{equation*}
    A = Q_1^{\alpha_1} \cdot Q_2^{\alpha_2} \cdot \dots \cdot Q_n^{\alpha_n}
\end{equation*}

where each of \(\alpha_1, \alpha_2, \dots, \alpha_n\) are constants. Then we simply use dimensional 
analysis to compute the result. 

There are catches, however, suppose that we were asked to find the kinetic 
energy of the particle which depends on its mass and velocity. We might guess 
by dimensional analysis alone that it is \(mv^2\), which is obviously false. What dimensional
analysis gives us is that,
\begin{equation*}
    E \propto mv^2
\end{equation*} 

And we must attach a constant for a definite answer, 

\begin{equation*}
    E = kmv^2
\end{equation*}

The constants comes out to be \(1/2\), but we can't determine it with 
dimensional analysis alone. 

Let us take a particular example now, the pendulum: its period depends on \(L\), \(g\), and the
amplitude \(\theta_0\).  

We note that the amplitude and angles in general are dimensionless since they're arguments 
of the trig functions. If we use dimensional analysis, we run into a bit of a problem.

If we use dimensional analysis, we see that we can use any combination of \(\theta_0\), 
since they're all dimensionless. In particular, we can use any function of \(\theta_0\).

If we ignore \(\theta_0\) since its dimensionless, we get that 

\begin{equation*}
    T \propto \sqrt{\frac{L}{g}}
\end{equation*}
We hence attach a dimensionless group, a function of \(\theta_0\) to get the period of a pendulum as \(T = f (\theta_0)\sqrt{L/g}.\) 

In general this can be done for any dimensionless group, for instance the 
resistance of conductor depends on its resistivity, the cross-sectional area, 
and its length. Say we know their dimensions and get that, 

\begin{equation*}
    R = \rho \frac{L}{A}
\end{equation*}

We could have also formulated by dimensional analysis, that \(R = \rho/\sqrt{A}\), the 
dimensions are the same. In fact, we could multiply it by any function of 
\(L^2/A\) since it is dimensionless.

\begin{theorem}[Buckingham Pi Theorem]
    Dimensional analysis can't always pin down the form of the answer. If one has \(N\) 
    quantities with \(D\) independent dimensions, then one can form \(N - D\) 
    independent dimensionless quantities. Dimensional analysis can't say how the answer depends 
    on them.
\end{theorem}

This is one of the limitations of dimensional analysis.

\section{Limiting Cases}

Sometimes we can verify our answers using limiting cases.

Suppose that we don't remember if the acceleration of a block 
on a smooth, non-accelerating wedge is \(g\sin\theta\) or \(g\cos\theta\).

We can then consider the limiting cases of \(\theta\). For instance, 
if \(\theta \to 0\) then the block rests on a horizontal plane and experiences no 
acceleration, which is only satisfied by \(g\sin\theta\). We could have 
also looked at \(\theta \to \pi/2\) as another limiting case.

