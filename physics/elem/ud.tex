\chapter{Unit and Dimensions}

Units are dimension are in many ways what give the abstract ideas of physics some
ground in the real life. The dimensions of a physical quantity represent what
kind of physical quantity is, while the unit is a measure of that physical quantity.

The distinction is appreciable to know, just not always useful.

\section{Dimensions}    

There are seven fundamental quantities which are assigned specific dimensions. It is not
necessary to take these as fundamental, in some perfectly fine formulations like the Planck quantities
this is not the case. But we'll not be working in these for now.

\begin{table}[H]
    \centering
    \begin{tabular}{llcc} \toprule
        Quantity & Unit Name & Unit symbol & Dimension \\ \midrule
        Mass & kilogram & \unit{\kilogram} & M  \\
        Length & meter & \si{\meter} & L\\
        Time & second & \unit{\second}& T \\
        Electric Current & ampere &\unit{\ampere} & I \\
        Luminous Intensity & candela & \unit{\candela}  & J \\
        Temperature & kelvin & \unit{\kelvin} & \(\Theta\) \\
        Amount of substance & mole & \unit{\mole} & N \\ \bottomrule
    \end{tabular}
    \caption{The \(7\) fundamental quantities.}
    \label{tbl: fundamental quantities}
\end{table}

For any physical quantity \(A\), we use the notation \([A]\) to denote
its dimension. For instance, \[
    [F] = MLT^{-2} 
    \]Evaluating the dimensions has a really neat use, it allows us to check
    our answers, and establish some relations through the use of \emph{dimensional}
    analysis.   