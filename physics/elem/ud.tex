\chapter{Unit and Dimensions}

Units are dimension are in many ways what give the abstract ideas of physics some
ground in the real life. A unit is defined to be the fundamental unit of some physical
quantity. The metre for length, second for time and so on. The dimension of a physical 
quantity is something along the lines of a unit, but represented as the powers of 
the fundamental quantities. 

\section{Dimensions}    

There are seven fundamental quantities which are assigned specific dimensions. It is not
necessary to take these as fundamental, in some perfectly fine formulations like the Planck quantities
this is not the case. But we'll be working in for now.

\begin{table}[H]
    \centering
    \begin{tabular}{lll} \toprule
        Quantity & Unit & Dimension \\ \midrule
        Mass & kilogram & M \\
        Length & meter & L \\
        Time & second & T \\
        Electric Current & ampere & A \\
        Luminous Intensity & candela & Cd \\
        Temperature & kelvin & \theta \\
        Amount of substance & mole & mol \\ \bottomrule
    \end{tabular}
    \caption{The \(7\) fundamental quantities.}
    \label{tbl: fundamental quantities}
\end{table}

For any physical quantity \(A\), we use the notation \([A]\) to denote
its dimension. For instance, \[
    [F] = MLT^{-2} 
    \]Evaluating the dimensions has a really neat use, it allows us to check
    our answers, and establish some relations through the use of \emph{dimensional}
    analysis.   