\chapter{Kinematics}

Kinematics is the study of motion without concerning its cause. It allows
us to calculate and find out the evolution of a body. Everything in our world undergoes
motion, one way or the other. To begin our study of motion, let us first
concern ourselves with some definitions.

\section{Reference Frame and Point Particle}

Motion of any object is considered to relative to an observer. The observer defines
a particular co-ordinate system called the reference frame

\begin{figure}[ht]
        \centering
        \incfig{3daxes}    
    \caption{A Reference Frame}\label{fig: 3daxes}
\end{figure}

We use a reference frame according to our convenience. In general, we use the 3-dimensional cartesian system like the one
in \cref{fig: 3daxes}.

\index{Reference Frame}

\begin{definition}
    [Reference Frame]
    A reference frame is a co-ordinate frame relative to which motion of a particle is considered.
\end{definition}

We shall, until we encounter rigid body systems consider the motion of a point particle. 
Well, why? It is done so to simplify the calculation of our system. The motion of a (rigid)body is discussed later.

\index{Particle}
\begin{definition}
    [Point Particle]
    A particle whose size is negligible in the study of its motion is called a point particle.
\end{definition}

To discuss the motion, state of our body, it is also necessary to setup some time co-ordinate.
It is the reading of the clock in the observer's frame. In general, we setup our clocks
at \(0\) at the start of an event.

\marginnote{One thing of importance is that a position vector of a particle is a function
of time. We generally denote this as \(r(t)\) to show that is a function.}

\section{Position Vector and Displacement}

\subsection{Position Vector}
\index{Position Vector}

We define the position of a particle relative to a reference frame using a \emph{position 
vector}. One end of the position vector is the origin of our reference frame. The other is 
the particle itself. Consider \cref{fig: position}, the 
vector \(\vec{r}\) from \(O\) to \(A\). It describes the position of \(A\) relative to the
co-ordinate frame. Let \(A = (x,y,z)\). Then we denoted \(\vec{r}\) as \(\vec{r}(x,y,z)\).

The position to any point is written as, \[
    \vec{r} = (x,y,z) = x\unitv{i} + y\unitv{j} + z\unitv{k}
\]

\begin{marginfigure}
    \centering
    \scalebox{2.2}{\incfig{position}}
    \caption{A position vector, \(\mathbf{r}\).}
    \label{fig: position}
\end{marginfigure}

\subsection{Displacement}
\index{Displacement}

\marginnote{Note that the position vector is not a \emph{true} vector. It is tied to particular
    reference frame.}

Consider the movement of a particle from \(A = (x_1,y_1,z_1)\) to \(B = (x_2,y_2,z_2)\). 
The \emph{displacement} is the change in the position vector,
and defines a true vector \(\vec{s} = (x_2-x_1,y_2-y_1,z_2-z_1)\).
\(\vec{s}\) is called the displacement vector. 
Note that it contains no information about the
individual positions, but only about the relative position of each. Thus, our choice
of reference frame, and thus position vector does not matter when we're concerned with displacement. 
In \cref{fig: displacement}, the vector
\(\vec{S}_{AB}\) defines a displacement from \(A\) to \(B\).

\begin{figure}
    [H]
    \centering
    \scalebox{1.4}{\incfig{displacement}}
    \caption{Displacement from \(A\) to \(B\)}
    \label{fig: displacement}
\end{figure}

The difference between displacement and distance is that of the path they describe. The distance covered is the length
of the actual path, while the magnitude of the displacement is simply the length of the vector between the final and initial positions.

\begin{example}
    Consider the case of displacement and distance of a particle in circular motion with radius \(r\) Let it move from \(A\) to \(B\), inscribing
    an angle \(\theta\) (in radians) Then, we have,
    \begin{itemize}
        \item Distance between \(A\), \(B\) = \(r\theta\)
        \item Displacement between \(A\), \(B\) = \(2r\sin\left(\dfrac{\theta}{2}\right)\)
    \end{itemize}
\end{example}

\begin{marginfigure}
    \centering
    \scalebox{2}{\incfig{distdisp}}
    \caption{Displacement in a circle}
\end{marginfigure}

\section{Speed and Velocity}

\index{Speed}

We are already familiar with the notion of speed. In elementary terms, speed is \({distance}/{time}\).
Speed itself is an elementary concept. It tells us nothing about the direction of the object whose speed we are
talking into consideration. Thus, it is a \emph{scalar} quantity.

\parbreak
\marginnote{The symbol \(\equiv\) stands for `defined as'.}

\index{Velocity}

The Velocity of a particle, contrastingly, contains information about both the speed and direction of the
object in consideration. 
It is a vector. Similar to position, it is also a function of time \(v(t)\), though it
may also be dependent on the position \(v(x)\). Even when that is the case,
it can still be written as function of time.

\begin{definition}
    [Velocity]
    \label{def: velocity}
    Velocity is defined as,
    \begin{equation}
        \vec{v} \equiv \ndiv{\vec{r}}    
    \end{equation}
\end{definition}

In particular, velocity is a function that maps three values --- the velocity components 
of the particle to a particle time instant, 

\[
    \vv : \RR \to \RR^3.
\] 

This is actually independent of the co-ordinate system we adopt. For any 
particle, we require merely three co-ordinates to describe its velocity. 

The same holds for position as well,

\[
    \vec{x} : \RR \to \RR^3.
\]

\sidenote{Infact to describe the motion of any body, we merely require its initial position and velocity functions.
Well, not quite, we also need the equations of motion, of how the system evolves. These 
end up being second order differential equations in Lagrangian Mechanics with which we'll deal
later.}
One thing of note is that velocity can also be expressed as the derivative of 
displacement, which is just \(\vec{r} - \vec{r}_0\). Since the latter 
term is constant, we simply get that \(\ndiv{\vec{s}} = \ndiv{\vec{r}}\).

From this definition, we may gather that,

\[\int_{r_0}^{r} \dd{\vec{r}} = \int_{0}^{t} \vv \dd{t}\]

Or, 
\begin{equation}
    \vec{r} = \vec{r}_0 + \int_{0}^{t} \vv \dd{t}
\end{equation}

Sometimes, we do not need to compute the derivative of the position vector, and simply need an average
estimate, or in the case of uniformly accelerated motion, it is not required that we computer derivatives.
In such a case we use the concept of \textbf{average velocity}, \[
    \bavg{\vec{v}}_{12} = \frac{\increment\vec{s}}{t_2 - t_1}  
\]

\marginnote{The dot over \(r\) in \(\dot{\vec{r}}\) tells us that \(r\) has been differentiated
with respect to time.}

\begin{algorithm}
    Let us consider the cases of average velocity,
    \begin{casework}
        \ii When there are \(n\) equal time intervals with velocity \(\vec{v}_1, \vec{v}_2, \ldots, \vec{v}_n\).
        Then, \[
            \bavg{\vec{v}} = \frac{\vec{v}_1 + \vec{v}_2 + \dots + \vec{v}_n}{n} 
        \]
        \ii When there are \(n\) equal intervals of distance/displacement with velocity \(\vec{v}_1, \vec{v}_2, \ldots, \vec{v}_n\).
        Then, \[
            \bavg{\vec{v}} = \frac{n}{\frac{1}{\vec{v}_1} + \frac{1}{\vec{v}_2} + \dots + \frac{1}{\vec{v}_n}} 
        \]
    \end{casework}
\end{algorithm}


In general velocity is also referred to as `instantaneous velocity'. This is done to remind of the difference
with average velocity. We will not refer to it that way since by velocity we shall always mean \(\vec{v}\) as defined in
\cref{def: velocity}. The same goes for speed. Notably, speed is not defined as \(\dv{\abs{r}}{t}\).

While the average speed is very well that, speed is actually the magnitude of the velocity of that point. 
Consider the distance between \(a\) and \(b\) as 
\(a \to b\). The path between them approaches a straight line. Thus, the distance and displacement become the same
in the limiting case. Therefore, the \emph{magnitude} of velocity is the speed at that instant.

Therefore, 

\[\text{speed} = \abs{\dv{\vec{r}}{t}}\]


\begin{algorithm}
    Let us consider some techniques for solving equations of velocity. Consider the case when velocity and time are given
    and we have to figure out displacement or when displacement and and velocity are given, but time is not. 
    \begin{enumerate}
        \ii \(v = c\) : This is a trivial case. Just use equations of uniformly accelerated motion.
        \ii \(v = f(x)\) : \(\displaystyle \dv{x}{t} = f(x)\). Then, \(\dfrac{\dd{x}}{f(x)} = \dd{t}\).
        Now, we simply have, \[
            \int^{x_2}_{x_1} \frac{\dd{x}}{f(x)} = \int^{t_2}_{t_1} \dd{t}
        \] to get \(x_1\) and \(x_2\) or \(t_1\) and \(t_2\). We may also set our co-ordinates such that \(x_2 = x\), \(x_1 = 0\) and \(t_2 = t\), \(t_1 = 0\).
        \ii \(v = f(t)\) : This is also easy, just, \(\displaystyle f(x) = \dv{x}{t}\), then,
        \[
            \int^{t_2}_{t_1} f(t) \dd{t} = \int^{x_2}_{x_1} \dd{x}
        \] We simply get \(x_1\) and \(x_2\) or \(t_1\) and \(t_2\) by solving the integral.
    \end{enumerate}
\end{algorithm}

\section{Acceleration}

If a body does not move with uniform velocity, it accelerates. Acceleration is also a vector and
is the rate of change of velocity.

\index{Acceleration}

\begin{definition}
    [Acceleration]
    \label{def: acceleration}
    Acceleration is defined as,
    \begin{equation}
        \vec{a} \equiv \ndiv{\vec{v}} = \nddiv{\vec{r}}    
    \end{equation}
\end{definition}

Much like the velocity definition, we have,

\begin{equation}
    \vv = \vv_0 + \int_0^t \vec{a} \dd{t}
\end{equation}

There is also another way to represent acceleration, as, \[
    \vec{a} = \vec{v}\dv{\vec{v}}{x}
\]

Such a representation is particularly useful when the velocity of a particle 
is a function of its position. In such a case, derivating with time is quite a hassle.

\section{Uniformly Accelerated Motion}

When the acceleration of a particle, \(a\) is constant, it undergoes uniformly accelerated motion.
We use special cases of integrals of \(a\) and \(v\) with respect to time and position for deriving
the equations of importance.

\begin{theorem}
    For a particle with initial velocity \(u\) undergoing uniformly accelerated motion of acceleration \(a\),
    \begin{align}
        v &= u + at \\
        v^2 &= u^2 + 2as \\
        s &= ut + \frac{1}{2} at^2 \\
        s_{n} &= u + \frac{a}{2}(2n - 1)
    \end{align}
\end{theorem}

\para{The derivation of these is quite easy, integrate \(\int a \dd{x}\) for the first, 
\(\int a \dd{t}\) for the second (both with constant acceleration), substitute from the former into the latter for the third,
and finally for \(n^{th}\) second just do \(s_n - s_{n-1}\).} 

Some particularly elementary examples demonstrating some tricks are now given.

\begin{example}
    Consider a particle moving with acceleration \(\alpha\) initially. After some time, it starts
    to `de-accelerate' at \(\beta\) acceleration. After time \(t = T\), the particle comes to a stop.
    Find the displacement covered by particle in time \(T\).
\begin{soln}
        It might be tempting to start calculating after setting up time \(t = t_0\) when acceleration
        changed from \(\alpha\) to \(\beta\), i.e., the particle moves with acceleration \(\alpha\) till
        \(t = t_0\).\\
        However, let us try to do a cleaner solution. Consider the \(v-t\) graph of the particle,

        \begin{figure}
            [H]
            \centering
            \begin{tikzpicture}
                \begin{axis}[funcgraphbare, xmax=5.8, ymax=10, xlabel={$t$}, ylabel={$v$}]
                    \addplot[red,domain=0:3]{3*x} node[left,pos=1/2] {$\alpha$};
                    \addplot[blue,domain=3:4.8]{(-5)*x+24} node[right,pos=1/2] {$\beta$};
                    \draw [black,dashed] (axis cs:3,0)--(axis cs:3,9) node[left,pos=1/25] {$t_0$};
                    \draw [black] (axis cs:4.8,-1)--(axis cs:4.8,0.5) node[right,pos=1] {$T$};
                    \draw [black,dashed] (axis cs:0,9)--(axis cs:4.8,9) node[right,pos=1] {$v_{max}$};
                \end{axis}
            \end{tikzpicture}
        \end{figure}
        Note that \(\alpha(t_0) = \beta(T-t_0) = v_{max}\). Thus,
        \begin{align*}
            \alpha(t_0) &= \beta(T-t_0) \\
            \alpha(t_0) &= \beta(T) - \beta(t_0) \\
            \alpha(t_0) + \beta(t_0) &= \beta(T) \\
            t_0 &= \frac{\beta T}{\alpha + \beta} \\
            v_{max} &= \alpha t_0 \\
            v_{max} &= \frac{\alpha\beta T}{\alpha + \beta}
        \end{align*}
        Now, note that \(\dfrac{1}{2} \times v_{max} \times T\) is the area under the graph.
       
        The thing to note here is that since the physical quantities are 
        often related in a differential form, they can be converted 
        into an integral form and thus, we can often consider the 
        area under the curve for measures. 

        Thus, displacement, \(s = \dfrac{\alpha\beta T^2}{2(\alpha + \beta)}\).
\end{soln}
\end{example}

\begin{example}
    A car is at distance \(d\) from a boy. It starts accelerating at \(a\) \unit{\metre\per\second\squared}. What is the minimum velocity that the boy should
    have to catch up with the car.
    \begin{soln}
        Consider separation of boy and car, \(s_{c,b}\). Using the equations of motion, 
        we have \(s_{b} = vt\), \(s_{c} = \dfrac{1}{2}at^2\). Thus, 
        \begin{equation}
            \label{eq: vel}
            s_{c,b} = d + \frac{1}{2}at^2 - vt        
        \end{equation}
    
        The most important thing to note here is that, 
        \begin{moral}
            For two objects to meet, the solution for time must be real.
        \end{moral}
        Therefore, from the simple equation, \eqref{eq: vel}, we must have, for real solutions,
        \(b^2 - 4ac\) for the equation \(at^2 + bt + c\). Solving for this by substituting values from
        \eqref{eq: vel}, we have,
            \begin{equation}
                v \ge \sqrt{2ad}
            \end{equation}
    \end{soln}
\end{example}

\begin{remark}
    One thing of note is that the converse of above is true as well,
    \begin{moral}
        For two objects to \emph{not} meet, the solution of time must be imaginary or negative.    
    \end{moral}
\end{remark}

\para{One more key note to takeaway is that,}

\begin{moral}
    At instant of maximum separation, relative velocity of particles is \(0\).
\end{moral}

\begin{example}
    A body is dropped at \(t = 0\), after time \(t = t_0\), another body is thrown downwards with
    velocity \(u\) \unit{\meter\per\second}. Assuming first body reaches ground first, plot graph
    of separation.

    \begin{soln}
        At instant \(t_0\), displacement of first particle = \(\dfrac{1}{2}gt_{0}^2\). Note that here
        we set up co-ordinates such that positive \(y\) is downwards from point of drop. The displacement 
        of first body at \(t = t\) after \(t_0\) but before \framebox{reaching ground} is
        \begin{equation}
            s_1 = \frac{1}{2}gt_{0}^2 + \frac{1}{2}g(t-t_0)^2
        \end{equation}
        While for second body is,
        \begin{equation}
            s_2 = ut + \frac{1}{2}g(t-t_0)^2
        \end{equation}

        Thus, 
        \begin{equation}
            s_{1,2} = \frac{1}{2}g(t_0)^2 + t(gt_0 - u)
        \end{equation}

        Thus, before reaching ground, \(s-t\) graph is linear. However, after first body reaches ground,

        \begin{equation}
            s_{1,2} = ut + \frac{1}{2}g(t)^2
        \end{equation}
        which is a parabola. Thus, overall graph is,
        \begin{figure}[H]
            \centering
            \begin{tikzpicture}
                \begin{axis}[funcgraphbare, xlabel={$t$}, ylabel={$s$}]
                    \addplot[red, domain=0:2] {x+2};
                    \addplot[red,domain=2:3] {1.5*(-x^2)+5*x};
                \end{axis}
            \end{tikzpicture}
        \end{figure}
    \end{soln}
\end{example}

\begin{example}
    Find the total displacement if the \(v-t\) graph is as follows,
    \begin{figure}
        [H]
        \centering
        \begin{tikzpicture}
            \begin{scope}[xshift=10cm]
                % Axes
                \draw (0,0) node[below left] {$O$}
                    (0,0) -- (7,0) node[below] {$t$}
                    (0,-0.5) -- (0,3.5) node[left] {$v$};
                
                \draw (6.4,0) arc[start angle=0, end angle=180, radius=3.2];
            \end{scope}
        \end{tikzpicture}
    \end{figure}
    The idea behind this is simple, the graph here is \textbf{not} a semicircle. It is an
    ellipse, because of dimensional limitations! The equations of circle, \(v^2 + t^2 = r^2\)
    doesn't work because we cannot add two dimensionally different quantities.

    Therefore, the graph is an ellipse, and we conclude by computing the area of the ellipse. I haven't
    given the data, since I wanted to mention the idea.
\end{example}

\section{Motion in Polar Co-ordinates}

The polar co-ordinate axes are the two-dimesional subset of the 3-d cylindrical co-ordinates. 
Any point in polar co-ordinates is depicted by a system of two unit vectors, \(\unitv{r}\) and
\(\unitv{\theta}\). However, each of these are \framebox{dependent on the position of the particle} and
maybe written explicitly as \(\unitv{r}(\theta)\) and \(\unitv{\theta}(\theta)\).  

Here, \(\unitv{r}\) points in the direction of increasing radius(along the radial vector)
\(\unitv{\theta}\) points in the direction of increasing \(\theta\) (tangent to the radial vector).
These two unit vectors are \emph{orthogonal} at any point. 

\begin{figure}
    [H]
    \centering
    \incfig{polarunit}
    \caption{Unit vectors in polar co-ordinates}
    \label{fig: polarunit}
\end{figure}

\noindent In \cref{fig: polarunit} if \(\vec{r}\) makes
an angle \(\theta\) with the horizontal, then,

\begin{align}
    \label{eq: polardef}
    \unitv{r} &= \cos \theta \unitv{i} + \sin \theta \unitv{j}\\
    \unitv{\theta} &= -\sin \theta \unitv{i} + \cos \theta \unitv{j}
\end{align}

\section{Kinematics in Polar Co-ordinates}

Now let us formulate kinematics in polar co-ordinates. The position \(\vec{r}\) can be written as

\begin{equation}
    \vec{r} = r \unitv{r}
\end{equation}

\noindent Velocity follows normally, 
\begin{equation}
    \vec{v} = \ndiv{\vec{r}} = \ndiv{r} \unitv{r} + r\dv{\unitv{r}}{t} 
\end{equation}
or does it? What even is \(\displaystyle\dv{\unitv{r}}{t}\) ? Well the answer lies in 
the time-derivative of the unit vectors. Using the definitions in \eqref{eq: polardef}, we get,

\begin{align}
    \dv{\unitv{r}}{t} &= \ndiv{\theta} \unitv{\theta}\\
    \dv{\unitv{\theta}}{t} &= -\ndiv{\theta} \unitv{r}
\end{align}

\noindent Using these results, we get that

\begin{empheq}[box=\widefbox]{align}    
    \vec{v} & = \ndiv{r} \unitv{r} + r\ndiv{\theta} \unitv{\theta}\\
    \label{polar: accel}
    \vec{a} & = (\nddiv{r} - r\ndiv{\theta}^2) \unitv{r} + (r\nddiv{\theta} + 2\ndiv{r}\ndiv{\theta}) \unitv{\theta}
\end{empheq}

\noindent The most interesting result are the various terms of acceleration in \eqref{polar: accel}. First let 
us consider the terms along \(\unitv{r}\). \(\nddiv{r}\) is the \emph{radial} acceleration. It is
the change is radial speed. The second term \(-r\ndiv{\theta}^2\) is the \emph{centripetal} acceleration.
It arises from a change in tangential velocity.

Now let us consider the terms along \(\unitv{\theta}\). The \(r\nddiv{\theta}\) term is the tangential
acceleration resulting from change in tangential speed. The other term \(2\ndiv{r}\ndiv{\theta}\) is the
\emph{coriolis} acceleration. I shall discuss it when discussing rotating frames.

\section{Free Fall}

When an object is dropped downwards freely, it undergoes \emph{free fall}. We ignore
air resistance, until we're told not to and in general also talk about cases of the
object being thrown instead of dropped. An object dropped from any height undergoes
accelerations \(\vec{a}\) where \(\norm{\vec{a}} = g\). The direction and sign of acceleration
is entirely dependent on us.

In general, it is a good idea to setup our axes such that the direction above ground is positive
and towards the ground is negative. The acceleration during free fall is always towards
the ground. Therefore, for these cases \(\vec{a} = -g \unitv{j}\). 

Some important points of note are that ``\(g\)'' does not become positive or negative or whatever,
this is a common misconception. Instead, we just decide where exactly the positive direction is
and setup \(g\) according to that.
\parbreak
When a ball is thrown with some velocity \(u\) say in the positive directions
(change it into \(-u\) if in the negative direction), then we have,

\begin{algorithm}
    The following are special cases of equations of motion.
    \begin{enumerate}
        \ii \(v = u - gt\)
        \ii \(v^2 = u^2 -2gh\)
        \ii \(h = ut - \dfrac{1}{2}gt^2\)
    \end{enumerate} 
\end{algorithm}

The case when a ball is thrown upwards is exactly symmetrical to this one, but there
the ball goes up, reachs \(v = 0\) and then goes down. Also note that all of these equations are vector
equations.

\begin{example}
    Two balls are thrown in intervals of \(2\) seconds with velocity \(u\) each. Find the
    time when they collide.

    \begin{soln}
        Note that when they collide, their displacement from ground must be the same! Setting 
        the origin at ground, and direction of away from ground as positive, we have,
        
        For the first ball, \(h_{1} = ut - \dfrac{1}{2}gt^2\). Since the second ball
        is thrown \(2\) seconds later, \(h_{2} = u(t-2) - \dfrac{1}{2}g(t-2)^2\). Now, we just have
        \(h_{1} = h_{2}\) and the calculation is trivial.
    \end{soln}
\end{example}

The thing to note in this example is that you do not need to setup two cases, one in which
\(g\) is negative and one in which \(g\) is positive since it is in the direction of velocity. 
Just note that these equations are vectorial and we are done.

Some quick formulas to remember because they help a lot,

\begin{algorithm}
    For an object dropped from height \(h\),
    \begin{enumerate}
        \ii Time to reach ground, \(t = \sqrt{\dfrac{2h}{g}}\).
        \ii Velocity when it reaches ground \(v = \sqrt{2gh}\).
    \end{enumerate}
\end{algorithm}

Deriving them is easy, and they can be generalised as when it travels height \(h\) between
the point of drop to anywhere, instead of ground. Clearly, both still follow there.

Also, always remember,

\begin{moral}
    When an object is thrown/dropped from another object traveling with some velocity and
    acceleration, it inherits its velocity but not its acceleration.
\end{moral}

\section{Graphs in Motion}

In the graphs we are going to discuss, acceleration will either be \(0\) or constant.
These graphs help to quickly solve problems, especially related to total distance and 
displacement. 

Let us consider the direction away from origin positive, and towards origin negative.
Then, consider constant acceleration, \(a = c\). We have the following displacement time
graphs,

\begin{figure}[H]
    \centering
    \begin{subfigure}{0.5\textwidth}
        \centering
        \begin{tikzpicture}
            \begin{axis}
            [funcgraphbare, xlabel=$t$, ylabel=$s$]
            \addplot[red, domain=0:16]{x^2};
            \end{axis}
        \end{tikzpicture}
        \caption{For \(c > 0\). }
    \end{subfigure}%
    \begin{subfigure}{0.5\textwidth}
        \centering
        \begin{tikzpicture}
            \begin{axis}
            [funcgraphbare, xlabel=$t$, ylabel=$s$]
            \addplot[red, domain=0:16]{-x^2 + 256};
            \end{axis}
        \end{tikzpicture}
        \caption{For \(c < 0\)}
    \end{subfigure}%
\end{figure}

I use these in particular to demonstrate concavity and convexity. 
For a twice differentiable function \(f\), it is concave if \(f''(x) > 0\) and convex 
otherwise. 

\subsubsection{Displacement Time}

Displacement time graphs can be generally interpreted also as position 
time graphs since the two differ only by a constant initial value.

Their area under the curve, \(\int x \dd{t}\) is not of much use but their 
slope, \(\ndiv{x}\) represents the velocity. We can also analyze the graphs 
to figure out whether their roc circle lies above or below the graph to figure out 
the sign of acceleration at any given moment.

\subsubsection{Velocity Time}

Their slope is simply \(\ndiv{v}\) or the acceleration. Their area is 
\(\int v \dd{t}\) which is the displacement. We need to remember that 
we need to setup definite integrals since these actually give the value 
we need and not a family of solutions. 

\subsubsection{Acceleration Time}

Acceleration time graphs are useful for their area which represents the difference 
in velocities in an interval,

\begin{equation*}
    \int_{0}^{t} a \dd{t} = v(t) - v_0
\end{equation*}

\subsubsection{Velocity Displacement}

These are the more esoteric and interesting graphs. The slope is simply,

\begin{equation*}
    \dv{v}{x} = \frac{a}{v}
\end{equation*}

where we use the fact that \(a = v \dv*{v}{x}\). Thus given a velocity time 
graph we can figure out the acceleration at any given \(x\) as, 

\begin{equation*}
    a = v_{x_0} \eval{\dv{v}{x}}_{x_0} = v_{x_0} \tan\theta
\end{equation*}

\begin{marginfigure}
    \centering
    \begin{tikzpicture}
        \begin{axis}
        [funcgraphbare, xlabel=$t$, ylabel=$\vec{v}$]
        \addplot[red, domain=0:4]{-3*x + 12};
        \end{axis}
    \end{tikzpicture}
    \caption{A linear velocity time graph.}
    \label{fig: vt to at example}
\end{marginfigure}

\begin{example}
    Given the velocity time graph in \cref{fig: vt to at example}, sketch out the acceleration time graph.

    \begin{soln}
        We see that \(v = mx + c\) where \(m < 0\). Now, 

        \begin{equation*}
            a = v \dv{v}{x} = m(mx + c) = m^2x + mc
        \end{equation*}

        Thus, it is of form \(ax + b\) where \(a > 0\), \(b < 0\).

        There-fore its graph is,
        
    \begin{figure}[H]
    \centering
        \begin{tikzpicture}
            \begin{axis}
            [funcgraphbare, xlabel=$t$, ylabel=$\vec{a}$]
            \addplot[red, domain=0:4]{9*x - 36};
            \end{axis}
        \end{tikzpicture}
        \caption{The corresponding acceleration time graph.}
    \end{figure}%
    \end{soln}
\end{example}

\subsubsection{Velocity squared time}

This is the last family of graphs we'll look at.
The integral is largely useless but we note that 

\begin{equation*}
    \dv{v^2}{x} = 2v \dv{v}{x} = 2a
\end{equation*}

Thus, its slope, \(\tan\theta = 2a\) and we can easily figure out its acceleration 
if we figure out its slope. yay.

\section{Relative Motion}

There is a very neat technique to solve questions of kinematics, it is to switch our 
reference frames. Let the base reference frame be that of earth, then,

\begin{figure}
    [H]
    \centering
    \incfig{reference}
\end{figure}

If \(\vec{r}_1\) evolves with time, \(r_1(t)\). And so does \(r_2(t)\). Then,

\begin{align*}
    \vec{r}_1 &= \vec{r}_{1,2} + \vec{r}_2 \\
    \ndiv{\vec{r}}_1 &= \ndiv{\vec{r}}_{1,2} + \ndiv{\vec{r}}_2 \\
    \vec{v}_1 &= \vec{v}_{1,2} + \vec{v}_{2}\\
    \vec{v}_{1,2} &= \vec{v}_1 - \vec{v}_{2}
\end{align*}

Obviously if the vectors face the same direction in the same plane, then, the components subtract. If they face
opposite direction, then the components add. 

\(\vec{v}_{1,2}\) is called the velocity of \(\mathit{1}\) w.r.t \(\mathit{2}\). When doing
problems, try to setup a reference frame such that velocity of any single object vanishes.
This allows for a very neat simplification of problems. 

\section{Motion in higher dimensions}

We can mostly safely extend our results from one dimension to higher dimensions.
So for instances, velocity of a particle whose trajectory is defined as 
\(x\unitv{i} + y\unitv{j} + z\unitv{k}\) can simply be computed as,
\[v = \ndiv{x}\unitv{i} + \ndiv{y}\unitv{j} + \ndiv{z}\unitv{k}\] Such a thing
is not true for non-cartesian co-ordinates, as we say in the case of polar forms.
Due to the orthogonality of the cartesian unit vectors, and using dot products
we simply have, 
\[v_x = \ndiv{x}, v_y = \ndiv{y}, v_z = \ndiv{z}\]

And the same is true for all other vectors. This gives us the power to
resolve our vector components into components alongside unit vectors
and consider independent one dimensional motion for each of them.

\section{Projectile Motion}

Projectile motion is one of the most important motion to consider in 
2 dimensions. It is the motion of a projected body. Generally, we
only care about the motion of a body projected from ground under the influence
of gravity. 

Let a body be projected at an angle \(\theta\) from the ground at velocity \(\vec{v}\)
along \(\theta\). Then, by resolving \(\vv\) into \(v_x \unitv{i} + v_y\unitv{j}\),
we have, 
\begin{align*}
    x &= v_xt \unitv{i} \\
    y &= v_yt - \frac{1}{2}gt^2 \unitv{j}
\end{align*}

Doing some manipulations, we may note that \(v_x = v\sin(\theta)\), \(v_y = v\cos(\theta)\).
Also, substituting \(t = x/v\sin(\theta)\), we have,
\begin{equation}
    \boxed{y = x\tan(\theta) - \frac{gx^2}{2v^2\cos^2(\theta)}}
\end{equation}

We can also use the identity \(1/\cos^2\theta = \tan^2\theta + 1\) and get another and 
perhaps nicer form of the equations,

\begin{equation}
    \boxed{y = x\tan\theta\left(1 - \frac{x}{R}\right)}
\end{equation}

after substituting with \(R\)\footnote{calculated just a bit later.}.

The path of the projectile is parabolic as shown in \cref{fig: projectile}.

\begin{figure}[H]
    \centering
    \incfig{projectile}
    \caption{Trajectory of a projectile}
    \label{fig: projectile}
\end{figure}

Here, \(H\) is the maximum height of the particle. Clearly,
when the projectile is at \(H\), \(v_y = 0\). We have, 
\(v_y^2 = 2gH\),
\begin{equation}
    \boxed{H = \frac{u^2\sin^2(\theta)}{2g}}
\end{equation}

The time period of the projectile can be simply calculated as well,
since at the final time, \(T\), \(y = v_yt - 1/2gt^2 = 0\) and,

\begin{equation}
    \boxed{T = \frac{2v\sin(\theta)}{g}}
\end{equation}

\(R\) is the range the particle covers horizontally in time \(T\) and is
clearly,

\begin{equation}
    \boxed{R = \frac{v^2\sin(2\theta)}{g}}
\end{equation}

Note that if the particle is projected at \(\theta = \pi/4\), 
the range is maximum and is \(u^2/2g\). Otherwise,
for \(0 < \theta < \pi/2\), \(\sin(2\theta)\) attains the same 
value for two \(\theta_1, \theta_2\). These \(\theta_1\), \(\theta_2\) are complementary
by trigonometry.

More generally, if we consider the projectile to have an initial position of 
\(\qty(x_0, y_0)\), then, by substituting for time, we find that the 
vertex of the parabola is \(\qty(y_0 + \frac{u^2\sin^2\theta}{2g}, x_0 + \frac{u^2\sin2\theta}{g})\).

The range is a little more difficult to computer here, as the particle will transverse to 
the ground which is lower than \(y_0\). Freely setting up our co-ordinates 
as \(\qty(h, 0)\), we get by the equation of trajectory,

\[
    0 = h + R\tan(\theta) - \frac{gx^2}{2v^2\cos^2(\theta)}
\]

Derivating this with respect to theta to find the maxima, and then using the attained
value, we have, 

\begin{equation}
    R = \frac{u}{g}\sqrt{u^2 + 2gh}
\end{equation}

\begin{marginfigure}
    \centering
    \scalebox{2.5}{\incfig{projectilewedge}}
    \caption{Projectile grazing a wedge.}
\end{marginfigure}

\begin{example}
    A projectile starting from one vertex of a wedge, grazes the other 
    vertex and ends upon the other vertex, with an initial velocity 
    \(v_0\) making an angle \(\theta\) with the horizontal. If the
    angles of the vertices are \(\alpha\) and \(\beta\), figure 
    out a relation between \(\tan \theta\), \(\tan\alpha\) and \(\tan\beta\).

    \begin{soln}
        Divide the range of the projectile into two lengths, \(x\) and 
        \(x'\) such that \(x + x' = R\). Now, clearly, 
        \begin{equation*}
            \tan\alpha = \frac{y_0}{x} \quad \tan\beta = \frac{y_0}{x'}
        \end{equation*}

        Also, by the alternate equation of trajectory,

        \begin{equation*}
            y_0 = x\tan\theta\left(1 - \frac{x}{R}\right) = x\tan\theta\left(1 - \frac{x}{x + x'}\right) 
        \end{equation*}

        Which gives us, 

        \begin{equation*}
            \tan\theta = y_0 \left(\frac{x + x'}{xx'}\right) = \frac{y_0}{x} + \frac{y_0}{x'}
        \end{equation*}

        Finally substituting the values, we get, 

        \begin{equation}
            \tan\theta = \tan\alpha + \tan\beta
        \end{equation}
        
        Pretty nice and useful result.

    \end{soln}

\end{example}

Some peculiar things about projectile motion is that, \begin{inparaenum}[a)]
    \ii it is reversible, and
    \ii the time to attain some vertical displacement can be found in terms of 
    the maximum height alone.
\end{inparaenum}

Consider the vertical position of the particle to be 
\(s\) at some time \(t\). Then, 
\[s = u_yt - \frac{1}{2}gt^2\] 
Also, note that if the maximum height is \(h\), 
\[u_y = \pm \sqrt{2gh}\]

Where the positive value is during the ascent and the negative is during the descent. Then,

\[s = \pm\sqrt{2gh}t - \frac{1}{2}gt^2\]

Solving for \(t\) gives us,
\[t = \pm \sqrt{\frac{2h}{g}} \pm \sqrt{\frac{2(h-s)}{g}}\]

If the initial velocity is positive and \(s > 0\), 

\begin{equation}
    t = \sqrt{\frac{2h}{g}} \pm \sqrt{\frac{2(h-s)}{g}}
\end{equation}
the two values correspond to the fact that the particle will have the same vertical
displacement at two instances of time.

If the initial velocity is positive but \(s < 0\), i.e the particle 
starts at some height but then transverses below it, 

\begin{equation}
    t = \sqrt{\frac{2h}{g}} + \sqrt{\frac{2(h-s)}{g}}
\end{equation}
is the only reasonable solution.

If the initial velocity is negative, then clearly the 
particle will attain a lower position. The maximum height in this 
case is simply the initial height of the particle.

\begin{equation}
    t = - \sqrt{\frac{2h}{g}} + \sqrt{\frac{2(h-s)}{g}}
\end{equation}

Somewhat qualitatively, projectile motion is independent of its horizontal range. 
If some projectiles have same maximum height, their time of flight \emph{will be the 
same}.

\subsection{Projectile Motion and Vectors}

If we write the equations of projectile motion as a vector equation, it 
corresponds to, 

\begin{equation*}
    \vec{v}(t) = \vec{v}_0 + \vec{g}t
\end{equation*}

and for the displacement,

\begin{equation*}
    \vec{s}(t) = \vec{v}_0t + \frac{1}{2}\vec{g}t^2
\end{equation*}

We see that the velocity at any further moment is just the resultant of a scaled 
gravitational acceleration vector and the initial velocity vector and the displacement is the resultant of 
the scaled velocity and gravitational acceleration vectors. 

Since both of them are multiplied by a positive scalar the time \(t\) or that squared, we simply
scale both of the vectors to find the resultant. These allow for some very slick derivations regarding
projectile motion and are a nicer way to solve some of the problems regarding 
the evolution of say the velocity vector.

\begin{marginfigure}
    \centering
    \scalebox{2}{\incfig{projectilevector1}}
    \caption{Scaled up \(\vec{v}_0\) and \(\vec{g}\) vectors in projectile motion.}
    \label{fig: rangewithvectors}
\end{marginfigure}

In \cref{fig: rangewithvectors}, we can clearly note that since \(\vec{v}_0\) is tangent 
to the curve at the initial point, so is \(\vec{v}_0t\). Now, we can calculate 
the time taken to reach the following point since the displacement at that instant, 
\(R\), is going to be the resultant of the two vectors as shown in \(\cref{fig: rangewithvectors}\).

Thus,

\begin{equation*}
    \sin\theta = \frac{gt^2}{2v_0t} \implies t = \frac{2v_0\sin\theta}{g}
\end{equation*}

which is the desired flight time. We can also use,

\begin{equation*}
    \cos\theta = \frac{2R}{v_0t} \implies 2R = v_0t\cos\theta
\end{equation*}

by plugging in the expression for time we just got, we also get the range of the projectile. 

This is an example of a really slick derivation. We can use such an idea to often solve from 
problems where the final velocity or displacement vector has a known direction. 

I would not recommend using it completely by itself, it is more slick than its often useful
and while useful can also sometimes hold you back.

\subsection{Projectile Motion with rotated axes}

\begin{marginfigure}
    \centering
    \scalebox{2.2}{\incfig{projectilevector2}}
    \caption{Projectile along a wedge}
    \label{fig: projectilealongwedge}
\end{marginfigure}

In this particular section we'll particularly look at projectile motion along a wedge. Consider 
\cref{fig: projectilealongwedge}. If we were to figure out \(d\), we could 
use a number of ways. The locus of the projectile is,

\begin{equation*}
    y = x\tan(\theta+\alpha) - \frac{gx^2}{2v_0^2\cos^2(\theta+\alpha)}
\end{equation*}

and of the wedge is,

\begin{equation*}
    y = x\tan\theta 
\end{equation*}

We could simply equate them to figure out the distance \(d\), which is the range of the projectile.

\begin{marginfigure}
    \centering
    \scalebox{2.2}{\incfig{projectilevector3}}
    \caption{The cartesian axes rotated by \(\theta\).}
    \label{fig: projectilerotatedaxes}
\end{marginfigure}

However, more generally, we have some other information than simply this which means that we 
can use a different way to look at things. Instead of the standard axes, consider them rotated 
by the angle of the wedge, \(\theta\). Consider \cref{fig: projectilerotatedaxes}.
We can resolve \(\vec{v}_0\) vector into two components, along and perpendicular to the wedge,
and thus along the \(x\) and \(y\) axis respectively. 

The acceleration along the \(y\) axis is simply \(g\cos\theta\) and along the \(x\) axis 
is \(g\sin\theta\). Thus, our equations in each direction are,

\begin{align*}
    v_x(t) = v_0\cos\alpha - g\sin\theta t\\
    v_y(t) = v_0\sin\alpha - g\cos\theta t 
\end{align*}

and we may solve each of these to get any desired result. Due to the dependence 
on both \(\theta\) and \(\alpha\), it is a bit difficult to things like maximum range
and the loci method maybe preferred.  

\begin{marginfigure}
    \centering
    \scalebox{2}{\incfig{projectilealongwedgeex}}
    \caption{Projectile thrown from a wedge.}
    \label{fig: projectile wedge example}
\end{marginfigure}

\begin{example}
Consider a projectile thrown with a velocity \(v_0\) making an angle \(\alpha\) with 
a wedge of inclination \(\theta\) as shown in the \cref{fig: projectile wedge example}. Find out the angle \(\alpha\) 
such that the projectile has maximum range.    

\begin{soln}
    The locus of the projectile is, 
    \begin{equation*}
        y = x\tan\alpha - \frac{gx^2}{2v_0^2\cos^2\alpha}
    \end{equation*}

    and of the wedge is, 

    \begin{equation*}
        y = -x\tan\theta
    \end{equation*}

    At the range, they will be equal,

    \begin{equation*}
        x\tan\alpha - \frac{gx^2}{2v_0^2\cos^2\alpha} = -x\tan\theta \implies 
        \tan\alpha + \tan\theta = \frac{gx}{2v_0^2\cos^2\alpha}
    \end{equation*}

    which gives us the desired range, 

    \begin{equation*}
        x = \frac{2v_0^2(\tan\alpha + \tan\theta)\cos^2\alpha}{g}
    \end{equation*}

    So the condition for maximum range is simply, \(\dv*{x}{\alpha} = 0\). This is also 
    kind of illustrates a nice idea, since we need to get the derivative \(0\), we
    can safely discard any multiplicative or additive constants right away which will 
    either have a zero differentiation, or can be divided by since the other side of the 
    equation is simply \(0\). 
    
    We get that, 

    \begin{align*}
        &\dv{(\tan\alpha + \tan\theta)\cos^2\alpha}{\alpha} = 0 \\
        &\dv{\sin 2\alpha/2 + \tan\theta\cos^2\alpha}{\alpha} = 0 \\
        &\cos 2\alpha - \tan\theta\sin 2\alpha = 0 \\
        &\tan 2\alpha = \cot\theta \\
        &\tan 2\alpha = \tan (\pi/2 - \theta) \\
        &\alpha = \frac{\pi}{4} - \frac{\theta}{2}
    \end{align*}

    And we're done.
\end{soln}

\end{example}

\section{Drag Forces}

In a more realistic scenario, a particle undergoing motion in any 
media experiences a \vocab{drag force}, anti-parallel to its velocity.
In particular,  

\begin{equation*}
    \vec{a} = -\alpha v \unitv{v} - \beta v^2 \unitv{v}
\end{equation*}

where the linear term arises from a property of the media, 
the viscosity(we will talk about it in the later sections.)

The quadratic term is a result of the collision of the particles with atoms 
and molecules during its motion. In general, 
the quadratic term dominates for higher velocities.

It is very difficult to analyse motion with drag, but we shall have a look at some particular 
cases of linear drag.

\marginref{\Cref{sec: newton's second law} \nameref{sec: newton's second law}}

\begin{example}
    Consider a projectile of mass \(m\) thrown upwards with initial velocity \(v_0\). It undergoes a linear drag 
    force, \(\Vec{F} =  -kv \unitv{v}\). Find the velocity of the particle as 
    a function of time. 

    \begin{soln}
        Using newton's second law, also by noting that 
        it also undergoes gravitation, \(\vec{a} = -g -kv/m \unitv{v}\).

        Now, we note that, 
        \begin{equation*}
            \dv{v}{t} = -g  - \frac{kv}{m}
        \end{equation*}

        \begin{align*}
            \frac{\dd{v}}{g + kv/m} &= - \dd{t} \\
            \int_{v_0}^{v_f} \frac{\dd{v}}{g + kv/m} &= \int_{0}^{t}\dd{t} \\
            \frac{m}{k} \eval{\log(g + kv/m)}_{v_0}^{v_f} &= t \\
        \end{align*}
        Which finally gives us,

        \begin{equation}
            v(t) = e^{-kt/m}v_0 + \frac{gm}{k}(e^{-kt/m} - 1)
        \end{equation}

    \end{soln}
\end{example}

The same idea can be extended to projectile --- \(2\)d motion 
by considering the components of the accelerations,

\[
    \vec{a} = -kv_x \unitv{i} - (g + kv_y) \unitv{j},
\]

and now considering the equations \(\dot{\vec{v}} = -kv_x\)
and the same for the \(y\) co-ordinate. 

Ulitmately, this leads to the equations,

\begin{align}
    v_x &= u_x e^{-kt} \\
    v_y &= \qty(u_y + \frac{g}{k})e^{-kt} - \frac{g}{k}
\end{align}

\subsection{A bag of techniques}

Although almost everything can be found with a bash-y mathematical solution, sometimes 
it is just nicer to use a particularly slick idea to solve problems. Some of these are discussed 
below.

\subsection{Fermat's Principle}

Given that in a medium, the body can move relative to it with some fixed 
velocity \(v_1\). Ultimately, it has to cross to a point in another medium, in which 
it can move with velocity \(v_2\) with respect to the first one. If it has to reach 
some fixed point in the other medium, what could be its \emph{optimal} trajectory?

\begin{marginfigure}
    \centering
    \scalebox{2.3}{\incfig{fermatprinciple}}
    \caption{The optimal trajectory in the two media.}
    \label{fig: fermat's principle}
\end{marginfigure}

Let the optimal trajectory be as in \cref{fig: fermat's principle}. The time in the first media
is \[t_1 = \frac{\sqrt{x^2 + y^2}}{v_1},\] and in the other media is \[t_2 = \frac{\sqrt{(L-y)^2 + (d-x)^2}}{v_2}\].

Note that \(y\) is fixed (since it is the vertical distance between the media.), Thus, 
the way to get minimum time is, to get for \(t = t_1 + t_2\) to be such that,

\begin{equation*}
    \dv{t}{x} = 0
\end{equation*}

Now, differentiating the whole equation, we will get the final result,

\begin{equation}
    \frac{-2x}{\sqrt{(x^2+y^2)}v_1} + \frac{-2(d-x)}{\sqrt{(L-y)^2 + (d-x)^2}v_2} = 0 
\end{equation}

Note however, that, 

\begin{equation*}
    \sin\theta_1 = \frac{x}{\sqrt{(x^2+y^2)}} \quad \sin\theta_2 = \frac{(d-x)}{\sqrt{(L-y)^2 + (d-x)^2}}
\end{equation*}

Replacing both of these and factoring out \(-2\), we get that, 

\begin{equation}
    \boxed{\frac{\sin\theta_1}{v_1} = \frac{\sin\theta_2}{v_2}}
\end{equation}

This is a very neat result and allows us to greatly simplify everything. Note 
here that both the velocities, \(v_1\) and \(v_2\) are relative to any common inertial
frame. We can use some frame moving at some velocity to observe this and it wouldn't change 
a thing. 


\begin{marginfigure}
    \centering
    \scalebox{2.2}{\incfig{fermatprincipleex}}
    \caption{Shortest time to reach opposite point in a stream.}
    \label{fig: fermat example}
\end{marginfigure}

\begin{example}
    If a man can swim relative to water with a velocity \(v_0\), and can walk on the ground 
    with velocity \(3v_0\) and the velocity of the stream is \(2v_0\), find out minimum time 
    in which the man can reach a point directly opposite to him. The width of the river is \(L\).
    Look at \cref{fig: fermat example} for clarity.
    
    \begin{soln}
        The man could walk in any direction on the ground, but since we want to minimize 
        the time, we consider him walking in the same direction as the river current to get 
        the velocity of man on ground relative to water being \(5v_0\). The velocity 
        of him in the river relative to the stream is \(v_0\). 

        From fermat's principle, the most optimal trajectory obeys 
        
        \begin{equation*}
            \frac{\sin\theta_1}{v_1} = \frac{\sin\theta_2}{v_2}
        \end{equation*}
        where the angles are taken relative to the normal. 
        
        Since \(\theta_2 = \dfrac{\pi}{2}\), \(v_1 = v_0\), and \(v_2 = 5v_0\),

        \begin{equation*}
            sin\theta_1 = \frac{1}{5}
        \end{equation*}

        After which we can easily find the time as by a simple angle chase, \(\theta_1\) is 
        the angle between the velocity of man and the vertical. From it we 
        can find the resultant velocity with water (using components) 
        and subsequently the drift and time required.
    \end{soln}
\end{example}