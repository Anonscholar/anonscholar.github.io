\chapter{Kinematics}

Kinematics is the study of motion without concerning its cause. It allows
us to calculate and find out the evolution of a body. Everything in our world undergoes
motion, one way or the other. To begin our study of motion, let us first
concern ourselves with some definitions.

\section{Reference Frame and Point Particle}

Motion of any object is considered to relative to a reference frame. We consider motion of
a body with respect to any other body or co-ordinate frame.

\begin{figure}[ht]
        \centering
        \incfig{3daxes}    
    \caption{A Reference Frame}\label{fig: 3daxes}
\end{figure}

We use a reference frame according to our convenience. In general, we use the 3-dimensional cartesian system like the one
in \cref{fig: 3daxes}.

\index{Reference Frame}

\begin{definition}
    [Reference Frame]
    A reference frame is a co-ordinate frame relative to which motion of a particle is considered.
\end{definition}

\para{We always consider the motion of a point particle. Well, why? It is done so to simplify
the calculation of our system. The motion of a (rigid)body is discussed later.}

\index{Particle}
\begin{definition}
    [Point Particle]
    A particle whose size is negligible in the study of its motion is called a point particle.
\end{definition}

\section{Position Vector and Displacement}

\subsection{Position Vector}
\index{Position Vector}

\marginnote{One thing of importance is that a position vector of a particle is `generally' a function
of time. We generally denote this as \(r(t)\) to show that is a function.}

\para{We define the position of a particle relative to a reference frame using a \emph{position 
vector}. One end of the position vector is the origin of our reference frame. The other is 
the particle itself. Consider \cref{fig: position}, the 
vector \(\vec{r}\) from \(O\) to \(A\). It describes the position of \(A\) relative to the
co-ordinate frame. Let \(A = (x,y,z)\). Then we denoted \(\vec{r}\) as \(\vec{r}(x,y,z)\).}

\para{The position to any point is written as, \[
    \vec{r} = (x,y,z) = x\unitv{i} + y\unitv{j} + z\unitv{k}
\]}



\begin{figure}
    [H]
    \centering
    \scalebox{0.6}{\incfig{position}}
    \caption{A position vector, \(\mathbf{r}\).}
    \label{fig: position}
\end{figure}



\subsection{Displacement}
\index{Displacement}

\marginnote{Note that the position vector is not a \emph{true} vector. It is tied to particular
    reference frame.}

\para{Consider the movement of a particle from \(A = (x_1,y_1,z_1)\) to \(B = (x_2,y_2,z_2)\). 
The \emph{displacement} defines a true vector \(\vec{S} = (x_2-x_1,y_2-y_1,z_2-z_1)\).
\(\vec{S}\) is called the displacement vector. Note that it contains no information about the
individual points, but only about the relative position of each. Thus, our choice
of reference frame, and thus position vector does not matter when we're concerned with displacement
. In \cref{fig: displacement}, the vector
\(\vec{S}_{AB}\) defines a displacement from \(A\) to \(B\).}

\begin{figure}
    [H]
    \centering
    \scalebox{0.8}{\incfig{displacement}}
    \caption{Displacement from \(A\) to \(B\)}
    \label{fig: displacement}
\end{figure}

\para{The difference between displacement and distance is that of the path they describe. The distance covered is the length
of the actual path, while the magnitude of the displacement is simply the length of the vector between the final and initial positions.}

\begin{figure}
    [H]
    \centering
    \incfig{distance}
    \caption{Distance vs Displacement}
\end{figure}

\begin{algorithm}
    Consider the case of displacement and distance of a particle in circular motion with radius \(r\) Let it move from \(A\) to \(B\), inscribing
    an angle \(\theta\) (in radians) Then, we have,\\
    ~\\
    Distance between \(A\), \(B\) = \(r\theta\)\\
    Displacement between \(A\), \(B\) = \(2r\sin\left(\dfrac{\theta}{2}\right)\)
\end{algorithm}

\begin{figure}
    [H]
    \centering
    \incfig{distdisp}
    \caption{Displacement in a circle}
\end{figure}

\section{Speed and Velocity}

\index{Speed}

\para{We are already familiar with the notion of speed. In elementary terms, speed is \({distance}/{time}\).
Speed itself is an elementary concept. It tells us nothing about the direction of the object whose speed we are
talking into consideration. Thus, it is a \emph{scalar} quantity.}

\parbreak
\marginnote{The symbol \(\equiv\) stands for `defined as'.}

\index{Velocity}

\para{The Velocity of a particle, contrastingly, contains information about both the speed and direction of the
object in consideration. It is a vector. Similar to position, it is also generally a function of time \(v(t)\), though it
may also be dependent on the position \(v(x)\).} 
~\\
\begin{definition}
    [Velocity]
    \label{def: velocity}
    Velocity is defined as,
    \begin{equation}
        \vec{v} \equiv \ndiv{\vec{r}}    
    \end{equation}
\end{definition}


\para{Sometimes, we do not need to compute the derivative of the position vector, and simply need an average
estimate, or in the case of uniformly accelerated motion, it is not required that we computer derivatives.
In such a case we use the concept of \textbf{average velocity}, \[
    \avg{\vec{v}}_{12} = \frac{\vec{r}_1-\vec{r}_2}{t_2 - t_1}  
\]}

\marginnote{The dot over \(r\) in \(\dot{\vec{r}}\) tells us that \(r\) has been differentiated
with respect to time.}

\begin{algorithm}
    Let us consider the cases of average velocity,
    \begin{casework}
        \ii When there are \(n\) equal time intervals with velocity \(\vec{v}_1\), \(\vec{v}_2\), \ldots \(\vec{v}_n\).
        Then, \[
            \avg{\vec{v}} = \frac{\vec{v}_1 + \vec{v}_2 + \dots + \vec{v}_n}{n} 
        \]
        \ii When there are \(n\) equal intervals of distance/displacement with velocity \(\vec{v}_1\), \(\vec{v}_2\), \ldots \(\vec{v}_n\).
        Then, \[
            \avg{\vec{v}} = \frac{n}{\frac{1}{\vec{v}_1} + \frac{1}{\vec{v}_2} + \dots + \frac{1}{\vec{v}_n}} 
        \]
    \end{casework}
\end{algorithm}

\para{In general velocity is also referred to as `instantaneous velocity'. This is done to remind of the difference
with average velocity. We will not refer to it that way since by velocity we shall always mean \(\vec{v}\) as defined in
\cref{def: velocity}. The same goes for speed. Speed is defined as \(\ndiv{d}\) where \(d\) is the distance.
Average speed is defined similar to average velocity.}

\marginnote{Magnitude refers to the absolute value of a vector, i.e., its modulus. \(\norm{\vec{a}}\) is the 
magnitude of \(\vec{a}\) and is often simply denoted \(a\).}

\para{There is an interesting relation between speed and velocity. Consider the distance between \(a\) and \(b\) as 
\(a \to b\). The path between them approaches a straight line. Thus, the distance and displacement become the same
in the limiting case. Therefore, the \emph{magnitude} of velocity is the speed at that instant.} 

\begin{algorithm}
    Let us consider some techniques for solving equations of velocity. Consider the case when velocity and time are given
    and we have to figure out displacement or when displacement and and velocity are given, but time is not. 
    \begin{enumerate}
        \ii \(v = c\) : This is a trivial case. Just use equations of uniformly accelerated motion.
        \ii \(v = f(x)\) : \(\displaystyle \dv{x}{t} = f(x)\). Then, \(\dfrac{\dd{x}}{f(x)} = \dd{t}\).
        Now, we simply have, \[
            \int^{x_2}_{x_1} \frac{\dd{x}}{f(x)} = \int^{t_2}_{t_1} \dd{t}
        \] to get \(x_1\) and \(x_2\) or \(t_1\) and \(t_2\). We may also set our co-ordinates such that \(x_2 = x\), \(x_1 = 0\) and \(t_2 = t\), \(t_1 = 0\).
        \ii \(v = f(t)\) : This is also easy, just, \(\displaystyle f(x) = \dv{x}{t}\), then,
        \[
            \int^{t_2}_{t_1} f(t) \dd{t} = \int^{x_2}_{x_1} \dd{x}
        \] We simply get \(x_1\) and \(x_2\) or \(t_1\) and \(t_2\) by solving the integral.
    \end{enumerate}
\end{algorithm}

\section{Acceleration}

If a body does not move with uniform velocity, it accelerates. Acceleration is also a vector and
is the rate of change of velocity.

\index{Acceleration}

\begin{definition}
    [Acceleration]
    \label{def: acceleration}
    Acceleration is defined as,
    \begin{equation}
        \vec{a} \equiv \ndiv{\vec{v}} = \nddiv{\vec{r}}    
    \end{equation}
\end{definition}

There is also another way to represent acceleration, as, \[
    \vec{a} = \vec{v}\dv{\vec{v}}{x}
\]

Such a representation is particularly useful when the velocity of a particle 
is a function of its position. In such a case, derivating with time is quite a hassle.

\section{Uniformly Accelerated Motion}

\para{When the acceleration of a particle, \(a\) is constant, it undergoes uniformly accelerated motion.
We use special cases of integrals of \(a\) and \(v\) with respect to time and position for deriving
the equations of importance.}

\begin{theorem}
    For a particle with initial velocity \(u\) undergoing uniformly accelerated motion of acceleration \(a\),
    \begin{align}
        v &= u + at \\
        v^2 &= u^2 + 2as \\
        s &= ut + \frac{1}{2} at^2 \\
        s_{n} &= u + \frac{a}{2}(2n - 1)
    \end{align}
\end{theorem}

\para{The derivation of these is quite easy, integrate \(\int a \dd{x}\) for the first, 
\(\int a \dd{t}\) for the second (both with constant acceleration), substitute from the former into the latter for the third,
and finally for \(n^{th}\) second just do \(s_n - s_{n-1}\).} 

\para{Now, we will be looking at some fascinating questions that I have encountered.}

\begin{example}
    Consider a particle moving with acceleration \(\alpha\) initially. After some time, it starts
    to `de-accelerate' at \(\beta\) acceleration. After time \(t = T\), the particle comes to a stop.
    Find the displacement covered by particle in time \(T\).
\begin{soln}
        It might be tempting to start calculating after setting up time \(t = t_0\) when acceleration
        changed from \(\alpha\) to \(\beta\), i.e., the particle moves with acceleration \(\alpha\) till
        \(t = t_0\).\\
        However, let us try to do a cleaner solution. Consider the \(v-t\) graph of the particle,

        \begin{figure}
            [H]
            \centering
            \begin{tikzpicture}
                \begin{axis}[funcgraphbare, xmax=5.8, ymax=10, xlabel={$t$}, ylabel={$v$}]
                    \addplot[red,domain=0:3]{3*x} node[left,pos=1/2] {$\alpha$};
                    \addplot[blue,domain=3:4.8]{(-5)*x+24} node[right,pos=1/2] {$\beta$};
                    \draw [black,dashed] (axis cs:3,0)--(axis cs:3,9) node[left,pos=1/25] {$t_0$};
                    \draw [black] (axis cs:4.8,-1)--(axis cs:4.8,0.5) node[right,pos=1] {$T$};
                    \draw [black,dashed] (axis cs:0,9)--(axis cs:4.8,9) node[right,pos=1] {$v_{max}$};
                \end{axis}
            \end{tikzpicture}
        \end{figure}
        Note that \(\alpha(t_0) = \beta(T-t_0) = v_{max}\). Thus,
        \begin{align*}
            \alpha(t_0) &= \beta(T-t_0) \\
            \alpha(t_0) &= \beta(T) - \beta(t_0) \\
            \alpha(t_0) + \beta(t_0) &= \beta(T) \\
            t_0 &= \frac{\beta T}{\alpha + \beta} \\
            v_{max} &= \alpha t_0 \\
            v_{max} &= \frac{\alpha\beta T}{\alpha + \beta}
        \end{align*}
        Now, note that \(\dfrac{1}{2} \times v_{max} \times T\) is the area under the graph.
        \begin{moral}
            The area under the \(v-t\) is the total displacement.
        \end{moral}
        Thus, displacement, \(s = \dfrac{\alpha\beta T^2}{2(\alpha + \beta)}\).
\end{soln}
\end{example}

\begin{example}
    A car is at distance \(d\) from a boy. It starts accelerating at \(a\) \si{\metre\per\second\squared}. What is the minimum velocity that the boy should
    have to catch up with the car.
    \begin{soln}
        Consider separation of boy and car, \(s_{c,b}\). Using the equations of motion, 
        we have \(s_{b} = vt\), \(s_{c} = \dfrac{1}{2}at^2\). Thus, 
        \begin{equation}
            \label{eq: vel}
            s_{c,b} = d + \frac{1}{2}at^2 - vt        
        \end{equation}
    
        The most important thing to note here is that, 
        \begin{moral}
            For two objects to meet, the solution for time must be real.
        \end{moral}
        Therefore, from the simple equation, \eqref{eq: vel}, we must have, for real solutions,
        \(b^2 - 4ac\) for the equation \(at^2 + bt + c\). Solving for this by substituting values from
        \eqref{eq: vel}, we have,
            \begin{equation}
                v \ge \sqrt{2ad}
            \end{equation}
    \end{soln}
\end{example}

\begin{remark}
    One thing of note is that the converse of above is true as well,
    \begin{moral}
        For two objects to \emph{not} meet, the solution of time must be imaginary or negative.    
    \end{moral}
\end{remark}

\para{One more key note to takeaway is that,}

\begin{moral}
    At instant of maximum separation, relative velocity of particles is \(0\).
\end{moral}

\begin{example}
    A body is dropped at \(t = 0\), after time \(t = t_0\), another body is thrown downwards with
    velocity \(u\) \si{\meter\per\second}. Assuming first body reaches ground first, plot graph
    of separation.

    \begin{soln}
        At instant \(t_0\), displacement of first particle = \(\dfrac{1}{2}gt_{0}^2\). Note that here
        we set up co-ordinates such that positive \(y\) is downwards from point of drop. The displacement 
        of first body at \(t = t\) after \(t_0\) but before \framebox{reaching ground} is
        \begin{equation}
            s_1 = \frac{1}{2}gt_{0}^2 + \frac{1}{2}g(t-t_0)^2
        \end{equation}
        While for second body is,
        \begin{equation}
            s_2 = ut + \frac{1}{2}g(t-t_0)^2
        \end{equation}

        Thus, 
        \begin{equation}
            s_{1,2} = \frac{1}{2}g(t_0)^2 + t(gt_0 - u)
        \end{equation}

        Thus, before reaching ground, \(s-t\) graph is linear. However, after first body reaches ground,

        \begin{equation}
            s_{1,2} = ut + \frac{1}{2}g(t)^2
        \end{equation}
        which is a parabola. Thus, overall graph is,
        \begin{figure}[H]
            \centering
            \begin{tikzpicture}
                \begin{axis}[funcgraphbare, xlabel={$t$}, ylabel={$s$}]
                    \addplot[red, domain=0:2] {x+2};
                    \addplot[red,domain=2:3] {1.5*(-x^2)+5*x};
                \end{axis}
            \end{tikzpicture}
        \end{figure}
    \end{soln}
\end{example}

\begin{example}
    Find the total displacement if the \(v-t\) graph is as follows,
    \begin{figure}
        [H]
        \centering
        \begin{tikzpicture}
            \begin{scope}[xshift=10cm]
                % Axes
                \draw (0,0) node[below left] {$O$}
                    (0,0) -- (7,0) node[below] {$t$}
                    (0,-0.5) -- (0,3.5) node[left] {$v$};
                
                \draw (6.4,0) arc[start angle=0, end angle=180, radius=3.2];
            \end{scope}
        \end{tikzpicture}
    \end{figure}
    The idea behind this is simple, the graph here is \textbf{not} a semicircle. It is an
    ellipse, because of dimensional limitations! The equations of circle, \(v^2 + t^2 = r^2\)
    doesn't work because we cannot add two dimensionally different quantities.

    Therefore, the graph is an ellipse, and we conclude by computing the area of the ellipse. I haven't
    given the data, since I wanted to mention the idea.
\end{example}

\section{Motion in Polar Co-ordinates}

The polar co-ordinate axes are the two-dimesional subset of the 3-d cylindrical co-ordinates. 
Any point in polar co-ordinates is depicted by a system of two unit vectors, \(\unitv{r}\) and
\(\unitv{\theta}\). However, each of these are \framebox{dependent on the position of the particle} and
maybe written explicitly as \(\unitv{r}(\theta)\) and \(\unitv{\theta}(\theta)\).  

\begin{figure}
    [H]
    \centering
    \scalebox{0.7}{\incfig{polar}}
    \caption{Polar axes}
    \label{fig: polar}
\end{figure}

Here, \(\unitv{r}\) points in the direction of increasing radius(along the straight lines in \cref{fig: polar})
\(\unitv{\theta}\) points in the direction of increasing \(\theta\) (along the tangent to the circles).
These two unit vectors are \emph{orthogonal} at any point. 

\begin{figure}
    [H]
    \centering
    \incfig{polarunit}
    \caption{Unit vectors in polar co-ordinates}
    \label{fig: polarunit}
\end{figure}

\noindent In \cref{fig: polarunit} if \(\vec{r}\) makes
an angle \(\theta\) with the horizontal, then,

\begin{align}
    \label{eq: polardef}
    \unitv{r} &= \cos \theta \unitv{i} + \sin \theta \unitv{j}\\
    \unitv{\theta} &= -\sin \theta \unitv{i} + \cos \theta \unitv{j}
\end{align}

\section{Kinematics in Polar Co-ordinates}

Now let us formulate kinematics in polar co-ordinates. The position \(\vec{r}\) can be written as

\begin{equation}
    \vec{r} = r \unitv{r}
\end{equation}

\noindent Velocity follows normally, 
\begin{equation}
    \vec{v} = \ndiv{\vec{r}} = \ndiv{r} \unitv{r} + r\dv{\unitv{r}}{t} 
\end{equation}
or does it? What even is \(\displaystyle\dv{\unitv{r}}{t}\) ? Well the answer lies in 
the time-derivative of the unit vectors. Using the definitions in \eqref{eq: polardef}, we get,

\begin{align}
    \dv{\unitv{r}}{t} &= \ndiv{\theta} \unitv{\theta}\\
    \dv{\unitv{\theta}}{t} &= -\ndiv{\theta} \unitv{r}
\end{align}

\noindent Using these results, we get that

\begin{empheq}[box=\widefbox]{align}    
    \vec{v} & = \ndiv{r} \unitv{r} + r\ndiv{\theta} \unitv{\theta}\\
    \label{polar: accel}
    \vec{a} & = (\nddiv{r} - r\ndiv{\theta}^2) \unitv{r} + (r\nddiv{\theta} + 2\ndiv{r}\ndiv{\theta}) \unitv{\theta}
\end{empheq}

\noindent The most interesting result are the various terms of acceleration in \eqref{polar: accel}. First let 
us consider the terms along \(\unitv{r}\). \(\nddiv{r}\) is the \emph{radial} acceleration. It is
the change is radial speed. The second term \(-r\ndiv{\theta}^2\) is the \emph{centripetal} acceleration.
It arises from a change in tangential velocity.

Now let us consider the terms along \(\unitv{\theta}\). The \(r\nddiv{\theta}\) term is the tangential
acceleration resulting from change in tangential speed. The other term \(2\ndiv{r}\ndiv{\theta}\) is the
\emph{coriolis} acceleration. I shall discuss it when discussing rotating frames.

\section{Free Fall}

When an object is dropped downwards freely, it undergoes \emph{free fall}. We ignore
air resistance, until we're told not to and in general also talk about cases of the
object being thrown instead of dropped. An object dropped from any height undergoes
accelerations \(\vec{a}\) where \(\norm{\vec{a}} = g\). The direction and sign of acceleration
is entirely dependent on us.

In general, it is a good idea to setup our axes such that the direction above ground is positive
and towards the ground is negative. The acceleration during free fall is always towards
the ground. Therefore, for these cases \(\vec{a} = -g \unitv{j}\). 

Some important points of note are that ``\(g\)'' does not become positive or negative or whatever,
this is a common misconception. Instead, we just decide where exactly the positive direction is
and setup \(g\) according to that.
\parbreak
When a ball is thrown with some velocity \(u\) say in the positive directions
(change it into \(-u\) if in the negative direction), then we have,

\begin{algorithm}
    The following are special cases of equations of motion.
    \begin{enumerate}
        \ii \(v = u - gt\)
        \ii \(v^2 = u^2 -2gh\)
        \ii \(h = ut - \dfrac{1}{2}gt^2\)
    \end{enumerate} 
\end{algorithm}

The case when a ball is thrown upwards is exactly symmetrical to this one, but there
the ball goes up, reachs \(v = 0\) and then goes down. Also note that all of these equations are vector
equations.

\begin{example}
    Two balls are thrown in intervals of \(2\) seconds with velocity \(u\) each. Find the
    time when they collide.

    \begin{soln}
        Note that when they collide, their displacement from ground must be the same! Setting 
        the origin at ground, and direction of away from ground as positive, we have,
        
        For the first ball, \(h_{1} = ut - \dfrac{1}{2}gt^2\). Since the second ball
        is thrown \(2\) seconds later, \(h_{2} = u(t-2) - \dfrac{1}{2}g(t-2)^2\). Now, we just have
        \(h_{1} = h_{2}\) and the calculation is trivial.
    \end{soln}
\end{example}

The thing to note in this example is that you do not need to setup two cases, one in which
\(g\) is negative and one in which \(g\) is positive since it is in the direction of velocity. 
Just note that these equations are vectorial and we are done.

Some quick formulas to remember because they help a lot,

\begin{algorithm}
    For an object dropped from height \(h\),
    \begin{enumerate}
        \ii Time to reach ground, \(t = \sqrt{\dfrac{2h}{g}}\).
        \ii Velocity when it reaches ground \(v = \sqrt{2gh}\).
    \end{enumerate}
\end{algorithm}

Deriving them is easy, and they can be generalised as when it travels height \(h\) between
the point of drop to anywhere, instead of ground. Clearly, both still follow there.

Also, always remember,

\begin{moral}
    When an object is thrown/dropped from another object traveling with some velocity and
    acceleration, it inherits its velocity but not its acceleration.
\end{moral}

\todo{add screw/lift example}

\section{Graphs in Motion}

In the graphs we are going to discuss, acceleration will either be \(0\) or constant.
These graphs help to quickly solve problems, especially related to total distance and 
displacement. 

Let us consider the direction away from origin positive, and towards origin negative.
Then, consider constant acceleration, \(a = c\). We have the following displacement time
graphs,

\begin{figure}[H]
    \centering
    \begin{subfigure}{0.5\textwidth}
        \centering
        \begin{tikzpicture}
            \begin{axis}
            [funcgraphbare, xlabel=$t$, ylabel=$s$]
            \addplot[red, domain=0:16]{x^2};
            \end{axis}
        \end{tikzpicture}
        \caption{For \(c > 0\). }
    \end{subfigure}%
    \begin{subfigure}{0.5\textwidth}
        \centering
        \begin{tikzpicture}
            \begin{axis}
            [funcgraphbare, xlabel=$t$, ylabel=$s$]
            \addplot[red, domain=0:16]{-x^2 + 256};
            \end{axis}
        \end{tikzpicture}
        \caption{For \(c < 0\)}
    \end{subfigure}%
\end{figure}

\section{Relative Motion}

There is a very neat technique to solve questions of kinematics, it is to switch our 
reference frames. Let the base reference frame be that of earth, then,

\begin{figure}
    [H]
    \centering
    \incfig{reference}
\end{figure}

If \(\vec{r}_1\) evolves with time, \(r_1(t)\). And so does \(r_2(t)\). Then,

\begin{align*}
    \vec{r}_1 &= \vec{r}_{1,2} + \vec{r}_2 \\
    \ndiv{\vec{r}}_1 &= \ndiv{\vec{r}}_{1,2} + \ndiv{\vec{r}}_2 \\
    \vec{v}_1 &= \vec{v}_{1,2} + \vec{v}_{2}\\
    \vec{v}_{1,2} &= \vec{v}_1 - \vec{v}_{2}
\end{align*}

Obviously if the vectors face the same direction in the same plane, then, the components subtract. If they face
opposite direction, then the components add. 

\(\vec{v}_{1,2}\) is called the velocity of \(\mathit{1}\) w.r.t \(\mathit{2}\). When doing
problems, try to setup a reference frame such that velocity of any single object vanishes.
This allows for a very neat simplification of problems. 

\section{Motion in higher dimensions}

We can mostly safely extend our results from one dimension to higher dimensions.
So for instances, velocity of a particle whose trajectory is defined as 
\(x\unitv{i} + y\unitv{j} + z\unitv{k}\) can simply be computed as,
\[v = \ndiv{x}\unitv{i} + \ndiv{y}\unitv{j} + \ndiv{z}\unitv{k}\] Such a thing
is not true for non-cartesian co-ordinates, as we say in the case of polar forms.
Due to the orthogonality of the cartesian unit vectors, and using dot products
we simply have, 
\[v_x = \ndiv{x}, v_y = \ndiv{y}, v_z = \ndiv{z}\]

And the same is true for all other vectors. This gives us the power to
resolve our vector components into components alongside unit vectors
and consider independent one dimensional motion for each of them.

\section{Projectile Motion}

Projectile motion is one of the most important motion to consider in 
2 dimensions. It is the motion of a projected body. Generally, we
only care about the motion of a body projected from ground under the influence
of gravity. 

Let a body be projected at an angle \(\theta\) from the ground at velocity \(\vec{v}\)
along \(\theta\). Then, by resolving \(\vv\) into \(v_x \unitv{i} + v_y\unitv{j}\),
we have, 
\begin{align*}
    x &= v_xt \unitv{i} \\
    y &= v_yt - \frac{1}{2}gt^2 \unitv{j}
\end{align*}

Doing some manipulations, we may note that \(v_x = v\sin(\theta)\), \(v_y = v\cos(\theta)\).
Also, substituting \(t = x/v\sin(\theta)\), we have,
\begin{equation}
    \boxed{y = x\tan(\theta) - \frac{gx^2}{2v^2\cos^2(\theta)}}
\end{equation}

The path of the projectile is parabolic as shown in \cref{fig: projectile}.

\begin{figure}[H]
    \centering
    \incfig{projectile}
    \caption{Trajectory of a projectile}
    \label{fig: projectile}
\end{figure}

Here, \(H\) is the maximum height of the particle. Clearly,
when the projectile is at \(H\), \(v_y = 0\). We have, 
\(v_y^2 = 2gH\),
\begin{equation}
    \boxed{H = \frac{u^2\sin^2(\theta)}{2g}}
\end{equation}

The time period of the projectile can be simply calculated as well,
since at the final time, \(T\), \(y = v_yt - 1/2gt^2 = 0\) and,

\begin{equation}
    \boxed{T = \frac{2v\sin(\theta)}{g}}
\end{equation}

\(R\) is the range the particle covers horizontally in time \(T\) and is
clearly,

\begin{equation}
    \boxed{R = \frac{v^2\sin(2\theta)}{g}}
\end{equation}

Note that if the particle is projected at \(\theta = \pi/4\), 
the range is maximum and is \(u^2/2g\). Otherwise,
for \(0 < \theta < \pi/2\), \(\sin(2\theta)\) attains the same 
value for two \(\theta_1, \theta_2\). These \(\theta_1\), \(\theta_2\) are complementary
by trigonometry.

