\chapter{Kinematics}

\para{Kinematics is the study of motion without concerning its cause. It allows
us to calculate and find out the evolution of a body. Everything in our world undergoes
motion, one way or the other. To begin our study of motion, let us first
concern ourselves with some definitions.}

\section{Reference Frame and Point Particle}

\para{Motion of any object is considered to relative to a reference frame. We consider motion of
a body with respect to any other body or co-ordinate frame.}

\begin{figure}[ht]
        \centering
        \incfig{3daxes}    
    \caption{A Reference Frame}\label{fig: 3daxes}
\end{figure}

\para{We use a reference frame according to our convenience. In general, we use the 3-dimensional cartesian system like the one
in \Cref{fig: 3daxes}.}

\index{Reference Frame}

\begin{definition}
    [Reference Frame]
    A reference frame is a co-ordinate frame relative to which motion of a particle is considered.
\end{definition}

\para{We always consider the motion of a point particle. Well, why? It is done so to simplify
the calculation of our system. The motion of a (rigid)body is discussed later.}

\index{Particle}
\begin{definition}
    [Point Particle]
    A particle whose size is negligible in the study of its motion is called a point particle.
\end{definition}

\section{Position Vector and Displacement}

\subsection{Position Vector}
\index{Position Vector}

\marginnote{One thing of importance is that a position vector of a particle is `generally' a function
of time. We generally denote this as \(r(t)\) to show that is a function.}

\para{We define the position of a particle relative to a reference frame using a \emph{position 
vector}. One end of the position vector is the origin of our reference frame. The other is 
the particle itself. Consider \Cref{fig: position}, the 
vector \(\vect{r}\) from \(O\) to \(A\). It describes the position of \(A\) relative to the
co-ordinate frame. Let \(A = (x,y,z)\). Then we denoted \(\vect{r}\) as \(\vect{r}(x,y,z)\).}

\para{The position to any point is written as, \[
    \vect{r} = (x,y,z) = x\unitv{i} + y\unitv{j} + z\unitv{k}
\]}



\begin{figure}
    [H]
    \centering
    \scalebox{0.6}{\incfig{position}}
    \caption{A position vector, \(\mathbf{r}\).}
    \label{fig: position}
\end{figure}


\marginnote{Note that the position vector is not a \emph{true} vector. It is tied to particular
    reference frame.}

\subsection{Displacement}
\index{Displacement}

\para{Consider the movement of a particle from \(A = (x_1,y_1,z_1)\) to \(B = (x_2,y_2,z_2)\). 
The \emph{displacement} defines a true vector \(\vect{S} = (x_2-x_1,y_2-y_1,z_2-z_1)\).
\(\vect{S}\) is called the displacement vector. Note that it contains no information about the
individual points, but only about the relative position of each. Thus, our choice
of reference frame, and thus position vector does not matter when we're concerned with displacement
. In \Cref{fig: displacement}, the vector
\(\vect{S}_{AB}\) defines a displacement from \(A\) to \(B\).}

\begin{figure}
    [H]
    \centering
    \scalebox{0.8}{\incfig{displacement}}
    \caption{Displacement from \(A\) to \(B\)}
    \label{fig: displacement}
\end{figure}

\para{The difference between displacement and distance is that of the path they describe. The distance covered is the length
of the actual path, while the magnitude of the displacement is simply the length of the vector between the final and initial positions.}

\begin{figure}
    [H]
    \centering
    \incfig{distance}
    \caption{Distance vs Displacement}
\end{figure}

\begin{algorithm}
    Consider the case of displacement and distance of a particle in circular motion with radius \(r\) Let it move from \(A\) to \(B\), inscribing
    an angle \(\theta\) (in radians) Then, we have,\\
    ~\\
    Distance between \(A\), \(B\) = \(r\theta\)\\
    Displacement between \(A\), \(B\) = \(2r\sin\left(\dfrac{\theta}{2}\right)\)
\end{algorithm}

\begin{figure}
    [H]
    \centering
    \incfig{distdisp}
    \caption{}
\end{figure}

\section{Speed and Velocity}

\index{Speed}

\para{We are already familiar with the notion of speed. In elementary terms, speed is \({distance}/{time}\).
Speed itself is an elementary concept. It tells us nothing about the direction of the object whose speed we are
talking into consideration. Thus, it is a \emph{scalar} quantity.}

\parbreak
\marginnote{The symbol \(\equiv\) stands for `defined as'.}

\index{Velocity}

\para{The Velocity of a particle, contrastingly, contains information about both the speed and direction of the
object in consideration. It is a vector. Similar to position, it is also generally a function of time \(v(t)\), though it
may also be dependent on the position \(v(x)\).} 

\begin{definition}
    [Velocity]
    \label{def: velocity}
    Velocity is defined as,
    \begin{equation}
        \vect{v} \equiv \ndiv{\vect{r}}    
    \end{equation}
\end{definition}
\marginnote{The dot over \(r\) in \(\dot{\vect{r}}\) tells us that \(r\) has been differentiated
with respect to time.}

\para{Sometimes, we do not need to compute the derivative of the position vector, and simply need an average
estimate, or in the case of uniformly accelerated motion, it is not required that we computer derivatives.
In such a case we use the concept of \textbf{average velocity}, \[
    \avg{\vect{v}}_{12} = \frac{\vect{r}_1-\vect{r}_2}{t_2 - t_1}  
\]}

\begin{algorithm}
    Let us consider the cases of average velocity,
    \begin{enumerate}
        \ii[\textbf{Case 1}] When there are \(n\) equal time intervals with velocity \(\vect{v}_1\), \(\vect{v}_2\), \ldots \(\vect{v}_n\).
        Then, \[
            \avg{\vect{v}} = \frac{\vect{v}_1 + \vect{v}_2 + \dots + \vect{v}_n}{n} 
        \]
        \ii[\textbf{Case 2}] When there are \(n\) equal intervals of distance/displacement with velocity \(\vect{v}_1\), \(\vect{v}_2\), \ldots \(\vect{v}_n\).
        Then, \[
            \avg{\vect{v}} = \frac{n}{\frac{1}{\vect{v}_1} + \frac{1}{\vect{v}_2} + \dots + \frac{1}{\vect{v}_n}} 
        \]
    \end{enumerate}
\end{algorithm}

\para{In general velocity is also referred to as `instantaneous velocity'. This is done to remind of the difference
with average velocity. We will not refer to it that way since by velocity we shall always mean \(\vect{v}\) as defined in
\Cref{def: velocity}. The same goes for speed. Speed is defined as \(\ndiv{d}\) where \(d\) is the distance.
Average speed is defined similar to average velocity.}

\marginnote{Magnitude refers to the absolute value of a vector, i.e., its modulus. \(\abs{\vect{a}}\) is the 
magnitude of \(\vect{a}\) and is often simply denoted \(a\).}

\para{There is an interesting relation between speed and velocity. Consider the distance between \(a\) and \(b\) as 
\(a \to b\). The path between them approaches a straight line. Thus, the distance and displacement become the same
in the limiting case. Therefore, the \emph{magnitude} of velocity is the speed at that instant.} 

\begin{algorithm}
    Let us consider some techniques for solving equations of velocity. Consider the case when velocity and time are given
    and we have to figure out displacement or when displacement and and velocity are given, but time is not. 
    \begin{enumerate}
        \ii \(v = c\) : This is a trivial case. Just use equations of uniformly accelerated motion.
        \ii \(v = f(x)\) : \(\displaystyle \derivative{x}{t} = f(x)\). Then, \(\dfrac{\differential{x}}{f(x)} = \differential{t}\).
        Now, we simply have, \[
            \int^{x_2}_{x_1} \frac{\differential{x}}{f(x)} = \int^{t_2}_{t_1} \differential{t}
        \] to get \(x_1\) and \(x_2\) or \(t_1\) and \(t_2\). We may also set our co-ordinates such that \(x_2 = x\), \(x_1 = 0\) and \(t_2 = t\), \(t_1 = 0\).
        \ii \(v = f(t)\) : This is also easy, just, \(\displaystyle f(x) = \derivative{x}{t}\), then,
        \[
            \int^{t_2}_{t_1} f(t) \differential{t} = \int^{x_2}_{x_1} \differential{x}
        \] We simply get \(x_1\) and \(x_2\) or \(t_1\) and \(t_2\) by solving the integral.
    \end{enumerate}
\end{algorithm}

\section{Acceleration}

\para{If a body does not move with uniform velocity, it accelerates. Acceleration is also a vector.}

\index{Acceleration}

\begin{definition}
    [Acceleration]
    \label{def: acceleration}
    Acceleration is defined as,
    \begin{equation}
        \vect{a} \equiv \ndiv{\vect{v}} = \nddiv{\vect{r}}    
    \end{equation}
\end{definition}

\para{There is also another way to represent acceleration, as, \[
    \vect{a} = \vect{v}\derivative{\vect{v}}{x}
\]
I must be held accountable here, because I do not, particularly, know the use of this representation, if it is worth
remembering, or should be discarded.}

\section{Uniformly Accelerated Motion}

\para{When the acceleration of a particle, \(a\) is constant, it undergoes uniformly accelerated motion.
We use special cases of integrals of \(a\) and \(v\) with respect to time and position for deriving
the equations of importance.}

\begin{theorem}
    For a particle with initial velocity \(u\) undergoing uniformly accelerated motion of acceleration \(a\),
    \begin{align}
        v &= u + at \\
        v^2 &= u^2 + 2as \\
        s &= ut + \frac{1}{2} at^2 \\
        s_{n} &= u + \frac{a}{2}(2n - 1)
    \end{align}
\end{theorem}

\para{The derivation of these is quite easy, integrate \(\int a \differential{x}\) for the first, 
\(\int a \differential{t}\) for the second (both with constant acceleration), substitute from the former into the latter for the third,
and finally for \(n^{th}\) second just do \(s_n - s_{n-1}\).} 

\para{Now, we will be looking at some fascinating questions that I have encountered.}

\begin{example}
    Consider a particle moving with acceleration \(\alpha\) initially. After some time, it starts
    to `de-accelerate' at \(\beta\) acceleration. After time \(t = T\), the particle comes to a stop.
    Find the displacement covered by particle in time \(T\).
\begin{soln}
        It might be tempting to start calculating after setting up time \(t = t_0\) when acceleration
        changed from \(\alpha\) to \(\beta\), i.e., the particle moves with acceleration \(\alpha\) till
        \(t = t_0\).\\
        However, let us try to do a cleaner solution. Consider the \(v-t\) graph of the particle,

        \begin{figure}
            [H]
            \centering
            \begin{tikzpicture}
                \begin{axis}[funcgraphbare, xmax=5.8, ymax=10, xlabel={$t$}, ylabel={$v$}]
                    \addplot[red,domain=0:3]{3*x} node[left,pos=1/2] {$\alpha$};
                    \addplot[blue,domain=3:4.8]{(-5)*x+24} node[right,pos=1/2] {$\beta$};
                    \draw [black,dashed] (axis cs:3,0)--(axis cs:3,9) node[left,pos=1/25] {$t_0$};
                    \draw [black] (axis cs:4.8,-1)--(axis cs:4.8,0.5) node[right,pos=1] {$T$};
                    \draw [black,dashed] (axis cs:0,9)--(axis cs:4.8,9) node[right,pos=1] {$v_{max}$};
                \end{axis}
            \end{tikzpicture}
        \end{figure}
        Note that \(\alpha(t_0) = \beta(T-t_0) = v_{max}\). Thus,
        \begin{align*}
            \alpha(t_0) &= \beta(T-t_0) \\
            \alpha(t_0) &= \beta(T) - \beta(t_0) \\
            alpha(t_0) + \beta(t_0) &= \beta(T) \\
            t_0 &= \frac{\beta T}{\alpha + \beta} \\
            v_{max} &= \alpha t_0 \\
            v_{max} &= \frac{\alpha\beta T}{\alpha + \beta}
        \end{align*}
        Now, note that \(\dfrac{1}{2} \times v_{max} \times T\) is the area under the graph.
        \begin{moral}
            The area under the \(v-t\) is the total displacement.
        \end{moral}
        Thus, displacement, \(s = \dfrac{\alpha\beta T^2}{2(\alpha + \beta)}\).
\end{soln}
\end{example}

\begin{example}
    A car is at distance \(d\) from a boy. It starts accelerating at \(a\) \si{\metre\per\second\squared}. What is the minimum velocity that the boy should
    have to catch up with the car.
    \begin{soln}
        Consider separation of boy and car, \(s_{c,b}\). Using the equations of motion, 
        we have \(s_{b} = vt\), \(s_{c} = \dfrac{1}{2}at^2\). Thus, 
        \begin{equation}
            \label{eq: vel}
            s_{c,b} = d + \frac{1}{2}at^2 - vt        
        \end{equation}
    
        The most important thing to note here is that, 
        \begin{moral}
            For two objects to meet, the solution for time must be real.
        \end{moral}
        Therefore, from the simple equation, \eqref{eq: vel}, we must have, for real solutions,
        \(b^2 - 4ac\) for the equation \(at^2 + bt + c\). Solving for this by substituting values from
        \eqref{eq: vel}, we have,
            \begin{equation}
                v \ge \sqrt{2ad}
            \end{equation}
    \end{soln}
\end{example}

\begin{remark}
    One thing of note is that the converse of above is true as well,
    \begin{moral}
        For two objects to \emph{not} meet, the solution of time must be imaginary or negative.    
    \end{moral}
\end{remark}

\para{One more key note to takeaway is that,}

\begin{moral}
    At instant of maximum separation, relative velocity of particles is \(0\).
\end{moral}

\begin{example}
    A body is dropped at \(t = 0\), after time \(t = t_0\), another body is thrown downwards with
    velocity \(u\) \si{\meter\per\second}. Assuming first body reaches ground first, plot graph
    of separation.

    \begin{soln}
        At instant \(t_0\), displacement of first particle = \(\dfrac{1}{2}gt_{0}^2\). Note that here
        we set up co-ordinates such that positive \(y\) is downwards from point of drop. The displacement 
        of first body at \(t = t\) after \(t_0\) but before \framebox{reaching ground} is
        \begin{equation}
            s_1 = \frac{1}{2}gt_{0}^2 + \frac{1}{2}g(t-t_0)^2
        \end{equation}
        While for second body is,
        \begin{equation}
            s_2 = ut + \frac{1}{2}g(t-t_0)^2
        \end{equation}

        Thus, 
        \begin{equation}
            s_{1,2} = \frac{1}{2}g(t_0)^2 + t(gt_0 - u)
        \end{equation}

        Thus, before reaching ground, \(s-t\) graph is linear. However, after first body reaches ground,

        \begin{equation}
            s_{1,2} = ut + \frac{1}{2}g(t)^2
        \end{equation}
        which is a parabola. Thus, overall graph is,
        \begin{figure}[H]
            \centering
            \begin{tikzpicture}
                \begin{axis}[funcgraphbare, xlabel={$t$}, ylabel={$s$}]
                    \addplot[red, domain=0:2] {x+2};
                    \addplot[red,domain=2:3] {1.5*(-x^2)+5*x};
                \end{axis}
            \end{tikzpicture}
        \end{figure}
    \end{soln}
\end{example}

\section{Time derivative of a vector}

\para{The time derivative of \(r\) and \(v\) have already been observed in \Cref{def: velocity} and 
\Cref{def: acceleration} already. However, let us have a better discussion of what it really entails.}

\todo{Add vectors, atleast elem stuff and rename jee to general or olymp.}

\para{When we write, \[
    \derivative{\vect{r}}{t}
\] The derivative differs from the derivative of a scalar. It has both magnitude and direction.
One way to compute it is to break it down to its components, \[
    \derivative{\vect{r}}{t} = \derivative{x}{t}  \unitv{i} + \derivative{y}{t}  \unitv{j} + \derivative{z}{t}  \unitv{k}
\]. But since the components of the vector depend on the reference frame, we may generalise the differentiation of a vector
in two parts.}

\begin{enumerate}
        \ii[\textbf{Case I}] \(\displaystyle\derivative{\vect{A}}{t} \perp \vect{A}\)
        \ii[\textbf{Case II}] \(\displaystyle\derivative{\vect{A}}{t} \parallel \vect{A}\)
\end{enumerate}

\para{For the first case, let us first consider a geometric argument for \(\increment \vect{A}\)
which is perpendicular to \(\vect{A}\). For sufficiently small values of \(\increment \vect{A}\),
the change in magnitude is negligible, we only observe a change in direction. This becomes
definitive in the limiting case. Consider the following figure,}

\begin{figure}
    [H]
    \centering
    \scalebox{0.7}{\incfig{timedev}}
\end{figure}

\para{\noindent This allows us to consider a simple geometric argument which gives \(\abs{\increment \vect{A}} = 2A\sin(\theta/2)\)
where \(\theta\) is the angle between \(\vect{A}_1\) and \(\vect{A}_2\).}

Since we're considering values of small \(\theta\), we may use the small angle approximation,
\(\sin(\theta) \approx \theta\). Therefore,

\begin{align}
    \abs{\vect{\increment A}} &\approx 2A \times \frac{\theta}{2}\\ 
    &\approx A \times \theta
\end{align}

Which is equivalent to \(A \increment \theta\). In the limit \(\increment \theta \to 0\),

\begin{equation}
    \diffd{A_{\perp}} = A \diffd{\theta}
\end{equation}

Which finally gives us,
\begin{equation}
    \abs{\derivative{\vect{A}_{\perp}}{t}} = A\derivative{\theta}{t}
\end{equation}

The second case is easy as it is just 
\begin{equation}
    \abs{\derivative{\vect{A}_{\parallel}}{t}} = \derivative{A}{t}
\end{equation}
    
This completes our formulation of the time derivative of any vector, independent of its co-ordinate
axes.

\section{Motion in Polar Co-ordinates}

The polar co-ordinate axes are the two-dimesional subset of the 3-d cylindrical co-ordinates. 
Any point in polar co-ordinates is depicted by a system of two unit vectors, \(\unitv{r}\) and
\(\unitv{\theta}\). However, each of these are \framebox{dependent on the position of the particle} and
maybe written explicitly as \(\unitv{r}(\theta)\) and \(\unitv{\theta}(\theta)\).  

\begin{figure}
    [H]
    \centering
    \scalebox{0.7}{\incfig{polar}}
    \caption{Polar axes}
    \label{fig: polar}
\end{figure}

Here, \(\unitv{r}\) points in the direction of increasing radius(along the straight lines in \Cref{fig: polar})
\(\unitv{\theta}\) points in the direction of increasing \(\theta\) (along the tangent to the circles).
These two unit vectors are \emph{orthogonal} at any point. 

\begin{figure}
    [H]
    \centering
    \incfig{polarunit}
    \caption{Unit vectors in polar co-ordinates}
    \label{fig: polarunit}
\end{figure}

\noindent In \Cref{fig: polarunit} if \(\vect{r}\) makes
an angle \(\theta\) with the horizontal, then,

\begin{align}
    \label{eq: polardef}
    \unitv{r} &= \cos \theta \unitv{i} + \sin \theta \unitv{j}\\
    \unitv{\theta} &= -\sin \theta \unitv{i} + \cos \theta \unitv{j}
\end{align}

\section{Kinematics in Polar Co-ordinates}

Now let us formulate kinematics in polar co-ordinates. The position \(\vect{r}\) can be written as

\begin{equation}
    \vect{r} = r \unitv{r}
\end{equation}

\noindent Velocity follows normally, 
\begin{equation}
    \vect{v} = \ndiv{\vect{r}} = \ndiv{r} \unitv{r} + r\derivative{\unitv{r}}{t} 
\end{equation}
or does it? What even is \(\displaystyle\derivative{\unitv{r}}{t}\) ? Well the answer lies in 
the time-derivative of the unit vectors. Using the definitions in \eqref{eq: polardef}, we get,

\begin{align}
    \derivative{\unitv{r}}{t} &= \ndiv{\theta} \unitv{\theta}\\
    \derivative{\unitv{\theta}}{t} &= -\ndiv{\theta} \unitv{r}
\end{align}