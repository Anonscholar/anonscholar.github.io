\chapter{Kinematics}
Kinematics is the study of motion without concerning its cause. It allows
us to calculate and find out the evolution of a body. Everything in our world undergoes
motion, one way or the other. To begin our study of motion, let us first
concern ourselves with some definitions.

\section{Reference Frame and Point Particle}

Motion of any object is considered to relative to an observer. The observer defines
a particular co-ordinate system called the reference frame

\begin{marginfigure}
        \centering
        \vspace{-9em}
        \scalebox{1.8}{\incfig{3daxes}}    
    \caption{A Reference Frame}\label{fig: 3daxes}
\end{marginfigure}

We use a reference frame according to our convenience. Most of the times we will work in a right 
handed cartesian co-ordinate system, the one in \cref{fig: 3daxes} (or a rotation of it).

\begin{definition}
    [Reference Frame]
    A \vocab{reference frame} is a co-ordinate frame relative to which motion of a particle is considered.
\end{definition}

We shall, until we encounter rigid body systems consider the motion of a point particle. 
Well, why? It is done so to simplify the calculation of our system. The motion of a (rigid) body is discussed later.

\index{Particle}
\begin{definition}
    [Point Particle]
    A particle whose size is negligible in the study of its motion is called a \vocab{point particle}.
\end{definition}

To discuss the motion, state of our body, it is also necessary to setup some time co-ordinate.
It is the reading of the clock in the observer's frame. In general, we setup our clocks
at \(0\) at the start of an event.

\sidenote{One thing of importance is that a position vector of a particle is a function
of time. We may denote this as \(r(t)\) to show that is a function.}

\section{Position Vector and Displacement}

\subsection{Position Vector}
\index{Position Vector}

We define the position of a particle relative to a reference frame using a \emph{position 
vector}. One end of the position vector is the origin of our reference frame. The other is 
the particle itself. Consider \cref{fig: position}, the 
vector \(\vec{r}_A\) from \(\mathcal{O}\) to \(A\). It describes the position of \(A\) relative to the
co-ordinate frame. Let \(A = (x,y,z)\). Then we denote \(\vec{r}_A\) as \(\vec{r}(x,y,z)\).

The position to any point is written as, \[
    \vec{r} = (x,y,z) = x\ihat + y\jhat + z\khat
\]

\begin{marginfigure}
    \centering
    \scalebox{2}{\incfig{position}}
    \caption{A position vector, \(\mathbf{r}\).}
    \label{fig: position}
\end{marginfigure}

\subsection{Displacement}
\index{Displacement}

\sidenote{Note that the position vector is not a \emph{true} vector. It is tied to particular
    reference frame.}

Consider the movement of a particle from \(A = (x_1,y_1,z_1)\) to \(B = (x_2,y_2,z_2)\). 
The \emph{displacement} is the change in the position vector,
and defines a true vector \(\vec{s} = (x_2-x_1,y_2-y_1,z_2-z_1)\).
\(\vec{s}\) is called the displacement vector. 

Note that it contains no information about the
individual positions, but only about the relative position of each. Thus, our choice
of reference frame, and thus position vector does not matter when we're concerned with displacement. 
In \cref{fig: displacement}, the vector
\(\vec{S}_{AB}\) defines a displacement from \(A\) to \(B\).

\begin{marginfigure}
    \centering
    \scalebox{2}{\incfig{displacement}}
    \caption{}
    \label{fig: displacement}
\end{marginfigure}

The difference between displacement and distance is that of the path they describe. The distance covered is the length
of the actual path, while the magnitude of the displacement is simply the length of the vector between the final and initial positions.

\section{Velocity and Acceleration}

\index{Speed}

We are already familiar with the notion of speed. In elementary terms, speed is {distance}/{time}.
It tells us nothing about the direction of the object whose speed we are
talking into consideration. Thus, it is a \emph{scalar} quantity.

\index{Velocity}

The Velocity of a particle, contrastingly, contains information about both the speed and direction of the
object in consideration. 

It is a vector. Similar to position, it is also a function of time \(v(t)\), though it
may also be dependent on the position \(v(x)\). Even when that is the case,
it can still be written as function of time.

\begin{definition}
    [Velocity]
    \label{def: velocity}
    Velocity is defined as,
    \begin{equation}
        \vec{v} \equiv \ndiv{\vec{r}}    
    \end{equation}
\end{definition}

\sidenote{The symbol \(\equiv\) stands for `defined as'.}

In particular, velocity is a function that is parametrized by time, 
it is a map,

\[
    \vv : \RR \to \RR^3,
\] 

which acts on the parameter \(t\) as, 

\begin{equation*}
    t \mapsto (v_x(t), v_y(t), v_z(t)) = \vv(t).
\end{equation*}

The same holds for position as well,

\[
    \vec{x} : \RR \to \RR^3.
\]

One thing of note is that velocity can also be expressed as the derivative of 
displacement, which is just \(\vec{r} - \vec{r}_0\). Since the latter 
term is constant, we simply get that \(\ndiv{\vec{s}} = \ndiv{\vec{r}}\).

From this definition, we may gather that,

\[\int_{r_0}^{r} \dd{\vec{r}} = \int_{0}^{t} \vv \dd{t},\]

Or, 
\begin{equation}
    \vec{r} = \vec{r}_0 + \int_{0}^{t} \vv \dd{t}.
\end{equation}

Sometimes, we do not need to compute the derivative of the position vector, and simply need an average
estimate, or in the case of uniformly accelerated motion, it is not required that we computer derivatives.
In such a case we use the concept of \textbf{average velocity}, \[
    \bavg{\vec{v}}_{12} = \frac{\increment\vec{s}}{t_2 - t_1}  
\]

\sidenote{The dot over \(r\) in \(\dot{\vec{r}}\) tells us that \(r\) has been differentiated
with respect to time.}

Velocity is also referred to as `instantaneous velocity'. This is done to remind of the difference
with average velocity. We will not refer to it that way since by velocity we shall always mean \(\vec{v}\) as defined in
\cref{def: velocity}. The same goes for speed. Notably, speed is not defined as \(\dv*{\abs{r}}{t}\).

While the average speed is very well that, speed is actually the magnitude of the velocity of that point. 
Consider the distance between \(a\) and \(b\) as 
\(a \to b\). The path between them approaches a straight line. Thus, the distance and displacement become the same
in the limiting case. Therefore, the \emph{magnitude} of velocity is the speed at that instant.

Therefore, 

\[\text{speed} = \abs{\dv{\vec{r}}{t}}\]

\section{Acceleration}

If a body does not move with uniform velocity, it accelerates. Acceleration is also a vector and
is the rate of change of velocity.

\index{Acceleration}

\begin{definition}
    [Acceleration]
    \label{def: acceleration}
    Acceleration is defined as,
    \begin{equation}
        \vec{a} \equiv \ndiv{\vec{v}} = \nddiv{\vec{r}}    
    \end{equation}
\end{definition}

Much like the velocity definition, we have,

\begin{equation}
    \vv = \vv_0 + \int_0^t \vec{a} \dd{t}
\end{equation}

There is also another way to represent acceleration, as, \[
    \vec{a} = \vec{v}\dv{\vec{v}}{x}
\]

Such a representation is particularly useful when the velocity of a particle 
is a function of its position. In such a case, differentiating with time is quite a hassle.

\section{Uniformly Accelerated Motion}

When the acceleration of a particle, \(a\) is constant, it undergoes uniformly accelerated motion.
We use special cases of integrals of \(a\) and \(v\) with respect to time and position for deriving
the equations of importance.

\begin{theorem}
    For a particle with initial velocity \(u\) undergoing uniformly accelerated motion of acceleration \(a\),
    \begin{align}
        v &= u + at \\
        v^2 &= u^2 + 2as \\
        s &= ut + \frac{1}{2} at^2 \\
        s_{n} &= u + \frac{a}{2}(2n - 1)
    \end{align}
\end{theorem}

The derivation of these is quite easy, integrate \(\int a \dd{x}\) for the first, 
\(\int a \dd{t}\) for the second (both with constant acceleration), substitute from the former into the latter for the third,
and finally for \(n^{th}\) second just do \(s_n - s_{n-1}\). 

\begin{example}
    A car is at distance \(d\) from a boy. It starts accelerating at \(a\) \unit{\metre\per\second\squared}. What is the minimum velocity that the boy should
    have to catch up with the car?

\begin{soln}
        Consider separation of boy and car, \(s_{c,b}\). Using the equations of motion, 
        we have \(s_{b} = vt\), \(s_{c} = \dfrac{1}{2}at^2\). Thus, 
        \begin{equation}
            \label{eq: vel}
            s_{c,b} = d + \frac{1}{2}at^2 - vt        
        \end{equation}
    
        From an inspection of the co-effecients of the \(x, x^2\) terms and the constant, 
        we see that if a real solution to this exists, it must be positive (if this may not be 
        apparent, recall vieta's relation and note \(a, v, d\) are all positive).

        So they must always meet if this has a real solution!

        Therefore, from the simple equation, \eqref{eq: vel}, we must have, for real solutions,
        \(b^2 - 4ac \ge 0\) for the equation \(at^2 + bt + c\). Solving for this by substituting values from
        \eqref{eq: vel}, we have,
            \begin{equation}
                v \ge \sqrt{2ad}
            \end{equation}
\end{soln}
\end{example}

\begin{example}
    A body is dropped at \(t = 0\), after time \(t = t_0\), another body is thrown downwards with
    velocity \(u\) \unit{\meter\per\second}. Assuming first body reaches ground first, plot graph
    of separation.

    \begin{soln}
        At instant \(t_0\), displacement of first particle = \(\dfrac{1}{2}gt_{0}^2\). Note that here
        we set up co-ordinates such that positive \(y\) is downwards from point of drop. The displacement 
        of first body at \(t = t\) after \(t_0\) but before \framebox{reaching ground} is
        \begin{equation}
            s_1 = \frac{1}{2}gt_{0}^2 + \frac{1}{2}g(t-t_0)^2
        \end{equation}
        While for second body is,
        \begin{equation}
            s_2 = ut + \frac{1}{2}g(t-t_0)^2
        \end{equation}

        Thus, 
        \begin{equation}
            s_{1,2} = \frac{1}{2}g(t_0)^2 + t(gt_0 - u)
        \end{equation}

        Thus, before reaching ground, \(s-t\) graph is linear. However, after first body reaches ground,

        \begin{equation}
            s_{1,2} = ut + \frac{1}{2}g(t)^2
        \end{equation}
        which is a parabola. Thus, overall graph is,
        \begin{figure}[H]
            \centering
            \begin{tikzpicture}
                \begin{axis}[funcgraphbare, xlabel={$t$}, ylabel={$s$}]
                    \addplot[red, domain=0:2] {x+2};
                    \addplot[red,domain=2:3] {1.5*(-x^2)+5*x};
                \end{axis}
            \end{tikzpicture}
        \end{figure}
    \end{soln}
\end{example}

\subsection{Motion in higher dimensions}
\label{sec: kin3d}

We can mostly safely extend our results from one dimension to higher dimensions.
So for instances, velocity of a particle whose trajectory is defined as 
\(x\ihat + y\jhat + z\khat\) can simply be computed as,
\[v = \ndiv{x}\ihat + \ndiv{y}\jhat + \ndiv{z}\khat\] Such a thing
is not true for non-cartesian co-ordinates, as we say in the case of polar forms.
Due to the orthogonality of the cartesian unit vectors, and using dot products
we simply have, 
\[v_x = \ndiv{x}, v_y = \ndiv{y}, v_z = \ndiv{z}\]


And the same is true for all other vectors. This gives us the power to
resolve our vector components into components alongside unit vectors
and consider independent one dimensional motion for each of them.

\subsection{Free Fall}

When an object is dropped downwards freely, it undergoes \emph{free fall}. We ignore
air resistance, until we're told not to and in general also talk about cases of the
object being thrown instead of dropped. An object dropped from any height undergoes
accelerations \(\vec{a}\) where \(\norm{\vec{a}} = g\). The direction and sign of acceleration
is entirely dependent on us.

In general, it is a good idea to setup our axes such that the direction above ground is positive
and towards the ground is negative. The acceleration during free fall is always towards
the ground. Therefore, for these cases \(\vec{a} = -g \jhat\). 

Some important points of note are that ``\(g\)'' does not become positive or negative or whatever,
this is a common misconception. Instead, we just decide where exactly the positive direction is
and setup \(g\) according to that.

\begin{example}
    Two balls are thrown in intervals of \(2\) seconds with velocity \(u\) each. Find the
    time when they collide.
    \begin{soln}
        Note that when they collide, their displacement from ground must be the same! Setting 
        the origin at ground, and direction of away from ground as positive, we have,
        
        For the first ball, \(h_{1} = ut - \dfrac{1}{2}gt^2\). Since the second ball
        is thrown \(2\) seconds later, \(h_{2} = u(t-2) - \dfrac{1}{2}g(t-2)^2\). Now, we just have
        \(h_{1} = h_{2}\) and the calculation is trivial.
    \end{soln}
\end{example}

The thing to note in this example is that you do not need to setup two cases, one in which
\(g\) is negative and one in which \(g\) is positive since it is in the direction of velocity. 
Just note that these equations are vectorial and we are done.


\subsection{Projectile Motion}

Projectile motion is one of the most important motion to consider in 
2 dimensions. It is the motion of a projected body. Generally, we
only care about the motion of a body projected from ground under the influence
of gravity. 

\begin{marginfigure}
    \centering
    \scalebox{2}{\incfig{projectile}}
    \caption{Trajectory of a projectile}
    \label{fig: projectile}
\end{marginfigure}

Let a body be projected at an angle \(\theta\) from the ground at velocity \(\vec{v}\)
along \(\theta\). Then, by resolving \(\vv\) into \(v_x \ihat + v_y\jhat\),
we have, 
\begin{align*}
    x &= v_xt \ihat \\
    y &= v_yt - \frac{1}{2}gt^2 \jhat
\end{align*}

Doing some manipulations, we may note that \(v_x = v\sin(\theta)\), \(v_y = v\cos(\theta)\).
Also, substituting \(t = x/v\sin(\theta)\), we have,
\begin{equation}
    \boxed{y = x\tan(\theta) - \frac{gx^2}{2v^2\cos^2(\theta)}}
\end{equation}

We can also use the identity \(1/\cos^2\theta = \tan^2\theta + 1\) and get another and 
perhaps nicer form of the equations,

\begin{equation}
    \boxed{y = x\tan\theta\left(1 - \frac{x}{R}\right)}
\end{equation}

after substituting with \(R\)\footnote{calculated just a bit later.}.

The path of the projectile is parabolic as shown in \cref{fig: projectile}.

Here, \(H\) is the maximum height of the particle. Clearly,
when the projectile is at \(H\), \(v_y = 0\). We have, 
\(v_y^2 = 2gH\),
\begin{equation}
    H = \frac{u^2\sin^2(\theta)}{2g}
\end{equation}

The time period of the projectile can be simply calculated as well,
since at the final time, \(T\), \(y = v_yt - 1/2gt^2 = 0\) and,

\begin{equation}
    T = \frac{2v\sin(\theta)}{g}
\end{equation}

\(R\) is the range the particle covers horizontally in time \(T\) and is
clearly,

\begin{equation}
    R = \frac{v^2\sin(2\theta)}{g}
\end{equation}

Note that if the particle is projected at \(\theta = \pi/4\), 
the range is maximum and is \(u^2/2g\). Otherwise,
for \(0 < \theta < \pi/2\), \(\sin(2\theta)\) attains the same 
value for two \(\theta_1, \theta_2\). These \(\theta_1\), \(\theta_2\) are complementary
by trigonometry.

More generally, if we consider the projectile to have an initial position of 
\(\qty(x_0, y_0)\), then, by substituting for time, we find that the 
vertex of the parabola is \(\qty(y_0 + \frac{u^2\sin^2\theta}{2g}, x_0 + \frac{u^2\sin2\theta}{g})\).

The range is a little more difficult to computer here, as the particle will transverse to 
the ground which is lower than \(y_0\). Freely setting up our co-ordinates 
as \(\qty(h, 0)\), we get by the equation of trajectory,

\[
    0 = h + R\tan(\theta) - \frac{gx^2}{2v^2\cos^2(\theta)}
\]

Derivating this with respect to theta to find the maxima, and then using the attained
value, we have, 

\begin{equation}
    R = \frac{u}{g}\sqrt{u^2 + 2gh}
\end{equation}

\begin{marginfigure}
    \scalebox{2}{\incfig{projectilewedge}}
    \caption{Projectile grazing a wedge.}
\end{marginfigure}

\begin{example}
    A projectile starting from one vertex of a wedge, grazes the other 
    vertex and ends upon the other vertex, with an initial velocity 
    \(v_0\) making an angle \(\theta\) with the horizontal. If the
    angles of the vertices are \(\alpha\) and \(\beta\), figure 
    out a relation between \(\tan \theta\), \(\tan\alpha\) and \(\tan\beta\).

    \begin{soln}
        Divide the range of the projectile into two lengths, \(x\) and 
        \(x'\) such that \(x + x' = R\). Now, clearly, 
        \begin{equation*}
            \tan\alpha = \frac{y_0}{x} \quad \tan\beta = \frac{y_0}{x'}
        \end{equation*}

        Also, by the alternate equation of trajectory,

        \begin{equation*}
            y_0 = x\tan\theta\left(1 - \frac{x}{R}\right) = x\tan\theta\left(1 - \frac{x}{x + x'}\right) 
        \end{equation*}

        Which gives us, 

        \begin{equation*}
            \tan\theta = y_0 \left(\frac{x + x'}{xx'}\right) = \frac{y_0}{x} + \frac{y_0}{x'}
        \end{equation*}

        Finally substituting the values, we get, 

        \begin{equation}
            \tan\theta = \tan\alpha + \tan\beta
        \end{equation}
        
        Pretty nice and useful result.

    \end{soln}
\end{example}

Some peculiar things about projectile motion is that, \begin{inparaenum}[a)]
    \ii it is reversible, and
    \ii the time to attain some vertical displacement can be found in terms of 
    the maximum height alone.
\end{inparaenum}

Consider the vertical position of the particle to be 
\(s\) at some time \(t\). Then, 
\[s = u_yt - \frac{1}{2}gt^2\] 
Also, note that if the maximum height is \(h\), 
\[u_y = \pm \sqrt{2gh}\]

Where the positive value is during the ascent and the negative is during the descent. Then,

\[s = \pm\sqrt{2gh}t - \frac{1}{2}gt^2\]

Solving for \(t\) gives us,
\[t = \pm \sqrt{\frac{2h}{g}} \pm \sqrt{\frac{2(h-s)}{g}}\]

If the initial velocity is positive and \(s > 0\), 

\begin{equation}
    t = \sqrt{\frac{2h}{g}} \pm \sqrt{\frac{2(h-s)}{g}}
\end{equation}
the two values correspond to the fact that the particle will have the same vertical
displacement at two instances of time.

If the initial velocity is positive but \(s < 0\), i.e the particle 
starts at some height but then transverses below it, 

\begin{equation}
    t = \sqrt{\frac{2h}{g}} + \sqrt{\frac{2(h-s)}{g}}
\end{equation}
is the only reasonable solution.

If the initial velocity is negative, then clearly the 
particle will attain a lower position. The maximum height in this 
case is simply the initial height of the particle.

\begin{equation}
    t = - \sqrt{\frac{2h}{g}} + \sqrt{\frac{2(h-s)}{g}}
\end{equation}

Somewhat qualitatively, projectile motion is independent of its horizontal range. 
If some projectiles have same maximum height, their time of flight \emph{will be the 
same}.

\begin{marginfigure}
    \vspace{15em}
    \centering
    \scalebox{2}{\incfig{projectilevector1}}
    \caption{Scaled up \(\vec{v}_0\) and \(\vec{g}\) vectors in projectile motion.}
    \label{fig: rangewithvectors}
\end{marginfigure}

\begin{exc}
    \begin{exercise}[subtitle={Projectile with vectors.}, points = 3]
        We can integrate the kinematic equations 
        vectorially, to get 
        \begin{equation*}
            \vec{v}(t) = \vec{v}_0 + \vec{g}t,
        \end{equation*}
        
        and for the displacement,
        
        \begin{equation*}
            \vec{s}(t) = \vec{v}_0t + \frac{1}{2}\vec{g}t^2.
        \end{equation*}

        \begin{enumerate}[label=(\alph*)]
            \item Show that these two equations are, in fact true. Hint: How can we integrate a vector? Try to use its projections. Refer to \Cref{sec: kin3d} \nameref{sec: kin3d} if needed.
            \item Note that the vectors \(\vec{v_0}t\), \(\vec{g}t^2/2\) are parallel to vector \(\vec{v_0}\) and \(\vec{g}\) respectively. Use this to find the range and time of flight. Hint: You will find \cref{fig: rangewithvectors} useful. 
        \end{enumerate}
    \end{exercise}

    \begin{solution}
        We see that the velocity at any further moment is just the resultant of a scaled 
        gravitational acceleration vector and the initial velocity vector and the displacement is the resultant of 
        the scaled velocity and gravitational acceleration vectors. 
        
        Since both of them are multiplied by a positive scalar the time \(t\) or that squared, we simply
        scale both of the vectors to find the resultant.
        
        In \cref{fig: rangewithvectors}, we can see that since \(\vec{v}_0\) is tangent 
        to the curve at the initial point, so is \(\vec{v}_0t\). Now, we can calculate 
        the time taken to reach the following point since the displacement at that instant, 
        \(R\), is going to be the resultant of the two vectors as shown in \cref{fig: rangewithvectors}.
        
        Thus,

        \begin{equation*}
            \sin\theta = \frac{gt^2}{2v_0t} \implies t = \frac{2v_0\sin\theta}{g}
        \end{equation*}

        which is the desired flight time. We can also use,

        \begin{equation*}
            \cos\theta = \frac{R}{v_0t} \implies R = v_0t\cos\theta
        \end{equation*}

        by plugging in the expression for time we just got, we also get the range of the projectile. 
    \end{solution}
    
    \begin{exercise}[subtitle={Projectile Motion in tilted axes.}, points = 3]
        Consider the case of a projectile launched along a wedge, as in,
        \cref{fig: projectilealongwedge}. If we were to figure out \(d\), we could 
        use a number of ways. The equation of the projectile is,

        \begin{equation*}
            y = x\tan(\theta+\alpha) - \frac{gx^2}{2v_0^2\cos^2(\theta+\alpha)}
        \end{equation*}

        and of the wedge is,

        \begin{equation*}
            y = x\tan\theta 
        \end{equation*}

        We could simply equate them to figure out the distance \(d\), which is the range of the projectile.
        However, let us look at alternate way of doing this.

        \begin{enumerate}[label=(\alph*)]
            \item Consider \cref{fig: projectilerotatedaxes}, we have rotated the axes by the angle of inclination of the wedge, \(\theta\). Now, find the kinematic equations along the two axes.
            \item Solve these to find the time at which the projectile hits the wedge, and the distance it travels. 
        \end{enumerate}

    \end{exercise}

    \begin{solution}
        The acceleration along the \(y\) axis is simply \(g\cos\theta\) and along the \(x\) axis 
        is \(g\sin\theta\). Thus, our equations in each direction are,
            \begin{align*}
                v_x(t) = v_0\cos\alpha - g\sin\theta t\\
                v_y(t) = v_0\sin\alpha - g\cos\theta t 
            \end{align*}

        We can solve this by noting that the displacement in \(y\) direction must be \(0\) at the time 
        it hits the wedge. So we integrate and the velocity to get, 
            \begin{equation*}
                y(t) = v_0\sin\alpha t - \frac{1}{2}g\cos\theta t = 0 \implies t = \frac{2v_0\sin\alpha}{g\cos\theta} 
            \end{equation*}
        
        Plug this in the equation obtained by integrating the velocity in the \(x\) direction to get the range.
    \end{solution}
\end{exc}

\begin{marginfigure}
    \vspace{-8em}
    \centering
    \scalebox{2}{\incfig{projectilevector2}}
    \caption{Projectile along a wedge}
    \label{fig: projectilealongwedge}
\end{marginfigure}


\begin{marginfigure}
    \centering
    \scalebox{2}{\incfig{projectilevector3}}
    \caption{The cartesian axes rotated by \(\theta\).}
    \label{fig: projectilerotatedaxes}
\end{marginfigure}

and we may solve each of these to get any desired result. Due to the dependence 
on both \(\theta\) and \(\alpha\), it is a bit difficult to things like maximum range
and the loci method maybe preferred.  

\begin{marginfigure}
    \vspace{2em}
    \centering
    \scalebox{2}{\incfig{projectilealongwedgeex}}
    \caption{Projectile thrown from a wedge.}
    \label{fig: projectile wedge example}
\end{marginfigure}

\begin{example}
Consider a projectile thrown with a velocity \(v_0\) making an angle \(\alpha\) with 
a wedge of inclination \(\theta\) as shown in the \cref{fig: projectile wedge example}. Find out the angle \(\alpha\) 
such that the projectile has maximum range.    

\begin{soln}
    The locus of the projectile is, 
    \begin{equation*}
        y = x\tan\alpha - \frac{gx^2}{2v_0^2\cos^2\alpha}
    \end{equation*}

    and of the wedge is, 

    \begin{equation*}
        y = -x\tan\theta
    \end{equation*}

    At the range, they will be equal,

    \begin{equation*}
        x\tan\alpha - \frac{gx^2}{2v_0^2\cos^2\alpha} = -x\tan\theta \implies 
        \tan\alpha + \tan\theta = \frac{gx}{2v_0^2\cos^2\alpha}
    \end{equation*}

    which gives us the desired range, 

    \begin{equation*}
        x = \frac{2v_0^2(\tan\alpha + \tan\theta)\cos^2\alpha}{g}
    \end{equation*}

    So the condition for maximum range is simply, \(\dv*{x}{\alpha} = 0\). This is also 
    kind of illustrates a nice idea, since we need to get the derivative \(0\), we
    can safely discard any multiplicative or additive constants right away which will 
    either have a zero differentiation, or can be divided by since the other side of the 
    equation is simply \(0\). 
    
    We get that, 

    \begin{align*}
        &\dv{\{(\tan\alpha + \tan\theta)\cos^2\alpha\}}{\alpha} = 0 \\
        &\dv{\{\sin 2\alpha/2 + \tan\theta\cos^2\alpha\}}{\alpha} = 0 \\
        &\cos 2\alpha - \tan\theta\sin 2\alpha = 0 \\
        &\tan 2\alpha = \cot\theta \\
        &\tan 2\alpha = \tan (\pi/2 - \theta) \\
        &\alpha = \frac{\pi}{4} - \frac{\theta}{2}
    \end{align*}

    And we're done.
\end{soln}
\end{example}


\section{Motion in Polar Co-ordinates}

The polar co-ordinate axes are the two-dimesional subset of the 3-d cylindrical co-ordinates. 
Any point in polar co-ordinates is depicted by a system of two unit vectors, \(\unitv{r}\) and
\(\gunitv{\theta}\). However, each of these are {dependent on the position of the particle} and
maybe written explicitly as \(\unitv{r}(\theta)\) and \(\gunitv{\theta}(\theta)\).  

Here, \(\unitv{r}\) points in the direction of increasing radius(along the radial vector)
\(\gunitv{\theta}\) points in the direction of increasing \(\theta\) (tangent to the radial vector).
These two unit vectors are \emph{orthogonal} at any point. 

\begin{marginfigure}
    \scalebox{1.5}{\incfig{polarunit}}
    \caption{Unit vectors in polar co-ordinates}
    \label{fig: polarunit}
\end{marginfigure}

In \cref{fig: polarunit} if \(\vec{r}\) makes
an angle \(\theta\) with the horizontal, then,

\begin{align}
    \label{eq: polardef}
    \unitv{r} &= \cos \theta \ihat + \sin \theta \jhat\\
    \gunitv{\theta} &= -\sin \theta \ihat + \cos \theta \jhat
\end{align}

Now let us formulate kinematics in polar co-ordinates. The position \(\vec{r}\) can be written as

\begin{equation}
    \vec{r} = r \unitv{r}
\end{equation}

\noindent Velocity follows normally, 
\begin{equation}
    \vec{v} = \ndiv{\vec{r}} = \ndiv{r} \unitv{r} + r\dv{\unitv{r}}{t} 
\end{equation}
or does it? What even is \(\displaystyle\dv{\unitv{r}}{t}\)? Well the answer lies in 
the time-derivative of the unit vectors. Using the definitions in \eqref{eq: polardef}, we get,

\begin{align}
    \dv{\unitv{r}}{t} &= \ndiv{\theta} \gunitv{\theta}\\
    \dv{\gunitv{\theta}}{t} &= -\ndiv{\theta} \unitv{r}
\end{align}

\noindent Using these results, we get that

\begin{empheq}[box=\widefbox]{align}    
    \vec{v} & = \ndiv{r} \unitv{r} + r\ndiv{\theta} \gunitv{\theta}\\
    \label{polar: accel}
    \vec{a} & = (\nddiv{r} - r\ndiv{\theta}^2) \unitv{r} + (r\nddiv{\theta} + 2\ndiv{r}\ndiv{\theta}) \gunitv{\theta}
\end{empheq}

\noindent The most interesting result are the various terms of acceleration in \eqref{polar: accel}. First let 
us consider the terms along \(\unitv{r}\). \(\nddiv{r}\) is the \emph{radial} acceleration. It is
the change in the radial speed(and thus the radius). The second term \(-r\ndiv{\theta}^2\) is the \emph{centripetal} acceleration, this is 
term responsible for change in the direction of the tangential velocity. In essence, it is 
the component which accounts for the rotation of the particle. 

Now let us consider the terms along \(\gunitv{\theta}\). The \(r\nddiv{\theta}\) term is the tangential
acceleration resulting from change in tangential speed, i.e if the radial vector is of 
constant magnitude, and speed \(v\) is changing, \[a_t = r\ddot{\theta} = \ndiv{v}.\] The other term \(2\ndiv{r}\ndiv{\theta}\) is the
\emph{coriolis} acceleration. I shall discuss it when discussing rotating frames.

\section{Reference Frames}

There is a very neat technique to solve questions of kinematics, it is to switch our 
reference frames. Let the base reference frame be that of earth, then,

\begin{figure}
    [H]
    \centering
    \incfig{reference}
\end{figure}

If \(\vec{r}_1\) evolves with time, \(r_1(t)\). And so does \(r_2(t)\). Then,

\begin{align*}
    \vec{r}_1 &= \vec{r}_{1,2} + \vec{r}_2 \\
    \ndiv{\vec{r}}_1 &= \ndiv{\vec{r}}_{1,2} + \ndiv{\vec{r}}_2 \\
    \vec{v}_1 &= \vec{v}_{1,2} + \vec{v}_{2}\\
    \vec{v}_{1,2} &= \vec{v}_1 - \vec{v}_{2}
\end{align*}

Obviously if the vectors face the same direction in the same plane, then, the components subtract. If they face
opposite direction, then the components add. 

\(\vec{v}_{1,2}\) is called the velocity of \(\mathit{1}\) w.r.t \(\mathit{2}\). When doing
problems, try to setup a reference frame such that velocity of any single object vanishes.
This allows for a very neat simplification of problems. 

\section{Drag Forces}

In a more realistic scenario, a particle undergoing motion in any 
media experiences a \vocab{drag force}, anti-parallel to its velocity.
In particular,  

\begin{equation*}
    \vec{a} = -\alpha v \unitv{v} - \beta v^2 \unitv{v}
\end{equation*}

where the linear term arises from a property of the media, 
the viscosity(we will talk about it in the later sections.)

The quadratic term is a result of the collision of the particles with atoms 
and molecules during its motion. In general, 
the quadratic term dominates for higher velocities.

It is very difficult to analyse motion with drag, but we shall have a look at some particular 
cases of linear drag.

\sideref{\Cref{sec: newton's second law} \nameref{sec: newton's second law}}

\begin{example}
    Consider a projectile of mass \(m\) thrown upwards with initial velocity \(v_0\). It undergoes a linear drag 
    force, \(\Vec{F} =  -kv \unitv{v}\). Find the velocity of the particle as 
    a function of time. 
    \begin{soln}
        Using newton's second law, also by noting that 
        it also undergoes gravitation, \(\vec{a} = -g -kv/m \unitv{v}\).

        Now, we note that, 
        \begin{equation*}
            \dv{v}{t} = -g  - \frac{kv}{m}
        \end{equation*}

        \begin{align*}
            \frac{\dd{v}}{g + kv/m} &= - \dd{t} \\
            \int_{v_0}^{v_f} \frac{\dd{v}}{g + kv/m} &= \int_{0}^{t}\dd{t} \\
            \frac{m}{k} \eval{\log(g + kv/m)}_{v_0}^{v_f} &= t \\
        \end{align*}
        Which finally gives us,

        \begin{equation}
            v(t) = e^{-kt/m}v_0 + \frac{mg}{k}(e^{-kt/m} - 1)
        \end{equation}

    \end{soln}
\end{example}

The same idea can be extended to projectile --- \(2\)d motion 
by considering the components of the accelerations,

\[
    \vec{a} = -kv_x \ihat - (g + kv_y) \jhat,
\]

and now considering the equations \(\dot{\vec{v}} = -kv_x\)
and the same for the \(y\) co-ordinate. 

Ulitmately, this leads to the equations,

\begin{align}
    v_x &= u_x e^{-kt} \\
    v_y &= \qty(u_y + \frac{g}{k})e^{-kt} - \frac{g}{k}
\end{align}

\section{A bag of techniques}

Although almost everything can be found with a bash-y mathematical solution, sometimes 
it is just nicer to use a particularly slick idea to solve problems. Some of these are discussed 
below.

\subsection{Fermat's Principle}

Given that in a medium, the body can move relative to it with some fixed 
velocity \(v_1\). Ultimately, it has to cross to a point in another medium, in which 
it can move with velocity \(v_2\) with respect to the first one. If it has to reach 
some fixed point in the other medium, what could be its \emph{optimal} trajectory?

\begin{marginfigure}
    \centering
    \scalebox{2.3}{\incfig{fermatprinciple}}
    \caption{The optimal trajectory in the two media.}
    \label{fig: fermat's principle}
\end{marginfigure}

Let the optimal trajectory be as in \cref{fig: fermat's principle}. The time in the first media
is \[t_1 = \frac{\sqrt{x^2 + y^2}}{v_1},\] and in the other media is \[t_2 = \frac{\sqrt{(L-y)^2 + (d-x)^2}}{v_2}\].

Note that \(y\) is fixed (since it is the vertical distance between the media.), Thus, 
the way to get minimum time is, to get for \(t = t_1 + t_2\) to be such that,

\begin{equation*}
    \dv{t}{x} = 0
\end{equation*}

Now, differentiating the whole equation, we will get the final result,

\begin{equation}
    \frac{-2x}{\sqrt{(x^2+y^2)}v_1} + \frac{-2(d-x)}{\sqrt{(L-y)^2 + (d-x)^2}v_2} = 0 
\end{equation}

Note however, that, 

\begin{equation*}
    \sin\theta_1 = \frac{x}{\sqrt{(x^2+y^2)}} \quad \sin\theta_2 = \frac{(d-x)}{\sqrt{(L-y)^2 + (d-x)^2}}
\end{equation*}

Replacing both of these and factoring out \(-2\), we get that, 

\begin{equation}
    \boxed{\frac{\sin\theta_1}{v_1} = \frac{\sin\theta_2}{v_2}}
\end{equation}

This is a very neat result and allows us to greatly simplify everything. Note 
here that both the velocities, \(v_1\) and \(v_2\) are relative to any common inertial
frame. We can use some frame moving at some velocity to observe this and it wouldn't change 
a thing. 

\begin{marginfigure}
    \centering
    \scalebox{2.2}{\incfig{fermatprincipleex}}
    \caption{Shortest time to reach opposite point in a stream.}
    \label{fig: fermat example}
\end{marginfigure}

\begin{example}
    If a man can swim relative to water with a velocity \(v_0\), and can walk on the ground 
    with velocity \(3v_0\) and the velocity of the stream is \(2v_0\), find out minimum time 
    in which the man can reach a point directly opposite to him. The width of the river is \(L\).
    Look at \cref{fig: fermat example} for clarity

    \begin{soln}
        The man could walk in any direction on the ground, but since we want to minimize 
        the time, we consider him walking in the same direction as the river current to get 
        the velocity of man on ground relative to water being \(5v_0\). The velocity 
        of him in the river relative to the stream is \(v_0\). 

        From fermat's principle, the most optimal trajectory obeys 
        
        \begin{equation*}
            \frac{\sin\theta_1}{v_1} = \frac{\sin\theta_2}{v_2}
        \end{equation*}
        where the angles are taken relative to the normal. 
        
        Since \(\theta_2 = \dfrac{\pi}{2}\), \(v_1 = v_0\), and \(v_2 = 5v_0\),

        \begin{equation*}
            sin\theta_1 = \frac{1}{5}
        \end{equation*}

        After which we can easily find the time as by a simple angle chase, \(\theta_1\) is 
        the angle between the velocity of man and the vertical. From it we 
        can find the resultant velocity with water (using components) 
        and subsequently the drift and time required.
    \end{soln}
\end{example}
