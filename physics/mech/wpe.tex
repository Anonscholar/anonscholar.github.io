\chapter{Energy and Momenta}

\motiv{While we could very well stop our discussion of 
mechanics with newton's laws, it is often difficult to 
describe physical phenomenon with just them. Thus, we develop 
an equivalent way of describing motion --- using the concept of energy.}

\section{Momentum}

Newton's laws essentially hold for point particles. However, we have, and without 
justification, often used for instance the second law, on an extended system of particles ---
a \vocab{body}.

If we were to consider the motion of such a body, the only way seems to be 
to account for all interactions of each particle of the body. This, however, is a very tedious task.
Hope is not lost, however, and the problem is much simpler than what we might think!

If we apply some external force on the body, the total force is actually the sum of the external 
force, plus the internal forces between the particles themselves. Let us denote 
the force on some particle \(i\) by \(j\) as,
\[\vec{f}_{ij}\]

\sidenote{This is actually the usual convention when writing forces, \(\vec{F}_{ij}\) is the 
force on \(i\) due to (interaction between \(i\) and)\(j\).}

The total internal force on \(i\) is thus, 

\[\sum_{j \ne i} \vec{f}_{ij}\]

And the total net force on the system is,

\[\sum_{i}\sum_{j \ne i} \vec{f}_{ij}\]

The neat thing here is that, this sum sums over over all pairs \(\vec{f}_{ij}\), \(\vec{f}_{ji}\), which, by 
the third law is zero and thus, 

\begin{equation}
    \sum_{i}\sum_{j \ne i} \vec{f}_{ij} = \sum (\vec{f}_{ij} + \vec{f}_{ji}) = \oo
\end{equation}

So when considering the force on a system, we only need to care about external force! Cool, 
but considering the interaction is still a tedious task. 

The force on the system by the second law must be, 

\begin{equation*}
    \vec{F} = \vec{F}_{ext} = \sum m_i\nddiv{\vec{r}}_i = \sum \ndiv{\vec{p}}_i
\end{equation*}

Now, we can simplify the equation a little by abbreviating the sum as, 

\begin{equation}
    \vec{P} = \sum \vec{p}_i
\end{equation}

So our equation becomes, 

\begin{equation}
    \vec{F} = \ndiv{\vec{P}}
\end{equation}

This looks very similar to the case we had for a single particle. Consider the mass to 
be constant, can we push this analogy to \(F = ma\)? Well, we can try,

\begin{equation*}
    \vec{F} = \ndiv{\vec{P}} = M\nddiv{\vec{R}}
\end{equation*}

where \(M = \sum m_i\) is the total mass of the system. Then,

\begin{equation*}
    M\ddot{\vec{R}} = \sum m_i \nddiv{\vec{r_i}} \iff \vec{R} = \frac{\sum m_i \vec{r}_i}{M}
\end{equation*}

The vector \(\vec{R}\) locates the center of mass of the system, the point about which, 
the \emph{weighted}\footnote{Weighted using the mass at that point} position vectors 
of the points of the body sum to zero. This is easy to verify. In essence, we can reduce the 
whole system to a particle, of the mass of the whole body, located at its center of mass.

\begin{definition}
    The \vocab{center of mass} of the system is located at the position \(\vec{R}\) from origin, where,
    \begin{equation}
        \vec{R} \equiv \frac{\sum m_i \vec{r}_i}{\sum m_i}
    \end{equation}
\end{definition}


If the mass distribution is continuous, them,

\begin{equation}
    \vec{R} = \frac{\int \vec{r} \dd{m}}{\int \dd{m}}
\end{equation}

Let us calculate the center of mass of a few bodies, 

\begin{example}
    Calculate the center of mass of a rod of length \(L\) and mass \(M\), with uniform mass density, \(\lambda\).

    \begin{soln}
        Setup the origin at one end of the rod. Since the mass density is uniform, 
        \begin{equation*}
            \lambda = \frac{M}{L}
        \end{equation*}

        Now, note that \(\dd{m} = \lambda \dd{x}\), if \(x\) is distance from the origin. Thus,

        \begin{equation*}
            \int_{0}^{L} \vec{r} \dd{m} = \int_{0}^{L} x\unitv{x}\;\lambda \dd{x}
        \end{equation*}

        Using the value of \(\lambda\), the integral becomes

        \begin{equation*}
            \frac{M}{L} \unitv{x} \int_{0}^{L} x \dd{x} = \frac{ML}{2} \unitv{x} 
        \end{equation*}

        Thus, the com is at, 

        \begin{equation}
            \vec{R} = \frac{L}{2}
        \end{equation}

    \end{soln}
\end{example}

A better way to do this is choose a nice origin, about which the body is symmetrical. For instance, 
the rod is symmetrical about its midpoint. Now for two points equal distance apart from the mid point,
their masses are equal, but their position vectors are opposite and thus \(\sum m_i\vec{r_1}\) evaluates 
to \(\vec{0}\), and our center of mass is at origin.

Lets consider another example, this time we will show a neat idea on avoiding double 
integrals.


\begin{marginfigure}
    \centering
    \scalebox{1.5}{\incfig{triangle com}}
\end{marginfigure}

\begin{example}
    Consider a right-angled triangular plate of height \(h\), and breadth \(b\). If the mass density 
    of the disk, \(\sigma\) is uniform, find the center of mass of the disk.


    \begin{soln}
        Let us divide the plate into thin vertical strips, of breadth \(dx\). If the horizontal and vertical distancea of 
        the strip from the origin, at one of the non-perpendicular vertices, are \(x\) and \(y\), then
        \begin{equation*}
            \frac{x}{y} = \frac{b}{h}
        \end{equation*} 
        The center of mass of this strip is located at the midpoint of this disk, at \(y/2\).
        So, we can reduce each of these strips into their com and then calculate the center of mass of the plate.

        \begin{equation*}
            \int \vec{r} \dd{m} = \int x \unitv{x} + y/2 \unitv{y} \dd{m}
        \end{equation*}

        But, \(\dd{m} = \sigma \dd{A}\) and \(\dd{A} = y \dd{x}\). 

        However, we also have, \(x/y = b/h \iff y = xh/b\). Thus, 

        \begin{equation*}
            \dd{m} = \sigma \frac{xh}{b} \dd{x}
        \end{equation*}

        And our integral is,

        \begin{equation*}
            \int_{0}^{b} x \unitv{x} \sigma \frac{xh}{b} \dd{x}+ \int_{0}^b \frac{xh}{2b} \sigma \unitv{y} \frac{xh}{b} \dd{x}
        \end{equation*}

        Also, \(\sigma = 2M/bh\), substituting and evaluating the integral, we have,

        \begin{equation}
            \vec{R} = \frac{2b}{3} \unitv{x} + \frac{h}{3} \unitv{y}
        \end{equation}

    \end{soln}
\end{example}
