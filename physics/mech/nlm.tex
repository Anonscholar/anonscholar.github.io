\chapter{Netwon's Laws}

In this chapter we shall look at newton's laws of motion. These form the basis of \vocab{dynamics} which is the 
study of a motion's cause. In general, we solve mechanical problems by first considering its 
dynamics and then reduce it to a purely kinematical problem.

First let us concern ourselves with the concept of {linear momentum}.

\section{Linear Momentum}

\begin{definition}
    The \vocab{Linear Momentum}, \(\vec{p}\) of a particle of mass \(m\) moving 
    with velocity \(\vv\) is defined as,
    \[\vec{p} = m\vv\] 
\end{definition}

The total momentum of a system of particles is simply the sum of momentum of each
particle. Momentum is a useful quantity because it is conserved,
we'll look at that latter. But for now, it has certain unique properties 
that make the concept of momentum extremely important.

\section{Newton's Three Laws}

\subsection{Newton's First law}

\begin{axioms}
    \ii Objects tend to continue in a state of constant
    velocity unless acted upon by a net external force.
\end{axioms}

The first law actually is much better stated as the 
assertion that \vocab{inertial frames} exist.

It defines inertial frames as such reference frames where 
objects move at constant velocity without the action of an external force.
It also makes an empirical observation that such frames exist. 
Thus, its part definition and part experimental fact.

In fact an inertial frame may also be stated as a frame which has no acceleration.
Here, the fact that we can say that it has no acceleration refers
to the newtonian concept that acceleration is absolute. That in any frame 
of reference, we may know that a body is accelerating, even when velocity is 
relative and an object may be at rest in certain frames and moving in others.

We may measure the acceleration in any reference frame by placing 
an accelerometer and the acceleration must come out the same no matter
what frame we choose.

Since the inertial frames are non-accelerating, they travel with some constant velocity 
relative to each other. Thus, a frame that travels with constant velocity 
to an inertial frame is also a reference frame.

Another way to talk about inertial frames is that they are the frames in which newton's laws
hold. Certainly, if our frame is non-inertial, none of the laws will hold in this frame.

\section{Newton's Second Law}

I'll recommend you read Kleppner for a very nice discussion of this.

\begin{axioms}
    \ii The net external force on a body is proportional to its change in 
    momentum.
\end{axioms}

Thus, our claim is that
\[\sum \vec{F} = \dv{(m\vv)}{t} = m\vec{a},\]

if the mass is constant. 

Let us take an interlude here and talk about what mass and force really are.

\subsubsection{Mass}

Assume that there are two bodies, of masses \(m_1\) and \(m_2\).
If they say, undergo an ``equal amount'' of physical interaction,
for instance two blocks are accelerated with the same amount 
of ``push'', and are accelerated with accelerations \(a_1\) and \(a_2\),
then,

\[m_2 \equiv m_1\frac{a_1}{a_2}.\]

We define the mass of the first body relative to the other. The accelerations
can be absolutely measured using an accelerometer. In such a manner we define 
the acceleration of any \(i\)th body relative to the first one as, 

\[m_i \equiv m_1\frac{a_1}{a_i}\]

We define some standard mass as \(1\) unit and then define the other masses 
relative to this. Such a definition is an \vocab{operational definition}.

It is not always useful, and is not theoretical but it does give us a way to measure the mass
of any body. 

The talk about equal physical interaction is vague and purely experimental concept.
It is entirely physical and has no theoretical basis on which we can rely. It
may seem a little wooly, but it is a truth.
