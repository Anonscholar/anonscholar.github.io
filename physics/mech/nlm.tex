\chapter{Netwon's Laws}

In this chapter we shall look at newton's laws of motion. These form the basis of \vocab{dynamics} which is the 
study of a motion's cause. In general, we solve mechanical problems by first considering its 
dynamics and then reduce it to a purely kinematical problem.

First let us concern ourselves with the concept of {linear momentum}.

\section{Linear Momentum}

\begin{definition}
    The \vocab{Linear Momentum}, \(\vec{p}\) of a particle of mass \(m\) moving 
    with velocity \(\vv\) is defined as,
    \[\vec{p} = m\vv\] 
\end{definition}

The total momentum of a system of particles is simply the sum of momentum of each
particle. Momentum is a useful quantity because it is conserved,
we'll look at that latter. But for now, it has certain unique properties 
that make the concept of momentum extremely important.

\section{Newton's Three Laws}

\subsection{Newton's First law}

\begin{axioms}
    \ii Objects tend to continue in a state of constant
    velocity unless acted upon by a net external force.
\end{axioms}

The first law actually is much better stated as the 
assertion that \vocab{inertial frames} exist.

It defines inertial frames as such reference frames where 
objects move at constant velocity without the action of an external force.
It also makes an empirical observation that such frames exist. 
Thus, its part definition and part experimental fact.

In fact an inertial frame may also be stated as a frame which has no acceleration.
Here, the fact that we can say that it has no acceleration refers
to the newtonian concept that acceleration is absolute. That in any frame 
of reference, we may know that a body is accelerating, even when velocity is 
relative and an object may be at rest in certain frames and moving in others.

We may measure the acceleration in any reference frame by placing 
an accelerometer and the acceleration must come out the same no matter
what frame we choose.

Since the inertial frames are non-accelerating, they travel with some constant velocity 
relative to each other. Thus, a frame that travels with constant velocity 
to an inertial frame is also a reference frame.

Another way to talk about inertial frames is that they are the frames in which newton's laws
hold. Certainly, if our frame is non-inertial, none of the laws will hold in this frame.

\subsection{Newton's Second Law}
\label{sec: newton's second law}
A very nice reference which discusses interaction in detail is \cite{kleppner}

\begin{axioms}
    \ii[\textbf{Axiom 2.}] The net external force on a body is proportional to its change in 
    momentum.
\end{axioms}

Thus, our claim is that
\[\sum \Vec{F} = \dv{(m\vv)}{t} = m\vec{a},\]

if the mass is constant. 

Let us take an interlude here and talk about what mass and force really are.

\subsubsection{Mass}

Assume that there are two bodies, of masses \(m_1\) and \(m_2\).
If they say, undergo an ``equal amount'' of physical interaction,
for instance two blocks are accelerated with the same amount 
of ``push'', and are accelerated with accelerations \(a_1\) and \(a_2\),
then,

\[m_2 \equiv m_1\frac{a_1}{a_2}.\]

We define the mass of the first body relative to the other. The accelerations
can be absolutely measured using an accelerometer. In such a manner we define 
the acceleration of any \(i\)th body relative to the first one as, 

\[m_i \equiv m_1\frac{a_1}{a_i}\]

We define some standard mass as \(1\) unit and then define the other masses 
relative to this. Such a definition is an \vocab{operational definition}.

It is not always useful, and is not theoretical but it does give us a way to measure the mass
of any body. 

The talk about equal physical interaction is vague and purely experimental concept.
It is entirely physical and has no theoretical basis on which we can rely. It
may seem a little wooly, but it works.

\subsubsection{Force}

We say a \vocab{force} is produced by a physical interaction. We dont 
talk about what forces ``are''. Merely, how they come to be. Whenever 
we interact physically with an object, we apply a force on it.  

It might be tempting to ask in case of forces where such an interaction 
is not visible, say gravity, does this really hold? In fact, it does
the interaction is mediated through the use of fields. We'll have 
a look at them later.

\subsection{Newton's Third Law}

The final law describes how forces come in pairs and result of a physical interaction.

\begin{axioms}
    \ii[\textbf{Axiom 3.}] If a body \(A\) applies a force on a body \(B\), then the body 
    \(B\) applies an equal and opposite force on \(A\). These forces are 
    called \vocab{action-reaction pairs}.
\end{axioms}

That is to say, 

\[\Vec{F}_{AB} = - \Vec{F}_{BA}.\]

The third law is also known as the weak law of action and reaction ---
in direct contrast with the strong law which requires the equal and opposite
forces to lie on the same line of action. Most forces such as gravity and the
normal force in fact obey the strong law of action and reaction.

\section{Force on a system of particles}

Consider a system of \(N\) particles. For an \(i\)th particle, by the second law,

\begin{equation*}
    \qty(\sum \Vec{F})_i + \sum_{j, j \ne i} \vec{f}_{ij} = \ndiv{\vec{p}_i}
\end{equation*}

where \(\Vec{F}\) is the external force and \(\vec{f}_{ij}\) is the internal force
on \(i\) due to \(j\). Summing this for all \(N\) particles, 

\begin{equation*}
    \sum\qty(\sum \Vec{F})_i + \sum_{i}\sum_{j, j \ne i} \vec{f}_{ij} = \sum_i\ndiv{\vec{p}_i}
\end{equation*}

Note that the first term is simply the net external force, \(\sum\Vec{F}_{ext}\). The second 
term actually goes to zero since if we split the sum into action-reaction pairs,
\(\vec{f}_{ij} + \vec{f}_{ji}\), they cancel out by the third law. 

So we simply have, 

\begin{equation*}
    \sum \Vec{F}_{ext} = \sum_{i=1}^N\ndiv{\vec{p}_i}
\end{equation*}

It's very difficult to compute the sum of momentums for large values of \(N\). Thus
we define something called the \emph{centre of mass}.

\begin{definition}
    \label{def: com}
    The position vector, \(\Vec{R}\), of the \vocab{Centre of Mass} of a (system)body of \(N\) particles is,
    \begin{equation}
        \Vec{R} = \frac{\sum_{i=1}^N m_i\vec{r}_i}{\sum_{i=1}^N m_i}
    \end{equation} 
\end{definition}

If the body has a mass \(M\), 

\[M = \sum_{i=1}^N m_i,\]

Then, for infintisemal \(\dd{\vec{r}}\), 

\begin{equation*}
    M\dd{\vec{R}} = \sum_{i=1}^N m_i\dd{\vec{r_i}}
\end{equation*}

Note that, 

\begin{equation*}
    \sum_{i=1}^N \vec{p}_i = \sum_{i=1}^N m_i \dv{\vec{r}_i}{t}
\end{equation*}

Thus,

\begin{equation}
    \boxed{\sum_{i=1}^N \vec{p}_i = M\ndiv{\Vec{R}}}
\end{equation}

Where let \(\vec{R}\) is the position vector of the centre of mass(CM). Thus, 

\begin{equation*}
    {\sum_{i=1}^N \vec{p}_i = M\vec{v}_{CM}}
\end{equation*}

And, 

\begin{equation*}
    {\sum_{i=1}^N \ndiv{\vec{p}}_i = M\vec{a}_{CM}}
\end{equation*}

Which finally gives us,

\begin{equation}
    \boxed{\sum \Vec{F}_{ext} = M\vec{a}_{CM}.}
\end{equation}

This means that a system of particles evolves translationally due to a net external force as if
it were a point particle of mass \(M\) located at the center of mass of the system, with 
a position vector \(\Vec{R}\).

This means that we can simply analyze the translational motion of the center of mass which 
defines the translational motion of the entire rigid body, as the 
position vector of the particles are fixed relative to each other.

For continous mass distributions, 

\begin{equation}
    \Vec{R} = \frac{\int\vec{r}\dd{m}}{\int\dd{m}}
\end{equation}

\subsection{Calculating the Centre of Mass}

A good reference here is \cite{wangricardo}.

Let us first calculate the centre of mass of two particles of masses 
\(m\) and \(M\) located at \(\qty(x_1, y_1, z_1)\) and \(\qty(x_2, y_2, z_2)\). 
By \cref{def: com}, 

\begin{equation*}
    \Vec{R} = \frac{1}{m+M}\qty[\begin{pmatrix} mx_1 \\ my_1 \\ mz_1\end{pmatrix} + \begin{pmatrix} Mx_2 \\ My_2 \\ Mz_2\end{pmatrix}]
    = \frac{1}{m + M}\begin{pmatrix} mx_1 + Mx_2 \\ my_1 + My_2 \\ mz_1 + Mz_2\end{pmatrix}
\end{equation*}

One special thing we may note is that instead of calculating the centre of mass 
for each particle, we reduce some say \(n\) particles or distributions into 
point particles by first calculating their centre of mass. For instance,
instead of calculating the com of three them at once, we 
can calculate the com of the first two, and then that of the third.

This works because suppose the com of any \(n\) particles or distributions is located at 
\(\Vec{R}\). Then, if we calculate the com of this with another say \(n+1\)th particle, 

\begin{equation*}
    \Vec{R'} = \frac{M\Vec{R} + m_{n+1}\vec{r}_{n+1}}{M + m_{n+1}},
\end{equation*}

where \(M\) is the total mass of the \(n\) particles. By the definition of com, 

\begin{equation*}
    M\Vec{R} = \sum^N_{i=1} m_i\vec{r}_i
\end{equation*}

Replacing this, 

\begin{equation*}
    \Vec{R'} = \frac{\sum^N_{i=1} m_i\vec{r}_i + m_{n+1}\vec{r}_{n+1}}{\sum_{i=1}^N m_i + m_{n+1}},
\end{equation*}

Which is simply the com of the \(n+1\) particles, at once calculated!

\begin{equation*}
    \Vec{R'} = \frac{\sum^{N+1}_{i=1} m_i\vec{r}_i}{\sum_{i=1}^{N+1}}
\end{equation*}

The same can be extended to mass distributions.

For example, to calculate the com of two spheres of uniform mass distribution, 
we may calculate their com, which is at their centre (by a symmetry argument), and then just 
use our previous result of com of two point masses.

This can also be used to find the com for ``missing masses''. Where we replace 
our incomplete system with a complete system having the total mass of the system 
if it were complete, and a negative mass. This might be a little difficult 
to understand, so let us look at an example:


\begin{marginfigure}
    \centering
    \scalebox{2}{\incfig{imaginarycom}}
    \caption{Missing mass}\label{fig: missingmasscom}
\end{marginfigure}

\begin{example}
    A uniform circle of radius \(R\) has a circular hole of radius \(R/2\). The surface 
    density of the circle is \(\sigma\). Find the centre of mass of the system.
    \begin{soln}
        To calculate the com, we first ``fill up'' or replace the incomplete the circle with 
        a complete circle. This can then be reduced to a point mass of mass \(M = \sigma\pi r^2\).

        To counter effect this addition of mass by filling up the circle, we associate 
        a negative mass with the smaller circle, \(-m = -\frac{\sigma\pi r^2}{4}\).

        Let the origin the com of the first circle, then the first circle, which we have 
        reduced to be a particle at its com, has a position vector, \(\vec{0}\). 
        Thus, com of the system is,

        \begin{equation*}
            \Vec{R} = \frac{M \times 0 -mR/2}{M - m} = \frac{-mR/2}{M-m} = \frac{R}{6}
        \end{equation*}
    \end{soln}
\end{example}   

For continuous distributions, we may use integration to calculate the centre of mass,

\begin{example}
    Determine the center of mass of a right-angled triangle of a uni-
    form surface mass density \(\sigma\) and with an angle of inclination θ in the \(xy\)
    plane. Let the length of its base be \(l\).

    \begin{soln}
        Note that \(\sigma = \dv*{m}{A} \iff \dd{m} = \sigma\dd{A} = \sigma\dd{x}\dd{y}\).

        So, we can setup a double integral, 

        \begin{equation*}
            \Vec{R} = \frac{1}{M} \int_{0}^{l} \int_{0}^{x\tan\theta} x \unitv{i} + y\unitv{j} \sigma \dd{y} \dd{x}
        \end{equation*}

        Splitting up the integrals, 

        \begin{align*}
            R_x &= \frac{1}{M} \int_{0}^{l} \int_{0}^{x\tan\theta} x \sigma \dd{y} \dd{x} \\
            R_x &= \frac{1}{M} \int_{0}^{l} \sigma \eval{xy}_{0}^{x\tan\theta} \dd{x} \\
            R_x &= \frac{1}{M} \int_{0}^{l} x^2 \tan\theta \sigma \dd{x} \\
            R_x &= \frac{1}{M} \eval*{\frac{x^3\sigma\tan\theta}{3}}_{0}^{l} \\
            R_x &= \frac{2}{l^2 \sigma\tan\theta} \frac{l^3\sigma\tan\theta}{3} \\
            R_x &= \frac{2l}{3}
        \end{align*}

        And by a similar procedure, \(R_y = l\tan\theta/3\). Which gives you \(\Vec{R}\).

    \end{soln}
\end{example}

\section{Some Phenomenological Forces}

\subsection{The Normal Force}

The contact force on a body can be resolved to two mutually perpendicular forces, the \vocab{normal force},
perpendicular to the object, and \vocab{friction}, tangential to the object. 

These contact forces arise due to atomic interactions. If we say keep a book on a table,
the book pushes against the table. On a microscopic scale, the molecules push against each other.
After a certain point, the molecules start repulsing each other and the table applies a normal force,
perpendicular to the body, to stop it. 

Since no object is perfectly rigid, there always occurs a deformation on its surface. However, 
this is usually negligible, and we may safely assume that the object is perfectly rigid. 

\subsection{Friction}

Friction much like the normal force, arises from molecular interactions.
It is, however, way too complex to analyse. Thus, our study of friction is 
pure phenomenological. Here, by friction we refer to dry friction, and not as is, drag. 

Friction opposes the relative motion of objects --- and this is in fact the most we 
can say safely, and generally. It is independent of the contact area.

This might feel a little weird, but consider the following argument --- say we have an object 
who has contact area with another object. This contact area is actually a very small fraction 
of its overall surface area. It is, also, proportional to the pressure.

If we double the surface area of the object, let us assume that the contact area doubles as well.
If we keep the normal force constant, the pressure halves. Friction, however, the 
product of the now half pressure, and the double contact area, remains the same.

This is a very ideal approximation of friction, but we'll leave it here for now. 

We differentiate between \vocab{static friction} and \vocab{kinetic friction}. When we apply 
a force on an object, and yet it does not undergo motion, the static friction assumes a value 
to balance out the force. This friction is actually related to the normal by \(\abs{f_s} \le \mu_sN\).
Where \(\mu_s\) is the coefficient of static friction. 

However, it can not keep the object at rest forever, after a certain amount of force is applied, 
the object undergoes motion and experiences kinetic friction. It is approximately constant 
and related to the normal by \(\abs{f_k} = \mu_kN\), where \(\mu_k\) is the coefficient of kinetic friction.

Generally, \(\mu_k < \mu_s\) and they're both less than \(1\). Though for some exceptionally rough surfaces, they
may have a value greater than \(1\).

\subsection{Spring Force}

The relation between the \vocab{spring force} and the displacement of the spring
from its equilibrium position \(x_0\) was empirically found by Robert Hooke and is popularly 
known as \vocab{Hooke's Law}, which states that,

\begin{equation}
    F_s = -k(x - x_0)
\end{equation}

where \(x\) is the displacement from its equilibrium position, \(x_0\). \(k\) is the
spring constant and depends on the spring used. 

Hooke's law actually fails for rather large displacements but is remarkably good for 
small displacements. It works not only here, but in a lot more places where the restoring forces 
are concerned.

\subsection{Tension}
