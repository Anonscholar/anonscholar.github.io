\chapter{Basics of Modular Arithmetic}
\label{nt: modbasic}
\motiv{Modular arithmetic allows us to expand the notion of equality.}

\section{``Modulo''}

It is very hard to and rarely needed to establish equality in number theory. What may do instead
is the build on the notion of \emph{modular} arithmetic. Consider an analog clock, if it starts
at any arbritary time, then no matter how many rotations it goes through, whenever it points at 
\(1\), the time is \(1\) a.m or p.m. Thus even if we start counting hours, we may note that
\(13 = 1 = 25 = \dots\)                                                                                                                                                                                       
\parbreak

The equality symbol is not apt here. What should be much better used is the symbol of equivalence,
``\(\equiv\)''. Thus, what we mean is \(13 \equiv 1 \pmod{n}\). In general,

\begin{definition}
    [Congruent]
    \(a \equiv b \pmod{n}\) if and only if \(p \mid a - b\). In such a case \(a\) is said to be 
    congruent to \(b\) modulo \(n\).
\end{definition}
