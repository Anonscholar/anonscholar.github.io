\chapter{Basics of Modular Arithmetic}
\label{nt: modbasic}
\motiv{Modular arithmetic allows us to expand the notion of equality.}

\section{``Modulo''}

It is very hard to and rarely needed to establish equality in number theory. What may do instead
is the build on the notion of \emph{modular} arithmetic. Consider an analog clock, if it starts
at any arbritary time, then no matter how many rotations it goes through, whenever it points at 
\(1\), the time is \(1\) a.m or p.m. Thus even if we start counting hours, we may note that
\(13 = 1 = 25 = \dots\)                                                                                                                                                                                       
\parbreak

The equality symbol is not apt here. What should be much better used is the symbol of equivalence,
``\(\equiv\)''. Thus, what we mean is \(13 \equiv 1 \pmod{n}\). In general,

\begin{definition}
    [Congruent]
    \(a \equiv b \pmod{n}\) if and only if \(p \mid a - b\). In such a case \(a\) is said to be 
    congruent to \(b\) modulo \(n\).
\end{definition}

A neat little property of this is that \(a + n \equiv a \pmod{n}\). Also, we may note that
the values of \(b\) such that \(b \equiv a \pmod{n}\) form an A.P.

\begin{proof}
    \(b \equiv a \pmod{n} \iff n \mid b - a \iff b - a = nk\) where \(k\) is an integer. Thus,
    \(b = a + nk\), which is an A.P.
\end{proof}

\section{Remainder and Modulo}

Consider the set of values for which \(b \equiv 3 \pmod{5}\). They form the set, \(\left\{\dots
-7, -2, 3, 8, \dots \right\}\). It is difficult to work with such infinite sets which expand
in both directions, so let us concern ourselves only with non-negative elements, \(\left\{
3, 8, 13, 18 \dots \right\}\). 

Through this example, we may see a neat pattern,

\begin{lemma}
    The lowest non-negative value of \(a\) such that \(a \equiv b \pmod{n}\) is the remainder \(r\)
    such that \(b = nk + r\), \(0 \le r < n\). 
\end{lemma}

\section{Residue class}

Consider a number \(3\). If we concern ourselves only with the remainder such that \(a \equiv r \pmod{n}\),
you may note that \(r\) can only have three values here, \(0\), \(1\), and \(2\). Thus, either
\(a \equiv 0\), \(a \equiv 1\) or \(a \equiv 2 \pmod{n}\). This is always true. These \(3\) sets
that \(a\) can belong to, are called the \(3\) residue classes modulo \(3\).

In general,

\begin{definition}
    For any natural number \(n\), \(r\) such that \(0 < r < n\), the \(r^{th}\) residue class is
    the set of all integers \(a\) such that \(a \equiv r \pmod{n}\). This set is,
    \[
        \left\{\dots, r - 2n, r - n, r, r + n, r + 2n, \dots\right\}
    \]
\end{definition}

There are exactly \(n\) residue classes modulo \(n\) since \(r\) can take \(n\) values, from \(0\)
to \(n - 1\).

\section{Arithmetic over modulo}

Consider \(a + b \pmod{n}\) for \(a \equiv r \pmod{n}\) and \(b = m \pmod{n}\). Does \(a + b \equiv r + m
\pmod{n}\)? Yes. The idea behind the proof is to just have \(a = nk + r\) and \(b = nd + m\). 

\begin{proposition}
    From the definition and the previous discussion of modular arithmetic, we have,
    \begin{enumerate}
        \ii For \(a \equiv r\), \(b \equiv m \pmod{n}\), \(a + b \equiv r + m \pmod{n}\)
        \ii \(a \equiv r\), \(b \equiv m \pmod{n} \implies\) \(ab \equiv rm \pmod{n}\)
        \ii For \(a \equiv b \pmod{n}\), \(a + c \equiv b + c \pmod{n}\)
        \ii For \(a \equiv b \pmod{n}\), \(ac \equiv bc \pmod{n}\)
        \ii For \(a \equiv b \pmod{n}\), \(a^c \equiv b^c \pmod{n}\)
    \end{enumerate}
\end{proposition}