\chapter{Divisibility}

\motiv{Divisibility forms the basis of all of Number Theory.}

\section{Basic Properties}



\para{The idea of divisibility is something most people are familiar with. Yet, it is a notion so useful, that we may
    yet again acquaint ourselves with it and its properties.
    This time, however, we are going to approach it with formality and use its property in a general, and abstract manner.
    For this, let us first define division for the integers.}

\index{division}

\marginref{If you are unfamiliar with these symbols, have a look at \Cref{ch: term}}

\begin{definition}
    [Division]
    The operation division is defined as:
    \begin{enumerate}
        \item An integer \(y\) is said to be a multiple of \(x\) if \[y\in \{kx : k\in \ZZ \} \]
        \item If \(y\) is a multiple of \(x\) then, \(y\) is \emph{divisible} by 
        \(x\) and \(x\) divides \(y\) which is written as: \[x \mid y.\]
    \end{enumerate}
\end{definition}


\para{Immediately, we can put forth some \emph{propositions} from these definitions. They may not be obvious from first glance
but they are indeed very easy to get an intuitive grasp on after we have seen them.}

\begin{plainprop*}
    For \(x\), \(y\), \(z\) \(\in\) \(\ZZ\)
    \begin{enumerate}
        \item \(\forall x \ne 0\), \(x \mid x\) and \(x \mid 0\).
        \item \(1 \mid x\).
        \item \(x\mid y\) and \(y \mid z \implies \) \(x \mid z\).
        \item \(x \mid y \iff \dfrac{y}{x} \in \ZZ\)
    \end{enumerate}        
\end{plainprop*}

\para{Most of the proofs here are trivial, but let us show that
using the last proposition we can prove the third one.}

\begin{proof}
    \( x \mid y \iff \dfrac{y}{x} \in \ZZ \) and \(  y \mid z \iff \dfrac{z}{y} \in \ZZ \). 
    But \( \dfrac{z}{x}= \dfrac{y}{x}\cdot\dfrac{z}{y} \in \ZZ \iff x\mid z\).
\end{proof}


\para{Let \(x \mid y\). Since \(y\) is a multiple of \(x\), it might seem to be the case that
\(y \ge x\). But is that really true? Well, not really. However, there is something along those lines
that is.}

\begin{theorem}
    If \(x \mid y\), then either \(\abs{y} \ge \abs{x} \) or \(y=0\).
\end{theorem}

\marginnote{The \(0\) case is extremely important and often hard to see. 
Remember to look out for this case.}

\begin{proof}
    Since \(x \mid y\), by the definition of division, we must have, \[
        y = kx : k \in \ZZ
    \] Since the case of \(0\) has already been established, let us concern ourselves with
    \(k \in \ZZ \setminus 0\), Thus,\\
    \(\abs{y}=\abs{k\cdot x}= \abs{k}\cdot \abs{x}\). But \( \abs{k}\cdot \abs{x} \ge \abs{x} \iff \abs{y} \ge \abs{x}\).
\end{proof}

\para{The following is not of a severe importance in the view that it is hard to see or prove. 
But it allows us to quickly formalise our solutions and is thus useful for proofs.}

\begin{lemma}
    \label{lem: unimp}
    If \(x \mid a\), then \(x \mid ac+b \iff x\mid b\).
\end{lemma}

\begin{proof}
    The proof is trivial as, \(x \mid a \iff a = kx : k \in \ZZ\).
    \par 
    \textbf{Direction 1}
    Note, \(x \mid ac+b \iff ac+b=nx : n \in \ZZ\), hence, \(b=nx-ac=nx-kcx=(n-kc)x \iff x \mid b\).
    \par
    \textbf{Direction 2}
    And, \(x\mid b \iff b=mx : m \in \ZZ \), hence, \(ac+b=kcx+mx=(kc+m)x \iff x \mid ac+b \).
\end{proof}

\para{This proof kind of sheds light on how to prove statement of the type \(P \iff Q\), or in words
\(P\) holds if and only if \(Q\) does. We show that \(P \implies Q\) and \(Q \implies P\). You can verify
that this is the same \(P \iff Q\). Now, let us end this section with some properties of division and examples.}

\begin{theorem}
    Let \(x\), \(y\), \(z\) \(\in\) \(\ZZ\)
    \begin{enumerate}
        \item \(\forall x \ne 0\), \(x \mid x\) and \(x \mid 0\).
        \item \(1 \mid x\).
        \item \(x\mid y\) and \(y \mid z \implies \) \(x \mid z\).
        \item \(x \mid y \iff \dfrac{y}{x} \in \ZZ\).
        \item \(\forall z \ne 0\), \(x \mid y \iff xz \mid yz\). 
        \item \(x\mid a\), \(x\mid b \implies x\mid an+bm\).
        \item \(x\mid y\), \(y\mid x \iff y=\pm x\).
        \item \(x \mid y \implies x \mid yz\).
        
    \end{enumerate}
    
\end{theorem}

\begin{example}
    Show that if \(n \in \ZZ, n > 1\), then \(n \nmid 2n^2 + 3n + 1\).
    \begin{proof}
        Here we can see the use of \Cref{lem: unimp}. Suppose \(n \mid 2n^2+3n+2\), then \(n \mid 2n^2 + 3n\), and \(n \mid 2n^2 + 3n + 1 \iff n \mid 1\). But, \( n \mid 1 \implies \abs{1} \ge \abs{n} \iff 1 \ge n\) which contradicts our earlier statement that \(n > 1\). Hence, \(n \nmid 2n^2+3n+1\).
    \end{proof}
\end{example}

\begin{remark}
    \( \abs{1} \ge \abs{n} \iff 1 \ge n\) because \(n>1>0\).
\end{remark}


\begin{example}
    Show that for any two natural numbers, \(a\), \(b : a > b\), \(a \nmid 2a+b\).
    \begin{proof}
        Again, we can use \Cref{lem: unimp}. Suppose \(a \mid 2a+b\), then \(a \mid 2a\) and \(a \mid 2a + b \iff a \mid b\). But, \(a \mid b \implies \abs{b} \ge \abs{a} \) and \(a, b \in \NN \implies b \ge a \). This contradicts our the fact \(a > b\). Hence \(a \nmid 2a+b\).
    \end{proof}
\end{example}

\section{Euclid's Division Lemma}

\para{Euclid's division lemma is something we will make much use of, especially later. For now we shall be satisfied with the unproven statement.}

\marginnote{By unique we really mean one and only one in the language of mathematics. This means
that there has to be at least one and at most one such thing.}

\begin{theorem}
    \label{thm: euclid}
    For any integers \(a\), \(b\), we can find \textit{unique} integers \(q, r\), such that: \[b=aq + r, \quad 0 \le r <a. \]
    Here, \(r\) is called the remainder, and \(q\) is called the quotient.
\end{theorem}

\section{Primes}

\index{Primes}

\marginnote{Note that there have to be exactly two factors. Therefore, \(1\) which has only one
factor is not prime.}

\begin{definition}
    [Primes]
    Any \(x \in \NN\) is called a prime, if and only if it has exactly \(2\) \textit{divisors}.
\end{definition}

\para{Primes are a very interesting subset of naturals. There properties will be explored later. 
For now, we use primes to define one of the most important theorem in all of number theory.}

\subsection{Fundamental Theorem of Arithmetic}

\begin{theorem}
    [Fundamental Theorem of Arithmetic]
    \label{thm: fta}
    Any natural number can be \textit{uniquely} expressed as a product of primes upto order.
\end{theorem}

\para{The order in which the product is expressed is irrelevant because of the commutativity of multiplication. \par 
We can further extend this theorem to include all integers by using some claims.}

\subsection{Integers as Multisets}

\begin{claim}
    Any non-zero \(x \in \ZZ\) can be expressed a multiset of primes(and \(-1\)). If the number is positive, then we can simply use \Cref{thm: fta} and included all the prime factors in its multiset. If the number is negative, then we express it as the multiset of its absolute value and include \(-1\) in the multiset.
\end{claim}

\begin{remark}
    A \textit{multiset} is a set where repeating elements are counted seperately. Each element has a \textit{mulitplicity} which indicates the number of times that element appears in the multiset.
\end{remark}

We could, for example, express \(20\) as \(\{2, 2, 5\}\), and \(-20\) as \(\{-1, 2, 2, 5\}\).
\\
This allows us to restate \Cref{thm: fta}.

\begin{theorem}
    [Fundamental theorem of Arithmetic in multisets]
    Any natural number can expressed as a multiset of primes.
\end{theorem}

\begin{remark}
    This multiset, is, by definition \textit{unique}.
\end{remark}

It also allows us to restate divisibility for positive integers.

\begin{theorem}
    [Divisibility in multisets]
    \(\forall a, b \in \NN\), \[a \mid b \iff A \subseteq B \]
\end{theorem}

\para{Where \(A\) and \(B\) are multisets of \(a\) and \(b\) respectively. We can't extend this
to the negative integers because of the rather annoying presence of \(-1\).}

\begin{remark}
    Henceforth, all variables are integers unless explicitly mentioned.
\end{remark}

\section{GCD and LCM}

\subsection{GCD}

\begin{definition}
    \( \gcd(a,b)\) is the multiset of all \textit{common} prime factors of \(a\) and \(b\).
\end{definition}

This gives us the following properties of \(\gcd\).

\begin{proposition}
    \hfill
    \begin{itemize}
        \item \(\gcd(a,b)\) is the greatest integer which divides both \(a\) and \(b\). In particular, \(\gcd(a,b) \le a, b\).
        \item \(c \mid a\), \(c \mid b \iff c \mid \gcd(a,b)\).
    \end{itemize}
\end{proposition}

We can, in fact, show that the two definition of gcd are equivalent. 

\begin{proof}
    Let \( \SC = \{p : p \mid a, b \) and \(p \) is a prime\( \} \). Suppose that \(\gcd(a,b) \ne \SC \). Then \( \gcd(a,b) > \SC \). 

    Note that \( \gcd(a,b)\) must contain an element \(x \notin \SC \) where \(x\) is prime.(Since it is an element in the multiset of \(\gcd(a,b)\)). 

    But, \(x \in \gcd(a,b) \iff x \mid \gcd(a,b)\), and \(x \mid \gcd(a,b) \), \(\gcd(a,b) \mid a, b \implies x \mid a,b\). \(x\) is prime, and \(x \mid a,b\ \implies x \in \SC\). This contradicts our earlier deduction that \(x \notin \SC\). By \textit{reductio ad absurdum}, we have \(\gcd(a,b) = \SC \).
\end{proof}

\begin{claim}
    
    Let \(a = p_{1} ^{\alpha_1} p_{2} ^{\alpha_2} \ldots p_{n}^{\alpha_n}\), \(b = p_{1} ^{\beta_1} p_{2} ^{\beta_2} \ldots p_{n}^{\beta_n}\). Then, \[\gcd(a,b)= p_{1} ^{\min(\beta_1, \alpha_1)} p_{2} ^{\min(\beta_2, \alpha_2)} \ldots p_{n}^{\min(\beta_n, \alpha_n)} \]
    Where, \(\alpha_i, \beta_i \in \NN_0 \).
    
\end{claim}

This is simply a consequence of the definition of \(\gcd\).

\begin{definition}
    [GCD]
    We may now state that the following are equivalent.
    \begin{itemize}
        \ii \( \gcd(a,b) = A \cap B  \)     
        \ii \(\gcd(a,b)\) is the greatest integer which divides both \(a\) and \(b\). In particular, \(\gcd(a,b) \le a, b\).
        \ii Let \(a = p_{1} ^{\alpha_1} p_{2} ^{\alpha_2} \ldots p_{n}^{\alpha_n}\), \(b = p_{1} ^{\beta_1} p_{2} ^{\beta_2} \ldots p_{n}^{\beta_n}\). Then, \[\gcd(a,b)= p_{1} ^{\min(\beta_1, \alpha_1)} p_{2} ^{\min(\beta_2, \alpha_2)} \ldots p_{n}^{\min(\beta_n, \alpha_n)} \]
        Where, \(\alpha_i, \beta_i \in \NN_0 \).
    \end{itemize}
    
\end{definition}

\subsection{LCM}

\begin{definition}
    [LCM]
    The following three definitions are equivalent.
    \begin{itemize}
        \ii \(\lcm(a,b) = A \cup B\)
        \ii \(\lcm(a,b)\) is the least number divisible by both \(a\) and \(b\). Particularly, \(\lcm(a,b) \ge a,b\).
        \ii Let \(a = p_{1} ^{\alpha_1} p_{2} ^{\alpha_2} \ldots p_{n}^{\alpha_n}\), \(b = p_{1} ^{\beta_1} p_{2} ^{\beta_2} \ldots p_{n}^{\beta_n}\). Then, \[\lcm(a,b)= p_{1} ^{\max(\beta_1, \alpha_1)} p_{2} ^{\max(\beta_2, \alpha_2)} \ldots p_{n}^{\max(\beta_n, \alpha_n)} \]
    \end{itemize}
    
    
\end{definition}


From the definitions of \(\gcd\) and \(\lcm\) we obtain the following very important theorem.

\begin{theorem}
    \[\lcm(a,b)\cdot \gcd(a,b) = a\cdot b \]
\end{theorem}

\begin{proof}
    
    
    \begin{align*}
        \gcd(a,b) \cdot \lcm(a,b) &= p_{1}^{\min(\alpha_1,\beta_1) + max(\alpha_1,\beta_1)} \ldots p_{n}^{(\max(\alpha_n, \beta_n) + \min(\alpha_n, \beta_n))}\\
        &= p_{1}^{\alpha_1 + \beta_1} \ldots p_2^{\alpha_n + \beta_n}\\
        &= a\cdot b
    \end{align*}    
    
    
\end{proof}



\section{Co-prime}

\begin{definition}
    Two numbers, \(a, b\) are called co-prime if \(\gcd(a,b)=1\).
\end{definition}

\begin{definition}
    A number \(x\) is pairwise co-prime to \(n\) other numbers \(p_{1}, p_{2} \ldots p_{n}\), if \( \gcd(x, p_1) = \gcd(x, p_2) \ldots = \gcd(x, p_n) =1 \).
\end{definition}

\section{Infinitude of Primes}



\begin{theorem}
    [Euclid]
    There are infinitely many primes.
\end{theorem}

\begin{proof}
    Let there be finitely many primes, \(\manys{p}{n}\). Define \( N \) as \( \many{p}{n} + 1 \).
    \par
    Note that \(N\) is pairwise co-prime to all, \(\many{p}{n}\) because if \(p_i \in \manys{p}{n} \mid N \) 
    and \(p_i \mid \many{p}{n} \iff p_i \mid 1\). But, \( p_i \in \NN \implies p_i = 1 \) which is absurd.
    \\
    \par Since \(N \in \NN\), \(N\) must have a prime factor, \(p\) because of \Cref{thm: fta}. 
    And \(p \notin \manys{p}{n}\) because \(N \) is pairwise co-prime to all primes in \(\manys{p}{n}\). 
    This contradicts our assumption that \(\manys{p}{n}\) was the set of all primes, 
    proving that there are infinitely many primes.

\end{proof}
\para{This type of proof is called the method of contradiction. We assume the negation of what we
want to prove, say to prove \(P\), assume \(\neg P\) and then show that the negation leads to a contradiction. Thus,
\(\neg P\) cannot hold, and therefore \(P\) must hold. }

\section{Euclid's Division Algorithm}

\para{Euclid's division algorithm is very useful in a lot of cases. It allows to compute 
\(\gcd(a,b)\) faster, and can be helpful in some proofs as well. The basic idea of it is that
since we can represent \(a = bq + r\) for any \(a\) and \(b\), We can substitute this as
\(a\) while computing \(\gcd(a,b)\).}

\begin{theorem}
    [Euclid's Division Algorithm]
    For any two integers \(a\), \(b\), such that \(a = bq + r\), where \(0 \le r < a\),
    \[
        \gcd(a,b) = \gcd(r,b)
    \]
\end{theorem}

\begin{proof}
    The idea here is to notice that \(\gcd(a,b) \mid a\) and \(\gcd(a,b) \mid b\). Therefore,
    since \(\gcd(a,b) \mid a\),
    \[
        \gcd(a,b) \mid bq + r
    \]
    But, \(\gcd (a,b) \mid b\). Citing \Cref{lem: unimp}, we must have \(\gcd(a,b) \mid r\).
    Now that we have shown \(\gcd(a,b) \mid r, b\), we must also show that it is the greatest number to do
    so. Let us assume that is not the case, let \(\ell \mid r, b : \ell > \gcd(a,b)\). Which means that
    \(\ell = \gcd(r,b)\)\par
    But the thing of note here is that if \(\ell \mid r\) and \(\ell \mid b\). Then \(\ell \mid a\).
    But that can not happen, since \(\ell\) is greater than \(\gcd(a,b)\) and divides both \(a,b\).
    Therefore, \(\ell = \gcd (a,b) = \gcd(r,b)\).
\end{proof}

\para{One thing to note here, is that since \(\gcd(a,b) = \gcd(r,b)\), \(\gcd(a,b) = \gcd(a-bq,r)\).
\begin{moral}
    Basically, \(\gcd(a,b) = \gcd(a \pm kb, r)\). 
\end{moral}
This statement is equally, if not more important than the formal one.}

\caution{While it is true that \(\gcd(a,b) = \gcd(r,b)\), it is not in any way true that
\(\lcm(a,b) = \lcm(r,b)\).}

\section{Bezout's Lemma}

\para{We are now nearing the end of this chapter, and bezout's lemma is one of the last
topics we'll cover. Consider the values of \(ax + by\), for \(a, b, x , y \in \ZZ\). For some
constant \(a\) and \(b\), say \(2\) and \(4\), you may notice that the \(\gcd(a,b)\)
is the smallest non-negative value of \(ax + by\)!}

\para{Is this always true? In this case, yes. Although I will not go much into the motivation, it
is an extremely useful tool.}

\begin{theorem}
    [Bezout's Lemma]
    \label{theo: bezout}
    For two integers \(a\) and \(b\), we can always find integers \(x\) and \(y\) such that,
    \[
        ax + by = \gcd(a,b)
    \]
\end{theorem}

\todo{add proof}

\section{Base systems}

\para{Whenever we write something like \(28\) what we really mean is \(28\) in base \(10\) or
\(28_{(10)}\). To represent it in an expanded form, we may write something like,}

\[
    28 = 8\times10^0 + 2\times10^1 
\]

\para{In general, for a number \(n\) in base \(\ell\), we can write something like,}

\[
    n = a_0 \times \ell^0 + a_1 \times \ell^1 + \dots \times a_k \times \ell^k
\]

\para{Where \( 0 \le a_0, a_1, \dots , a_k < \ell \). This representation is unique. If there is another such representation 
in the same base, it must violate the conditions established. Note that \(a_k \ne 0\), because
if it is, then we can simply reduce to \(k-1\) and rewrite similarly. Thus \(n \ge \ell^k\).
But n \(n < \ell^{k+1}\) because otherwise the representation would be not be unique. If \(n\)
is represent in such a manner, then it is written with digits \((a_k a_{k-1} \dots a_1)_{(\ell)}\).}

\parbreak

\para{Also similarly, you may notice that \(a\ell^k \le n < (a+1)\ell^k\). These help us devise
an algorithm to represent \(n\) in base \(\ell\).}

