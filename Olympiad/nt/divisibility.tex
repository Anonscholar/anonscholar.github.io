\chapter{Divisibility}
\label{nt: div}


\begin{chapquote}{Euclid of \source{Alexandaria}}
``Give him threepence, since he must make gain out of what he learns''
\end{chapquote}

\motiv{Divisibility forms the basis of all of Number Theory.}

\section{Basic Properties}

\para{
    The idea of divisibility is something most people are familiar with. Yet, it is a notion so useful, that we may
    yet again acquaint ourselves with it and its properties.
    This time, however, we are going to approach it with formality and use its property in a general, and abstract manner.
    For this, let us first define division for the integers.
}

\index{division}

\begin{definition}
    [Division]
    The operation division is defined as:
    \begin{enumerate}
        \item An integer \(y\) is said to be a multiple of \(x\) if \[y\in \{kx : k\in \ZZ \} \]
        \item If \(y\) is a multiple of \(x\) then, \(y\) is \emph{divisible} by 
        \(x\) and \(x\) divides \(y\) which is written as: \[x \mid y.\]
    \end{enumerate}
\end{definition}

\para{
    Immediately, we can put forth some \emph{propositions} from these definitions. They may not be obvious from first glance
    but they are indeed very easy to get an intuitive grasp on after we have seen them.
}

\begin{plainprop*}
    For \(x\), \(y\), \(z\) \(\in\) \(\ZZ\)
    \begin{enumerate}
        \item \(\forall x \ne 0\), \(x \mid x\) and \(x \mid 0\).
        \item \(1 \mid x\).
        \item \(x\mid y\) and \(y \mid z \implies \) \(x \mid z\).
        \item \(x \mid y \iff \dfrac{y}{x} \in \ZZ\)
    \end{enumerate}        
\end{plainprop*}

\para{Most of the proofs here are trivial, but let us show that
using the last proposition we can prove the third one.}

\begin{proof}
    \( x \mid y \iff \dfrac{y}{x} \in \ZZ \) and \(  y \mid z \iff \dfrac{z}{y} \in \ZZ \). But \( \dfrac{z}{x}= \dfrac{y}{x}\cdot\dfrac{z}{y} \in \ZZ \iff x\mid z\).
\end{proof}


\para{We may note that because of the definition of division, we get the following theorem:}

\begin{theorem}
    If \(x \mid y\), then either \(\abs{y} \ge \abs{x} \) or \(y=0\).
\end{theorem}

\para{The \(0\) case is extremely important and often hard to see. 
Remember to look out for this case.}

\begin{proof}
    \(\abs{y}=\abs{k\cdot x}= \abs{k}\cdot \abs{x}\). But, \( \abs{k}\cdot \abs{x} \ge \abs{x} \iff \abs{y} \ge \abs{x}\). Note here that \(k\) is not taken to be \(0\) since, \(y=k\cdot x=0\cdot x=0\). But this case is already mentioned.
\end{proof}



\para{The following is not of a severe importance in the view that it is hard to see or prove. 
But it allows us to quickly formalise our solutions and is thus useful for proofs.}

\begin{lemma}
    \label{lem: unimp}
    If \(x \mid a\), then \(x \mid ac+b \iff x\mid b\).
\end{lemma}

\begin{proof}
    The proof is trivial as, \(x \mid a \iff a = kx : k \in \ZZ\).
    \par 
    \textbf{Direction 1}
    Note, \(x \mid ac+b \iff ac+b=nx : n \in \ZZ\), hence, \(b=nx-ac=nx-kcx=(n-kc)x \iff x \mid b\).
    \par
    \textbf{Direction 2}
    And, \(x\mid b \iff b=mx : m \in \ZZ \), hence, \(ac+b=kcx+mx=(kc+m)x \iff x \mid ac+b \).
\end{proof}

Finally, we may conclude this section of properties by stating all the previous properties and some more. 
\begin{theorem}
    Let \(x\), \(y\), \(z\) \(\in\) \(\ZZ\)
    \begin{enumerate}
        \item \(\forall x \ne 0\), \(x \mid x\) and \(x \mid 0\).
        \item \(1 \mid x\).
        \item \(x\mid y\) and \(y \mid z \implies \) \(x \mid z\).
        \item \(x \mid y \iff \dfrac{y}{x} \in \ZZ\).
        \item \(\forall z \ne 0\), \(x \mid y \iff xz \mid yz\). 
        \item \(x\mid a\), \(x\mid b \implies x\mid an+bm\).
        \item \(x\mid y\), \(y\mid x \iff y=\pm x\).
        \item \(x \mid y \implies x \mid yz\).
        
    \end{enumerate}
    
\end{theorem}

\begin{example}
    Show that if \(n \in \ZZ, n > 1\), then \(n \nmid 2n^2 + 3n + 1\).
    \begin{proof}
        Here we can see the use of \Cref{lem: unimp}. Suppose \(n \mid 2n^2+3n+2\), then \(n \mid 2n^2 + 3n\), and \(n \mid 2n^2 + 3n + 1 \iff n \mid 1\). But, \( n \mid 1 \implies \abs{1} \ge \abs{n} \iff 1 \ge n\) which contradicts our earlier statement that \(n > 1\). Hence, \(n \nmid 2n^2+3n+1\).
    \end{proof}
    
    
\end{example}

\begin{remark}
    \( \abs{1} \ge \abs{n} \iff 1 \ge n\) because \(n>1>0\).
\end{remark}


\begin{example}
    Show that for any two natural numbers, \(a\), \(b : a > b\), \(a \nmid 2a+b\).
    
    \begin{proof}
        Again, we can use \Cref{lem: unimp}. Suppose \(a \mid 2a+b\), then \(a \mid 2a\) and \(a \mid 2a + b \iff a \mid b\). But, \(a \mid b \implies \abs{b} \ge \abs{a} \) and \(a, b \in \NN \implies b \ge a \). This contradicts our the fact \(a > b\). Hence \(a \nmid 2a+b\).
    \end{proof}
\end{example}

\section{Euclid's Division Lemma}

Euclid's division lemma is something we will make much use of, especially later. For now the unproven statement shall suffice.

\begin{theorem}
    \label{thm: euclid}
    For any integers \(a\), \(b\), we can find \textit{unique} integers \(q, r\), such that: \[b=aq + r, \quad 0 \le r <a. \]
    Here, \(r\) is called the remainder, and \(q\) is called the quotient.
\end{theorem}

\section{Primes}

\index{Primes}

\begin{definition}
    [Primes]
    Any \(x \in \NN\) is called a prime, if and only if it has exactly \(2\) \textit{divisors}.
\end{definition}

Primes are a very interesting subset of naturals. There properties will be explored later. For now, we use primes to define one of the most important theorem in all of number theory.

\subsection{Fundamental Theorem of Arithmetic}

\begin{theorem}
    [Fundamental Theorem of Arithmetic]
    \label{thm: fta}
    Any natural number can be \textit{uniquely} expressed as a product of primes upto order.
\end{theorem}

By uniquely, we mean there is \textit{at most} and \textit{at least} one way to express any natural as a product of primes. The order in which the product is expressed is irrelevant because of the commutativity of multiplication. \par 
We can further extend this theorem to include all integers by using some claims.

\subsection{Integers as Multisets}

\begin{claim}
    Any non-zero \(x \in \ZZ\) can be expressed a multiset of primes(and \(-1\)). If the number is positive, then we can simply use \Cref{thm: fta} and included all the prime factors in its multiset. If the number is negative, then we express it as the multiset of its absolute value and include \(-1\) in the multiset.
\end{claim}

\begin{remark}
    A \textit{multiset} is a set where repeating elements are counted seperately. Each element has a \textit{mulitplicity} which indicates the number of times that element appears in the multiset.
\end{remark}

We could, for example, express \(20\) as \(\{2, 2, 5\}\), and \(-20\) as \(\{-1, 2, 2, 5\}\).
\\
This allows us to restate \Cref{thm: fta}.

\begin{theorem}
    [Fundamental theorem of Arithmetic in multisets]
    Any natural number can expressed as a multiset of primes.
\end{theorem}

\begin{remark}
    This multiset, is, by definition \textit{unique}.
\end{remark}

It also allows us to restate divisibility for positive integers.

\begin{theorem}
    [Divisibility in multisets]
    \(\forall a, b \in \NN\), \[a \mid b \iff A \subseteq B \]
\end{theorem}

\begin{remark}
    Henceforth, all variables are integers unless explicitly mentioned.
\end{remark}

\section{GCD and LCM}

\subsection{GCD}

\begin{definition}
    \( \gcd(a,b)\) is the multiset of all \textit{common} prime factors of \(a\) and \(b\).
\end{definition}

This gives us the following properties of \(\gcd\).

\begin{proposition}
    \hfill
    \begin{itemize}
        \item \(\gcd(a,b)\) is the greatest integer which divides both \(a\) and \(b\). In particular, \(\gcd(a,b) \le a, b\).
        \item \(c \mid a\), \(c \mid b \iff c \mid \gcd(a,b)\).
    \end{itemize}
\end{proposition}

We can, in fact, show that the two definition of gcd are equivalent. 

\begin{proof}
    Let \( \SC = \{p : p \mid a, b \) and \(p \) is a prime\( \} \). Suppose that \(\gcd(a,b) \ne \SC \). Then \( \gcd(a,b) > \SC \). 
    \\
    \par
    Note that \( \gcd(a,b)\) must contain an element \(x \notin \SC \) where \(x\) is prime.(Since it is an element in the multiset of \(\gcd(a,b)\)). 
    \par
    But, \(x \in \gcd(a,b) \iff x \mid \gcd(a,b)\), and \(x \mid \gcd(a,b) \), \(\gcd(a,b) \mid a, b \implies x \mid a,b\). \(x\) is prime, and \(x \mid a,b\ \implies x \in \SC\). This contradicts our earlier deduction that \(x \notin \SC\). By \textit{reductio ad absurdum}, we have \(\gcd(a,b) = \SC \).
\end{proof}

\begin{claim}
    
    Let \(a = p_{1} ^{\alpha_1} p_{2} ^{\alpha_2} \ldots p_{n}^{\alpha_n}\), \(b = p_{1} ^{\beta_1} p_{2} ^{\beta_2} \ldots p_{n}^{\beta_n}\). Then, \[\gcd(a,b)= p_{1} ^{\min(\beta_1, \alpha_1)} p_{2} ^{\min(\beta_2, \alpha_2)} \ldots p_{n}^{\min(\beta_n, \alpha_n)} \]
    Where, \(\alpha_i, \beta_i \in \NN_0 \).
    
\end{claim}

This is simply a consequence of the definition of \(\gcd\).

\begin{definition}
    [GCD]
    We may now state that the following are equivalent.
    \begin{itemize}
        \ii \( \gcd(a,b) = A \cap B  \)     
        \ii \(\gcd(a,b)\) is the greatest integer which divides both \(a\) and \(b\). In particular, \(\gcd(a,b) \le a, b\).
        \ii Let \(a = p_{1} ^{\alpha_1} p_{2} ^{\alpha_2} \ldots p_{n}^{\alpha_n}\), \(b = p_{1} ^{\beta_1} p_{2} ^{\beta_2} \ldots p_{n}^{\beta_n}\). Then, \[\gcd(a,b)= p_{1} ^{\min(\beta_1, \alpha_1)} p_{2} ^{\min(\beta_2, \alpha_2)} \ldots p_{n}^{\min(\beta_n, \alpha_n)} \]
        Where, \(\alpha_i, \beta_i \in \NN_0 \).
    \end{itemize}
    
\end{definition}

\subsection{LCM}

\begin{definition}
    [LCM]
    The following three definitions are equivalent.
    \begin{itemize}
        \ii \(\lcm(a,b) = A \cup B\)
        \ii \(\lcm(a,b)\) is the least number divisible by both \(a\) and \(b\). Particularly, \(\lcm(a,b) \ge a,b\).
        \ii Let \(a = p_{1} ^{\alpha_1} p_{2} ^{\alpha_2} \ldots p_{n}^{\alpha_n}\), \(b = p_{1} ^{\beta_1} p_{2} ^{\beta_2} \ldots p_{n}^{\beta_n}\). Then, \[\lcm(a,b)= p_{1} ^{\max(\beta_1, \alpha_1)} p_{2} ^{\max(\beta_2, \alpha_2)} \ldots p_{n}^{\max(\beta_n, \alpha_n)} \]
    \end{itemize}
    
    
\end{definition}


From the definitions of \(\gcd\) and \(\lcm\) we obtain the following very important theorem.

\begin{theorem}
    \[\lcm(a,b)\cdot \gcd(a,b) = a\cdot b \]
\end{theorem}

\begin{proof}
    
    
    \begin{align*}
        \gcd(a,b) \cdot \lcm(a,b) &= p_{1}^{\min(\alpha_1,\beta_1) + max(\alpha_1,\beta_1)} \ldots p_{n}^{(\max(\alpha_n, \beta_n) + \min(\alpha_n, \beta_n))}\\
        &= p_{1}^{\alpha_1 + \beta_1} \ldots p_2^{\alpha_n + \beta_n}\\
        &= a\cdot b
    \end{align*}    
    
    
\end{proof}



\section{Co-prime}

\begin{definition}
    Two numbers, \(a, b\) are called co-prime if \(\gcd(a,b)=1\).
\end{definition}

\begin{definition}
    A number \(x\) is pairwise co-prime to \(n\) other numbers \(p_{1}, p_{2} \ldots p_{n}\), if \( \gcd(x, p_1) = \gcd(x, p_2) \ldots = \gcd(x, p_n) =1 \).
\end{definition}

\section{Infinitude of Primes}



\begin{theorem}
    [Euclid]
    There are infinitely many primes.
\end{theorem}

\begin{proof}
    Let there be finitely many primes, \(\manys{p}{n}\). Define \( N \) as \( \many{p}{n} + 1 \).
    \par
    Note that \(N\) is pairwise co-prime to all, \(\many{p}{n}\) because if \(p_i \in \manys{p}{n} \mid N \) and \(p_i \mid \many{p}{n} \iff p_i \mid 1\). But, \( p_i \in \NN \implies p_i = 1 \) which is absurd.
    \\
    \par Since \(N \in \NN\), \(N\) must have a prime factor, \(p\) because of \Cref{thm: fta}. And \(p \notin \manys{p}{n}\) because \(N \) is pairwise co-prime to all primes in \(\manys{p}{n}\). This contradicts our assumption that \(\manys{p}{n}\) was the set of all primes, proving that there are infintely many primes.
    


\end{proof}

\todo[inline]{Add motivation for stuff}




