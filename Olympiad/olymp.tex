\documentclass[twoside,a4paper,10pt]{memoir}


\usepackage[no-math]{fontspec}
\setmainfont{Baskerville}[
  BoldFont={* Semibold},
]

\usepackage{amsmath, amssymb}

\nonzeroparskip
\usepackage[right=7.5cm,left=2.5cm,bottom=1cm,top=1cm,marginparwidth=5.5cm,marginparsep=1cm,includeheadfoot,asymmetric]{geometry}

\usepackage[texindy]{imakeidx}
\makeindex[intoc]


\usepackage[dvipsnames]{xcolor}

\usepackage{etoolbox}

\usepackage{booktabs, tabularx}
\setlength{\parindent}{0em}

\usepackage{dsfont}

%Chapter Style
\usepackage{calc}
%chapter style

\makeatletter
\def\@makechapterhead#1{%
  \vspace*{2\p@}%
  {\parindent \z@ \raggedleft \reset@font
            \scshape \@chapapp{} \thechapter
        \par\nobreak
        \interlinepenalty\@M
    \Huge \bfseries #1\par\nobreak
    %\vspace*{1\p@}%
    \hrulefill
    \par\nobreak
    \vskip 10\p@
  }}
\def\@makeschapterhead#1{%
  \vspace*{2\p@}%
  {\parindent \z@ \raggedleft \reset@font
            \scshape \vphantom{\@chapapp{} \thechapter}
        \par\nobreak
        \interlinepenalty\@M
    \Huge \bfseries #1\par\nobreak
    %\vspace*{1\p@}%
    \par\nobreak
    \vskip 10\p@
  }}

\usepackage{xparse}
\usepackage{lipsum}

%%%%%%%%%%%%%%%%%%%%%%%%%%%%%%%%%%%%%%%%%%%%%%%%%%%%%%%%%%%%%%%%%%%%%%%%%%%%%%%%%%%%%%%%%%%%%%%%%%%%%%%%%%%%%%%%%%%%%%%%

%page style

\nouppercaseheads
\makepagestyle{mystyle}
\makeevenhead{mystyle}{\textbf{\thepage}\;\;\;{\emph{\leftmark}}}{}{}
\makeoddhead{mystyle}{}{}{\emph{\rightmark}\;\;\;\textbf{\thepage}}

\makeevenfoot{mystyle}{}{}{}
\makeoddfoot{mystyle}{}{}{}
\makepsmarks{mystyle}{%
  \createmark{chapter}{left}{nonumber}{}{}
  \createmark{part}{right}{nonumber}{}{}}
  

\pagestyle{mystyle}

\copypagestyle{chapter}{plain}
\makeoddfoot{chapter}{}{}{\rule[0cm]{0.8cm}{0.12cm}\\\emph{\thepage}}

\copypagestyle{part}{plain}
\makeoddfoot{part}{}{}{\rule[0cm]{0.8cm}{0.12cm}\\\emph{\thepage}}


%some graph stuff
\usepackage{todonotes}
\usepackage{tikz}
\usepackage{tikz-3dplot}
\usetikzlibrary{mindmap}
  \usetikzlibrary{arrows.meta,backgrounds}
  \usetikzlibrary{decorations.pathmorphing,patterns}
  \usetikzlibrary{angles, quotes, intersections}
\usepackage{pgfplots}
\pgfplotsset{width=7cm,compat=1.18}
\pgfplotsset{
    funcgraphbare/.style={
        axis x line=center,
        axis y line=center,
        ticks=none,
    },
    funcgraph/.style={
        funcgraphbare,
        xlabel={\(x\)},
        ylabel={\(y\)},
    },
}
%%%%%%%%%%%%%%%%%%%%%%%%%%%%%%%%%%%%%%%%%%%%%%%%%%%%%%%%%%%%%%%%%%%%%%%%%%%%%%%%%%%%%%%%%%%%%%%%%%%%%%%%%%%%%%%%%%%%%%%%
%packages
\usepackage{physics}
\usepackage{tkz-euclide}
\usepackage{babel}
\usepackage{mhchem}
\usepackage[ddmmyyyy]{datetime}
\renewcommand{\dateseparator}{.}
\usepackage{caption}
\usepackage{paralist}
\usepackage{graphicx}
\usepackage{microtype}
\usepackage{wrapfig}
\usepackage{import}
\usepackage{xifthen}
\usepackage{appendix}

\usepackage[per-mode=symbol]{siunitx}
\usepackage{bohr}
\usepackage{modiagram}
\usepackage{subcaption}


\usepackage{nameref}
\usepackage{tzplot}
\usepackage{enumitem}
\newlist{axioms}{enumerate}{10}
\setlist[axioms]{label=\textbf{A\arabic*.},ref=Axiom\arabic*,leftmargin=*}

\newlist{casework}{enumerate}{10}
\setlist[casework]{label=\textbf{Case\,\Roman*},ref=Axiom\arabic*,leftmargin=*}


\usepackage{ragged2e}
\usepackage{empheq}

\usepackage[linktoc=page,hyperfootnotes=false]{hyperref}
\hypersetup{
colorlinks=true,
linkcolor=TealBlue!60!black,
citecolor=TealBlue!60!black,
urlcolor=TealBlue!60!black,
}
\usepackage[nameinlink]{cleveref}
\usepackage{float}


\usepackage[citestyle=alphabetic]{biblatex}

\usepackage{listings}
\definecolor{codegreen}{rgb}{0,0.6,0}
\definecolor{codegray}{rgb}{0.5,0.5,0.5}
\definecolor{codepurple}{rgb}{0.58,0,0.82}
\definecolor{backcolour}{rgb}{0.95,0.95,0.92}

\lstdefinestyle{mystyle}{
    backgroundcolor=\color{backcolour},   
    commentstyle=\color{codegreen},
    keywordstyle=\color{magenta},
    numberstyle=\tiny\color{codegray},
    stringstyle=\color{codepurple},
    basicstyle=\ttfamily\footnotesize,
    breakatwhitespace=false,         
    breaklines=true,                 
    captionpos=b,                    
    keepspaces=true,                 
    numbers=left,                    
    numbersep=5pt,                  
    showspaces=false,                
    showstringspaces=false,
    showtabs=false,                  
    tabsize=2
}

\lstset{style=mystyle}


\usetikzlibrary{decorations.markings}
\tikzset{
    arrow inside/.style = {
        postaction = {
            decorate,
            decoration={
                markings,
                mark=at position #1 with {\arrow{>}}
            }
        }
    },
    arrow inside/.default = 0.5
}

\raggedbottom
\newcommand{\incfig}[1]{%
    \def\svgwidth{2.4in}
    \scalebox{.5}{\import{../figures/}{#1.pdf_tex}}
}


\epigraphfontsize{\small\itshape}
\setlength\epigraphwidth{6cm}
\setlength\epigraphrule{0pt}

\usepackage{collectbox}

\makeatletter
\newcommand{\mybox}{%
    \collectbox{%
        \setlength{\fboxsep}{1pt}%
        \fbox{\BOXCONTENT}%
    }%
}
\makeatother

\usepackage{cleveref}


\crefformat{section}{\S#2#1#3} % see manual of cleveref, section 8.2.1
\crefformat{subsection}{\S#2#1#3}
\crefformat{subsubsection}{\S#2#1#3}


%%%%%%%%%%%%%%%%%%%%%%%%%%%%%%%%%%%%%%%%%%%%%%%%%%%%%%%%%%%%%%%%%%%%%%%%%%%%%%%%%%%%%%%%%%%%%%%%%%%%%%%%%%%%%%%%%%%%%%%%

%section, part formatting

\renewcommand*{\thepart}{\color{black}\Alph{part}}
\renewcommand*{\parttitlefont}{\color{black}\normalfont\bfseries\HUGE}
\renewcommand*{\partnamefont}{\normalfont\LARGE\itshape}
\renewcommand*{\partnumfont}{\color{black}\normalfont\itshape}

\setsechook{\hangsecnum}

\makeatletter

\renewcommand{\hangsecnum}{%
  \def\@seccntformat##1{%
    \makebox[0pt][r]{%
      \color{black} 
      {%
      \csname the##1\endcsname
      }
      \, 
    }%
  }%
}

\setsecnumdepth{subsection}

\setsubsechook{\hangsecnum}
\setsubsubsechook{\hangsecnum}


\makeatletter                   
\def\printauthor{%                  
    {\large \@author}}          
\makeatother

\usepackage{fontawesome5}

\makeatletter
\newcommand{\github}[1]{%
   \href{#1}{\textcolor{black}{\faGithub}}%
}
\makeatother

\usepackage{authblk}

\author{
    \github{https://github.com/Anonscholar} Ascholar
}

\newcommand*{\email}[1]{%
    \normalsize\href{mailto:#1}{#1}\par
    }

\affil{\email{adyanshmishra@proton.me}}

%%%%%%%%%%%%%%%%%%%%%%%%%%%%%%%%%%%%%%%%%%%%%%%%%%%%%%%%%%%%%%%%%%%%%%%%%%%%%%%%%%%%%%%%%%%%%%%%%%%%%%%%%%%%%%%%%%%%%%%%

%title page

\newcommand*{\titleSW}{\begin{titlingpage}
\newgeometry{left=10cm}
\begingroup% Story of Writing
\raggedleft
\vspace*{\baselineskip}
{\HUGE\itshape \thetitle}\\[\baselineskip]
\vspace{3em}
{\printauthor ~ \\ Version: 0.1\today}
\vfill
\vspace*{\baselineskip}
\endgroup
\restoregeometry
\end{titlingpage}}

\usepackage{ellipsis}

\usetikzlibrary{patterns,decorations.pathmorphing}
\usetikzlibrary{arrows.meta}
\tikzset{>=latex}

\usepackage{framed}
% colors to be used
\definecolor{myred}{RGB}{127,0,0}
\definecolor{myyellow}{RGB}{169,121,69}

% a modification of the leftbar environment defined by the framed package
% will be used to place a vertical colored bar separating the page number and the
% title in chapter entries
\renewenvironment{leftbar}{%
  \def\FrameCommand{\textcolor{myyellow}{\vrule width 1.4pt depth 5pt}\hspace*{15pt}}%
  \MakeFramed{\advance\hsize-\width\FrameRestore}}%
 {\endMakeFramed}

% redefinition of the name of the ToC
\makeatletter
% redefinitions for chapter entries

% redefinitions for part entries
\renewcommand{\partnumberline}[1]{\mbox{\centering\normalfont\itshape\rmfamily Part\thepart~#1}\par\noindent\Large}
\renewcommand\cftpartafterpnum{\vskip1ex}

\makeatletter
\renewcommand*{\l@part}[2]{%
  \ifnum \c@tocdepth >-2\relax
    \cftpartbreak
    \begingroup
      {
        \setlength{\memRTLleftskip}{0pt}
        \setlength{\memRTLrightskip}{0pt}
        \interlinepenalty\@M
        \centering
        \cftpartfont #1
        \par
      }
      \nobreak
        \global\@nobreaktrue
        \everypar{\global\@nobreakfalse\everypar{}}%
    \endgroup
  \fi}
\makeatother
\renewcommand{\cftpartfont}{\bfseries}

% redefinitions for section entries
\renewcommand\cftsectionfont{\rmfamily}
\renewcommand\cftsectionpagefont{\rmfamily\itshape\color{myred}}

\cftsetindents{section}{1em}{2em}


% text styling of all side footnotes
% styling and placement of mark
\footmarkstyle{{\itshape\footnotesize#1. }}
\setlength{\footmarkwidth}{0em}
\setlength{\footmarksep}{-\footmarkwidth}
% memoir command - do all footnotes in margin
\footnotesinmargin

% SIDECAPTIONS
\setsidecaps{\marginparsep}{\marginparwidth}
\sidecapmargin{outer}
\setsidecappos{t}
\renewcommand*{\sidecapstyle}{%
\captionnamefont{\scshape}
\ifscapmargleft
\captionstyle{\RaggedLeft\footnotesize\foottextfont}%
\else
\captionstyle{\RaggedRight\footnotesize\foottextfont}%
\fi}

 \makeatletter
  \renewcommand{\fnum@figure}{
  {\large\scshape\figurename~\thefigure}}
\makeatother
\renewcommand{\footnotesize}{\fontsize{10pt}{12pt}\selectfont} % Change footnotesize to 10pt


%%%%%%%%%%%%%%%%%%%%%%%%%%%%%%%%%%%%%%%%%%%%%%%%%%%%%%%%%%%%%%%%%%%%%%%%%%%%%%%%%%%%%%%%%%%%%%%%%%%%%%%%%%%%%%%%%%%%%%%%
\usepackage{minitoc}     % For generating chapter-local TOCs

% Initialize the minitoc package
\dominitoc[n]
\setcounter{minitocdepth}{1}   % Show until sections in minitoc
\nomtcrule    % removes rules = horizontal lines  Define the command to create a local TOC in the margin without subsections

\undottedmtctrue

\newcommand{\margintoc}{
  \marginnote{
    \vspace{-10em}
    \begin{minipage}[t]{7.5cm}
      {\minitoc} % This will use the depth set previously in \dominitoc
    \end{minipage}
  }
}

% Start of subappendices environment
\AtBeginEnvironment{subappendices}{%
\chapter*{Appendices}
\addcontentsline{toc}{chapter}{Appendices}
\counterwithin{figure}{section}
\counterwithin{table}{section}
}

% End of subappendices environment
\AtEndEnvironment{subappendices}{%
\counterwithout{figure}{section}
\counterwithout{table}{section}
}

\tikzset{axis/.style={thick,-latex}}
\tikzset{vec/.style={thick,blue}}
\tikzset{univec/.style={thick,black,-latex}}

%Modified from Evan's evan.sty.
%
%Boost Software License - Version 1.0 - August 17th, 2003
%
%Permission is hereby granted, free of charge, to any person or organization
%obtaining a copy of the software and accompanying documentation covered by
%this license (the "Software") to use, reproduce, display, distribute,
%execute, and transmit the Software, and to prepare derivative works of the
%Software, and to permit third-parties to whom the Software is furnished to
%do so, all subject to the following:

%The copyright notices in the Software and this entire statement, including
%the above license grant, this restriction and the following disclaimer,
%must be included in all copies of the Software, in whole or in part, and
%all derivative works of the Software, unless such copies or derivative
%works are solely in the form of machine-executable object code generated by
%a source language processor.

%THE SOFTWARE IS PROVIDED "AS IS", WITHOUT WARRANTY OF ANY KIND, EXPRESS OR
%IMPLIED, INCLUDING BUT NOT LIMITED TO THE WARRANTIES OF MERCHANTABILITY,
%FITNESS FOR A PARTICULAR PURPOSE, TITLE AND NON-INFRINGEMENT. IN NO EVENT
%SHALL THE COPYRIGHT HOLDERS OR ANYONE DISTRIBUTING THE SOFTWARE BE LIABLE
%FOR ANY DAMAGES OR OTHER LIABILITY, WHETHER IN CONTRACT, TORT OR OTHERWISE,
%ARISING FROM, OUT OF OR IN CONNECTION WITH THE SOFTWARE OR THE USE OR OTHER
%DEALINGS IN THE SOFTWARE.



\usepackage{amsthm}
\usepackage{thmtools}
\usepackage[framemethod=TikZ]{mdframed}
\usetikzlibrary{shadows}
% https://tex.stackexchange.com/a/292090/76888
% https://github.com/marcodaniel/mdframed/issues/12
\xpatchcmd{\endmdframed}
{\aftergroup\endmdf@trivlist\color@endgroup}
{\endmdf@trivlist\color@endgroup\@doendpe}
{}{}

\mdfdefinestyle{mdbluebox}{%
    roundcorner=0pt,
    linewidth=1pt,
    skipabove=12pt,
    innertopmargin=9pt,
    innerbottommargin=9pt,
    skipbelow=2pt,
    linecolor=TealBlue!35,
    nobreak=false,
    backgroundcolor=TealBlue!5,
}
\declaretheoremstyle[
headfont=\rmfamily\bfseries\color{TealBlue},
mdframed={style=mdbluebox},
headpunct={\\[3pt]},
postheadspace={0pt},
postheadhook={\textcolor{TealBlue!80}{\rule[.6ex]{\linewidth}{0.4pt}} \\ }, % \\ removed
]{thmbluebox}

\mdfdefinestyle{mdredbox}{%
    roundcorner=0pt,
    linewidth=1pt,
    skipabove=12pt,
    skipbelow=12pt,
    innertopmargin=9pt,
    innerbottommargin=9pt,
    backgroundcolor=Salmon!5,
    linecolor=Salmon!35,
    frametitleaboveskip=8pt,
    frametitlebelowskip=8pt,
    frametitlebackgroundcolor=Violet!50!black,
    frametitlefont=\bfseries\sffamily\color{white},
    frametitlerule=true,    
    nobreak=false,
}
\declaretheoremstyle[
headfont=\bfseries\color{RawSienna},
mdframed={style=mdredbox},
headpunct={\\[3pt]},
postheadspace={0pt},
postheadhook={\textcolor{RawSienna!80}{\rule[.6ex]{\linewidth}{0.4pt}} \\ }, % \\ removed
]{thmredbox}

\mdfdefinestyle{mdgreenbox}{%
    skipabove=8pt,
    linewidth=2pt,
    innertopmargin=9pt,
    innerbottommargin=9pt,
    rightline=false,
    leftline=true,
    topline=false,
    bottomline=false,
    linecolor=ForestGreen!40,
    backgroundcolor=ForestGreen!4,
}
\declaretheoremstyle[
headfont=\bfseries\rmfamily\color{ForestGreen},
bodyfont=\normalfont,
postheadspace={0pt},
mdframed={style=mdgreenbox},
headpunct={ --- },
]{thmgreenbox}

\mdfdefinestyle{mdblackbox}{%
    skipabove=8pt,
    linewidth=3pt,
    rightline=false,
    leftline=true,
    innertopmargin=9pt,
    innerbottommargin=9pt,
    topline=false,
    bottomline=false,
    linecolor=black,
    backgroundcolor=RedViolet!5!gray!5,
}
\declaretheoremstyle[
headfont=\bfseries,
bodyfont=\normalfont\small,
spaceabove=2pt,
spacebelow=0pt,
mdframed={style=mdblackbox}
]{thmblackbox}

\mdfdefinestyle{mdpurplebox}{%
    roundcorner=0pt,
    linewidth=1pt,
    skipabove=12pt,
    skipbelow=12pt,
    innertopmargin=9pt,
    innerbottommargin=9pt,
    linecolor=Orchid!35,
    nobreak=false,
    backgroundcolor=Orchid!5,
    frametitleaboveskip=8pt,
    frametitlebelowskip=8pt,
    frametitlebackgroundcolor=Violet!50!black,
    frametitlefont=\bfseries\sffamily\color{white},
    frametitlerule=true,
}
\declaretheoremstyle[
headfont=\rmfamily\bfseries\color{Orchid},
mdframed={style=mdpurplebox},
headpunct={\\[3pt]},
postheadspace={0pt},
postheadhook={\textcolor{Orchid!80}{\rule[.6ex]{\linewidth}{0.4pt}} \\ }, % \\ removed
]{thmpurplebox}
\newcommand{\listhack}{$\empty$\vspace{-2em}}


\mdfdefinestyle{mdsiennamargin}{%
    skipabove=8pt,
    linewidth=2pt,
    rightline=false,
    leftline=true,
    topline=false,
    bottomline=false,
    linecolor=RoyalBlue!40,
    backgroundcolor=RoyalBlue!5,
}
\declaretheoremstyle[
headfont=\bfseries\rmfamily\color{purple},
bodyfont=\normalfont,
postheadspace={0pt},
mdframed={style=mdsiennamargin},
headpunct={ \\[3pt] },
]{thmmarginbox}

\mdfdefinestyle{mdgreymargin}{%
    skipabove=8pt,
    linewidth=2pt,
    rightline=false,
    leftline=false,
    topline=false,
    bottomline=false,
    backgroundcolor=JungleGreen!15,
}
\declaretheoremstyle[
headfont=\bfseries\rmfamily\color{purple},
bodyfont=\normalfont,
postheadspace={0pt},
mdframed={style=mdgreymargin},
headpunct={ \\[3pt] },
]{thmmarginref}


\theoremstyle{definition}
\declaretheorem[style=thmbluebox,name=Theorem,numberwithin=section]{theorem}


\declaretheorem[style=thmbluebox,name=Theorem,numbered=no]{theorem*}
\declaretheorem[style=thmbluebox,name=Lemma,numbered=no]{lemma*}
\declaretheorem[style=thmbluebox,name=Lemma,sibling=theorem]{lemma}


\declaretheorem[style=thmgreenbox,name=Proposition,numbered=no]{proposition*}
\declaretheorem[style=thmgreenbox,name=Corollary,numbered=no]{corollary*}
\declaretheorem[style=thmgreenbox,name=Assumption,numbered=no]{assume*}
\declaretheorem[style=thmgreenbox,name=Proposition,sibling=theorem]{proposition}
\declaretheorem[style=thmgreenbox,name=Corollary,sibling=theorem]{corollary}
\declaretheorem[style=thmgreenbox,name=Assumption,sibling=theorem]{assume}
\declaretheorem[style=thmgreenbox,name=Algorithm,sibling=theorem]{algorithm}
\declaretheorem[style=thmgreenbox,name=Algorithm,numbered=no]{algorithm*}
\declaretheorem[style=thmgreenbox,name=Claim,sibling=theorem]{claim}
\declaretheorem[style=thmgreenbox,name=Claim,numbered=no]{claim*}

\declaretheorem[style=thmredbox,name=Example,sibling=theorem]{example}
\declaretheorem[style=thmredbox,name=Example,numbered=no]{example*}


% Remark-style theorems
\declaretheorem[style=thmblackbox,name=Remark,sibling=theorem]{remark}
\declaretheorem[style=thmblackbox,name=Remark,numbered=no]{remark*}
\declaretheorem[style=thmblackbox,name=Problem,numbered=no]{problem*}
\declaretheorem[style=thmblackbox,name=Question,sibling=theorem]{ques}


\declaretheorem[style=thmmarginbox,name=\begingroup\color{RoyalBlue}\blacktriangleright\endgroup,numbered=no]{marginnotebox}
\declaretheorem[style=thmmarginref,name=\begingroup\color{JungleGreen}\blacksquare\endgroup,numbered=no]{marginrefbox}


\declaretheoremstyle[
headfont=\color{blue!40!black}\normalfont\bfseries,
spaceabove=8pt,
spacebelow=8pt,
bodyfont=\normalfont
]{basehead}

\declaretheoremstyle[spaceabove=6pt,spacebelow=6pt]{basehead}


\declaretheorem[style=basehead,name=Answer,sibling=theorem]{answer}
\declaretheorem[style=basehead,name=Answer,numbered=no]{answer*}
\declaretheorem[style=basehead,name=Proposition,sibling=theorem]{plainprop}
\declaretheorem[style=basehead,name=Proposition,numbered=no]{plainprop*}
\declaretheorem[style=basehead,name=Theorem,sibling=theorem]{plaintheo}
\declaretheorem[style=basehead,name=Theorem,numbered=no]{plaintheo*}
\declaretheorem[style=basehead,name=Lemma,sibling=theorem]{plainlem}
\declaretheorem[style=basehead,name=Lemma,numbered=no]{plainlem*}


\declaretheorem[style=thmpurplebox,name=Conjecture,sibling=theorem]{conjecture}
\declaretheorem[style=thmpurplebox,name=Conjecture,numbered=no]{conjecture*}
\declaretheorem[style=thmpurplebox,name=Definition,sibling=theorem]{definition}
\declaretheorem[style=thmpurplebox,name=Definition,numbered=no]{definition*}

\declaretheorem[style=basehead,name=Exercise,sibling=theorem]{exercise}
\declaretheorem[style=basehead,name=Exercise,numbered=no]{exercise*}
\declaretheorem[style=basehead,name=Fact,sibling=theorem]{fact}
\declaretheorem[style=basehead,name=Fact,numbered=no]{fact*}
\declaretheorem[style=basehead,name=Problem,sibling=theorem]{problem}
\declaretheorem[style=basehead,name=Question,numbered=no]{ques*}

\Crefname{answer}{Answer}{Answers}
\Crefname{assume}{Assumption}{Assumptions}
\Crefname{claim}{Claim}{Claims}
\Crefname{conjecture}{Conjecture}{Conjectures}
\Crefname{exercise}{Exercise}{Exercises}
\Crefname{fact}{Fact}{Facts}
\Crefname{problem}{Problem}{Problems}
\Crefname{ques}{Question}{Questions}

\newcommand{\motiv}[1]{
    \emph{{\color{red} Motivation:} #1} \par\medskip
}
\newenvironment{moral}{%
    \begin{mdframed}[linecolor=green!70!black]%
        \bfseries\color{green!50!black}}%
    {\end{mdframed}}

    \usepackage{exsheets}
%Modified from Evan's evan.sty.
%
%Boost Software License - Version 1.0 - August 17th, 2003
%
%Permission is hereby granted, free of charge, to any person or organization
%obtaining a copy of the software and accompanying documentation covered by
%this license (the "Software") to use, reproduce, display, distribute,
%execute, and transmit the Software, and to prepare derivative works of the
%Software, and to permit third-parties to whom the Software is furnished to
%do so, all subject to the following:

%The copyright notices in the Software and this entire statement, including
%the above license grant, this restriction and the following disclaimer,
%must be included in all copies of the Software, in whole or in part, and
%all derivative works of the Software, unless such copies or derivative
%works are solely in the form of machine-executable object code generated by
%a source language processor.

%THE SOFTWARE IS PROVIDED "AS IS", WITHOUT WARRANTY OF ANY KIND, EXPRESS OR
%IMPLIED, INCLUDING BUT NOT LIMITED TO THE WARRANTIES OF MERCHANTABILITY,
%FITNESS FOR A PARTICULAR PURPOSE, TITLE AND NON-INFRINGEMENT. IN NO EVENT
%SHALL THE COPYRIGHT HOLDERS OR ANYONE DISTRIBUTING THE SOFTWARE BE LIABLE
%FOR ANY DAMAGES OR OTHER LIABILITY, WHETHER IN CONTRACT, TORT OR OTHERWISE,
%ARISING FROM, OUT OF OR IN CONNECTION WITH THE SOFTWARE OR THE USE OR OTHER
%DEALINGS IN THE SOFTWARE.

%use l instead of ell in math mode.
\mathcode`l="8000
\begingroup
\makeatletter
\lccode`\~=`\l
\DeclareMathSymbol{\lsb@l}{\mathalpha}{letters}{`l}
\lowercase{\gdef~{\ifnum\the\mathgroup=\m@ne \ell \else \lsb@l \fi}}%
\endgroup

\usepackage{xargs}
\usepackage{centernot}
\usepackage{mathtools}
\usepackage{tikz-cd}

\renewcommand{\mathbb}[1]{\mathds{#1}}

\newcommand{\increment}{\Delta}
\renewcommand{\vec}[1]{\boldsymbol{\mathbf{#1}}}
\renewcommand{\Vec}[1]{\mathbf{#1}}
\newcommand{\unitv}[1]{\vec{\hat{#1}}}
\let\conjugate\overline
\newcommand{\conj}[1]{\overline{#1}}
\newcommand{\seq}[2]{\left\langle#1_1, #1_2, \dots, #1_{#2}\right\rangle}
\newcommand{\seqq}[1]{\left\langle#1\right\rangle}

\usepackage{braket}
\renewcommand*{\Set}[1]{\left\{#1\right\}}
\newcommand{\given}{\mid}
\newcommand{\cgiven}{:}

\newcommand{\avg}[1]{\langle#1\rangle}
\newcommand{\bavg}[1]{\overline{#1}}
\newcommand{\notiff}{%
  \mathrel{{\ooalign{\hidewidth$\not\phantom{"}$\hidewidth\cr$\iff$}}}}
  
\newcommand{\listvec}[2]{\left(#1_{1}, #1_{2}, \dots, #1_{#2}\right)}
\newcommand{\many}[2]{#1_{1}#1_{2}\cdots#1_{#2}}
\newcommand{\manys}[2]{\{#1_{1},#1_{2}\ldots,#1_{#2}\}}
\newcommand{\cbrt}[1]{\sqrt[3]{#1}}
\newcommand{\floor}[1]{\left\lfloor #1 \right\rfloor}
\newcommand{\ceiling}[1]{\left\lceil #1 \right\rceil}
\newcommand{\mailto}[1]{\href{mailto:#1}{\texttt{#1}}}
\newcommand{\ol}{\overline}
\newcommand{\ul}{\underline}
\newcommand{\wt}{\widetilde}
\newcommand{\wh}{\widehat}
\newcommand{\eps}{\varepsilon}
\newcommand{\vocab}[1]{\sffamily #1}
\providecommand{\alert}{\vocab}
\providecommand{\half}{\frac{1}{2}}
\newcommand{\catname}{\mathsf}
\newcommand{\hrulebar}{
    \par\hspace{\fill}\rule{0.95\linewidth}{.7pt}\hspace{\fill}
    \par\nointerlineskip \vspace{\baselineskip}
}
\providecommand{\arc}[1]{\wideparen{#1}}

%For use in author command
\newcommand{\plusemail}[1]{\\ \normalfont \texttt{\mailto{#1}}}

%More commands and math operators
\DeclareMathOperator{\cis}{cis}
\DeclareMathOperator*{\lcm}{lcm}
\DeclareMathOperator*{\argmin}{arg min}
\DeclareMathOperator*{\argmax}{arg max}

%Convenient Environments
\newenvironment{soln}{\begin{proof}[Solution]}{\end{proof} \hrule}
\newenvironment{parlist}{\begin{inparaenum}[(i)]}{\end{inparaenum}}
\newenvironment{gobble}{\setbox\z@\vbox\bgroup}{\egroup}

%Inequalities
\newcommand{\cycsum}{\sum_{\mathrm{cyc}}}
\newcommand{\symsum}{\sum_{\mathrm{sym}}}
\newcommand{\cycprod}{\prod_{\mathrm{cyc}}}
\newcommand{\symprod}{\prod_{\mathrm{sym}}}

%From H113 "Introduction to Abstract Algebra" at UC Berkeley
\newcommand{\CC}{\mathbb C}
\newcommand{\FF}{\mathbb F}
\newcommand{\NN}{\mathbb N}
\newcommand{\NNO}{\mathbb N_{0}}
\newcommand{\ZZO}{\mathbb Z_{\ge 0}}
\newcommand{\RRO}{\mathbb R_{\ge 0}}
\newcommand{\QQ}{\mathbb Q}
\newcommand{\RR}{\mathbb R}
\newcommand{\ZZ}{\mathbb Z}

\newcommand{\charin}{\text{ char }}
\DeclareMathOperator{\sign}{sign}
\DeclareMathOperator{\Aut}{Aut}
\DeclareMathOperator{\Inn}{Inn}
\DeclareMathOperator{\Syl}{Syl}
\DeclareMathOperator{\Gal}{Gal}
\DeclareMathOperator{\GL}{GL} % General linear group
\DeclareMathOperator{\SL}{SL} % Special linear group
\DeclareMathOperator{\Vol}{Vol} % Special linear group

%From Kiran Kedlaya's "Geometry Unbound"
\newcommand{\dang}{\measuredangle} %% Directed angle
\newcommand{\ray}[1]{\overrightarrow{#1}}
\newcommand{\seg}[1]{\overline{#1}}

\newcommand{\comp}[1]{\widebar{#1}}
\newcommand{\ndiv}[1]{\dot{#1}}
\newcommand{\nddiv}[1]{\ddot{#1}}

%From M275 "Topology" at SJSU
\DeclareMathOperator{\id}{id}
\newcommand{\taking}[1]{\xrightarrow{#1}}
\newcommand{\inv}{^{-1}}

%From M170 "Introduction to Graph Theory" at SJSU
\DeclareMathOperator{\diam}{diam}
\DeclareMathOperator{\ord}{ord}
\newcommand{\defeq}{\overset{\mathrm{def}}{=}}

%From the USAMO .tex files
\newcommand{\ts}{\textsuperscript}
\newcommand{\dg}{^\circ}
\newcommand{\ii}{\item}

% From Math 55 and Math 145 at Harvard
\newenvironment{subproof}[1][Proof]{%
    \begin{proof}[#1] \renewcommand{\qedsymbol}{$\blacksquare$}}%
    {\end{proof}}

\newcommand{\liff}{\leftrightarrow}
\newcommand{\lthen}{\rightarrow}

\DeclareMathOperator{\Img}{Im} % Image
\DeclareMathOperator{\coker}{coker} % Cokernel
\DeclareMathOperator{\Coker}{Coker} % Cokernel
\DeclareMathOperator{\Ker}{Ker} % Kernel
\DeclareMathOperator{\Spec}{Spec} % spectrum
\DeclareMathOperator{\pr}{pr} % projection
\DeclareMathOperator{\ext}{ext} % extension
\DeclareMathOperator{\pred}{pred} % predecessor
\DeclareMathOperator{\dom}{dom} % domain
\DeclareMathOperator{\ran}{ran} % range
\DeclareMathOperator{\Hom}{Hom} % homomorphism
\DeclareMathOperator{\Mor}{Mor} % morphisms
\DeclareMathOperator{\End}{End} % endomorphism

% Things Lie
\newcommand{\kb}{\mathfrak b}
\newcommand{\kg}{\mathfrak g}
\newcommand{\kh}{\mathfrak h}
\newcommand{\kn}{\mathfrak n}
\newcommand{\ku}{\mathfrak u}
\newcommand{\kz}{\mathfrak z}
\DeclareMathOperator{\Ext}{Ext} % Ext functor
\DeclareMathOperator{\Tor}{Tor} % Tor functor
\newcommand{\SC}{\mathcal{S}}
\newcommand{\SCF}{\mathscr F}
\newcommand{\SCG}{\mathscr G}
\newcommand{\SCH}{\mathscr H}

% Mathfrak primes
\newcommand{\km}{\mathfrak m}
\newcommand{\kp}{\mathfrak p}
\newcommand{\kq}{\mathfrak q}


%aliases
\renewcommand{\ge}{\geqslant}
\renewcommand{\le}{\leqslant}
\renewcommand{\subset}{\subsetneq}
\newcommand{\auth}[1]{\emph{#1}}
\newcommand{\para}[1]{#1 \par}
\newcommand{\lpara}[1]{\par}
\newcommand{\parbreak}{\smallskip}

%some stuff

%geo
\newcommand{\rantri}{\tkzDefPoints{0/1/A,4/3/B,5/1/C}%
\tkzDrawPolygon(A,B,C)}
\newcommand{\coor}[2]{\tkzDefPoint(#1){#2}}
\newcommand{\Triangle}[1]{\tkzDrawPolygon(#1)}
\newcommand{\polygon}[1]{\tkzDrawPolygon(#1)}



\newcommand{\Line}[1]{\tkzDrawSegment(#1)}

\newcommand{\equi}[1]{\tkzDefTriangle[equilateral](#1)}
\newcommand{\twoang}[2]{\tkzDefTriangle[two angles = #1](#2)}
\newcommand{\isoright}[1]{\tkzDefTriangle[isosceles right](#1)}
\newcommand{\getp}[1]{\tkzGetPoint{#1}}

\newcommand{\centroid}[2]{\tkzDefTriangleCenter[centroid](#1)
\tkzGetPoint{G}\tkzDrawPoints(G)\tkzLabelPoints[#2](G)}

\newcommand{\incentre}[2]{\tkzDefCircle[in](#1) \tkzGetPoints{I}{a}
\tkzDrawPoints(I)\tkzLabelPoints[#2](I)}

\newcommand{\incircle}[1]{\tkzDefCircle[in](#1) \tkzGetPoints{I}{a}
\tkzDrawCircle(I,a)}

\newcommand{\circumcentre}[2]{\tkzDefCircle[circum](#1) 
\tkzGetPoint{O}\tkzDrawPoints(I)\tkzLabelPoints[#2](I)}

\newcommand{\circumcircle}[2]{\tkzDefCircle[circum](#1) \tkzGetPoint{O}
\tkzDrawCircle(O,#2)}

\newcommand{\orthocentre}[2]{\tkzDefTriangleCenter[ortho](#1)
\tkzGetPoint{H}\tkzDrawPoints(H)\tkzLabelPoints[#2](H)}

\newcommand{\orthopoints}[3]{\tkzDefSpcTriangle[orthic](#1,#2,#3){H_#1,H_#2,H_#3}}
\newcommand{\ortho}[3]{\tkzDefSpcTriangle[orthic](#1,#2,#3){H_#1,H_#2,H_#3}
\tkzDrawSegments(#1,H_#1 #2,H_#2 #3,H_#3)
\tkzMarkRightAngles[fill=gray!20,
opacity=.5](#1,H_#1,#3 #2,H_#2,#1 #3,H_#3,#1)}

\newcommand{\rightang}[1]{\tkzMarkRightAngles[fill=gray!20,
opacity=.5](#1)}

\newcommand{\drawsquare}[1]{\tkzDefSquare(#1)
\tkzDrawPolygon(#1,tkzFirstPointResult,%
tkzSecondPointResult)}

\newcommand{\angname}[2]{\tkzLabelAngle[pos=1](#2){$#1$}}
\newcommand{\foot}[3]{\tkzDefLine[perpendicular=through #1,K=-.5](#2,#3)\tkzGetPoint{c}
\tkzDefPointBy[projection=onto #2--#3](c)\tkzGetPoint{h}}

\newcommand{\project}[3]{\tkzDefPointBy[projection=onto #2](#1) \tkzGetPoint{#3}}

\newcommandx{\empangle}[4][1=0.5,2=black,3=|]{\tkzMarkAngle[size=#1,color=#2,mark=#3](#4)}
\newcommandx{\vertice}[2][1=left]{\tkzDrawPoints(#2)\tkzLabelPoints[#1](#2)}

\newcommandx{\fillangle}[3][1=orange]{\tkzDrawSector[R with nodes,fill=#1!20](#2,0.25)(#3)}

\newcommand*\len[1]{\overline{#1}}

%framed
\mdfdefinestyle{MyFrame}{%
    linecolor=black,
    outerlinewidth=0.05pt,
    %roundcorner=20pt,
    %backgroundcolor=gray!50!white}
        }

\newcommand\header[1]{
  \newlength{\headerwidth}
  \setlength{\headerwidth}{\widthof{#1}}
  \addtolength{\headerwidth}{8pt}
  \begin{mdframed}[style=MyFrame,userdefinedwidth=\headerwidth]
    #1
  \end{mdframed}
}

\makeatletter
\let\save@mathaccent\mathaccent
\newcommand*\if@single[3]{%
  \setbox0\hbox{${\mathaccent"0362{#1}}^H$}%
  \setbox2\hbox{${\mathaccent"0362{\kern0pt#1}}^H$}%
  \ifdim\ht0=\ht2 #3\else #2\fi
  }
%The bar will be moved to the right by a half of \macc@kerna, which is computed by amsmath:
\newcommand*\rel@kern[1]{\kern#1\dimexpr\macc@kerna}
%If there's a superscript following the bar, then no negative kern may follow the bar;
%an additional {} makes sure that the superscript is high enough in this case:
\newcommand*\widebar[1]{\@ifnextchar^{{\wide@bar{#1}{0}}}{\wide@bar{#1}{1}}}
%Use a separate algorithm for single symbols:
\newcommand*\wide@bar[2]{\if@single{#1}{\wide@bar@{#1}{#2}{1}}{\wide@bar@{#1}{#2}{2}}}
\newcommand*\wide@bar@[3]{%
  \begingroup
  \def\mathaccent##1##2{%
%Enable nesting of accents:
    \let\mathaccent\save@mathaccent
%If there's more than a single symbol, use the first character instead (see below):
    \if#32 \let\macc@nucleus\first@char \fi
%Determine the italic correction:
    \setbox\z@\hbox{$\macc@style{\macc@nucleus}_{}$}%
    \setbox\tw@\hbox{$\macc@style{\macc@nucleus}{}_{}$}%
    \dimen@\wd\tw@
    \advance\dimen@-\wd\z@
%Now \dimen@ is the italic correction of the symbol.
    \divide\dimen@ 3
    \@tempdima\wd\tw@
    \advance\@tempdima-\scriptspace
%Now \@tempdima is the width of the symbol.
    \divide\@tempdima 10
    \advance\dimen@-\@tempdima
%Now \dimen@ = (italic correction / 3) - (Breite / 10)
    \ifdim\dimen@>\z@ \dimen@0pt\fi
%The bar will be shortened in the case \dimen@<0 !
    \rel@kern{0.6}\kern-\dimen@
    \if#31
      \overline{\rel@kern{-0.6}\kern\dimen@\macc@nucleus\rel@kern{0.4}\kern\dimen@}%
      \advance\dimen@0.4\dimexpr\macc@kerna
%Place the combined final kern (-\dimen@) if it is >0 or if a superscript follows:
      \let\final@kern#2%
      \ifdim\dimen@<\z@ \let\final@kern1\fi
      \if\final@kern1 \kern-\dimen@\fi
    \else
      \overline{\rel@kern{-0.6}\kern\dimen@#1}%
    \fi
  }%
  \macc@depth\@ne
  \let\math@bgroup\@empty \let\math@egroup\macc@set@skewchar
  \mathsurround\z@ \frozen@everymath{\mathgroup\macc@group\relax}%
  \macc@set@skewchar\relax
  \let\mathaccentV\macc@nested@a
%The following initialises \macc@kerna and calls \mathaccent:
  \if#31
    \macc@nested@a\relax111{#1}%
  \else
%If the argument consists of more than one symbol, and if the first token is
%a letter, use that letter for the computations:
    \def\gobble@till@marker##1\endmarker{}%
    \futurelet\first@char\gobble@till@marker#1\endmarker
    \ifcat\noexpand\first@char A\else
      \def\first@char{}%
    \fi
    \macc@nested@a\relax111{\first@char}%
  \fi
  \endgroup
}
\makeatother




\newcommand{\irrev}[1]{%
    \Ifthispageodd{%
    \reversemarginpar\marginpar{\RaggedLeft\large \bfseries \color{purple}Extra}\normalmarginpar}{%
    \reversemarginpar\marginpar{\RaggedLeft\large \bfseries \color{purple}Extra}\normalmarginpar}%
    }


\usepackage{scrextend}

\definecolor{boldcolor}{gray}{0.18} % range from [0,1]
\newcommand{\lightbold}[1]{\textcolor{boldcolor}{#1}}

\newcommand{\cautionmark}{{\Huge\color{red}!}}
\newlist{Caution}{enumerate}{1}
\setlist[Caution]{label=\raisebox{-0.5cm}[0pt][0pt]{\cautionmark},leftmargin=1cm}

\newcommand{\alignedmarginpar}[1]{%
    \Ifthispageodd{%
        \marginpar{\RaggedRight#1}}{%
        \marginpar{\RaggedLeft#1}}%
    }


\newcommand{\caution}[1]{  
  \alignedmarginpar{\bigskip \cautionmark}
  
  \begin{mdframed}[linecolor=red!70!black]%
      \bfseries\color{red!50!black}%
      #1
    \end{mdframed}
}

%From Knzhou

\newcommand{\union}{\cup}
\newcommand{\intersect}{\cap}
\newcommand{\subgr}{\subseteq}
\newcommand{\subr}{\subseteq}
\newcommand{\nsubgr}{\trianglelefteq} % normal subgroup
\newcommand{\dunion}{\sqcup}
\newcommand{\incl}{\iota}
\renewcommand{\mod}{\, \mathrm{mod}\, } % modular arithmetic
\newcommand{\sdprod}{\rtimes} % semidirect product

\def\rcurs{{\mbox{$\resizebox{.09in}{.08in}{\includegraphics[trim= 1em 0 14em 0,clip]{../script_r/ScriptR.pdf}}$}}}
\def\brcurs{{\mbox{$\resizebox{.09in}{.08in}{\includegraphics[trim= 1em 0 14em 0,clip]{../script_r/BoldR.pdf}}$}}}
\def\hrcurs{{\mbox{$\hat \brcurs$}}}

\newcommand*\widefbox[1]{\fbox{\hspace{2em}#1\hspace{2em}}}

\renewcommand*{\Re}{\mathfrak{R}}
\renewcommand*{\Im}{\mathfrak{I}}

\newcommand{\spart}[1]{\newgeometry{left=2cm,right=2cm} \part{#1} \restoregeometry} 

\renewcommand*{\vocab}[1]{{\sffamily#1}\index{#1}}
\newcommand{\chvocab}[2]{{\sffamily#2}\index{#1!#2}}
\newcommand{\vv}{\vec{v}}
\newcommand{\oo}{\vec{0}}
\newcommand{\Aa}{\vec{A}}
\newcommand{\BB}{\vec{B}}
\newcommand{\uu}{\vec{u}}
\newcommand{\ww}{\vec{w}}

\newcommand{\dtp}{\dotproduct}

\newcommand{\uch}{\,{\text{u}}}
\renewcommand{\dd}{\mathop{}\!{d}}
\renewcommand{\diffd}{\mathop{}\!{d}}

\newcommand{\sidenote}[1]{%
    \marginpar{\RaggedRight \itshape #1 }}

\newcommand{\marginnote}[1]{\sidenote{#1}}
\newcommand{\marginref}[1]{\sidenote{#1}}

\captionsetup[figure]{labelfont={bf},name={Fig.},labelsep=period}

\renewcommand{\figurename}{\normalsize\textbf{Fig.}}
\makeatletter
\makeatletter
\renewcommand{\@makecaption}[2]{%
  \vskip\abovecaptionskip
  \sbox\@tempboxa{#1. {\normalfont #2}}% Add a period after the figure name
  \ifdim \wd\@tempboxa >\hsize
    #1. \RaggedRight #2\par % Apply \RaggedRight here
  \else
    \global \@minipagefalse
    \hb@xt@\hsize{\hfil\box\@tempboxa\hfil}%
  \fi
  \vskip\belowcaptionskip
}
\makeatother
\addbibresource{resources.bib}

\author{
    Adyansh Mishra \\
    \mailto{adyanshmishra@proton.me}
}\date{\today}
\title{An Olympiad Notebook}


\begin{document}
%https://tex.stackexchange.com/questions/249475/index-hyperlink-not-pointing-to-correct-page
\pagenumbering{roman}

\begin{titlingpage}
    \BgThispage
    \newgeometry{left=1cm,right=6cm,bottom=2cm}
    \vspace*{0.4\textheight}
    \noindent
    \textcolor{white}{\Huge\textbf{\textsf{\thetitle}}}
    \vspace*{2cm}\par
    \noindent
    \begin{minipage}{0.35\linewidth}
        \begin{flushright}
            \printauthor
        \end{flushright}
    \end{minipage} \hspace{15pt}
    %
    \begin{minipage}{0.02\linewidth}
        \rule{1pt}{175pt}
    \end{minipage} \hspace{-10pt}
    %
    \begin{minipage}{0.63\linewidth}
    \vspace{5pt}
        \begin{abstract} 
    An abstract is a brief summary of a research article, thesis, review, conference proceeding or any in-depth analysis of a particular subject or discipline, and is often used to help the reader quickly ascertain the paper's purpose. When used, an abstract always appears at the beginning of a manuscript, acting as the point-of-entry for any given scientific paper or patent application. Abstracting and indexing services for various academic disciplines are aimed at compiling a body of literature for that particular subject.
        \end{abstract}
    \end{minipage}
\end{titlingpage}
    \restoregeometry

\frontmatter
    \fncytitle    
    \newgeometry{right=1.5cm,left=1.5cm}
    \renewcommand*{\contentsname}{Short contents}
    \setcounter{tocdepth}{0}% chapters and above
    \tableofcontents
    \clearpage
    \renewcommand*{\contentsname}{Contents}
    \setcounter{tocdepth}{1}% subsections and above
    \tableofcontents
    
    \clearpage
    \listoffigures
    \restoregeometry
    \pagestyle{mystyle}
    \chapter{Preface}
\label{preface}

\section{The Philosophy}

\para{This is a collection of notes written by me on all of the four subjects included in mainstream olympiads. 
Much of the contents is sourced from a various plethora of free and paid books.}

\parbreak

\para{This is not meant to be a one-spot go to for all your Olympiad preparation, 
even though I have tried to include every bit of information I could get my hands on. 
One important part that is missing from it are problems. 
The Primary reason for that being my laziness and the fact and these are meant to be my \emph{notes}, 
which shall capture my Olympiad journey. This is not an olympiad go-to book.
Be aware}

\section{Acknowledgments}

\para{First and foremost, my greatest thanks goes to Evan Chen for his \href{https://github.com/vEnhance/dotfiles/blob/main/texmf/tex/latex/evan/evan.sty}%
{awesome style file}, parts
of which I have modified and used in my notes. He has also been a source of much inspiration,
and for the EGMO \cite{egmo} book, upon which much of the Geometry is based.}

\para{I also must thank Aditya Khurmi for his freely available book MONT \cite{mont},
Although he did not have any particular ivolvement here, I must also thank Kevin Zhou, for 
inspiring me.}

\section{Notation}

\begin{itemize}
    \ii  \(\NN = \{1, 2, 3 \ldots \}\)
    \ii \(\NN_0 =\{0, 1, 2, \ldots\}\)
\end{itemize}

\clearpage
\thispagestyle{empty}
    \vspace*{\fill}
    {\em \centering \hspace{3cm} To her, whose death still haunts me.}
    \vspace*{\fill}


    \mainmatter 
    \spart{Number Theory}
    \chapter{Divisibility}

\section{Basic Properties}

\begin{definition}
	[Division]
	The operation division is defined as:
	\begin{enumerate}
		\item An integer \(y\) is said to be a multiple of \(x\) if \[y\in \{kx : k\in \ZZ \} \]
		\item If \(y\) is a multiple of \(x\) then, \(y\) is \textit{divisible} by \(x\) and \(x\) divides \(y\) which is written as: \[x \mid y.\]
	\end{enumerate}
\end{definition}

\begin{proposition}
	Let \(x\), \(y\), \(z\) \(\in\) \(\ZZ\)
	\begin{enumerate}
		\item \(\forall x \ne 0\), \(x \mid x\) and \(x \mid 0\).
		\item \(1 \mid x\).
		\item \(x\mid y\) and \(y \mid z \implies \) \(x \mid z\).
		\item \(x \mid y \iff \dfrac{y}{x} \in \ZZ\)
	\end{enumerate}
	
	
\end{proposition}

Most of the proofs here are trivial, but let us show that using the last proposition we can prove the third one.

\begin{proof}
	\( x \mid y \iff \dfrac{y}{x} \in \ZZ \) and \(  y \mid z \iff \dfrac{z}{y} \in \ZZ \). But \( \dfrac{z}{x}= \dfrac{y}{x}\cdot\dfrac{z}{y} \in \ZZ \iff x\mid z\).
\end{proof}


We may note that because of the definition of division, we get the following theorem:

\begin{theorem}
	If \(x \mid y\), then either \(\abs{y} \ge \abs{x} \) or \(y=0\).
\end{theorem}

The \(0\) case is extremely important and often hard to see. Remember to look out for this case.

\begin{proof}
	\(\abs{y}=\abs{k\cdot x}= \abs{k}\cdot \abs{x}\). But, \( \abs{k}\cdot \abs{x} \ge \abs{x} \iff \abs{y} \ge \abs{x}\). Note here that \(k\) is not taken to be \(0\) since, \(y=k\cdot x=0\cdot x=0\). But this case is already mentioned.
\end{proof}



The following is not of a severe importance in the view that it is hard to see or prove. But it allows us to quickly formalise our solutions and is thus useful for proofs.

\begin{lemma}
	\label{lem: unimp}
	If \(x \mid a\), then \(x \mid ac+b \iff x\mid b\).
\end{lemma}

\begin{proof}
	The proof is trivial as, \(x \mid a \iff a = kx : k \in \ZZ\).
	\par 
	\textbf{Direction 1}
	Note, \(x \mid ac+b \iff ac+b=nx : n \in \ZZ\), hence, \(b=nx-ac=nx-kcx=(n-kc)x \iff x \mid b\).
	\par
	\textbf{Direction 2}
	And, \(x\mid b \iff b=mx : m \in \ZZ \), hence, \(ac+b=kcx+mx=(kc+m)x \iff x \mid ac+b \).
\end{proof}

Finally, we may conclude this section of properties by stating all the previous properties and some more. 
\begin{theorem}
	Let \(x\), \(y\), \(z\) \(\in\) \(\ZZ\)
	\begin{enumerate}
		\item \(\forall x \ne 0\), \(x \mid x\) and \(x \mid 0\).
		\item \(1 \mid x\).
		\item \(x\mid y\) and \(y \mid z \implies \) \(x \mid z\).
		\item \(x \mid y \iff \dfrac{y}{x} \in \ZZ\).
		\item \(\forall z \ne 0\), \(x \mid y \iff xz \mid yz\). 
		\item \(x\mid a\), \(x\mid b \implies x\mid an+bm\).
		\item \(x\mid y\), \(y\mid x \iff y=\pm x\).
		\item \(x \mid y \implies x \mid yz\).
		
	\end{enumerate}
	
\end{theorem}

\begin{example}
	Show that if \(n \in \ZZ, n > 1\), then \(n \nmid 2n^2 + 3n + 1\).
	\begin{proof}
		Here we can see the use of \Cref{lem: unimp}. Suppose \(n \mid 2n^2+3n+2\), then \(n \mid 2n^2 + 3n\), and \(n \mid 2n^2 + 3n + 1 \iff n \mid 1\). But, \( n \mid 1 \implies \abs{1} \ge \abs{n} \iff 1 \ge n\) which contradicts our earlier statement that \(n > 1\). Hence, \(n \nmid 2n^2+3n+1\).
	\end{proof}
	
	
\end{example}

\begin{remark}
	\( \abs{1} \ge \abs{n} \iff 1 \ge n\) because \(n>1>0\).
\end{remark}


\begin{example}
	Show that for any two natural numbers, \(a\), \(b : a > b\), \(a \nmid 2a+b\).
	
	\begin{proof}
		Again, we can use \Cref{lem: unimp}. Suppose \(a \mid 2a+b\), then \(a \mid 2a\) and \(a \mid 2a + b \iff a \mid b\). But, \(a \mid b \implies \abs{b} \ge \abs{a} \) and \(a, b \in \NN \implies b \ge a \). This contradicts our the fact \(a > b\). Hence \(a \nmid 2a+b\).
	\end{proof}
\end{example}

\section{Euclid's Division Lemma}

Euclid's division lemma is something we will make much use of, especially later. For now the unproven statement shall suffice.

\begin{theorem}
	\label{thm: euclid}
	For any integers \(a\), \(b\), we can find \textit{unique} integers \(q, r\), such that: \[b=aq + r, \quad 0 \le r <a. \]
	Here, \(r\) is called the remainder, and \(q\) is called the quotient.
\end{theorem}

\section{Primes}

\index{Primes}

\begin{definition}
	[Primes]
	Any \(x \in \NN\) is called a prime, if and only if it has exactly \(2\) \textit{divisors}.
\end{definition}

Primes are a very interesting subset of naturals. There properties will be explored later. For now, we use primes to define one of the most important theorem in all of number theory.

\subsection{Fundamental Theorem of Arithmetic}

\begin{theorem}
	[Fundamental Theorem of Arithmetic]
	\label{thm: fta}
	Any natural number can be \textit{uniquely} expressed as a product of primes upto order.
\end{theorem}

By uniquely, we mean there is \textit{at most} and \textit{at least} one way to express any natural as a product of primes. The order in which the product is expressed is irrelevant because of the commutativity of multiplication. \par 
We can further extend this theorem to include all integers by using some claims.

\subsection{Integers as Multisets}

\begin{claim}
	Any non-zero \(x \in \ZZ\) can be expressed a multiset of primes(and \(-1\)). If the number is positive, then we can simply use \Cref{thm: fta} and included all the prime factors in its multiset. If the number is negative, then we express it as the multiset of its absolute value and include \(-1\) in the multiset.
\end{claim}

\begin{remark}
	A \textit{multiset} is a set where repeating elements are counted seperately. Each element has a \textit{mulitplicity} which indicates the number of times that element appears in the multiset.
\end{remark}

We could, for example, express \(20\) as \(\{2, 2, 5\}\), and \(-20\) as \(\{-1, 2, 2, 5\}\).
\\
This allows us to restate \Cref{thm: fta}.

\begin{theorem}
	[Fundamental theorem of Arithmetic in multisets]
	Any natural number can expressed as a multiset of primes.
\end{theorem}

\begin{remark}
	This multiset, is, by definition \textit{unique}.
\end{remark}

It also allows us to restate divisibility for positive integers.

\begin{theorem}
	[Divisibility in multisets]
	\(\forall a, b \in \NN\), \[a \mid b \iff A \subseteq B \]
\end{theorem}

\begin{remark}
	Henceforth, all variables are integers unless explicitly mentioned.
\end{remark}

\section{GCD and LCM}

\subsection{GCD}

\begin{definition}
	\( \gcd(a,b)\) is the multiset of all \textit{common} prime factors of \(a\) and \(b\).
\end{definition}

This gives us the following properties of \(\gcd\).

\begin{proposition}
	\hfill
	\begin{itemize}
		\item \(\gcd(a,b)\) is the greatest integer which divides both \(a\) and \(b\). In particular, \(\gcd(a,b) \le a, b\).
		\item \(c \mid a\), \(c \mid b \iff c \mid \gcd(a,b)\).
	\end{itemize}
\end{proposition}

We can, in fact, show that the two definition of gcd are equivalent. 

\begin{proof}
	Let \( \SC = \{p : p \mid a, b \) and \(p \) is a prime\( \} \). Suppose that \(\gcd(a,b) \ne \SC \). Then \( \gcd(a,b) > \SC \). 
	\\
	\par
	Note that \( \gcd(a,b)\) must contain an element \(x \notin \SC \) where \(x\) is prime.(Since it is an element in the multiset of \(\gcd(a,b)\)). 
	\par
	But, \(x \in \gcd(a,b) \iff x \mid \gcd(a,b)\), and \(x \mid \gcd(a,b) \), \(\gcd(a,b) \mid a, b \implies x \mid a,b\). \(x\) is prime, and \(x \mid a,b\ \implies x \in \SC\). This contradicts our earlier deduction that \(x \notin \SC\). By \textit{reductio ad absurdum}, we have \(\gcd(a,b) = \SC \).
\end{proof}

\begin{claim}
	
	Let \(a = p_{1} ^{\alpha_1} p_{2} ^{\alpha_2} \ldots p_{n}^{\alpha_n}\), \(b = p_{1} ^{\beta_1} p_{2} ^{\beta_2} \ldots p_{n}^{\beta_n}\). Then, \[\gcd(a,b)= p_{1} ^{\min(\beta_1, \alpha_1)} p_{2} ^{\min(\beta_2, \alpha_2)} \ldots p_{n}^{\min(\beta_n, \alpha_n)} \]
	Where, \(\alpha_i, \beta_i \in \NN_0 \).
	
\end{claim}

This is simply a consequence of the definition of \(\gcd\).

\begin{definition}
	[GCD]
	We may now state that the following are equivalent.
	\begin{itemize}
		\ii \( \gcd(a,b) = A \cap B  \)		
		\ii \(\gcd(a,b)\) is the greatest integer which divides both \(a\) and \(b\). In particular, \(\gcd(a,b) \le a, b\).
		\ii Let \(a = p_{1} ^{\alpha_1} p_{2} ^{\alpha_2} \ldots p_{n}^{\alpha_n}\), \(b = p_{1} ^{\beta_1} p_{2} ^{\beta_2} \ldots p_{n}^{\beta_n}\). Then, \[\gcd(a,b)= p_{1} ^{\min(\beta_1, \alpha_1)} p_{2} ^{\min(\beta_2, \alpha_2)} \ldots p_{n}^{\min(\beta_n, \alpha_n)} \]
		Where, \(\alpha_i, \beta_i \in \NN_0 \).
	\end{itemize}
	
\end{definition}

\begin{figure}
	\centering
	\begin{tikzpicture}
		\begin{axis}[funcgraph]
			\addplot [red, domain=-180:180, samples=100]{sin(x)};
		\end{axis}
	\end{tikzpicture}
	\label{sin}
	\caption{\(\sin(x)\)}
\end{figure}


\subsection{LCM}

\begin{definition}
	[LCM]
	The following three definitions are equivalent.
	\begin{itemize}
		\ii \(\lcm(a,b) = A \cup B\)
		\ii \(\lcm(a,b)\) is the least number divisible by both \(a\) and \(b\). Particularly, \(\lcm(a,b) \ge a,b\).
		\ii Let \(a = p_{1} ^{\alpha_1} p_{2} ^{\alpha_2} \ldots p_{n}^{\alpha_n}\), \(b = p_{1} ^{\beta_1} p_{2} ^{\beta_2} \ldots p_{n}^{\beta_n}\). Then, \[\lcm(a,b)= p_{1} ^{\max(\beta_1, \alpha_1)} p_{2} ^{\max(\beta_2, \alpha_2)} \ldots p_{n}^{\max(\beta_n, \alpha_n)} \]
	\end{itemize}
	
	
\end{definition}


From the definitions of \(\gcd\) and \(\lcm\) we obtain the following very important theorem.

\begin{theorem}
	\[\lcm(a,b)\cdot \gcd(a,b) = a\cdot b \]
\end{theorem}

\begin{proof}
	
	
	\begin{align*}
		\gcd(a,b) \cdot \lcm(a,b) &= p_{1}^{\min(\alpha_1,\beta_1) + max(\alpha_1,\beta_1)} \ldots p_{n}^{(\max(\alpha_n, \beta_n) + \min(\alpha_n, \beta_n))}\\
		&= p_{1}^{\alpha_1 + \beta_1} \ldots p_2^{\alpha_n + \beta_n}\\
		&= a\cdot b
	\end{align*}	
	
	
\end{proof}



\section{Co-prime}

\begin{definition}
	Two numbers, \(a, b\) are called co-prime if \(\gcd(a,b)=1\).
\end{definition}

\begin{definition}
	A number \(x\) is pairwise co-prime to \(n\) other numbers \(p_{1}, p_{2} \ldots p_{n}\), if \( \gcd(x, p_1) = \gcd(x, p_2) \ldots = \gcd(x, p_n) =1 \).
\end{definition}

\section{Infinitude of Primes}



\begin{theorem}
	[Euclid]
	There are infinitely many primes.
\end{theorem}

\begin{proof}
	Let there be finitely many primes, \(\manys{p}{n}\). Define \( N \) as \( \many{p}{n} + 1 \).
	\par
	Note that \(N\) is pairwise co-prime to all, \(\many{p}{n}\) because if \(p_i \in \manys{p}{n} \mid N \) and \(p_i \mid \many{p}{n} \iff p_i \mid 1\). But, \( p_i \in \NN \implies p_i = 1 \) which is absurd.
	\\
	\par Since \(N \in \NN\), \(N\) must have a prime factor, \(p\) because of \Cref{thm: fta}. And \(p \notin \manys{p}{n}\) because \(N \) is pairwise co-prime to all primes in \(\manys{p}{n}\). This contradicts our assumption that \(\manys{p}{n}\) was the set of all primes, proving that there are infintely many primes.
	


\end{proof}


\begin{tikzpicture}
	\tkzDefPoints{0/0/A,2/5/B,3/2/C}
	\tkzDrawPolygon(A,B,C)
\end{tikzpicture}

\todo{Add motivation for stuff}





    \chapter{Basics of Modular Arithmetic}
\label{nt: modbasic}
\motiv{Modular arithmetic allows us to expand the notion of equality.}

\section{``Modulo''}

It is very hard to and rarely needed to establish equality in number theory. What may do instead
is the build on the notion of \emph{modular} arithmetic. Consider an analog clock, if it starts
at any arbritary time, then no matter how many rotations it goes through, whenever it points at 
\(1\), the time is \(1\) a.m or p.m. Thus even if we start counting hours, we may note that
\(13 = 1 = 25 = \dots\)                                                                                                                                                                                       
\parbreak

The equality symbol is not apt here. What should be much better used is the symbol of equivalence,
``\(\equiv\)''. Thus, what we mean is \(13 \equiv 1 \pmod{n}\). In general,

\begin{definition}
    [Congruent]
    \(a \equiv b \pmod{n}\) if and only if \(p \mid a - b\). In such a case \(a\) is said to be 
    congruent to \(b\) modulo \(n\).
\end{definition}

  
    
    \spart{Combinatorics}
    
    \appendix
    \spart{Appendices}
    \chapter{Terminology}
\label{ch: term}

\section{Set Theory}
    
    \backmatter
    \printindex
    \printbibliography

\end{document}