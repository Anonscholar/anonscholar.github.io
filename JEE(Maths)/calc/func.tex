\chapter{Functions and Relations}
\section{Relation}

\marginpar{}

\marginpar{Cartesian Product \Cref{sec: ctp}}

\para{A relation between two sets \(A\), \(B\) is the subset of their \emph{cartesian
product} \(A \times B\).}
\para{Since, a cartesian product between \(A\) and \(B\) is also a set of the ordered pairs, so
is a relation \(R\),\[
    R \subseteq \{(a,b) \mid a \in A, b \in B\}
\]}
\para{There need not be any specific relation between the two elements. But often, it is useful
to talk about relations which employ a specific relationship between the two elements.
Hence, we can think of a relation as somehow connecting the two elements.}
\para{A relation on a set \(A\) is simply the subset of \(A \times A\).}

\begin{example}
    Consider sets \(A = \{2, 4, 6, 8, \dots\}\) and \(B = \{1,3,5,7,9,\dots\}\). Let \(R\) be
    the relation such that, \[R = \{(a,b) \mid a,b \text{ are divisible by 3 and}, a \in A, b \in B\}\]
    This relation is simply the set of all multiples of \(3\).
\end{example}

\begin{example}
    \label{ex: reltri}
    Let \(A = \{1,2,8\}\), \(B = \{3,5,6\}\). Let \(R\) be the relation, \[
        R = \{(a,b) \mid a > b, a \in A, b \in B\}
    \]
    Then \(R\) contains the ordered pairs \((8,3)\), \((8,5)\) and \((8,6)\)
\end{example}

\para{We show that \((a,b) \in R\) by saying \(a R b\).}

\begin{figure}[H]
    \centering
    \incfig{relation}
    \caption{Pictorial representation of \Cref{ex: reltri}}
\end{figure}

\subsection{Domain}

\para{The domain of a relation \(R \subseteq A \times B\) is the set of all first elements of 
the ordered pairs of the relation, that are in set \(A\). \[D_R = \{a \mid (a,b) \in R\}\]}

\subsection{Range and Co-domain}

\subsubsection{Range}

\para{Range of \(R\) is the set of all second elements of the ordered pairs in the relation, that
are in set \(B\). \[Range_R = \{b \mid (a,b) \in R\}\]}

\subsubsection{Co-domain}

\marginpar{\raggedleft You may note that \(Range_R \subset Co-domain_R\)}

\para{Co-domain of \(R \subseteq A \times B\) is the set \(B\) itself.} 

\section{Types Of Relations}

\para{Although it might seem like discussing the type of relations would be as trivial as sets,
there are actually a couple really important type of relations. I'll mention the ones
we really need to know about.}

\subsection{Trivial Types}

\para{Mainly, these are trivial in the way that they're of very low importance to us. The ones
that are useful later, have been especially marked.}

\subsubsection{Empty Relation}

\para{The relation \(R\) is an empty relation if it does not contain any elements.
\(R = \varnothing \subset A \times B\).}

\subsubsection{Universal Relation}

\para{Yes, it is the relation that contains all the elements. A universal relation \(R\) on \(A\)
is \(R = A \times A\).}

\subsubsection{Inverse Relation}
\index{Inverse Relation}

\para{Let \(R\) be the relation from \(A\) to \(B\). Then \(R\inv\) is the relation from \(B\)
to A such that, \[R\inv = \left\{ (b,a) \mid (a,b) \in R \right\}\]
Therefore, we may have \((a,b) \in R \iff (b,a) \in R\inv\).}

\subsubsection{Identity Relation}
\index{Identity Relation}

\para{The identity relation \(I_A\) on \(A\) is defined as
\[
    I_A = \left\{(a,a) \mid a \in A\right\}
\]}

\subsection{Equivalence Relation}

\subsubsection{Reflexive relation}

\marginpar{\raggedright All identity relations are reflexive relations but all reflexive relations are
not identity relations. See \Cref{ex: identity}.}

\para{A relation \(R\) is a reflexive relation on \(A\), if for every \(a \in A\),
\((a,a) \in R\).}

\index{Reflexive Relation}

\begin{example}
    \label{ex: identity}
    The relation \(I_A = \left\{(1,1), (2,2), (3,3)\right\}\) is an identity relation on the set
    \(A = \left\{1,2,3\right\}\). \\
    However, the relation \(R = \left\{(1,1), (2,3)(3,2), (2,2), (3,3)\right\}\)
    is not an identity relation on \(A\), but rather a reflexive relation. But \(I_A \subset R\).\par
    Here, \(R\) is a reflexive relation but not an identity relation.
\end{example}

\subsubsection{Symmetric Relation}
\index{Symmetric Relation}

\para{A relation \(R\) on \(A\) is symmetric if \((a,b) \in R \iff (b,a) \in R\).}

\begin{example}
    Let \(R\) be the relation on \(\RR\) such that \[
        R = \left\{(x,y) \mid x + y = 5\right\}
    \]
    This relation is symmetric, because suppose any two reals \(x\) and \(y \in R\). Then,
    \(x + y = 5\). But since addition is commutative, \(x + y = y + x = 5\). 
    Thus, \((x,y) \in R \iff (y,x) \in R\).
\end{example}

\subsubsection{Transitive Relation}
\index{Transitive Relation}

\para{A relation \(R\) on \(A\) is transitive if \((a,b) \in R\) and \((b,c) \in R \implies (a,c) \in R\). }

\begin{example}
    The relation \(<\) is a transitive relation on \(\RR\). For instance, if \(a < b\), \(b < c\),
    then \(a < c\).
\end{example}

\subsubsection{Equivalence Relation}

\para{An equivalence relation on \(A\) is a relation that is transitive, reflexive, and symmetric.}

\marginpar{\raggedright If you haven't already noticed, I have deliberately chosen examples that
are non conventional. I have done it in order to showcase that relations can be anything!}

\begin{example}
    The relation \(=\) is an equivalence relation on \(\RR\). It is symmetric as \(a = b \iff b = a\). 
    It is reflexive as \(a = a\). It is transitive as \(a = b, b = c \implies a = c\).
\end{example}

