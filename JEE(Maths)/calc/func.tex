\chapter{Functions and Relations}
\section{Relation}

\marginpar{}

\marginpar{Cartesian Product \Cref{sec: ctp}}

\para{A relation between two sets \(A\), \(B\) is the subset of their \emph{cartesian
product} \(A \times B\).}
\para{Since, a cartesian product between \(A\) and \(B\) is also a set of the ordered pairs, so
is a relation \(R\),\[
    R \subseteq \{(a,b) \mid a \in A, b \in B\}
\]}
\para{There need not be any specific relation between the two elements. But often, it is useful
to talk about relations which employ a specific relationship between the two elements.
Hence, we can think of a relation as somehow connecting the two elements.}
\para{A relation on a set \(A\) is simply the subset of \(A \times A\).}

\begin{example}
    Consider sets \(A = \{2, 4, 6, 8, \dots\}\) and \(B = \{1,3,5,7,9,\dots\}\). Let \(R\) be
    the relation such that, \[R = \{(a,b) \mid a,b \text{ are divisible by 3 and}, a \in A, b \in B\}\]
    This relation is simply the set of all multiples of \(3\).
\end{example}

\begin{example}
    \label{ex: reltri}
    Let \(A = \{1,2,8\}\), \(B = \{3,5,6\}\). Let \(R\) be the relation, \[
        R = \{(a,b) \mid a > b, a \in A, b \in B\}
    \]
    Then \(R\) contains the ordered pairs \((8,3)\), \((8,5)\) and \((8,6)\)
\end{example}

\para{We show that \((a,b) \in R\) by saying \(a R b\).}

\begin{figure}[H]
    \centering
    \incfig{relation}
    \caption{Pictorial representation of \Cref{ex: reltri}}
\end{figure}

\subsection{Domain}

\para{The domain of a relation \(R \subseteq A \times B\) is the set of all first elements of 
the ordered pairs of the relation, that are in set \(A\). \[D_R = \{a \mid (a,b) \in R\}\]}

\subsection{Range and Co-domain}

\paragraph{Range}

\para{Range of \(R\) is the set of all second elements of the ordered pairs in the relation, that
are in set \(B\). \[Range_R = \{b \mid (a,b) \in R\}\]}

\paragraph{Co-domain}

\para{Co-domain of \(R \subseteq A \times B\) is the set \(B\) itself.} 