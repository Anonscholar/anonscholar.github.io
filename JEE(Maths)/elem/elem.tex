\chapter{Sets and Manipulations}
\epigraph{``You don't have to be great to start, but you have to start
to be great.''}{\auth{Zig Ziglar}}

\section{Naive Set Theory}

\motiv{There is little Mathematics that does not use the language of sets.}

\para{Sets are perhaps the most important structures in all of Mathematics. Although
an indepth discussion of sets is beyond these notes, and their axiomation is complicated, we 
shall not let this affect us and discuss a less rigorous formalisation of sets. 
Namely, \emph{Naive Set Theory}.}

\marginpar{\raggedright These definitions are superficious We have not and will not define what
we mean by collection or elements. That is what makes this theory \emph{naive}.}

\begin{definition}
    A set is any collection of elements.\\
    It is denoted as \(\{a,b,c\}\) where \(a,b, c\) are elements of the set.
\end{definition}

\para{Well, what do we mean by any collection? To give a few examples, the set of flowers
would be \{roses, lilies, sunflowers, \ldots\} The trailing \ldots are used to show
that the elements continue on \emph{ad infinitum}. Hence, any collection of objects is a set.}

\para{Often, instead of manually writing out \{roses, lilies, sunflowers, \ldots\}, we will
use a condition that each element in the set follows. Here, the condition is that all
elements all flowers. We denoted this by saying that \(S\) is the set of all \(x\) such that 
\(x\) is a flower. We represent this in what is called \emph{Set-Builder} notation. Thus,
\[S = \{x : P(x)\} \quad \text{or} \quad S = \{x \mid P(x)\}\] 
Here, \( P(x)\) represents the condition. We use the different symbols depending on the context,
mostly for ensuring that the statements are clear.
}

\marginpar{\raggedright Do not be overwhelmed by terms like ``set builder notation''. They're only formal
names and not something to remember.}

\begin{remark}
    Notice the equality symbol between \(S\) and \( \{x : P(x)\} \). Generally, we denote sets
    as capital latin letters \(A, B, C, \ldots\), and their elements by little latin
    letters \(a, b, c, \ldots\).
\end{remark}

\begin{example}
    The sets of all Natural numbers is denoted by \(\NN\). Generally, it is either used to 
    represent \( \{0, 1, 2, \ldots\} \) or \( \{1, 2, 3, \ldots\} \). It matters little.
    We will let \(\NN = \{1, 2, 3, \ldots\}\) and \(\NNO = \{0, 1, 2, \ldots\}\) in this text 
\end{example}

\begin{example}
    The other common sets are similarly denoted by \emph{blackboard bold} letters as well.
    \begin{enumerate}
        \ii \(\ZZ\) is the set of all integers.
        \ii \(\QQ\) is the set of all rationals.
        \ii \(\RR\) is the set of all reals.
        \ii \(\CC\) is set of all complex numbers.
    \end{enumerate}
\end{example}

\marginpar{The symbol \(\in\) in \(x \in S\) is used to show that \(x\) is an element
of \(S\). It may be read as `in'.}

\begin{example}
    The cartesian plane is also a set. It defined as \[\{(x,y) \mid x, y \in \RR\}\]
\end{example}

\para{The next section is skippable, and I advise you to skip it if you're reviewing or are not interested
in extra stuff at all.}

\subsection{Why Naive?}

\para{One interesting formulation possible in naive set-theory is the ``Russel's Paradox''. 
It goes as follows.
Let \(S\) be a set \(\{x \mid x \notin S\}\). Basically, \(S\) is the set of all elements
not in \(S\). This leads to a paradox because for \(x\) to be in \(S\), it has to follow
the condition that it is not in \(S\)! And any element not in \(S\) must be in \(S\).}
\para{There is no particular solution to this using naive set-theory. This is one of the reasons
why this discussion of sets is so naive.}

\section{Subsets and Equality}