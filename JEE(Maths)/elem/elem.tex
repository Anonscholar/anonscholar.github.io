\chapter{Sets and Manipulations}
\epigraph{``You don't have to be great to start, but you have to start
to be great.''}{\auth{Zig Ziglar}}

\section{Naive Set Theory}

\motiv{There is little Mathematics that does not use the language of sets.}

\para{Sets are perhaps the most important structures in all of Mathematics. Although
an indepth discussion of sets is beyond these notes, and their axiomation is complicated, we 
shall not let this affect us and discuss a less rigorous formalisation of sets. 
Namely, \emph{Naive Set Theory}.}

\marginpar{\raggedright These definitions are superficious We have not and will not define what
we mean by collection or elements. That is what makes this theory \emph{naive}.}

\begin{definition}
    A set is any collection of elements.\\
    It is denoted as \(\{a,b,c\}\) where \(a,b, c\) are elements of the set.
    This representation of sets is called \emph{Roster form}.
\end{definition}

\para{Well, what do we mean by any collection? To give a few examples, the set of flowers
would be \{roses, lilies, sunflowers, \ldots\} The trailing \ldots are used to show
that the elements continue on \emph{ad infinitum}. Hence, any collection of objects is a set.}

\para{Often, instead of manually writing out \{roses, lilies, sunflowers, \ldots\}, we will
use a condition that each element in the set follows. Here, the condition is that all
elements all flowers. We denoted this by saying that \(S\) is the set of all \(x\) such that 
\(x\) is a flower. We represent this in what is called \emph{Set-Builder} notation. Thus,
\[S = \{x : P(x)\} \quad \text{or} \quad S = \{x \mid P(x)\}\] 
Here, \( P(x)\) represents the condition. We use the different symbols depending on the context,
mostly for ensuring that the statements are clear.
}

\marginpar{\raggedright Do not be overwhelmed by terms like ``set-builder''. They're only formal
names and not something to remember.}

\begin{remark}
    Notice the equality symbol between \(S\) and \( \{x : P(x)\} \). Generally, we denote sets
    as capital latin letters \(A, B, C, \ldots\), and their elements by little latin
    letters \(a, b, c, \ldots\).
\end{remark}

\begin{example}
    The sets of all Natural numbers is denoted by \(\NN\). Generally, it is either used to 
    represent \( \{0, 1, 2, \ldots\} \) or \( \{1, 2, 3, \ldots\} \). It matters little.
    We will let \(\NN = \{1, 2, 3, \ldots\}\) and \(\NNO = \{0, 1, 2, \ldots\}\) in this text 
\end{example}

\begin{example}
    The other common sets are similarly denoted by \emph{blackboard bold} letters as well.
    \begin{enumerate}
        \ii \(\ZZ\) is the set of all integers.
        \ii \(\QQ\) is the set of all rationals.
        \ii \(\RR\) is the set of all reals.
        \ii \(\CC\) is set of all complex numbers.
    \end{enumerate}
\end{example}

\marginpar{The symbol \(\in\) in \(x \in S\) is used to show that \(x\) is an element
of \(S\). It may be read as `belongs to'.}

\begin{example}
    The cartesian plane is also a set. It defined as \[\{(x,y) \mid x, y \in \RR\}\]
\end{example}

\subsection{Why Naive?}

\marginpar{\raggedleft \large \bfseries Extra}

\para{One interesting formulation possible in naive set-theory is the ``Russel's Paradox''. 
It goes as follows.
Let \(S\) be a set \(\{x \mid x \notin S\}\). Basically, \(S\) is the set of all elements
not in \(S\). This leads to a paradox because for \(x\) to be in \(S\), it has to follow
the condition that it is not in \(S\)! And any element not in \(S\) must be in \(S\).}
\para{There is no particular solution to this using naive set-theory. This is one of the reasons
why this discussion of sets is so naive.}

\subsection{Types of Sets}

\para{Although this division of sets is superfluous at best, we will still categorise them for 
a reference.}

\marginpar{\raggedleft The set \{\varnothing\} \,is not a null set! In-fact it is the set that contains
the empty or null set.}

\begin{itemize}
    \ii \textbf{Finite Set} : \para{A set having a finite number of elements is called a finite set.}
    \ii \textbf{Infinite Set} : \para{A set which has infinite elements is called an infinite set.}
    \ii \textbf{Null Set} : \para{A set which has no elements is called a null set. It is 
    represented as \{\} or \(\varnothing\).}
    \ii \textbf{Singleton Set} : \para{A set containing a single element is called a singleton set.}    
\end{itemize}

\section{Equality and Subsets}

\subsection{Sets of Sets}
\para{Remember that we referred to sets as a collection of \emph{any} objects? That might lead
to the question that can we also make sets of sets? In-fact, that is exactly right, we can!
A set of sets maybe represented as \(\{\{a,b,c\}, \{d,e,f\}, \{g, h, i\}\}\). All of the
elements of this set are themselves sets. One thing to note here is that we cannot form a set 
of all sets, and we will see why in a moment.}

\subsection{Subsets}
\para{Consider the sets \(\NN\) and \(\ZZ\). You might see that all the elements of \(\NN\) are also 
present in \(\ZZ\). If we were to describe it, we could say that \(\NN\) is contained in \(\ZZ\).
Such a relation between sets is expressed by saying \(\NN\) is a subset of \(\ZZ\).}

\begin{definition}
    A set \(A\) is a subset of \(B\) if all elements of \(A\) are also elements of \(B\).\\
    This is represented as \[A \subseteq B\]
\end{definition}

\begin{example}
    \(\NN \subseteq \ZZ \subseteq \QQ \subseteq \RR \subseteq \CC.\)
\end{example}

\begin{example}
    Let \(A = \{x \mid x \text{ is prime and } x \text{ is odd}\}\), and \(B = \{x \mid x \text{
        is prime}\}\), then \(A \subseteq B\). 
\end{example}

\para{One thing to note here is that for \(A\) to be a subset of \(B\), it does not need to 
have less elements than \(B\). Infact, \(A\) is a subset of \(A\) itself because all elements of 
\(A\) are in \(A\).}
\para{To show that each element of \(A\) is contained in \(B\) but all elements of \(B\) are not
contained in \(A\), we say that \(A\) is a proper subset of \(B\).}

\marginpar{\raggedright Note that \(A\) is not a proper subset of \(A\).}

\begin{definition}
    \(A\) is a proper subset of \(B\) if all elements of \(A\) are in \(B\) but all elements of
    \(B\) are not in \(A\).\\
    We represent this as \[A \subsetneq B\]
\end{definition}

\subsection{Superset}
\para{If \(A\) is a subset of \(B\), then \(B\) is a superset of \(A\). It is represented as\[
    B \supseteq A\]
    This is another way of saying the same thing. The definition of proper superset is also
    the same.}


\subsection{Equality}

\para{We may say that two sets are equal if they have the same elements.
However, there is one thing to consider. When discussing about sets, we do not care how many
times an element appears. The sets \(\{1, 1, 1, 1, \ldots\}\) is the same as \(\{1\}\).
Therefore, we always ignore the repeating terms. Thus, two sets are equal if they have the same 
number of elements, \textbf{regardless of the frequency they appear in.} The same thing applies
for subsets and supersets, \textbf{the frequency of elements does not matter.}}

\para{We can phrase the equality of sets in another manner, in terms of subsets.}

\begin{definition}
    [Equality in sets]
    Two sets \(A\) and \(B\) are equal if and only if \(A \subseteq B\) and \(B \subseteq A\).\\
    Therefore, \[A = B \iff A \subseteq B \text{ and } B \subseteq A\]
\end{definition}