\chapter{Elementary Prereqs}
\section{Intervals}

\marginpar{\raggedleft The notation \(\left]\right[\) is French, and is less commonly used in India, USA, etc.}

\begin{definition}
    For \(a, b, x \in \RR \) and \(a < b\),
    \begin{enumerate}
        \ii The subset \(\{x \mid a < x < b\}\) of \(\RR\) is called the open interval of \(a, b\)
        and is represented as \((a,b)\) or \(]a,b[\).
        \ii The subset \(\{x \mid a \le x \le b \}\) of \(\RR\) is called the closed interval
        of \(a, b\) and is represented as \([a,b]\).    
    \end{enumerate}
    
\end{definition}

\todo{Add figures for all}

\begin{figure}[ht]
    \centering
    \begin{subfigure}{0.31\textwidth}
        \begin{tikzpicture}
            \begin{axis}[funcgraph]
            \end{axis} 
            \draw[latex-] (-1,0) -- (9,0) ;
            \draw[-latex] (-1,0) -- (9,0) ;
            \foreach \x in  {0,1,2,3,4,5,6,7,8} \draw[shift={(\x,0)},color=black] (0pt,3pt) -- (0pt,-3pt);
            \foreach \x in {0,1,2,3,4,5,6,7,8} \draw[shift={(\x,0)},color=black] (0pt,0pt) -- (0pt,-3pt) node[below] {$\x$};
            \draw[*-o] (2.9,0) -- (5.1,0);
            \draw[very thick] (2.98,0) -- (5,0);
        \end{tikzpicture}    
    \end{subfigure}
    
    \caption{Closed-Open Interval}
\end{figure}

\para{We can make all kind of closed-open (\(a \le x < b\)) or open-closed (\(a < x \le b\))
intervals, but there is no point in giving them a name.}

\section{Inequalities}

\para{Real numbers and all its subsets have a common property. They're ordered under the
\(<, >, =\) relations. Thus, for any two real numbers, we can always compare them and have the
trichotomy :}
\begin{claim}
    For \(a, b \in \RR\) we will always have, 
    \begin{enumerate}
        \ii \(a < b\),
        \ii \(a = b\), or
        \ii \(a > b\).
    \end{enumerate}
\end{claim}

\para{This why inequalities are so useful for reals! We can always form inequalities 
for equations. Now, let us propose the following properties of inequalities.}

\begin{proposition}
    For \(a, b \in \RR\),
    \begin{enumerate}
        \ii \(a < b \iff a\pm k < b \pm k \)
        \ii \(a < b, c < d \iff a + c < b + d\)
        \ii \(a < b, c < d \iff a - d < b - c\)
        \ii \(a < b, k > 0 \implies ka<kb\)
        \ii \(a < b, k < 0 \implies ka>kb\) 
        \ii \(a < b, b < c \implies a < c\)
        \ii \(0 < a < b, r > 0 \implies a^r < b^r\)
        \ii \(0 < a < b, r < 0 \implies a^r > b^r\)
        \ii \(a > 0 \iff a + \dfrac{1}{a} \ge 2\)
        \ii \(a < 0 \iff a + \dfrac{1}{a} \le 2\)         
    \end{enumerate}
\end{proposition}

\subsection{Inequalities concerning squares}

\para{One thing of note to consider is how do the inequalities of \(x\) from
\(x^2\) follow? Consider \( a \le x \le b\). You might expect that for \(a \le x \le b\),
\(a^2 \le x \le b^2\). But that is simply incorrect! What we instead get is that \(x^2\) lies
between \(0\) and \(\max(a^2,b^2)\). Here, max represents that \(x^2\) lies between \(0\) and
whichever of the \(a^2\), \(b^2\) is greater.
In-fact, if \(a \le x \le b \iff a^2 \le x^2 \le b^2 \), then \(a\), \(b\) and hence \(x\)
must be \(\ge 0\).}

\begin{proposition}
    For \(x, a \in \RR\),\\
    If \(x^2 \le a^2\), then \(x \in [-a,a]\).\\
    And, if \(x^2 \ge a^2\), then \(x \in (-\infty,-a] \cup [a, \infty]\).   
\end{proposition}


\begin{example}
    For \(-5 \le x \le 4\), we have \( 0 \le x^2 \le \max(-5^2, 4^2)\). Therefore, \(0 \le 
    x^2 \le 25\).
\begin{figure}[H]
    \centering
    \begin{tikzpicture}[scale=0.86]
        \begin{axis}[funcgraph]
            \draw [red] (axis cs:-5,-1000)--(axis cs:-5,25);
            \draw [red] (axis cs:4,-1000)--(axis cs:4,16);
            \addplot[color=black,domain=-6:-5]{x^2};
            \addplot[color=red,domain=-5:4]{x^2};
            \addplot[color=black,domain=4:6]{x^2};
        \end{axis}
    \end{tikzpicture}
    \caption{ {\color{red}\(x^2\)} for \(-5 \le x \le 4\)}
\end{figure}
\end{example}

\section{Sign Scheme Method}

\index{Sign Scheme Method}

\para{A way of determining inequalities of the form \(F(x) = (x-a_1)^{k_1}(x-a_2)^{k_2}\dots(x-a_n)^{k_n}\)
where \(F(x) \le 0\), or \(F(x) \ge 0\) is the \emph{Sign Scheme Method}. Arrange
\(a_1,a_2,\dots,a_n\) such that \(a_1 \le a_2 \le \dots \le a_{n-1} \le a_n\).
Now follow the following steps:}

\begin{enumerate}
    \ii Mark \(a_1,a_2,\dots,a_n\) on the number line, and put the plus sign on the right of the
    largest of these, \(a_n\). Since when \(x \ge a_n\), \(x \ge a_1, a-2, \dots\). So \(F(x) \ge 0\).
    \ii Put the plus sign on the interval to left of \(a_n\) if \(k_n\) is even, and the minus 
    sign if it is odd. Similarly, when passing to the left of \(a_i\), plus sign if \(k_i\)
    is even, minus otherwise.
    \ii The union of all intervals with the plus sign is the solution for \(F(x) \ge 0\), and
    the union of the ones with minus sign, for \(F(x) \le 0\). 
\end{enumerate}

\begin{example}
    Consider \(\dfrac{2y-1}{y} \le 0\). First, we change it into the already known form,
    \begin{align}
        \dfrac{2y-1}{y} &\le 0\\
        2\dfrac{(y-1/2)}{y} &\le 0\\
        \dfrac{y-1/2}{y} &\le 0\\
        (y-1/2)(y-0)^{-1} &\le 0
    \end{align}
    Plotting these we get,
    \begin{figure}[H]
        \centering
        \incfig{signscheme}
        \caption{Sign Scheme Method in action}
    \end{figure}

    Since We are concerned with \(\dfrac{2y-1}{y} \le 0\) and can't have \(y=0\),
    \(y \in \left(0,\dfrac{1}{2}\right]\).
\end{example}