\chapter{Preface}
\section{The Philosophy}

\para{First, before jumping to any content, I shall discuss the philosophy behind these notes.
These notes are made because I did not like pen-paper notes, these have version control, can be
easily rewritten are more accessible(on phone/pc and if needed, on print).}
\para{Another thing to note is that these notes follow a specific philosophy, in that they try to be expository.
They are meant to explain things. I have noticed that books often talk circularly, create redundancies, and etc.}
\parbreak
\para{To combat this, I have tried to clearly lay out everything here. Also, because of how books are made,
generally in various volumes, content often goes repeated. These notes will try to remove that.}
\para{Lastly, I have often added some sort of additional non-JEE relevant content in many places, because \begin{inparaenum}[1)]
    \ii JEE is not the only thing I care for,
    \ii I am particularly interested in Mathematics/Physics so I study extra stuff,
    \ii They often help in exposition and better understanding.
    \end{inparaenum} 
    These are, however, mentioned in the text, and you can easily skip them.
    }
\parbreak

\para{Happy learning!}